\subsection{Generalised Laplace Equation} 

\subsection{Two Dimensional Anisotropic Solution}

Consider the problem shown in \figref{fig:femanisoplate}.

%\epstexfigure{AnalyticSolutions/ClassicalFieldClass/svgs/anisoplate.eps_tex}{Anisotropic plate solution
%  domain.}{Anisotropic plate solution domain.}{fig:anisoplate}{0.75}

For this test problem the conductivity tensor in fibre coordinates is
\begin{equation}
  \tensor{\sigma}^{*}=\begin{bmatrix}
    \sigma_{t} & 0 \\ 
    0 & \sigma_{n}
  \end{bmatrix}
\end{equation}

The conductivity tensor in  reference coordinates can be calculated from the
conductivity tensor in  fibre coordinates by the tensor transformation rule
$\tensor{\sigma}=\transpose{\tensor{Q}}\tensor{\sigma}^{*}\tensor{Q}$ where
$\tensor{Q}$ is given by
\begin{equation}
  \tensor{Q}=\begin{bmatrix}
    \cos(-\theta) & -\sin(-\theta) \\
    \sin(-\theta) & \cos(-\theta)
  \end{bmatrix} = \begin{bmatrix}
    \cos\theta & +\sin\theta \\
    -\sin\theta & \cos\theta
  \end{bmatrix}
\end{equation}
Hence
\begin{equation}
  \tensor{\sigma}=\begin{bmatrix}
    \sigma_{t}\cos^{2}\theta+\sigma_{n}\sin^{2}\theta & 
    \pbrac{\sigma_{t}-\sigma_{n}}\sin\theta\cos\theta \\
    \pbrac{\sigma_{t}-\sigma_{n}}\sin\theta\cos\theta &
    \sigma_{t}\sin^{2}\theta+\sigma_{n}\cos^{2}\theta
  \end{bmatrix} = \begin{bmatrix}
    \sigma_{11} & \sigma_{12} \\
    \sigma_{12} & \sigma_{22} 
  \end{bmatrix}
  \label{eqn:conductivtytransformation}
\end{equation}

The generalised Laplace equation for this problem in reference coordinates is
hence given by
\begin{equation}
  \divergence{}{\pbrac{\tensor{\sigma}\gradient{}\Phi}}=\sigma_{11}\deltwosqby{\Phi}
  {x}+2\sigma_{12}\deltwoby{\Phi}{x}{y}+\sigma_{22}\deltwosqby{\Phi}{y}
  =0
  \label{eqn:2DrcgenLaplace}
\end{equation}

\Eqnref{eqn:2DrcgenLaplace} cannot be solved using traditional eigenfunction
expansions as it is non-separable. In fact, for an arbitrary domain and
arbitrary boundary conditions, \eqnref{eqn:2DrcgenLaplace} is extremely
difficult to solve. It can be solved, however, by realising that from the
definition of $\sigma_{ij}$ in \eqnref{eqn:conductivtytransformation} the
equation is elliptic everywhere in the solution domain. From
\citeasnoun{clements:1981}, a system of second order elliptic partial
differential equations is of the form
\begin{equation}
  a_{ijkl}\deltwoby{\Phi_{k}}{x_{j}}{x_{l}}=0
  \label{eqn:general2ndorderellipticpde}
\end{equation}
where $\Phi_{k}$ are complex-valued functions of the dependent variables
$x_{1}$ and $x_{2}$ and $i,k=1,\ldots,N$. The $a_{ijkl}$ are real constants
which satisfy the symmetry condition $a_{ijkl}=a_{klij}$ and, since the system
is elliptic, satisfy $a_{ijkl}\xi_{ij}\xi_{kl}>0$ for every non-zero $N\times
2$ real matrix $\xi_{ij}$.

Real solutions to \eqnref{eqn:general2ndorderellipticpde} may be written in
the form 
\begin{equation}
  \Phi_{k}=\dsum_{\alpha}A_{k\alpha}\fnof{f_{\alpha}}{z_{\alpha}}+
  \dsum_{\alpha}\overline{A_{k\alpha}}\overline{\fnof{f_{\alpha}}{z_{\alpha}}}
\end{equation}
where $\alpha=1,\hdots,N$, $\fnof{f_{\alpha}}{z_{\alpha}}$ are arbitrary
analytic functions of a complex variable $z_{\alpha}$,
$\overline{\fnof{f_{\alpha}}{z_{\alpha}}}$ the complex conjugate of the
function $\fnof{f_{\alpha}}{z_{\alpha}}$, $A_{k\alpha}$ are complex constants
(to be determined), $z_{\alpha}=x_{1}+\tau_{\alpha} x_{2}$ and $\tau_{\alpha}$
are complex constants given by
\begin{equation}
  \det\sqbrac{a_{i1k1}+a_{i1k2}\tau+a_{i2k1}\tau+a_{i2k2}\tau^{2}}=0
\end{equation}

Comparing \eqnref{eqn:2DrcgenLaplace} with
\eqnref{eqn:general2ndorderellipticpde} it can be seen that $N=1$ and
$a_{1i1j}=\sigma_{ij}$ and thus $\tau$ is given by
\begin{equation}
  \tau=\dfrac{-\sigma_{12}+\imath\sqrt{\sigma_{11}\sigma_{22}-\sigma_{12}^{2}}}
  {\sigma_{22}}=\dfrac{-\sigma_{12}}{\sigma_{22}}+\dfrac{\imath\sqrt{
      \sigma_{11}\sigma_{22}-\sigma_{12}^{2}}}{\sigma_{22}}=-\lambda_{1}+
  \imath\lambda_{2}
\end{equation}
where $\imath$ is the square root of $-1$. With this definition it can be seen
that
\begin{equation}
  z=x+\tau y=\pbrac{x-\lambda_{1}y}+\imath\lambda_{2}y
\end{equation}

From \citeasnoun{clements:1981} $A_{11}=1$ for this problem and hence a
solution to \eqnref{eqn:2DrcgenLaplace} can be found by specifying an
arbitrary analytic function $\fnof{f}{z}$.

For the test problem in this chapter the analytic function is chosen to be
$\fnof{f}{z}=e^{z}$. Thus the complete analytic solution is given by
\begin{equation}
  \begin{split}
    \fnof{\Phi}{x,y}&=\fnof{f}{z}+\overline{\fnof{f}{z}} \\
    &= e^{\pbrac{x-\lambda_{1}y}+\imath\lambda_{2}y}+
    \overline{e^{\pbrac{x-\lambda_{1}y}+\imath\lambda_{2}y}} \\
    &= e^{\pbrac{x-\lambda_{1}y}}\pbrac{\cos\pbrac{\lambda_{2}y}+\imath
      \sin\pbrac{\lambda_{2}y}}+e^{\pbrac{x-\lambda_{1}y}}
    \pbrac{\cos\pbrac{\lambda_{2}y}-\imath\sin\pbrac{\lambda_{2}y}} \\
    &= 2e^{x}e^{-\lambda_{1}{y}}\cos\pbrac{\lambda_{2}y}
  \end{split}
  \label{eqn:femanisoanalyticsol}
\end{equation}
