\section{Operations}
\label{sec:MathsOperations}

\subsection{Variation}
\label{subsec:VariationOperator}

Given a function $\fnof{\vectr{u}}{\vectr{x}}$  and another function
$\fnof{\vectr{u}^{*}}{\vectr{x}}$ which is only infinitesimally different from
the first function at every point $\vectr{x}$ then the variation of the
function is defined as
\begin{equation}
  \fnof{\variationdir{\vectr{u}}}{\vectr{x}}=\fnof{\vectr{u}^{*}}{\vectr{x}}-\fnof{\vectr{u}}{\vectr{x}}
\end{equation}

The variation of a function is an infinitessimal change in the function at
$\vectr{x}$ and is different from the derivative of a function at a point.

Some properties of the variation operator include
\begin{equation}
  \begin{split}
    \variationdir{\delby{\vectr{u}}{\vectr{x}}}&=\delby{\variationdir{\vectr{u}}}{\vectr{x}} \\
    \variationdir{\gint{\vectr{x}_{1}}{\vectr{x}_{2}}{\fnof{\vectr{u}}{\vectr{x}}}{\vectr{x}}}&=\gint{\vectr{x}_{1}}{\vectr{x}_{2}}{\variationdir{\fnof{\vectr{u}}{\vectr{x}}}}{\vectr{x}} 
  \end{split}
\end{equation}

Consider a function of $\vectr{u}$, $\fnof{\vectr{f}}{\vectr{x}}$, and a
variation of $\vectr{u}$, $\variationdir{\vectr{u}}$. The variation in the
function is given by
\begin{equation}
  \variation{\fnof{\vectr{f}}{\vectr{u}}}{\vectr{u}}=\fnof{\vectr{f}}{\vectr{u}+\variationdir{\vectr{u}}}-\fnof{\vectr{f}}{\vectr{u}}
\end{equation}

Now
\begin{equation}
  \fnof{\vectr{f}}{\vectr{u}+\epsilon\variationdir{\vectr{u}}}
\end{equation}

\begin{equation}
  \variation{\fnof{\vectr{f}}{\vectr{u}}}{\vectr{u}}=\directionalderiv{}{\fnof{\vectr{f}}{\vectr{u}}}{\variationdir{\vectr{u}}}=\evalat{\dby{}{\epsilon}\fnof{\vectr{f}}{\vectr{u}+\epsilon\variationdir{\vectr{u}}}}{\epsilon=0}
\end{equation}
  
\subsection{Linearisation}
\label{subsec:Linearisation}

A linearisation of a function $\fnof{f}{\vectr{x}}$ in the direction of
$\Delta\vectr{x}$ is
\begin{equation}
  \linearisation{f}{\vectr{x}}{\vectr{x}}=\dby{}{\epsilon}\evalat{\fnof{f}{\vectr{x}+\epsilon\linearisationdir{\vectr{x}}}}{\epsilon=0}=\fnof{f}{\vectr{x}}+\directionalderiv{\vectr{x}}{\fnof{f}{\vectr{x}}}{\linearisationdir{\vectr{x}}}
\end{equation}
 
\subsection{Tensor Operations}
\label{subsec:TensorOperations}

\subsubsection{Trace}
\label{subsubsec:Trace}

The trace of a tensor, $\tensor{A}$ (with respect to a metric, $\tensor{g}$) is given by
\begin{equation}
  \trace{\tensor{g}}{\tensor{A}}=\doubledotprod{\inverse{\tensor{g}}}{\tensor{A}}=g^{ij}A_{ij}
\end{equation}

The trace operator may also be written without the metric \ie
$\trace{}{\tensor{A}}$. Note that the trace of a tensor is invariant with
respect to a change of basis.
