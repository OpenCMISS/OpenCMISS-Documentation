 
\subsection{Directional derivative}
\label{subsec:DirectionalDerivativeOperator}

The directional derivative of a scalar function of a vector,
$\fnof{f}{\vectr{x}}$, with respect to $\vectr{x}$ along the vector direction
$\vectr{u}$ is defined by the limit
\begin{equation}
  \begin{split}
    \directionalderiv{\vectr{x}}{\fnof{f}{\vectr{x}}}{\vectr{u}}&=\genlimit{\epsilon}{0}\dfrac{\fnof{f}{\vectr{x}+\epsilon\vectr{u}}-\fnof{f}{\vectr{x}}}{\epsilon}
    \\ &=
    \evalat{\dfrac{d}{d\epsilon}\fnof{f}{\vectr{x}+\epsilon\vectr{u}}}{\epsilon=0}\\ &=
    \dotprod{\gradient{\vectr{x}}{\fnof{f}{\vectr{x}}}}{\vectr{u}} \\ &=
    \dotprod{\delby{\fnof{f}{\vectr{x}}}{\vectr{x}}}{\vectr{u}}
  \end{split} 
\end{equation} 
 
Some properties of directional derivatives of a scalar function of a vector include
\begin{equation}
  \begin{split}
    \directionalderiv{\vectr{x}}{\fnof{f_{1}}{\vectr{x}}+\fnof{f_{2}}{\vectr{x}}}{\vectr{u}}&=\directionalderiv{\vectr{x}}{\fnof{f_{1}}{\vectr{x}}}{\vectr{u}}+\directionalderiv{\vectr{x}}{\fnof{f_{2}}{\vectr{x}}}{\vectr{u}}\\
    &=\dotprod{\pbrac{\delby{\fnof{f_{1}}{\vectr{x}}}{\vectr{x}}+\delby{\fnof{f_{2}}{\vectr{x}}}{\vectr{x}}}}{\vectr{u}}\\
    \directionalderiv{\vectr{x}}{\fnof{f_{1}}{\vectr{x}}\fnof{f_{2}}{\vectr{x}}}{\vectr{u}}&=\directionalderiv{\vectr{x}}{\fnof{f_{1}}{\vectr{x}}}{\vectr{u}}\fnof{f_{2}}{\vectr{x}}+\fnof{f_{1}}{\vectr{x}}\directionalderiv{\vectr{x}}{\fnof{f_{2}}{\vectr{x}}}{\vectr{u}}\\
    &=\pbrac{\dotprod{\delby{\fnof{f_{1}}{\vectr{x}}}{\vectr{x}}}{\vectr{u}}}\fnof{f_{2}}{\vectr{x}}+\fnof{f_{1}}{\vectr{x}}\pbrac{\dotprod{\delby{\fnof{f_{2}}{\vectr{x}}}{\vectr{x}}}{\vectr{u}}}\\
    \directionalderiv{\vectr{x}}{\fnof{f_{1}}{\fnof{f_{2}}{\vectr{x}}}}{\vectr{u}}&=\directionalderiv{f_{2}}{\fnof{f_{1}}{\fnof{f_{2}}{\vectr{x}}}}{\directionalderiv{\vectr{x}}{\fnof{f_{2}}{\vectr{x}}}{\vectr{u}}}\\
    &=\delby{\fnof{f_{1}}{\fnof{f_{2}}{\vectr{x}}}}{\fnof{f_{2}}{\vectr{x}}}\pbrac{\dotprod{\delby{\fnof{f_{2}}{\vectr{x}}}{\vectr{x}}}{\vectr{u}}}
  \end{split}
\end{equation}

Note that the normal derivative of a function is just the directional
derivative in the normal direction \ie
\begin{equation}
  \delby{\fnof{f}{\vectr{x}}}{\vectr{n}}=\dotprod{\gradient{}{\fnof{f}{\vectr{x}}}}{\vectr{n}}=\directionalderiv{\vectr{x}}{\fnof{f}{\vectr{x}}}{\vectr{n}}
\end{equation}

For a vector valued function of a vector, $\fnof{\vectr{f}}{\vectr{x}}$, the
directional derivative in the direction of a vector $\vectr{u}$ is
\begin{equation}
  \begin{split}
    \directionalderiv{\vectr{x}}{\fnof{\vectr{f}}{\vectr{x}}}{\vectr{u}}&=\evalat{\dfrac{d}{d\epsilon}\fnof{\vectr{f}}{\vectr{x}+\epsilon\vectr{u}}}{\epsilon=0}\\
    &=\dotprod{\gradient{\vectr{x}}{\fnof{\vectr{f}}{\vectr{x}}}}{\vectr{u}}\\
    &=\dotprod{\delby{\fnof{\vectr{f}}{\vectr{x}}}{\vectr{x}}}{\vectr{u}}\\
  \end{split}
\end{equation}

Some properties of directional derivatives of a vector function of a vector include
\begin{equation}
  \begin{split}
    \directionalderiv{\vectr{x}}{\fnof{\vectr{f}_{1}}{\vectr{x}}+\fnof{\vectr{f}_{2}}{\vectr{x}}}{\vectr{u}}&=\directionalderiv{\vectr{x}}{\fnof{\vectr{f}_{1}}{\vectr{x}}}{\vectr{u}}+\directionalderiv{\vectr{x}}{\fnof{\vectr{f}_{2}}{\vectr{x}}}{\vectr{u}}\\
    &=\dotprod{\pbrac{\delby{\fnof{\vectr{f}_{1}}{\vectr{x}}}{\vectr{x}}+\delby{\fnof{\vectr{f}_{2}}{\vectr{x}}}{\vectr{x}}}}{\vectr{u}}\\
    \directionalderiv{\vectr{x}}{\crossprod{\fnof{\vectr{f}_{1}}{\vectr{x}}}{\fnof{\vectr{f}_{2}}{\vectr{x}}}}{\vectr{u}}&=\crossprod{\directionalderiv{\vectr{x}}{\fnof{\vectr{f}_{1}}{\vectr{x}}}{\vectr{u}}}{\fnof{\vectr{f}_{2}}{\vectr{x}}}+\crossprod{\fnof{\vectr{f}_{1}}{\vectr{x}}}{\directionalderiv{\vectr{x}}{\fnof{\vectr{f}_{2}}{\vectr{x}}}{\vectr{u}}}\\
    &=\crossprod{\pbrac{\dotprod{\delby{\fnof{\vectr{f}_{1}}{\vectr{x}}}{\vectr{x}}}{\vectr{u}}}}{\fnof{\vectr{f}_{2}}{\vectr{x}}}+\crossprod{\fnof{\vectr{f}_{1}}{\vectr{x}}}{\pbrac{\dotprod{\delby{\fnof{\vectr{f}_{2}}{\vectr{x}}}{\vectr{x}}}{\vectr{u}}}}\\
    \directionalderiv{\vectr{x}}{\fnof{\vectr{f}_{1}}{\fnof{\vectr{f}_{2}}{\vectr{x}}}}{\vectr{u}}&=\directionalderiv{\vectr{f}_{2}}{\fnof{\vectr{f}_{1}}{\fnof{\vectr{f}_{2}}{\vectr{x}}}}{\directionalderiv{\vectr{x}}{\fnof{\vectr{f}_{2}}{\vectr{x}}}{\vectr{u}}}\\
    &=\dotprod{\delby{\fnof{\vectr{f}_{1}}{\fnof{\vectr{f}_{2}}{\vectr{x}}}}{\fnof{\vectr{f}_{2}}{\vectr{x}}}}{\pbrac{\dotprod{\delby{\vectr{f}_{2}}{\vectr{x}}}{\vectr{u}}}}
  \end{split}
\end{equation}

For a scalar valued function of a second order tensor, $\fnof{f}{\tensor{A}}$,
the directional derivative in the the direction of a second order tensor
$\tensor{U}$ is
\begin{equation}
  \begin{split}
    \directionalderiv{\tensor{A}}{\fnof{f}{\tensor{A}}}{\vectr{U}}&=\evalat{\dfrac{d}{d\epsilon}\fnof{f}{\tensor{A}+\epsilon\tensor{U}}}{\epsilon=0}\\
    &=\doubledotprod{\gradient{\tensor{A}}{\fnof{f}{\tensor{A}}}}{\tensor{U}}\\
    &=\doubledotprod{\delby{\fnof{f}{\tensor{A}}}{\tensor{A}}}{\tensor{U}}\\
  \end{split}  
\end{equation}

Some properties of directional derivatives of a scalar function of a second
order tensor include
\begin{equation}
  \begin{split}
    \directionalderiv{\tensor{A}}{\fnof{f_{1}}{\tensor{A}}+\fnof{f_{2}}{\tensor{A}}}{\tensor{U}}&=\directionalderiv{\tensor{A}}{\fnof{f_{1}}{\tensor{A}}}{\tensor{U}}+\directionalderiv{\tensor{A}}{\fnof{f_{2}}{\tensor{A}}}{\tensor{U}}\\
    &=\doubledotprod{\pbrac{\delby{\fnof{f_{1}}{\tensor{A}}}{\tensor{A}}+\delby{\fnof{f_{2}}{\tensor{A}}}{\tensor{A}}}}{\tensor{U}}\\
    \directionalderiv{\tensor{A}}{\fnof{f_{1}}{\tensor{A}}\fnof{f_{2}}{\tensor{A}}}{\tensor{U}}&=\directionalderiv{\vectr{x}}{\fnof{f_{1}}{\vectr{x}}}{\vectr{u}}\fnof{f_{2}}{\vectr{x}}+\fnof{f_{1}}{\vectr{x}}\directionalderiv{\vectr{x}}{\fnof{f_{2}}{\vectr{x}}}{\vectr{u}}\\
    &=\pbrac{\doubledotprod{\delby{\fnof{f_{1}}{\tensor{A}}}{\tensor{A}}}{\tensor{U}}}\fnof{f_{2}}{\tensor{A}}+\fnof{f_{1}}{\tensor{A}}\pbrac{\doubledotprod{\delby{\fnof{f_{2}}{\tensor{A}}}{\tensor{A}}}{\tensor{U}}} \\
    \directionalderiv{\tensor{A}}{\fnof{f_{1}}{\fnof{f_{2}}{\tensor{A}}}}{\tensor{U}}&=\directionalderiv{f_{2}}{\fnof{f_{1}}{\fnof{f_{2}}{\tensor{A}}}}{\directionalderiv{\tensor{A}}{\fnof{f_{2}}{\tensor{A}}}{\tensor{U}}}\\
    &=\delby{\fnof{f_{1}}{\fnof{f_{2}}{\tensor{A}}}}{\fnof{f_{2}}{\tensor{A}}}\pbrac{\doubledotprod{\delby{\fnof{f_{2}}{\tensor{A}}}{\tensor{A}}}{\tensor{U}}}
  \end{split}
\end{equation}

For a second order tensor valued function of a second order tensor, $\fnof{F}{\tensor{A}}$,
the directional derivative in the the direction of a second order tensor
$\tensor{U}$ is
\begin{equation}
  \begin{split}
    \directionalderiv{\tensor{A}}{\fnof{\tensor{F}}{\tensor{A}}}{\vectr{U}}&=\evalat{\dfrac{d}{d\epsilon}\fnof{\tensor{F}}{\tensor{A}+\epsilon\tensor{U}}}{\epsilon=0}\\
    &=\doubledotprod{\gradient{\tensor{A}}{\fnof{\tensor{F}}{\tensor{A}}}}{\tensor{U}}\\
    &=\doubledotprod{\delby{\fnof{\tensor{F}}{\tensor{A}}}{\tensor{A}}}{\tensor{U}}\\
  \end{split}  
\end{equation}

Some properties of directional derivatives of a second order tensor function
of a second order tensor include
\begin{equation}
  \begin{split}
    \directionalderiv{\tensor{A}}{\fnof{\tensor{F}_{1}}{\tensor{A}}+\fnof{\tensor{F}_{2}}{\tensor{A}}}{\tensor{U}}&=\directionalderiv{\tensor{A}}{\fnof{\tensor{F}_{1}}{\tensor{A}}}{\tensor{U}}+\directionalderiv{\tensor{A}}{\fnof{\tensor{F}_{2}}{\tensor{A}}}{\tensor{U}}\\
    &=\doubledotprod{\pbrac{\delby{\fnof{\tensor{F}_{1}}{\tensor{A}}}{\tensor{A}}+\delby{\fnof{\tensor{F}_{2}}{\tensor{A}}}{\tensor{A}}}}{\tensor{U}}\\
    \directionalderiv{\tensor{A}}{\dotprod{\fnof{\tensor{F}_{1}}{\tensor{A}}}{\fnof{\tensor{F}_{2}}{\tensor{A}}}}{\tensor{A}}&=\dotprod{\directionalderiv{\tensor{A}}{\fnof{\tensor{F}_{1}}{\tensor{A}}}{\tensor{U}}}{\fnof{\tensor{F}_{2}}{\tensor{A}}}+\dotprod{\fnof{\tensor{F}_{1}}{\tensor{A}}}{\directionalderiv{\tensor{A}}{\fnof{\tensor{F}_{2}}{\tensor{A}}}{\tensor{U}}}\\
    &=\dotprod{\pbrac{\doubledotprod{\delby{\fnof{\tensor{F}_{1}}{\tensor{A}}}{\tensor{A}}}{\tensor{U}}}}{\fnof{\tensor{F}_{2}}{\tensor{A}}}+\dotprod{\fnof{\tensor{F}_{1}}{\tensor{A}}}{\pbrac{\doubledotprod{\delby{\fnof{\tensor{F}_{2}}{\tensor{A}}}{\tensor{A}}}{\tensor{U}}}}\\
    \directionalderiv{\tensor{A}}{\fnof{\tensor{F}_{1}}{\fnof{\tensor{F}_{2}}{\tensor{A}}}}{\tensor{U}}&=\directionalderiv{\tensor{F}_{2}}{\fnof{\tensor{F}_{1}}{\fnof{\tensor{F}_{2}}{\tensor{A}}}}{\directionalderiv{\tensor{A}}{\fnof{\tensor{F}_{2}}{\tensor{A}}}{\tensor{U}}}\\
    &=\doubledotprod{\delby{\fnof{\tensor{F}_{1}}{\fnof{\tensor{F}_{2}}{\tensor{A}}}}{\fnof{\tensor{F}_{2}}{\tensor{A}}}}{\pbrac{\doubledotprod{\delby{\tensor{F}_{2}}{\tensor{A}}}{\tensor{U}}}}
  \end{split}
\end{equation}


Consider the directional derivative of the determinant of a tensor,
$\tensor{A}$ in the direction of another tensor, $\tensor{U}$
\begin{equation}
  \begin{split}
    \directionalderiv{}{\determinant{\tensor{A}}}{\tensor{U}}&=\evalat{\dby{}{\epsilon}\determinant{\tensor{A}+\epsilon\tensor{U}}}{\epsilon=0}\\
    &=\evalat{\dby{}{\epsilon}\determinant{\tensor{A}\pbrac{\tensor{I}+\epsilon\inverse{\tensor{A}}\tensor{U}}}}{\epsilon=0}\\
    &=\determinant{\tensor{A}}\evalat{\dby{}{\epsilon}\determinant{\pbrac{\tensor{I}+\epsilon\inverse{\tensor{A}}\tensor{U}}}}{\epsilon=0}
  \end{split}
\end{equation}

Now, the characteristic equation for a tensor, $\tensor{B}$ is
\begin{equation}
  \determinant{\pbrac{\tensor{B}-\lambda\tensor{I}}}=\pbrac{\lambda_{1}-\lambda}\pbrac{\lambda_{2}-\lambda}\pbrac{\lambda_{3}-\lambda}
\end{equation}
where $\lambda_{i}$ is the $\nth{i}$ eigenvalue of $\tensor{B}$.

We thus have
\begin{equation}
  \directionalderiv{}{\determinant{\tensor{A}}}{\tensor{U}}=\determinant{\tensor{A}}\evalat{\dby{}{\epsilon}\determinant{\pbrac{\pbrac{1+\epsilon\lambda_{1}}\pbrac{1+\epsilon\lambda_{2}}\pbrac{1+\epsilon\lambda_{3}}}}}{\epsilon=0}
\end{equation}
where $\lambda_{i}$ is the $\nth{i}$ eigenvalue of
$\inverse{\tensor{A}}\tensor{U}$. Now
\begin{equation}
  \begin{split}
    \directionalderiv{}{\determinant{\tensor{A}}}{\tensor{U}}&=\determinant{\tensor{A}}\evalat{\dby{}{\epsilon}\determinant{\pbrac{\pbrac{1+\epsilon\lambda_{1}}\pbrac{1+\epsilon\lambda_{2}}\pbrac{1+\epsilon\lambda_{3}}}}}{\epsilon=0}\\
    &=\determinant{\tensor{A}}\pbrac{\lambda_{1}+\lambda_{2}+\lambda_{3}}\\
    &=\determinant{\tensor{A}}\trace{\pbrac{\inverse{\tensor{A}}\tensor{U}}}\\
    &=\determinant{\tensor{A}}\doubledotprod{\invtranspose{\tensor{A}}}{\tensor{U}}
  \end{split}
\end{equation}

\subsection{Variation}
\label{subsec:VariationOperator}

Given a function $\fnof{\vectr{u}}{\vectr{x}}$  and another function
$\fnof{\vectr{u}^{*}}{\vectr{x}}$ which is only infinitesimally different from
the first function at every point $\vectr{x}$ then the variation of the
function is defined as
\begin{equation}
  \fnof{\variationdir{\vectr{u}}}{\vectr{x}}=\fnof{\vectr{u}^{*}}{\vectr{x}}-\fnof{\vectr{u}}{\vectr{x}}
\end{equation}

The variation of a function is an infinitessimal change in the function at
$\vectr{x}$ and is different from the derivative of a function at a point.

Some properties of the variation operator include
\begin{equation}
  \begin{split}
    \variationdir{\delby{\vectr{u}}{\vectr{x}}}&=\delby{\variationdir{\vectr{u}}}{\vectr{x}} \\
    \variationdir{\gint{\vectr{x}_{1}}{\vectr{x}_{2}}{\fnof{\vectr{u}}{\vectr{x}}}{\vectr{x}}}&=\gint{\vectr{x}_{1}}{\vectr{x}_{2}}{\variationdir{\fnof{\vectr{u}}{\vectr{x}}}}{\vectr{x}} 
  \end{split}
\end{equation}

Consider a function of $\vectr{u}$, $\fnof{\vectr{f}}{\vectr{x}}$, and a
variation of $\vectr{u}$, $\variationdir{\vectr{u}}$. The variation in the
function is given by
\begin{equation}
  \variation{\fnof{\vectr{f}}{\vectr{u}}}{\vectr{u}}=\fnof{\vectr{f}}{\vectr{u}+\variationdir{\vectr{u}}}-\fnof{\vectr{f}}{\vectr{u}}
\end{equation}

Now
\begin{equation}
  \fnof{\vectr{f}}{\vectr{u}+\epsilon\variationdir{\vectr{u}}}
\end{equation}

\begin{equation}
  \variation{\fnof{\vectr{f}}{\vectr{u}}}{\vectr{u}}=\directionalderiv{}{\fnof{\vectr{f}}{\vectr{u}}}{\variationdir{\vectr{u}}}=\evalat{\dby{}{\epsilon}\fnof{\vectr{f}}{\vectr{u}+\epsilon\variationdir{\vectr{u}}}}{\epsilon=0}
\end{equation}
  
\subsection{Linearisation}
\label{subsec:Linearisation}

A linearisation of a function $\fnof{f}{\vectr{x}}$ in the direction of
$\Delta\vectr{x}$ is
\begin{equation}
  \linearisation{f}{\vectr{x}}{\vectr{x}}=\dby{}{\epsilon}\evalat{\fnof{f}{\vectr{x}+\epsilon\linearisationdir{\vectr{x}}}}{\epsilon=0}=\fnof{f}{\vectr{x}}+\directionalderiv{\vectr{x}}{\fnof{f}{\vectr{x}}}{\linearisationdir{\vectr{x}}}
\end{equation}
 
\subsection{Tensor Operations}
\label{subsec:TensorOperations}

\subsubsection{Trace}
\label{subsubsec:Trace}

The trace of a tensor, $\tensor{A}$ (with respect to a metric, $\tensor{g}$) is given by
\begin{equation}
  \trace{\tensor{g}}{\tensor{A}}=\doubledotprod{\inverse{\tensor{g}}}{\tensor{A}}=g^{ij}A_{ij}
\end{equation}

The trace operator may also be written without the metric \ie
$\trace{}{\tensor{A}}$. Note that the trace of a tensor is invariant with
respect to a change of basis.

For tensors $\tensor{A}$ and $\tensor{B}$ and a scalar $\phi$, some properties of the trace operator include:
\begin{align}
  \trace{}{\pbrac{\tensor{A}+\tensor{B}}}&=\trace{}{\tensor{A}}+\trace{}{\tensor{B}}\\
  \trace{}{\pbrac{\phi\tensor{A}}}&=\phi\trace{}{\tensor{A}} \\
  \trace{}{\tensor{A}}&=\trace{}{\transpose{\tensor{A}}} \\
  \trace{}{\pbrac{\tensor{A}\tensor{B}}}&=\trace{}{\pbrac{\tensor{B}\tensor{A}}}\\
  \trace{}{\pbrac{\tensorprod{\tensor{A}}{\tensor{B}}}}&=\trace{}{\tensor{A}}\trace{}{\tensor{B}}\\
  \doubledotprod{\tensor{A}}{\tensor{B}}&=\trace{}{\pbrac{\transpose{\tensor{A}}\tensor{B}}}=\trace{}{\pbrac{\tensor{A}\transpose{\tensor{B}}}}=\trace{}{\pbrac{\transpose{\tensor{B}}\tensor{A}}}=\trace{}{\pbrac{\tensor{B}\transpose{\tensor{A}}}}
\end{align}

\subsubsection{Symmetric and Skew-symmetric}
\label{subsubsec:SymmetricSkew}

Consider a second order tensor, $\tensor{A}$. Any second order tensor can be
split into two parts \ie
\begin{align}
  \tensor{A}&=\frac{1}{2}\pbrac{\tensor{A}+\transpose{\tensor{A}}}+\frac{1}{2}\pbrac{\tensor{A}-\transpose{\tensor{A}}}
  \notag \\
  &=\symmetric{\tensor{A}}+\skewsym{\tensor{A}}
\end{align}
as first part of $\dfrac{1}{2}\pbrac{\tensor{A}+\transpose{\tensor{A}}}$ is a
strictly symmetric tensor and the second part of
$\dfrac{1}{2}\pbrac{\tensor{A}-\transpose{\tensor{A}}}$ is a strictly
skew-symmetric tensor. Here $\symmetric{}$ is the symmetric operation and $\skewsym{}$ the
skew-symmetric operator.

For second order tensors $\tensor{A}$ and $\tensor{B}$, some properties of a symmetric and skew-symmetric decomposition include:
\begin{align}
  \transpose{\pbrac{\symmetric{\tensor{A}}}}&=\symmetric{\tensor{A}} \\
  \transpose{\pbrac{\skewsym{\tensor{A}}}}&=-\skewsym{\tensor{A}} \\
  \determinant{\pbrac{\skewsym{\tensor{A}}}}&=0 \\
  \trace{}{\pbrac{\symmetric{\tensor{A}}\skewsym{\tensor{A}}}}&=0\\
  \doubledotprod{\symmetric{\tensor{A}}}{\skewsym{\tensor{A}}}&=0\\
  \doubledotprod{\tensor{A}}{\tensor{B}}&=\doubledotprod{\pbrac{\symmetric{\tensor{A}}+\skewsym{\tensor{A}}}}{\pbrac{\symmetric{\tensor{B}}+\skewsym{\tensor{B}}}} \\
  &=\doubledotprod{\symmetric{\tensor{A}}}{\symmetric{\tensor{B}}}+\doubledotprod{\skewsym{\tensor{A}}}{\skewsym{\tensor{B}}}
\end{align}
as the double contraction of a symmetric and skew-symmetric tensor is zero.

\subsubsection{Deviatoric and Spherical}
\label{subsubsec:DeviatoricSpherical}

Consider a second order tensor, $\tensor{A}$, on a manifold with a metric,
$\tensor{g}$. Any second order tensor can be split into two parts \ie
\begin{align}
  \tensor{A}&=\pbrac{\frac{1}{3}\trace{\tensor{g}}{\tensor{A}}}\tensor{I}+\pbrac{\tensor{A}-\pbrac{\frac{1}{3}\trace{\tensor{g}}{\tensor{A}}}\tensor{I}}
  \notag \\
  &=\spherical{\tensor{g}}{\tensor{A}}+\deviatoric{\tensor{g}}{\tensor{A}}
\end{align}

The spherical and deviatoric operators may also be written without the
optional metric \ie $\spherical{}{\tensor{A}}$ and $\deviatoric{}{\tensor{A}}$.

For tensors $\tensor{A}$ and $\tensor{B}$ and a scalar $c$, some properties of
the spherical and deviatoric operators include:
\begin{align}
  \determinant{\pbrac{\deviatoric{}{\tensor{A}}}}&=1\\
  \trace{}{\pbrac{\deviatoric{}{\tensor{A}}}}&=0\\
  \spherical{}{\pbrac{\deviatoric{}{\tensor{A}}}}&=0\\
  \doubledotprod{\deviatoric{}{\tensor{A}}}{\spherical{}{\tensor{B}}}&=0 \\
  \doubledotprod{\tensor{A}}{\tensor{B}}&=\doubledotprod{\pbrac{\spherical{}{\tensor{A}}+\deviatoric{}{\tensor{A}}}}{\pbrac{\spherical{}{\tensor{B}}+\deviatoric{}{\tensor{B}}}}
  \notag \\
  &=\doubledotprod{\spherical{}{\tensor{A}}}{\spherical{}{\tensor{B}}}+\doubledotprod{\deviatoric{}{\tensor{A}}}{\deviatoric{}{\tensor{B}}}
\end{align}
as the double contraction of a spherical and deviatoric tensor is zero.
