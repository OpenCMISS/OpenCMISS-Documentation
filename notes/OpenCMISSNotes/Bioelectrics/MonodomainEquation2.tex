\section{Monodomain Equation}
\label{sec:MonodomainEquation}

Under certain conditions the bidomain equations given in
\eqnref{eqn:BidomainEquation1} and \eqnref{eqn:BidomainEquation2} can
be simplified. If we assume that the intra- and extra-cellular
conductivity tensors have equal anisotropy ratios \ie
\begin{equation}
  \extracellularconductivitytensor=\lambda\intracellularconductivitytensor
  \label{eqn:MonodomainConductivityAssumption}
\end{equation}
where $\lambda$ is the conductivity scaling, we can substitute
\eqnref{eqn:MonodomainConductivityAssumption} into
\eqnref{eqn:FirstBidomainEquation} to obtain
\begin{equation}
  \begin{aligned}
    \divergence{}{\intracellularconductivitytensor\gradient{}{\transmembranevoltage}}&=
    -\divergence{}{\pbrac{\pbrac{\intracellularconductivitytensor+\extracellularconductivitytensor}
      \gradient{}{\extracellularpotential}}}\\&=
    -\divergence{}{\pbrac{\pbrac{\intracellularconductivitytensor+\lambda\intracellularconductivitytensor}
      \gradient{}{\extracellularpotential}}}\\
    &=-\divergence{}{\pbrac{\pbrac{1+\lambda}\intracellularconductivitytensor\gradient{}{\extracellularpotential}}}
  \end{aligned}
\end{equation}
or
\begin{equation}
  \divergence{}{\pbrac{\intracellularconductivitytensor\gradient{}{\extracellularpotential}}}=
  \dfrac{-1}{1+\lambda}\divergence{}{\intracellularconductivitytensor\gradient{}{\transmembranevoltage}}
  \label{eqn:MonodomainFirstBidomainEquation}
\end{equation}

If we now substitute \eqnref{eqn:MonodomainFirstBidomainEquation} into \eqnref{eqn:SecondBidomainEquation} we obtain
\begin{equation}
  \begin{aligned}
    \membraneareavolumeratio\pbrac{\membranecapacitance\delby{\transmembranevoltage}{t}+\ioniccurrent}&=
    \divergence{}{\intracellularconductivitytensor\gradient{}{\transmembranevoltage}} +
    \dfrac{-1}{1+\lambda}\divergence{}{\intracellularconductivitytensor\gradient{}{\transmembranevoltage}} \\
    &=\dfrac{\pbrac{1+\lambda}\divergence{}{\intracellularconductivitytensor\gradient{}{\transmembranevoltage}} -
      \divergence{}{\intracellularconductivitytensor\gradient{}{\transmembranevoltage}}}{1+\lambda} \\
    &=\dfrac{\divergence{}{\intracellularconductivitytensor\gradient{}{\transmembranevoltage}}+
      \lambda\divergence{}{\intracellularconductivitytensor\gradient{}{\transmembranevoltage}} -
      \divergence{}{\intracellularconductivitytensor\gradient{}{\transmembranevoltage}}}{1+\lambda} \\
    &=\dfrac{\lambda}{1+\lambda}\divergence{}{\intracellularconductivitytensor\gradient{}{\transmembranevoltage}}\\
    &=\divergence{}{\pbrac{\dfrac{\lambda}{1+\lambda}\intracellularconductivitytensor\gradient{}{\transmembranevoltage}}}
  \end{aligned}
\end{equation}
We can now define a \emph{monodomain conductivity tensor}\index{Monodomain conductivity tensor}, $\monodomainconductivitytensor$, as
\begin{equation}
  \monodomainconductivitytensor=\dfrac{\lambda\intracellularconductivitytensor}{1+\lambda}=
  \dfrac{\intracellularconductivitytensor\extracellularconductivitytensor}{\intracellularconductivitytensor+
    \extracellularconductivitytensor}
  \label{eqn:MonodomainConductivityTensor}  
\end{equation}

\Eqnref{eqn:MonodomainConductivityTensor} defines the monodomain
conductivity tensor as the arithmetic mean of the intra- and
extra-cellular conductivity tensors.

Subsituting \eqnref{eqn:MonodomainConductivityTensor} into
\eqnref{eqn:SecondBidomainEquation} and adding a stimulus current
gives us
\begin{equation}
  \addtolength{\fboxsep}{5pt}
  \boxed{
    \membraneareavolumeratio\membranecapacitance\delby{\transmembranevoltage}{t}-
    \divergence{}{\pbrac{\tensor{\sigma}_{i}\grad{\transmembranevoltage}}}=
    -\membraneareavolumeratio\ioniccurrent+\stimuluscurrent
  }
  \label{eqn:MonodomainEquation}
\end{equation}
where $\stimuluscurrent$ is the monodomain stimulus current.

\Eqnref{eqn:MonodomainEquation} is known as the \emph{monodomain equation}\index{Monodomain equation}.
