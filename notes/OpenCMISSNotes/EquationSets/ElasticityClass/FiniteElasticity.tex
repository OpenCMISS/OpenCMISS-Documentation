x\subsection{Finite Elasticity}
\label{subsec:FiniteElasticity}

%Deformation can be viewed three ways: A point to point transformation; A
%coordinate transformation; or as a transformation of metrics (convected coordinates).

\subsubsection{Kinematics}

As shown in \figref{fig:configurationsetting}, consider a \emph{material
  body} which is a three-dimensional smooth manifold with a boundary, $\manifold{B}$,
which consists of a set of points which are refered to as \emph{material
  points}. Consider also an ambient space manifold,
$\manifold{S}\in\rntopology{n}$. The material body is only accessible to
the observer when it moves through the ambient space. This motion is a
time-dependent embedding on the material body into the ambient space. The
embedding is known as a \emph{placement of the body}. It is given by the
mapping
\begin{equation}
  \mapping{\fnof{\kappa}{\mathcal{X},t}}{\manifold{B}}{\manifold{S}}
\end{equation}

The embedded submanifold occupying a location in the ambient space is is
called a \emph{configuration} of $\manifold{B}$ and is given by
\begin{equation}
  \embedmanifold{B}_{t}=\fnof{\kappa_{t}}{\manifold{B}}=\fnof{\kappa}{\manifold{B},t}
\end{equation}

The customary (but not necessary) \emph{reference placement} is given by
\begin{equation}
  \mapping{\kappa_{0}}{\manifold{B}}{\manifold{S}}
\end{equation}
and the region of space occupied by the reference placement \ie the
\emph{reference configuration} is given by
\begin{equation}
  \embedmanifold{B}_{0}=\fnof{\kappa_{0}}{\manifold{B}}
\end{equation}
Points in $\embedmanifold{B}_{0}$ are denoted by capital letters \ie $X, Y,
\ldots$. Points in $\embedmanifold{B}$ are denoted by lower case leters \ie
$x, y, \dots$.

\epstexfigure{svgs/EquationSets/Elasticity/FiniteElasticity/setup.eps_tex}{}{}{fig:configurationsetting}{0.75}

A new configuration of $\manifold{B}$ is given by the deformation mapping
\begin{equation}
  \mapping{\chi}{\embedmanifold{B}}{\rntopology{3}}
\end{equation}
where a configuration represents a deformed state of the body. As the body
moves we obtain a family of configurations. If we hold $X\in\embedmanifold{B}$
fixed can write $\fnof{V_{t}}{X}=\fnof{V}{X,t}$. We then have
\begin{equation}
  \fnof{V_{t}}{X}=\fnof{V}{X,t}=\delby{\fnof{\chi}{X,t}}{t}=\dby{\fnof{\chi_{X}}{t}}{t}
\end{equation}

Here $V_{t}$ is called the \emph{material velocity} of the motion. The
\emph{material acceleration} of the body is defined as
\begin{equation}
  \fnof{A_{t}}{X}=\fnof{A}{X,t}=\delby{\fnof{V}{X,t}}{t}=\dby{\fnof{V_{X}}{t}}{t}
\end{equation}

The \emph{spatial velocity} of the motion is defined by $v_{t}$ and the
\emph{spatial acceleration} of the motion is defined by $a_{t}$.

FIX BELOW

Consider the following line, area and volume forms in the reference configuration given by
\begin{align}
  \exteriorderiv{\covectr{X}}&=\sqrt{\determinant{G_{IJ}}}dX^{1} \\
  \exteriorderiv{V}&=\sqrt{\determinant{G_{IJ}}}\wedgeprod{\wedgeprod{dX^{1}}{dX^{2}}}{dX^{3}}
\end{align}
the corresponding volume form in the current configuration
\begin{equation}
  \exteriorderiv{v}=\sqrt{\determinant{g_{ij}}}\wedgeprod{\wedgeprod{dx^{1}}{\cdots}}{dx^{n}}
\end{equation}
is given by
\begin{equation}
  \pullback{\chi}{\exteriorderiv{v}}=J\exteriorderiv{V}
\end{equation}
where $J$ is the \emph{Jacobian} of the mapping and is given by
\begin{equation}
  \fnof{J}{X}=\sqrt{\dfrac{\determinant{g_{ij}}}{\determinant{G_{IJ}}}}\determinant{\pbrac{\delby{\fnof{\chi^{i}}{X}}{X^{I}}}}
\end{equation}

\subsubsection{Deformation Gradient}

Let
$\mapping{\chi}{\embedmanifold{B}_{0}}{\fnof{\chi}{\embedmanifold{B}_{0}}\subset\manifold{S}}$
be a deformation configuration of $\embedmanifold{B}$ in $\manifold{S}$. The
tangent of the mapping \ie $\tangentbundle{\chi}$ is denoted as $\tensor{F}$
and is called the \emph{deformation gradient} of $\chi$ \ie
$\tensor{F}=\tangentbundle{\chi}$. For $X\in\embedmanifold{B}_{0}$ we
have\symbolat{$\tensortwo{F}$}{Deformation gradient tensor}
\begin{equation}
  \tensor{F}_{X}=\mapping{\fnof{\tensor{F}}{X}}{\tangentspace{\embedmanifold{B}_{0}}{X}}{\tangentspace{\manifold{S}}{\fnof{\chi}{X}}}
\end{equation}
 
If $X^{I}$ and $x^{i}$ are the coordinates on $\embedmanifold{B}_{0}$ and
$\manifold{S}$ then the deformation gradient tensor with respect to the
coordinate bases are
\begin{equation}
  \fnof{\tensor{F}}{X}=\fnof{F^{i}_{I}}{X}\tensorprod{\vectr{g}_{i}}{\vectr{G}^{I}}=\delby{\fnof{\chi^{i}}{X}}{X^{I}}\tensorprod{\vectr{g}_{i}}{\vectr{G}^{I}}
\end{equation}

Note that $\tensor{F}$ is a two-point tensor. 

The polar decomposition

\begin{diagram}
 & & \tangentspace{B}{X} & & \\
 & \ruTo^{\tensor{U}} & & \rdTo^{\tensor{R}} \\
\tangentspace{B}{X} & & \rTo^{\tensor{F}} & & \tangentspace{S}{x}\\
 & \rdTo_{\tensor{R}} & & \ruTo_{\tensor{V}} \\
 & &  \tangentspace{S}{x} & &
\end{diagram}

is given by
\begin{equation}
  \tensor{F}=\dotprod{\tensor{R}}{\tensor{U}}=\dotprod{\tensor{V}}{\tensor{R}}
\end{equation}
where $\tensor{U}$ is the \emph{right stretch tensor}, $\tensor{V}$ is the
\emph{left stretch tensor} and $\tensor{R}$ is the \emph{rotation
  tensor}.\symbolat{$\tensor{U}$}{Right stretch
  tensor}\symbolat{$\tensor{V}$}{Left stretch
  tensor}\symbolat{$\tensor{R}$}{Rotation tensor}

If we let the deformed coordinates be given by the position vector,
$\fnof{\vectr{z}}{\vectr{x},t}$ then the deformation gradient tensor with
respect to the undeformed $\vectr{X}$ coordinates is given by
\begin{equation}
  \fnof{\tensor{F}}{\vectr{X}}=\delby{\vectr{z}}{\vectr{X}}
\end{equation}
or, in component form,
\begin{equation}
  F^{i}_{I}=\delby{z^{i}}{X^{I}}=\delby{z^{i}}{\xi^{r}}\delby{\xi^{r}}{X^{I}}
\end{equation}

The deformation gradient tensor can also be used to map lines, area and volume forms
between the reference and current configurations and vice versa \ie
\begin{align}
  \exteriorderiv{\covectr{x}} &= \tensor{F}\exteriorderiv{\covectr{X}} \\
  \exteriorderiv{\covectr{a}} &= J\invtranspose{\tensor{F}}\exteriorderiv{\covectr{A}} \\
  \exteriorderiv{v} &= J\exteriorderiv{V}
\end{align}
where $\exteriorderiv{\covectr{X}}$, $\exteriorderiv{\covectr{A}}$ and
$\exteriorderiv{V}$ are the line, area and volume forms in the reference
configuration and $\exteriorderiv{\covectr{x}}$, $\exteriorderiv{\covectr{a}}$ and
$\exteriorderiv{v}$ are the line, area and volume forms in the current
configuration. $J$ is the \emph{Jacobian} and is given by
\begin{equation}
  J=\sqrt{\dfrac{\determinant{\tensor{g}}}{\determinant{\tensor{G}}}}\determinant{\tensor{F}}
\end{equation}

The formula for mapping areas is known as \emph{Nanson's formula} and is
given by
\begin{equation}
  \covectr{n}\exteriorderiv{\covectr{a}}=J\pushforward{\chi}{\covectr{N}\exteriorderiv{\covectr{A}}}
\end{equation}
or
\begin{equation}
  \covectr{n}\exteriorderiv{\covectr{a}}=J\invtranspose{\tensor{F}}\covectr{N}\exteriorderiv{\covectr{A}}
\end{equation}
where $\covectr{N}$ and $\exteriorderiv{\covectr{A}}$ are the unit normal and
area form in the reference configuration,  $\covectr{n}$ and $\exteriorderiv{\covectr{a}}$ are the unit normal and
area form in the current configuration and $J$ is the Jacobian. 
  
\subsubsection{Strain and deformation tensors}

Strain and deformation tensors quantify the amount of strain or deformation
\ie the amount of stretch or distance between material points.  

There are three strain tensors in the \emph{material} coordinate system. The
\emph{right Cauchy-Green (or Green) deformation tensor}, $\tensor{C}$, is
defined by\symbolat{$\tensortwo{C}$}{Right Cauchy-Green (or Green) deformation
  tensor}
\begin{equation}
  \mapping{\fnof{\tensor{C}}{X}}{\tangentspace{\embedmanifold{B}}{X}}{\tangentspace{\embedmanifold{B}}{X}}
\end{equation}
as the pullback of the spatial metric tensor \ie $\fnof{\tensor{C}}{X}=\transpose{\fnof{\tensor{F}}{X}}\fnof{\tensor{g}}{x}\fnof{\tensor{F}}{X}$
or $\tensor{C}=\transpose{\tensor{F}}\tensor{g}\tensor{F}$ where $x=\fnof{\chi}{X}$. In terms of coordinates we
have
\begin{equation}
  \tensor{C}=C_{IJ}\tensorprod{\vectr{G}^{I}}{\vectr{G}^{J}}=g_{ij}F^{i}_{I}F^{j}_{J}\,\dotprodthree{\tensorprod{\vectr{G}^{I}}{\vectr{g}_{i}}}{\tensorprod{\vectr{g}^{i}}{\vectr{g}^{j}}}{\tensorprod{\vectr{g}_{j}}{\vectr{G}^{J}}}=\transpose{\tensor{F}}\tensor{g}\tensor{F}
\end{equation}

If $\tensor{C}$ is invertible we also have $\tensor{B}=\inverse{\tensor{C}}$
where $\tensor{B}$ is called the \emph{Piola deformation
  tensor}.\symbolat{$\tensortwo{B}$}{Piola deformation tensor}

We also have the \emph{Green-Lagrange strain tensor} is given by the difference
in metric tensors\symbolat{$\tensortwo{E}$}{Green-Lagrange strain tensor}
\begin{equation}
  \fnof{\tensor{E}}{X}=\frac{1}{2}\pbrac{\fnof{\tensor{C}}{X}-\tensor{G}}
\end{equation}
of, in component form
\begin{equation}
  \tensor{E}=E_{IJ}\tensorprod{\vectr{G}^{I}}{\vectr{G}^{J}}=\frac{1}{2}\pbrac{C_{IJ}-G_{IJ}}\tensorprod{\vectr{G}^{I}}{\vectr{G}^{J}}
\end{equation}

There are also three strain tensors in the \emph{spatial} coordinate
sytsem. The \emph{left Cauchy-Green (or Finger) deformation tensor},
$\tensor{b}$, is defined by\symbolat{$\tensortwo{b}$}{Left Cauchy-Green (or
  Finger) deformation tensor}
\begin{equation}
  \mapping{\fnof{\tensor{b}}{x}}{\tangentspace{\fnof{\chi}{\embedmanifold{B}}}{x}}{\tangentspace{\fnof{\chi}{\embedmanifold{B}}}{x}}
\end{equation}
as the push forward of the material metric tensor \ie
$\fnof{\tensor{b}}{x}=\fnof{\tensor{F}}{X}\fnof{\tensor{G}}{X}\transpose{\fnof{\tensor{F}}{X}}$
or $\tensor{b}=\tensor{F}\tensor{G}\transpose{\tensor{F}}$ where
$X=\fnof{\inverse{\chi}}{x}$. In terms of coordinates we have
\begin{equation}
  \tensor{b}=b^{ij}\tensorprod{\vectr{g}_{i}}{\vectr{g}_{j}}=G^{IJ}F^{i}_{I}F^{j}_{J}\dotprodthree{\tensorprod{\vectr{g}_{i}}{\vectr{G}^{I}}}{\tensorprod{\vectr{G}_{I}}{\vectr{G}_{J}}}{\tensorprod{\vectr{G}^{J}}{\vectr{g}_{j}}}=\tensor{F}\tensor{G}\transpose{\tensor{F}}
\end{equation}

The left and right Cauchy-Green deformation tensors get their left and right
names from their relationship to the left and right stretch tensors \ie in
cartesian coordinates
\begin{equation}
  \tensor{U}=\sqrt{\tensor{C}}
\end{equation}
and
\begin{equation}
  \tensor{V}=\sqrt{\tensor{b}}
\end{equation}

We also have $\tensor{c}=\inverse{\tensor{b}}$ where $\tensor{c}$ is the
\emph{Cauchy deformation tensor} \ie \symbolat{$\tensor{c}$}{Cauchy deformation tensor}
\begin{equation}
	  \tensor{c}=c_{ij}\tensorprod{\vectr{g}^{i}}{\vectr{g}^{j}}
\end{equation}

The final spatial strain tensor is the \emph{Euler-Almansi strain tensor},
$\tensor{e}$, is defined as the difference in metrics
by\symbolat{$\tensor{e}$}{Euler-Almansi strain tensor}
\begin{equation}
  \mapping{\fnof{\tensor{e}}{x}}{\tangentspace{\fnof{\chi}{\embedmanifold{B}}}{x}}{\tangentspace{\fnof{\chi}{\embedmanifold{B}}}{x}}
\end{equation}
and
\begin{equation}
  \begin{split}
    \fnof{\tensor{e}}{x}&=\frac{1}{2}\pbrac{\tensor{g}-\inverse{\fnof{\tensor{b}}{x}}}\\
    &= \frac{1}{2}\pbrac{\tensor{g}-\fnof{\tensor{c}}{x}}
  \end{split}
\end{equation}
where $\tensor{c}=\inverse{\tensor{b}}$. In component form we have
\begin{equation}
  \tensor{e}=e_{ij}\tensorprod{\vectr{g}^{i}}{\vectr{g}^{j}}=\frac{1}{2}\pbrac{g_{ij}-c_{ij}}\tensorprod{\vectr{g}^{i}}{\vectr{g}^{j}}
\end{equation}

Note that we have the following relationships between the deformation tensors
and metric tensors
\begin{alignat}{7}
  \flattensor{\tensor{C}}&=&\pullback{\chi}{\flattensor{\tensor{g}}} &\quad&
  \flattensor{\tensor{c}}&=&\pushforward{\chi}{\flattensor{\tensor{G}}} \\
  \sharptensor{\tensor{B}}&=&\pullback{\chi}{\sharptensor{\tensor{g}}} &\quad&
  \sharptensor{\tensor{b}}&=&\pushforward{\chi}{\sharptensor{\tensor{G}}} \\
  \flattensor{\tensor{E}}&=&\pullback{\chi}{\flattensor{\tensor{e}}} &\quad&
  \flattensor{\tensor{e}}&=&\pushforward{\chi}{\flattensor{\tensor{E}}} \\ 
  &=&\frac{1}{2}\pbrac{\pullback{\chi}{\flattensor{\tensor{g}}}-\flattensor{\tensor{G}}}&\quad&
  &=&\frac{1}{2}\pbrac{\flattensor{\tensor{g}}-\pushforward{\chi}{\flattensor{\tensor{G}}}}
\end{alignat}
\ie
\begin{alignat}{7}
  \flattensor{\tensor{C}}&=&\transpose{\tensor{F}}\flattensor{\tensor{g}}\tensor{F} &\quad&
  \flattensor{\tensor{c}}&=&\invtranspose{\tensor{F}}\flattensor{\tensor{G}}\inverse{\tensor{F}} \\
  \sharptensor{\tensor{B}}&=&\inverse{\tensor{F}}\sharptensor{\tensor{g}}\invtranspose{\tensor{F}} &\quad&
  \sharptensor{\tensor{b}}&=&\tensor{F}\sharptensor{\tensor{G}}\transpose{\tensor{F}} \\
  \flattensor{\tensor{E}}&=&\transpose{\tensor{F}}\flattensor{\tensor{e}}\tensor{F} &\quad&
  \flattensor{\tensor{e}}&=&\invtranspose{\tensor{F}}\flattensor{\tensor{E}}\inverse{\tensor{F}}   
\end{alignat}
or, in component form,
\begin{alignat}{3}
  C_{IJ}=g_{ij}F^{i}_{I}F^{j}_{J} &\quad& 
  c_{ij}=G_{IJ}\pbrac{\inverse{F}}^{I}_{i}\pbrac{\inverse{F}}^{J}_{j} \\
  B^{IJ}=g^{ij}\pbrac{\inverse{F}}^{I}_{i}\pbrac{\inverse{F}}^{J}_{j} &\quad&
  b^{ij}=G^{IJ}F^{i}_{I}F^{j}_{J} \\
  E_{IJ}=e_{ij}F^{i}_{I}F^{j}_{J} &\quad&
  e_{ij}=E_{IJ}\pbrac{\inverse{F}}^{I}_{i}\pbrac{\inverse{F}}^{J}_{j}   
\end{alignat}

\subsubsection{Stress tensors}

Stress is amount of force over an area. We can thus form a number of different
stress tensors depending on what configuration we base the force and the area
in.

The \emph{Cauchy stress tensor}\symbolat{$\tensortwo{\sigma}$}{Cauchy stress tensor}, $\tensor{\sigma}$, has both the force and
area referred to the \emph{spatial} configuration. The Cauchy stress tensor is
important as it quantifies the physical stress that exists in current
configuration and can be measured. It is defined by
\begin{equation}
  \mapping{\fnof{\tensor{\sigma}}{x}}{\spaceprod{\tangentspace{\embedmanifold{B}}{x}}{\tangentspace{\embedmanifold{B}}{x}}}{\realnums}
\end{equation}
\ie
\begin{equation}
  \fnof{\tensor{\sigma}}{x}=\sigma^{ij}\tensorprod{\vectr{g}_{i}}{\vectr{g}_{j}}
\end{equation}

Cauchy's law....

We can also define the \emph{Kirchoff stress tensor}\symbolat{$\tensortwo{\tau}$}{Kirchoff stress tensor}, $\tensor{\tau}$, by
scaling the Cauchy stress by the Jacobian \ie
\begin{equation}
  \tensor{\tau}=J\tensor{\sigma}
\end{equation}

To find other stress measures we need to make use of Piola's transform and
idenity.

The \emph{Piola transform} allows mapping of vector fields on the current
configuration to equivalent vector fields in the reference configuration. The
transform is given by
\begin{equation}
  \vectr{Y}=J\pullback{\chi}{\vectr{y}}=J\inverse{\tensor{F}}\vectr{y}
\end{equation}
where $\vectr{y}$ is the vector field on the current configuration and
$\vectr{Y}$ is the equivalent vector field on the reference configuration. In
component form we have
\begin{equation}
  Y^{I}=J\pbrac{\inverse{F}}^{I}_{i}y^{i}
\end{equation}

Now if $\vectr{Y}$ is the Piola transform of $\vectr{y}$ then we have the
\emph{Piola identity} given by
\begin{equation}
  \Divop{\vectr{Y}}=J\divop{\circcomposition{\vectr{y}}{\chi}}
\end{equation}
where $\Divop$ is the divergence operator in the reference configuration and
$\divop$ is the divergence operator in the current configurtion. 

Using the Piola transform we can construct two new stress tensors. If we apply
the Piola transform to the second index of the Cauchy stress tensor we obtain
the \emph{First Piola-Kirchoff Stress Tensor}\symbolat{$\tensortwo{P}$}{First Piola-Kirchoff stress tensor} \ie
\begin{equation}
  \tensor{P}=P^{iI}\tensorprod{\vectr{g}_{i}}{\vectr{G}_{I}}=J\pbrac{\inverse{F}}^{I}_{j}\sigma^{ij}\dotprod{\tensorprod{\vectr{g}_{i}}{\vectr{g}_{j}}}{\tensorprod{\vectr{g}^{j}}{\vectr{G}_{I}}}=J\tensor{\sigma}\invtranspose{\tensor{F}}
\end{equation}

The first Piola-Kirchoff stress tensor relates forces in the current
configuration with areas in the reference configuration. Note that
$\tensor{P}$ is a two point tensor and is not symmetric.

The relationship between Cauchy stress and the First Piola-Kirchoff stress is
given by
\begin{equation}
  \tensor{\sigma}=\inverse{J}\tensor{P}\transpose{\tensor{F}}
\end{equation}

If we now apply the Piola transform to the first index of the First Piola-Kirchoff stress tensor we obtain
the \emph{Second Piola-Kirchoff Stress Tensor}\symbolat{$\tensortwo{S}$}{Second Piola-Kirchoff stress tensor} \ie
\begin{equation}
  \tensor{S}=\inverse{\tensor{F}}\tensor{P}
\end{equation}
or, in terms of the Cauchy stress,
\begin{equation}
  \tensor{S}=S^{IJ}\tensorprod{\vectr{G}_{I}}{\vectr{G}_{J}}=J\pbrac{\inverse{F}}^{I}_{i}\pbrac{\inverse{F}}^{J}_{j}\sigma^{ij}\dotprod{\dotprod{\tensorprod{\vectr{G}_{I}}{\vectr{g}^{i}}}{\tensorprod{\vectr{g}_{i}}{\vectr{g}_{j}}}}{\tensorprod{\vectr{g}^{j}}{\vectr{G}_{J}}}=J\inverse{\tensor{F}}\tensor{\sigma}\invtranspose{\tensor{F}}
\end{equation}

The second Piola-Kirchoff stress tensor relates forces in the reference
configuration with areas in the reference configuration.

Note that the Kirchoff stress is just the push forward of the second
Piola-Kirchoff stress \ie
\begin{equation}
  \tensor{\tau}=\tensor{F}\tensor{S}\transpose{\tensor{F}}
\end{equation}
and
\begin{equation}
  \tensor{S}=\inverse{\tensor{F}}\tensor{\tau}\invtranspose{\tensor{F}}
\end{equation}

BIOT STRESS

COVECTED STRESSS


The relationships between the stress tensors are given in \Tabref{tab:RelationshipBetweenStressTensors}.

\begin{table}[htb] \centering
  \begin{tabular}{|c|c|c|c|c|} \hline
    & $\tensor{\sigma}$ & $\tensor{\tau}$ & $\tensor{P}$ & $\tensor{S}$
    \\ \hline \hline
    $\tensor{\sigma}$ & - & $\inverse{J}\tensor{\tau}$ &
    $\inverse{J}\tensor{P}\transpose{\tensor{F}}$ &
    $\inverse{J}\tensor{F}\tensor{S}\transpose{\tensor{F}}$ \\
    $\tensor{\tau}$ & $J\tensor{\sigma}$ & - &
    $\tensor{P}\transpose{\tensor{F}}$ &
    $\tensor{F}\tensor{S}\transpose{\tensor{F}}$ \\
    $\tensor{P}$ & $J\tensor{\sigma}\invtranspose{\tensor{F}}$ &
    $\tensor{\tau}\invtranspose{\tensor{F}}$ & - & $\tensor{F}\tensor{S}$ \\
    $\tensor{S}$ &
    $J\inverse{\tensor{F}}\tensor{\sigma}\invtranspose{\tensor{F}}$ &
    $\inverse{\tensor{F}}\tensor{\sigma}\invtranspose{\tensor{F}}$ &
    $\inverse{\tensor{F}}\tensor{P}$ & - \\ \hline
  \end{tabular}
  \caption{Reltionships between stress tensors.}
  \label{tab:RelationshipBetweenStressTensors}
\end{table}

\subsubsection{Incompressibility}

For incompressible materials we need an additional volumetric stress. The
\emph{hydrostatic stress} is a Cauchy stress and so we have
\begin{equation}
  \tensor{\sigma}_{p}=-p\tensor{g}
\end{equation}
or in component form
\begin{equation}
  \tensor{\sigma}_{p}=\sigma^{ij}_{p}\tensorprod{\vectr{g}_{i}}{\vectr{g}_{j}}=-pg^{ij}\tensorprod{\vectr{g}_{i}}{\vectr{g}_{j}}
\end{equation}
where $p$ is known as the \emph{hydrostatic pressure}.

We can pull this stress back to give a second Piola-Kirchoff stress via the
pullback operation for a second order tensor \ie
\begin{equation}
  \tensor{S}_{p}=J\inverse{\tensor{F}}\tensor{\sigma}_{p}\invtranspose{\tensor{F}}=-pJ\inverse{\tensor{C}}
\end{equation}
or, in component form,
\begin{equation}
  \tensor{S}_{p}=S^{IJ}_{p}\tensorprod{\vectr{G}_{I}}{\vectr{G}_{J}}=-JF^{I}_{i}pg^{ij}F^{J}_{j}\tensorprod{\vectr{G}_{I}}{\vectr{G}_{J}}=-pJ\pbrac{\inverse{C}}^{IJ}\tensorprod{\vectr{G}_{I}}{\vectr{G}_{J}}
\end{equation}

RESTRICTIONS ON STRESS TENSOR E.G. SYMMETRY

\subsubsection{Constituative Relationships}

The relationship between stress and strain is known as a \emph{consituative
  relationship}. Materials for which the constituative relationship is just a
function of the current state of deformation are known as \emph{elastic}.

\subsubsubsection{Hyperelasticity}

For the special case whereby the work done by stresses during deformation is
just a function of the initial configuration and the current configuration are
known as \emph{hyperelastic}. As a consequence hyperelastic materials are
independent of the path of deformation and just depend on a \emph{stored
  energy function} or \emph{elastic potential}, $W$.

WORK CONJUGATES: P and F, S and C/E.

The first Piola-Kirchoff stress tensor is thus a function of postion and the
deformation gradient tensor \ie
\begin{equation}
  \tensor{P}=\fnof{\tensor{P}}{\vectr{X},\fnof{\tensor{F}}{\vectr{X}}}
\end{equation}

In terms of the stored energy function we have
\begin{equation}
  \fnof{\tensor{P}}{\vectr{X}}=\delby{\fnof{\psi}{\fnof{\tensor{F}}{\vectr{X}},\vectr{X}}}{\fnof{\tensor{F}}{\vectr{X}}}
\end{equation}
or, in component form,
\begin{equation}
  \tensor{P}=P^{iI}\tensorprod{\vectr{g}_{i}}{\vectr{G}_{I}}=\delby{\psi}{F^{i}_{I}}\tensorprod{\vectr{g}_{i}}{\vectr{G}_{I}}
\end{equation}
ABOVE IS NOT RIGHT IN TERMS OF POSITION OF INDICES.

The second Piola-Kirchoff stress tensor is thus a function of position and
the right Cauchy-Green deformation tensor (or, equivalently, the
Green-Lagrange strain tensor) \ie
\begin{equation}
  \tensor{S}=\fnof{\tensor{S}}{\vectr{X},\fnof{\tensor{C}}{\vectr{X}}}
\end{equation}


The constitutive law can then be used to derive the second Piola Kirchhoff
stress tensor in fibre coordinates, $\fnof{\tensor{S}}{\vectr{N}}$, from
either the right Cauchy-Green deformation tensor or the Green-Lagrange strain
tensor \ie
\begin{equation}
  \fnof{\tensor{S}}{\vectr{N}}=2\delby{\fnof{W}{\fnof{\tensor{C}}{\vectr{N}}}}{\fnof{\tensor{C}}{\vectr{N}}}
\end{equation}
or
\begin{equation}
  \fnof{\tensor{S}}{\vectr{N}}=\delby{\fnof{W}{\fnof{\tensor{E}}{\vectr{N}}}}{\fnof{\tensor{E}}{\vectr{N}}}
\end{equation}
where $\fnof{W}{\fnof{\tensor{C}}{\vectr{N}}}$ or
$\fnof{W}{\fnof{\tensor{E}}{\vectr{N}}}$ is the strain energy
function. In component form we have
\begin{equation}
  \tensor{S}=S^{AB}\tensorprod{\vectr{N}_{A}}{\vectr{N}_{B}}=2\delby{\fnof{W}{\tensor{C}}}{\tensor{C}}=2\delby{\fnof{W}{\tensor{C}}}{C_{AB}}\tensorprod{\vectr{N}_{A}}{\vectr{N}_{B}}
\end{equation}
or
\begin{equation}
  \tensor{S}=S^{AB}\tensorprod{\vectr{N}_{A}}{\vectr{N}_{B}}=\delby{\fnof{W}{\tensor{E}}}{\tensor{E}}=\delby{\fnof{W}{\tensor{E}}}{E_{AB}}\tensorprod{\vectr{N}_{A}}{\vectr{N}_{B}}
\end{equation}

\clearpage

Now it is often useful to consider the strain energy (or stress) in terms of a
component that is purely volumetric (or spherical) and a component that is
purely isochoric (or deviatoric). Following \citeasnoun{federico:2012} we start with
a \emph{modified deformation gradient
  tensor}\symbolat{$\bar{\tensortwo{F}}$}{modified deformation gradient tensor} \ie
\begin{equation}
  \tensor{F}=J^{\frac{1}{3}}\bar{\tensor{F}} \quad\Rightarrow\quad \bar{\tensor{F}} =J^{-\frac{1}{3}}\tensor{F}
  \label{eqn:modifiedDeformationGradientTensor}
\end{equation}
Note that $\determinant{\bar{\tensor{F}}}=1$ and so represents deformation that is
purely distortional rather than dialational. 

From the modified deformation gradient tensor we can construct the various
deformation and strain tensors as before. In the reference configuration we
have
\begin{equation}
  \tensor{C} = J^{\frac{1}{3}}J^{\frac{1}{3}}\transpose{\bar{\tensor{F}}}\tensor{g}\bar{\tensor{F}}=J^{\frac{2}{3}}\bar{\tensor{C}} \quad\Rightarrow\quad \bar{\tensor{C}} =J^{-\frac{2}{3}}\tensor{C}
\end{equation}
where $\bar{\tensor{C}}$ is the \emph{modified right Cauchy-Green deformation
  tensor}\symbolat{$\bar{\tensortwo{C}}$}{modified right Cauchy-Green deformation tensor} and
\begin{equation}
  \tensor{E} =
  J^{\frac{2}{3}}{\bar{\tensor{E}}}+\dfrac{1}{2}\pbrac{J^{\frac{2}{3}}-1}\tensor{G}
  \quad\Rightarrow\quad \bar{\tensor{E}} = \dfrac{1}{2}\pbrac{\bar{\tensor{C}}-\tensor{G}}
\end{equation}
where $\bar{\tensor{E}}$ is the \emph{modified right Green-Lagrange strain
  tensor}\symbolat{$\bar{\tensortwo{E}}$}{modified Green-Lagrange strain tensor}. Note that we also have
$\bar{\tensor{B}}=\inverse{\bar{\tensor{C}}}$ the \emph{modified Piola
  deformation tensor}\symbolat{$\bar{\tensortwo{B}}$}{modified Piola deformation tensor}.

The modified strain tensors in the current configuration are 
\begin{equation}
  \tensor{b} = J^{\frac{1}{3}}J^{\frac{1}{3}}\bar{\tensor{F}}\tensor{G}\transpose{\bar{\tensor{F}}}=J^{\frac{2}{3}}\bar{\tensor{b}} \quad\Rightarrow\quad \bar{\tensor{b}} =J^{-\frac{2}{3}}\tensor{b}
\end{equation}
where $\bar{\tensor{b}}$ is the \emph{modified left Cauchy-Green deformation
  tensor}\symbolat{$\bar{\tensortwo{b}}$}{modified left Cauchy-Green deformation tensor} and
\begin{equation}
  \tensor{e} =
  J^{-\frac{2}{3}}{\bar{\tensor{e}}}+\dfrac{1}{2}\pbrac{1-J^{-\frac{2}{3}}}\tensor{g}
  \quad\Rightarrow\quad \bar{\tensor{e}} = \dfrac{1}{2}\pbrac{\tensor{g}-\bar{\tensor{c}}}
\end{equation}
where $\bar{\tensor{e}}$ is the \emph{modified Almansi strain
  tensor}\symbolat{$\bar{\tensortwo{e}}$}{modified Almansi strain tensor}. Note that we also have
$\bar{\tensor{c}}=\inverse{\bar{\tensor{b}}}$ the \emph{modified Cauchy
  deformation tensor}\symbolat{$\bar{\tensortwo{e}}$}{modified Cauchy deformation tensor}.

Consider now decomposing the strain energy function into a
isochoric/deviatoric part and a volumetric/spherical part \ie
\begin{equation}
  \begin{split}
    \fnof{W}{\tensor{C}} &=
    \fnof{W_{\Devop}}{\tensor{C}}+\fnof{W_{\Sphop}}{\tensor{C}} \\
    &=\fnof{W_{\Devop}}{\fnof{\bar{\tensor{C}}}{\tensor{C}}}+\fnof{W_{\Sphop}}{\fnof{J}{\tensor{C}}}\\
    &=\fnof{\bar{W}}{\fnof{\bar{\tensor{C}}}{\tensor{C}},\fnof{J}{\tensor{C}}}
  \end{split}
\end{equation}
where $\fnof{\bar{W}}{\bar{\tensor{C}},J}$ is the \emph{modified strain energy function},
$\fnof{J}{\tensor{C}}$ is the Jacobian which quantifies volume change and
$\fnof{\bar{\tensor{C}}}{\tensor{C}}$, is the modified right Cauchy-Green
deformation tensor.

The second Piola-Kirchoff stress can now be obtained from
\begin{equation}
  \begin{split}
    \fnof{\tensor{S}}{\tensor{C}} &= 2\delby{\fnof{W}{\tensor{C}}}{\tensor{C}} \\
    &=2\delby{\fnof{\bar{W}}{\bar{\tensor{C}},J}}{\tensor{C}}\\
    &=2\doubledotprod{\delby{\fnof{\bar{W}}{\bar{\tensor{C}},J}}{\bar{\tensor{C}}}}{\delby{\bar{\tensor{C}}}{\tensor{C}}}+2\delby{\fnof{\bar{W}}{\bar{\tensor{C}},J}}{J}\delby{J}{\tensor{C}}
  \end{split}
\end{equation}
or equivalently
\begin{equation}
  \begin{split}
    \fnof{\tensor{S}}{\tensor{E}} &= \delby{\fnof{W}{\tensor{E}}}{\tensor{E}} \\
    &=\delby{\fnof{\bar{W}}{\bar{\tensor{E}},J}}{\tensor{E}}\\
    &=\doubledotprod{\delby{\fnof{\bar{W}}{\bar{\tensor{E}},J}}{\bar{\tensor{E}}}}{\delby{\bar{\tensor{E}}}{\tensor{E}}}+\delby{\fnof{\bar{W}}{\bar{\tensor{E}},J}}{J}\delby{J}{\tensor{E}}
  \end{split}
\end{equation}

We now define $\bar{\tensor{S}}$ as the \emph{second Piola-Kirchoff psuedo
  stress tensor}\symbolat{$\bar{\tensortwo{S}}$}{second Piola-Kirchoff
  psuedo stress tensor} \ie 
\begin{equation}
  \bar{\tensor{S}}=2\delby{\fnof{\bar{W}}{\bar{\tensor{C}},J}}{\bar{\tensor{C}}}=\delby{\fnof{\bar{W}}{\bar{\tensor{E}},J}}{\bar{\tensor{E}}}
\end{equation}
and
\begin{equation}
  p = -\delby{\fnof{\bar{W}}{\bar{\tensor{C}},J}}{J}= -\delby{\fnof{\bar{W}}{\bar{\tensor{E}},J}}{J}
\end{equation}
as the \emph{hydrostatic pressure} to give
\begin{align}
  \tensor{S}&=\doubledotprod{\bar{\tensor{S}}}{\delby{\bar{\tensor{C}}}{\tensor{C}}}-2p\delby{J}{\tensor{C}}\\
  &=\doubledotprod{\bar{\tensor{S}}}{\delby{\bar{\tensor{E}}}{\tensor{E}}}-p\delby{J}{\tensor{E}}
\end{align}
 
Using 
\begin{align}
  \delby{J}{\tensor{C}}&=\dfrac{J}{2}\tensor{B} \\
  \delby{J}{\tensor{E}}&=J\tensor{B}\\
  \delby{\bar{\tensor{C}}}{\tensor{C}}&=J^{-\frac{2}{3}}\pbrac{\transpose{\tensorfour{I}}-\dfrac{\tensorprod{\tensor{C}}{\tensor{B}}}{3}}\\
  \delby{\bar{\tensor{E}}}{\tensor{E}}&=J^{-\frac{2}{3}}\pbrac{\transpose{\tensorfour{I}}-\dfrac{\tensorprod{\tensor{C}}{\tensor{B}}}{3}}\\\end{align}
we can thus find that
\begin{equation}
  \begin{split}
    \fnof{\tensor{S}}{\tensor{C}}&=\doubledotprod{\bar{\tensor{S}}}{\delby{\bar{\tensor{C}}}{\tensor{C}}}-2p\delby{J}{\tensor{C}}\\
    &=
    \doubledotprod{\bar{\tensor{S}}}{J^{-\frac{2}{3}}\pbrac{\transpose{\tensorfour{I}}-\dfrac{\tensorprod{\tensor{C}}{\tensor{B}}}{3}}}-pJ\tensor{B}\\
    &=\tensor{S}_{\pullback{\Devop}{}}+\tensor{S}_{\pullback{\Sphop}{}}
  \end{split}
  \label{eqn:decomposedSecondPKinC}
\end{equation}
or
\begin{equation}
  \begin{split}
    \fnof{\tensor{S}}{\tensor{E}}&=\doubledotprod{\bar{\tensor{S}}}{\delby{\bar{\tensor{E}}}{\tensor{E}}}-p\delby{J}{\tensor{E}}\\
    &=
    \doubledotprod{\bar{\tensor{S}}}{J^{-\frac{2}{3}}\pbrac{\transpose{\tensorfour{I}}-\dfrac{\tensorprod{\tensor{C}}{\tensor{B}}}{3}}}-pJ\tensor{B}\\
    &=\tensor{S}_{\pullback{\Devop}{}}+\tensor{S}_{\pullback{\Sphop}{}}
  \end{split}
  \label{eqn:decomposedSecondPKinE}
\end{equation}

It should be noted that $\tensor{S}_{\pullback{\Devop}{}}$ and
$\tensor{S}_{\pullback{\Sphop}{}}$ are not the deviatoric and spherical parts
of $\tensor{S}$. However, we do not require the deviatoric and spherical parts
of the second Piola-Kirchoff stress tensor in the reference configuration (\ie
with respect to the metric $\tensor{G}$) but rather we require that when we
push $\tensor{S}_{\pullback{\Devop}{}}$ and
$\tensor{S}_{\pullback{\Sphop}{}}$ forward to give $\tensor{\sigma}_{\devop}$
and $\tensor{\sigma}_{\sphop}$ that these tensors are deviartoric and
spherical respectively (\ie with respect to the metric $\tensor{g}$).

We can now define for the current configuration (\ie with respect to the
metric tensor $\tensor{g}$) the fourth order identity,
$\tensorfour{i}$, spherical, $\tensorfour{k}$, and deviatoric,
$\tensorfour{m}$, tensors as
\begin{align}
  \tensorfour{i}=\symtensorprod{\tensortwo{i}}{\tensortwo{i}}&=
  i^{ab}_{cd}\,\tensorprodfour{\vectr{g}_{a}}{\vectr{g}_{b}}{\vectr{g}^{c}}{\vectr{g}^{d}}\\ \nonumber
  &=
  \dfrac{1}{2}\pbrac{\mixedkronecker{a}{c}\mixedkronecker{b}{d}+\mixedkronecker{a}{d}\mixedkronecker{b}{c}}\tensorprodfour{\vectr{g}_{a}}{\vectr{g}_{b}}{\vectr{g}^{c}}{\vectr{g}^{d}}
  \\
  \tensorfour{k}=\dfrac{1}{3}\tensorprod{\inverse{\tensor{g}}}{\tensor{g}}&=
  k^{ab}_{cd}\,\tensorprodfour{\vectr{g}_{a}}{\vectr{g}_{b}}{\vectr{g}^{c}}{\vectr{g}^{d}}\\ \nonumber
  &=
  \dfrac{1}{3}g^{ab}g_{cd}\,\tensorprodfour{\vectr{g}_{a}}{\vectr{g}_{b}}{\vectr{g}^{c}}{\vectr{g}^{d}}\\
  \tensorfour{m}=\tensorfour{i}-\tensorfour{k}&=
  m^{ab}_{cd}\,\tensorprodfour{\vectr{g}_{a}}{\vectr{g}_{b}}{\vectr{g}^{c}}{\vectr{g}^{d}}\\ \nonumber
  &=
  \pbrac{\dfrac{1}{2}\pbrac{\mixedkronecker{a}{c}\mixedkronecker{b}{d}+\mixedkronecker{a}{d}\mixedkronecker{b}{c}}-\dfrac{1}{3}g^{ab}g_{cd}}\tensorprodfour{\vectr{g}_{a}}{\vectr{g}_{b}}{\vectr{g}^{c}}{\vectr{g}^{d}}
\end{align}
where $\tensortwo{i}$ is the spatial identity tensor \ie
$\tensortwo{i}=\inverse{\tensortwo{g}}\tensortwo{g}$.

Note that these tensors act on a second order symmetric (contravariant) tensor
in the current configuration, $\tensortwo{q}$, such that
\begin{align}
  \doubledotprod{\tensorfour{i}}{\tensortwo{q}}&=\tensortwo{q} \\
  \doubledotprod{\tensorfour{k}}{\tensortwo{q}}&=\dfrac{1}{3}\trop\tensortwo{q}\inverse{\tensortwo{g}} \\
  \doubledotprod{\tensorfour{m}}{\tensortwo{q}}&=\tensortwo{q}-\dfrac{1}{3}\trop\tensortwo{q}\inverse{\tensortwo{g}}
\end{align}
where the trace operator is defined as $\trop\tensortwo{q}=\doubledotprod{\tensortwo{g}}{\tensortwo{q}}$ \ie
with respect to the current configuration metric tensor, $\tensortwo{g}$.

The musical isomorphisms of these tensors are
\begin{align}
  \sharptensor{\tensorfour{i}}=
  \symtensorprod{\inverse{\tensortwo{g}}}{\inverse{\tensortwo{g}}} &=
  i^{abcd}\,\tensorprodfour{\vectr{g}_{a}}{\vectr{g}_{b}}{\vectr{g}_{c}}{\vectr{g}_{d}}
  \\ \nonumber
  &=
  \dfrac{1}{2}\pbrac{g^{ac}g^{bd}+g^{ad}g^{bc}}\tensorprodfour{\vectr{g}_{a}}{\vectr{g}_{b}}{\vectr{g}_{c}}{\vectr{g}_{d}}
  \\
  \sharptensor{\tensorfour{k}}=
  \dfrac{1}{3}\tensorprod{\inverse{\tensor{g}}}{\inverse{\tensor{g}}}&=
  k^{abcd}\,\tensorprodfour{\vectr{g}_{a}}{\vectr{g}_{b}}{\vectr{g}_{c}}{\vectr{g}_{d}}
  \\ \nonumber
  &=\dfrac{1}{3}g^{ab}g^{cd}\,\tensorprodfour{\vectr{g}_{a}}{\vectr{g}_{b}}{\vectr{g}_{c}}{\vectr{g}_{d}}
  \\
  \sharptensor{\tensorfour{m}}=
  \sharptensor{\tensorfour{i}}-\sharptensor{\tensorfour{k}}&=
  m^{abcd}\,\tensorprodfour{\vectr{g}_{a}}{\vectr{g}_{b}}{\vectr{g}_{c}}{\vectr{g}_{d}}
  \\ \nonumber
  &=\pbrac{\dfrac{1}{2}\pbrac{g^{ac}g^{bd}+g^{ad}g^{bc}}-\dfrac{1}{3}g^{ab}g^{cd}}\tensorprodfour{\vectr{g}_{a}}{\vectr{g}_{b}}{\vectr{g}_{c}}{\vectr{g}_{d}}
\end{align}
and
\begin{align}
  \flattensor{\tensorfour{i}}=
  \symtensorprod{\tensortwo{g}}{\tensortwo{g}}&=
  i_{abcd}\,\tensorprodfour{\vectr{g}^{a}}{\vectr{g}^{b}}{\vectr{g}^{c}}{\vectr{g}^{d}}
  \\ \nonumber
  &=\dfrac{1}{2}\pbrac{g_{ac}g_{bd}+g_{ad}g_{bc}}\tensorprodfour{\vectr{g}^{a}}{\vectr{g}^{b}}{\vectr{g}^{c}}{\vectr{g}^{d}}
  \\
  \flattensor{\tensorfour{k}}=
  \dfrac{1}{3}\tensorprod{\tensor{g}}{\tensor{g}}&=
  k_{abcd}\,\tensorprodfour{\vectr{g}^{a}}{\vectr{g}^{b}}{\vectr{g}^{c}}{\vectr{g}^{d}}
  \\ \nonumber
  &=\dfrac{1}{3}g_{ab}g_{cd}\,\tensorprodfour{\vectr{g}^{a}}{\vectr{g}^{b}}{\vectr{g}^{c}}{\vectr{g}^{d}}
  \\
  \flattensor{\tensorfour{m}}=
  \flattensor{\tensorfour{i}}-\flattensor{\tensorfour{k}}&=
  m_{abcd}\,\tensorprodfour{\vectr{g}^{a}}{\vectr{g}^{b}}{\vectr{g}^{c}}{\vectr{g}^{d}}
  \\ \nonumber
  &=\pbrac{\dfrac{1}{2}\pbrac{g_{ac}g_{bd}+g_{ad}g_{bc}}-\dfrac{1}{3}g_{ab}g_{cd}}\tensorprodfour{\vectr{g}^{a}}{\vectr{g}^{b}}{\vectr{g}^{c}}{\vectr{g}^{d}}
\end{align}

We can pull-back these spatial tensors into the material configuration \ie
\begin{align}
  \tensorfour{I}&=\pullback{\chi}{\tensorfour{i}}=\symtensorprod{\tensortwo{I}}{\tensortwo{I}}
  \\
  \pullback{\tensorfour{K}}{}&=\pullback{\chi}{\tensorfour{k}}=\dfrac{1}{3}\tensorprod{\tensortwo{B}}{\tensortwo{C}}\\
  \pullback{\tensorfour{M}}{}&=\pullback{\chi}{\tensorfour{m}}=\tensorfour{I}-\pullback{\tensorfour{K}}{}
\end{align}
where $\tensortwo{I}$ is the material identity tensor \ie
$\tensortwo{I}=\inverse{\tensortwo{G}}\tensortwo{G}$.

The operation of these tensors on
$\tensortwo{Q}=\pullback{\chi}{\tensortwo{q}}=\inverse{\tensortwo{F}}\tensortwo{q}\invtranspose{\tensortwo{F}}$
is given by
\begin{align}
  \doubledotprod{\tensorfour{I}}{\tensortwo{Q}}&=\tensortwo{Q} \\
  \doubledotprod{\pullback{\tensorfour{K}}{}}{\tensortwo{Q}}&=\dfrac{1}{3}\pullback{\Trop}{}\tensortwo{Q}\tensortwo{B} \\
  \doubledotprod{\pullback{\tensorfour{M}}{}}{\tensortwo{Q}}&=\tensortwo{Q}-\dfrac{1}{3}\pullback{\Trop}{}\tensortwo{Q}\tensortwo{B}
\end{align}
where the pulled back trace operator in the reference configuration is defined
as
$\pullback{\Trop}{}\tensortwo{Q}=\doubledotprod{\tensortwo{C}}{\tensortwo{Q}}$
\ie with respect to the pulled back current configuration metric tensor,
$\tensortwo{C}=\pullback{\chi}{\tensortwo{g}}$. In other words
$\doubledotprod{\pullback{\tensorfour{K}}{}}{\tensortwo{Q}}$ and
$\doubledotprod{\pullback{\tensorfour{M}}{}}{\tensortwo{Q}}$ are the pulled
back spherical and deviatoric components of $\tensor{Q}$ with respect to the
pulled back current configuration metric tensor $\tensortwo{C}=\pullback{\chi}{\tensortwo{g}}$.

The musical isomorphisms of these pulled back tensors are
\begin{align}
  \tensorfour{I}^{\sharp *}&=\pullback{\chi}{\sharptensor{\tensorfour{i}}}=\symtensorprod{\tensortwo{B}}{\tensortwo{B}}\\
  \tensorfour{K}^{\sharp *}&=\pullback{\chi}{\sharptensor{\tensorfour{k}}}=\dfrac{1}{3}\tensorprod{\tensortwo{B}}{\tensortwo{B}}\\
  \tensorfour{M}^{\sharp
    *}&=\pullback{\chi}{\sharptensor{\tensorfour{m}}}=\tensorfour{I}^{\sharp *}-\tensorfour{K}^{\sharp *}
\end{align}
and
\begin{align}
  \tensorfour{I}^{\flat *}&=\pullback{\chi}{\flattensor{\tensorfour{i}}}=\symtensorprod{\tensortwo{C}}{\tensortwo{C}}\\
  \tensorfour{K}^{\flat *}&=\pullback{\chi}{\flattensor{\tensorfour{k}}}=\dfrac{1}{3}\tensorprod{\tensortwo{C}}{\tensortwo{C}}\\
  \tensorfour{M}^{\flat
    *}&=\pullback{\chi}{\flattensor{\tensorfour{m}}}=\tensorfour{I}^{\flat
    *}-\tensorfour{K}^{\flat *}
\end{align}

Note that
\begin{align}
  \tensorfour{I}^{\sharp
    *}&=-\delby{\tensortwo{B}}{\tensortwo{C}}=-\delby{\inverse{\tensortwo{C}}}{\tensortwo{C}} \\
  \tensorfour{I}^{\flat
    *}&=-\delby{\tensortwo{C}}{\tensortwo{B}}=-\delby{\inverse{\tensortwo{B}}}{\tensortwo{B}} \\
\end{align}

We can now define for the reference configuration (\ie with respect to the
metric tensor $\tensor{G}$) the fourth order identity,
$\tensorfour{I}$, spherical, $\tensorfour{K}$, and deviatoric,
$\tensorfour{M}$, tensors as
\begin{align}
  \tensorfour{I}= \symtensorprod{\tensortwo{I}}{\tensortwo{I}}&=
  I^{AB}_{CD}\,\tensorprodfour{\vectr{G}_{A}}{\vectr{G}_{B}}{\vectr{G}^{C}}{\vectr{G}^{D}}
  \\ \nonumber
  &=\dfrac{1}{2}\pbrac{\mixedkronecker{A}{C}\mixedkronecker{B}{D}+\mixedkronecker{A}{D}\mixedkronecker{B}{C}}\tensorprodfour{\vectr{TG}_{A}}{\vectr{G}_{B}}{\vectr{G}^{C}}{\vectr{G}^{D}}
  \\
  \tensorfour{K}=\dfrac{1}{3}\tensorprod{\inverse{\tensor{G}}}{\tensor{G}}&=
  K^{AB}_{CD}\,\tensorprodfour{\vectr{G}_{A}}{\vectr{G}_{B}}{\vectr{G}^{C}}{\vectr{G}^{D}}
  \\ \nonumber
  &=\dfrac{1}{3}G^{AB}G_{CD}\,\tensorprodfour{\vectr{G}_{A}}{\vectr{G}_{B}}{\vectr{G}^{C}}{\vectr{G}^{D}}\\
  \tensorfour{M}=\tensorfour{I}-\tensorfour{K}&=
  M^{AB}_{CD}\,\tensorprodfour{\vectr{G}_{A}}{\vectr{G}_{B}}{\vectr{G}^{C}}{\vectr{G}^{D}}
  \\ \nonumber
 &= \pbrac{\dfrac{1}{2}\pbrac{\mixedkronecker{A}{C}\mixedkronecker{B}{D}+\mixedkronecker{A}{D}\mixedkronecker{B}{C}}-\dfrac{1}{3}G^{AB}G_{CD}}\tensorprodfour{\vectr{G}_{A}}{\vectr{G}_{B}}{\vectr{G}^{C}}{\vectr{G}^{D}}
\end{align}

Note that these tensors act on a second order symmetric (contravariant) tensor
in the reference configuration, $\tensortwo{Q}$, such that
\begin{align}
  \doubledotprod{\tensorfour{I}}{\tensortwo{Q}}&=\tensortwo{Q} \\
  \doubledotprod{\tensorfour{K}}{\tensortwo{Q}}&=\dfrac{1}{3}\Trop\tensortwo{Q}\inverse{\tensortwo{G}} \\
  \doubledotprod{\tensorfour{M}}{\tensortwo{Q}}&=\tensortwo{Q}-\dfrac{1}{3}\Trop\tensortwo{Q}\inverse{\tensortwo{G}}
\end{align}
where the trace operator is defined as $\Trop\tensortwo{Q}=\doubledotprod{\tensortwo{Q}}{\tensortwo{G}}$ \ie
with respect to the reference configuration metric tensor, $\tensortwo{G}$.

The musical isomorphisms of these tensors are
\begin{align}
  \sharptensor{\tensorfour{I}}=
  \symtensorprod{\inverse{\tensortwo{G}}}{\inverse{\tensortwo{G}}} &=
  I^{ABCD}\,\tensorprodfour{\vectr{G}_{A}}{\vectr{G}_{B}}{\vectr{G}_{C}}{\vectr{G}_{D}}
  \\ \nonumber
  &=
  \dfrac{1}{2}\pbrac{G^{AC}G^{BD}+G^{AD}G^{BC}}\tensorprodfour{\vectr{G}_{A}}{\vectr{G}_{B}}{\vectr{G}_{C}}{\vectr{G}_{D}}
  \\
  \sharptensor{\tensorfour{K}}=
  \dfrac{1}{3}\tensorprod{\inverse{\tensor{G}}}{\inverse{\tensor{G}}}&=
  K^{ABCD}\,\tensorprodfour{\vectr{G}_{A}}{\vectr{G}_{B}}{\vectr{G}_{B}}{\vectr{G}_{D}}
  \\ \nonumber
  &=\dfrac{1}{3}G^{AB}G^{CD}\,\tensorprodfour{\vectr{G}_{A}}{\vectr{G}_{B}}{\vectr{G}_{C}}{\vectr{G}_{D}}
  \\
  \sharptensor{\tensorfour{M}}=
  \sharptensor{\tensorfour{I}}-\sharptensor{\tensorfour{K}}&=
  M^{ABCD}\,\tensorprodfour{\vectr{G}_{A}}{\vectr{G}_{B}}{\vectr{G}_{C}}{\vectr{G}_{D}}
  \\ \nonumber
  &=\pbrac{\dfrac{1}{2}\pbrac{G^{AC}G^{BD}+G^{AD}G^{BC}}-\dfrac{1}{3}G^{AB}G^{CD}}\tensorprodfour{\vectr{G}_{A}}{\vectr{G}_{B}}{\vectr{G}_{C}}{\vectr{G}_{D}}
\end{align}
and
\begin{align}
  \flattensor{\tensorfour{I}}=
  \symtensorprod{\tensortwo{G}}{\tensortwo{G}}&=
  I_{ABCD}\,\tensorprodfour{\vectr{G}^{A}}{\vectr{G}^{B}}{\vectr{G}^{C}}{\vectr{G}^{D}}
  \\ \nonumber
  &=\dfrac{1}{2}\pbrac{G_{AC}G_{BD}+G_{AD}G_{BC}}\tensorprodfour{\vectr{G}^{A}}{\vectr{G}^{B}}{\vectr{G}^{C}}{\vectr{G}^{D}}
  \\
  \flattensor{\tensorfour{K}}=
  \dfrac{1}{3}\tensorprod{\tensor{G}}{\tensor{G}}&=
  K_{ABCD}\,\tensorprodfour{\vectr{G}^{A}}{\vectr{G}^{B}}{\vectr{G}^{C}}{\vectr{G}^{D}}
  \\ \nonumber
  &=\dfrac{1}{3}G_{AB}G_{CD}\,\tensorprodfour{\vectr{G}^{A}}{\vectr{G}^{B}}{\vectr{G}^{C}}{\vectr{G}^{D}}
  \\
  \flattensor{\tensorfour{M}}=
  \flattensor{\tensorfour{I}}-\flattensor{\tensorfour{K}}&=
  M_{ABCD}\,\tensorprodfour{\vectr{G}^{A}}{\vectr{G}^{B}}{\vectr{G}^{C}}{\vectr{G}^{D}}
  \\ \nonumber
  &=\pbrac{\dfrac{1}{2}\pbrac{G_{AC}G_{BD}+G_{AD}G_{BC}}-\dfrac{1}{3}G_{AB}G_{CD}}\tensorprodfour{\vectr{G}^{A}}{\vectr{G}^{B}}{\vectr{G}^{C}}{\vectr{G}^{D}}
\end{align}

We can push-forward these material tensors into the spatial configuration \ie
\begin{align}
  \tensorfour{i}&=\pushforward{\chi}{\tensorfour{I}}=\symtensorprod{\tensortwo{i}}{\tensortwo{i}}
  \\
  \pushforward{\tensorfour{k}}{}&=\pushforward{\chi}{\tensorfour{K}}=\dfrac{1}{3}\tensorprod{\tensortwo{b}}{\tensortwo{c}}\\
  \pushforward{\tensorfour{m}}{}&=\pushforward{\chi}{\tensorfour{M}}=\tensorfour{i}-\pushforward{\tensorfour{k}}{}
\end{align}

The operation of these tensors on
$\tensortwo{q}=\pushforward{\chi}{\tensortwo{Q}}=\tensortwo{F}\tensortwo{Q}\transpose{\tensortwo{F}}$
is given by
\begin{align}
  \doubledotprod{\tensorfour{i}}{\tensortwo{q}}&=\tensortwo{q} \\
  \doubledotprod{\pushforward{\tensorfour{k}}{}}{\tensortwo{q}}&=\dfrac{1}{3}\pushforward{\trop}{}\tensortwo{q}\tensortwo{b} \\
  \doubledotprod{\pushforward{\tensorfour{m}}{}}{\tensortwo{q}}&=\tensortwo{q}-\dfrac{1}{3}\pushforward{\trop}{}\tensortwo{q}\tensortwo{b}
\end{align}
where the pushed forward trace operator in the current configuration is
defined as
$\pushforward{\trop}{}\tensortwo{q}=\doubledotprod{\tensortwo{c}}{\tensortwo{q}}$
\ie with respect to the pushed forward reference configuration metric tensor,
$\tensortwo{c}=\inverse{\tensortwo{b}}=\pushforward{\chi}{\tensortwo{G}}$. In other words
$\doubledotprod{\pushforward{\tensorfour{k}}{}}{\tensortwo{q}}$ and
$\doubledotprod{\pushforward{\tensorfour{m}}{}}{\tensortwo{q}}$ are the push
forward spherical and deviatoric components of $\tensor{q}$ with respect to
the pushed forward reference configuration metric tensor
$\tensortwo{c}=\inverse{\tensortwo{b}}=\pushforward{\chi}{\tensortwo{G}}$.

The musical isomorphisms of these pushed forward tensors are
\begin{align}
  \tensorfour{i}^{\sharp}_{*}&=\pushforward{\chi}{\sharptensor{\tensorfour{I}}}=\symtensorprod{\tensortwo{b}}{\tensortwo{b}}\\
  \tensorfour{k}^{\sharp}_{*}&=\pushforward{\chi}{\sharptensor{\tensorfour{K}}}=\dfrac{1}{3}\tensorprod{\tensortwo{b}}{\tensortwo{b}}\\
  \tensorfour{m}^{\sharp}_{*}&=\pushforward{\chi}{\sharptensor{\tensorfour{M}}}=\tensorfour{i}^{\sharp}_{*}-\tensorfour{k}^{\sharp}_{*}
\end{align}
and
\begin{align}
  \tensorfour{i}^{\flat}_{*}&=\pushforward{\chi}{\flattensor{\tensorfour{I}}}=\symtensorprod{\tensortwo{c}}{\tensortwo{c}}\\
  \tensorfour{k}^{\flat}_{*}&=\pushforward{\chi}{\flattensor{\tensorfour{K}}}=\dfrac{1}{3}\tensorprod{\tensortwo{c}}{\tensortwo{c}}\\
  \tensorfour{m}^{\flat}_{*}&=\pushforward{\chi}{\flattensor{\tensorfour{M}}}=\tensorfour{i}^{\flat}_{*}-\tensorfour{k}^{\flat}_{*}
\end{align}

Note that
\begin{align}
  \tensorfour{i}^{\sharp}_{*}&=-\delby{\tensortwo{b}}{\tensortwo{b}}=-\delby{\inverse{\tensortwo{c}}}{\tensortwo{c}} \\
  \tensorfour{i}^{\flat}_{*}&=-\delby{\tensortwo{c}}{\tensortwo{b}}=-\delby{\inverse{\tensortwo{b}}}{\tensortwo{b}} \\
\end{align}

We are now in a position to reconsider \eqnref{eqn:decomposedSecondPKinC} or \eqnref{eqn:decomposedSecondPKinE} \ie
using
$\doubledotprod{\bar{\tensortwo{S}}}{\tensorfour{M}^{*T}}=\doubledotprod{\pullback{\tensorfour{M}}{}}{\bar{\tensortwo{S}}}$
we can write
\begin{equation}
  \begin{split}
    \tensor{S}&=\doubledotprod{\bar{\tensor{S}}}{\delby{\bar{\tensor{C}}}{\tensor{C}}}-2p\delby{J}{\tensor{C}}=
    \doubledotprod{\bar{\tensor{S}}}{\delby{\bar{\tensor{E}}}{\tensor{E}}}-p\delby{J}{\tensor{E}}\\
    &=
    \doubledotprod{\bar{\tensor{S}}}{J^{-\frac{2}{3}}\pbrac{\tensorfour{I}-\dfrac{\tensorprod{\tensor{C}}{\tensor{B}}}{3}}}-pJ\tensor{B}\\
    &=J^{-\frac{2}{3}}\doubledotprod{\pullback{\tensorfour{M}}{}}{\bar{\tensortwo{S}}}-pJ\tensor{B}\\
    &=\doubledotprod{\pullback{\tensorfour{M}}{}}{\tensortwo{S}}+\doubledotprod{\pullback{\tensorfour{K}}{}}{\tensortwo{S}}\\
    &=\tensor{S}_{\pullback{\Devop}{}}+\tensor{S}_{\pullback{\Sphop}{}}
  \end{split}
\end{equation}
where
\begin{equation}
  \tensor{S}_{\pullback{\Devop}{}}=\doubledotprod{\pullback{\tensorfour{M}}{}}{\tensortwo{S}}=J^{-\frac{2}{3}}\doubledotprod{\pullback{\tensorfour{M}}{}}{\bar{\tensortwo{S}}}
\end{equation}
and
\begin{equation}
  \tensor{S}_{\pullback{\Sphop}{}}=\doubledotprod{\pullback{\tensorfour{K}}{}}{\tensortwo{S}}=-pJ\tensor{B}
\end{equation}

We can now push the second Piola-Kirchoff stress forward to give the Cauchy
stress \ie
\begin{equation}
  \begin{split}
    \tensor{\sigma}&=\inverse{J}\pushforward{\chi}{\tensor{S}}=\inverse{J}\tensor{F}\tensor{S}\transpose{\tensor{F}}
    \\
    &=\inverse{J}\pushforward{\chi}{\pbrac{\tensor{S}_{\pullback{\Devop}{}}+\tensor{S}_{\pullback{\Sphop}{}}}}\\
    &=J^{-\frac{2}{3}}\doubledotprod{\tensorfour{m}}{\bar{\tensortwo{\sigma}}}-p\inverse{\tensor{g}}\\
    &=\doubledotprod{\tensorfour{m}}{\tensortwo{\sigma}}+\doubledotprod{\tensorfour{k}}{\tensortwo{\sigma}}\\
    &=\tensor{\sigma}_{\devop}+\tensor{\sigma}_{\sphop}
  \end{split}
\end{equation}
where
\begin{equation}
  \tensor{\sigma}_{\devop}=\doubledotprod{\tensorfour{m}}{\tensortwo{\sigma}}=J^{-\frac{2}{3}}\doubledotprod{\tensorfour{m}}{\bar{\tensortwo{\sigma}}}
\end{equation}
and
\begin{equation}
  \tensor{\sigma}_{\sphop}=\doubledotprod{\tensorfour{k}}{\tensortwo{\sigma}}=-p\inverse{\tensor{g}}
\end{equation}
with $\bar{\tensor{\sigma}}=\inverse{J}\pushforward{\chi}{\bar{\tensor{S}}}$
being the \emph{Cauchy psuedo stress
  tensor}\symbolat{$\bar{\tensortwo{\sigma}}$}{Cauchy psuedo stress
  tensor}.

We can also use the constitutive law to obtain the elasticity tensor \ie
\begin{equation}
  \fnof{\tensorfour{C}}{\tensor{C}}=\delby{\fnof{\tensor{S}}{\tensor{C}}}{\tensor{C}}=\deltwosqby{\fnof{W}{\tensor{C}}}{\tensor{C}}=\deltwosqby{\fnof{\bar{W}}{\bar{\tensor{C}},J}}{\tensor{C}}
\end{equation}
or
\begin{equation}
  \fnof{\tensorfour{C}}{\tensor{E}}=\delby{\fnof{\tensor{S}}{\tensor{E}}}{\tensor{E}}=\deltwosqby{\fnof{W}{\tensor{E}}}{\tensor{E}}=\deltwosqby{\fnof{\bar{W}}{\bar{\tensor{E}},J}}{\tensor{E}}
\end{equation}

TODO CHECK THE EXPRESSIONS INVOLVING C AND CHECK FOR THE ``2'' FACTORS?

Now if we decompose the strain energy function into volumetric/spherical and
isochoric/deviatoric components we have
\begin{multline}
  \fnof{\tensorfour{C}}{\tensor{C}}=2\doubledotprodthree{\transpose{\pbrac{\delby{\bar{\tensor{C}}}{\tensor{C}}}}}{\deltwosqby{\fnof{\bar{W}}{\bar{\tensor{C}},J}}{\bar{\tensor{C}}}}{\delby{\bar{\tensor{C}}}{\tensor{C}}}+2\doubledotprod{\delby{\fnof{\bar{W}}{\bar{\tensor{C}},J}}{\bar{\tensor{C}}}}{\deltwosqby{\bar{\tensor{C}}}{\tensor{C}}}\\
  +2\deltwosqby{\fnof{\bar{W}}{\bar{\tensor{C}},J}}{J}\tensorprod{\delby{J}{\tensor{C}}}{\delby{J}{\tensor{C}}}+2\delby{\fnof{\bar{W}}{\bar{\tensor{C}},J}}{J}\deltwosqby{J}{\tensor{C}}\\
  +2\tensorprod{\delby{J}{\tensor{C}}}{\pbrac{\doubledotprod{\deltwoby{\fnof{\bar{W}}{\bar{\tensor{C}},J}}{J}{\bar{\tensor{C}}}}{\delby{\bar{\tensor{C}}}{\tensor{C}}}}}+2\tensorprod{\pbrac{\doubledotprod{\deltwoby{\fnof{\bar{W}}{\bar{\tensor{C}},J}}{J}{\bar{\tensor{C}}}}{\delby{\bar{\tensor{C}}}{\tensor{C}}}}}{\delby{J}{\tensor{C}}}
\end{multline}
or
\begin{multline}
  \fnof{\tensorfour{C}}{\tensor{E}}=\doubledotprodthree{\transpose{\pbrac{\delby{\bar{\tensor{E}}}{\tensor{E}}}}}{\deltwosqby{\fnof{\bar{W}}{\bar{\tensor{E}},J}}{\bar{\tensor{E}}}}{\delby{\bar{\tensor{E}}}{\tensor{E}}}+\doubledotprod{\delby{\fnof{\bar{W}}{\bar{\tensor{E}},J}}{\bar{\tensor{E}}}}{\deltwosqby{\bar{\tensor{E}}}{\tensor{E}}}\\
  +\deltwosqby{\fnof{\bar{W}}{\bar{\tensor{E}},J}}{J}\tensorprod{\delby{J}{\tensor{E}}}{\delby{J}{\tensor{E}}}+\delby{\fnof{\bar{W}}{\bar{\tensor{E}},J}}{J}\deltwosqby{J}{\tensor{E}}\\
  +\tensorprod{\delby{J}{\tensor{E}}}{\pbrac{\doubledotprod{\deltwoby{\fnof{\bar{W}}{\bar{\tensor{E}},J}}{J}{\bar{\tensor{E}}}}{\delby{\bar{\tensor{E}}}{\tensor{E}}}}}+\tensorprod{\pbrac{\doubledotprod{\deltwoby{\fnof{\bar{W}}{\bar{\tensor{E}},J}}{J}{\bar{\tensor{E}}}}{\delby{\bar{\tensor{E}}}{\tensor{E}}}}}{\delby{J}{\tensor{E}}}
\end{multline}

If we now define
\begin{equation}
  \bar{\tensorfour{C}}=2\deltwosqby{\fnof{\bar{W}}{\bar{\tensor{C}},J}}{\bar{\tensor{C}}}=\deltwosqby{\fnof{\bar{W}}{\bar{\tensor{E}},J}}{\bar{\tensor{E}}}
\end{equation}
as the \emph{material psuedo second elasticity
  tensor}\symbolat{$\bar{\tensorfour{C}}$}{Material pseudo second elasticity tensor} and
\begin{equation}
  \tensor{Y}=2\deltwoby{\fnof{\bar{W}}{\bar{\tensor{C}},J}}{J}{\bar{\tensor{C}}}=\deltwoby{\fnof{\bar{W}}{\bar{\tensor{E}},J}}{J}{\bar{\tensor{E}}}
\end{equation}
as the \emph{material coupling tensor}\symbolat{$\tensortwo{Y}$}{Material coupling tensor} (\ie representing any coupling between
deviatoric and spherical parts of the stress) and
\begin{equation}
  K=\deltwosqby{\fnof{\bar{W}}{\bar{\tensor{C}},J}}{J}=\deltwosqby{\fnof{\bar{W}}{\bar{\tensor{E}},J}}{J}
\end{equation}
as the \emph{large strain bulk modulus}\symbolat{$K$}{Large strain bulk modulus}, then we obtain
\begin{multline}
  \fnof{\tensorfour{C}}{\tensor{C}}=\doubledotprodthree{\transpose{\pbrac{\delby{\bar{\tensor{C}}}{\tensor{C}}}}}{\bar{\tensorfour{C}}}{\delby{\bar{\tensor{C}}}{\tensor{C}}}+\doubledotprod{\bar{\tensor{S}}}{\deltwosqby{\bar{\tensor{C}}}{\tensor{C}}}\\
  +2K\tensorprod{\delby{J}{\tensor{C}}}{\delby{J}{\tensor{C}}}-2p\deltwosqby{J}{\tensor{C}}\\
  +\tensorprod{\delby{J}{\tensor{C}}}{\pbrac{\doubledotprod{\tensor{Y}}{\delby{\bar{\tensor{C}}}{\tensor{C}}}}}+\tensorprod{\pbrac{\doubledotprod{\tensor{Y}}{\delby{\bar{\tensor{C}}}{\tensor{C}}}}}{\delby{J}{\tensor{C}}}
\end{multline}
or
\begin{multline}
  \fnof{\tensorfour{C}}{\tensor{E}}=\doubledotprodthree{\transpose{\pbrac{\delby{\bar{\tensor{E}}}{\tensor{E}}}}}{\bar{\tensorfour{C}}}{\delby{\bar{\tensor{E}}}{\tensor{E}}}+\doubledotprod{\bar{\tensor{S}}}{\deltwosqby{\bar{\tensor{E}}}{\tensor{E}}}\\
  +K\tensorprod{\delby{J}{\tensor{E}}}{\delby{J}{\tensor{E}}}-p\deltwosqby{J}{\tensor{E}}\\
  +\tensorprod{\delby{J}{\tensor{E}}}{\pbrac{\doubledotprod{\tensor{Y}}{\delby{\bar{\tensor{E}}}{\tensor{E}}}}}+\tensorprod{\pbrac{\doubledotprod{\tensor{Y}}{\delby{\bar{\tensor{E}}}{\tensor{E}}}}}{\delby{J}{\tensor{E}}}
\end{multline}

Using
\begin{align}
  2\delby{J}{\tensor{C}}&=\delby{J}{\tensor{E}}=J\tensor{B} \\
  4\tensorprod{\delby{J}{\tensor{C}}}{\delby{J}{\tensor{C}}}&=\tensorprod{\delby{J}{\tensor{E}}}{\delby{J}{\tensor{E}}}=\tensorprod{\pbrac{J\tensor{B}}}{\pbrac{J\tensor{B}}}=3J^{2}\tensorfour{K}^{\sharp
    *} \\
  \delby{\bar{\tensor{C}}}{\tensor{C}}&=\delby{\bar{\tensor{E}}}{\tensor{E}}=J^{-\frac{2}{3}}\pbrac{\transpose{\tensorfour{I}}-\dfrac{\tensorprod{\tensor{C}}{\tensor{B}}}{3}}=J^{-\frac{2}{3}}\tensorfour{M}^{*T} \\
  4\deltwosqby{J}{\tensor{C}}&=\deltwosqby{J}{\tensor{E}}=\tensorprod{\tensor{B}}{\pbrac{J\tensor{B}}}+2J\delby{\tensor{B}}{\tensor{C}}=3J\tensorfour{M}^{\sharp
    *}-2J\tensorfour{I}^{\sharp
    *} \\
  \doubledotprod{\tensor{Q}}{\deltwosqby{\bar{\tensor{C}}}{\tensor{C}}}&=\doubledotprod{\tensor{Q}}{\deltwosqby{\bar{\tensor{E}}}{\tensor{E}}}=\dfrac{2J^{-\frac{2}{3}}\pullback{\Trop}{\tensor{Q}}}{3}\tensorfour{M}^{\sharp
  *}-\dfrac{2J^{-\frac{2}{3}}}{3}\sqbrac{\tensorprod{\tensor{B}}{\pbrac{\doubledotprod{\pullback{\tensorfour{M}}{}}{\tensor{Q}}}}+\tensorprod{\pbrac{\doubledotprod{\pullback{\tensorfour{M}}{}}{\tensor{Q}}}}{\tensor{B}}}
\end{align}
we can derive the second material elasticity tensor
\begin{multline}
  \tensorfour{C}=3J\pbrac{JK-p}\tensorfour{K}^{\sharp
    *}+2Jp\tensorfour{I}^{\sharp
    *}\\
  +J^{\frac{1}{3}}\sqbrac{\tensorprod{\tensortwo{B}}{\pbrac{\doubledotprod{\pullback{\tensorfour{M}}}{\tensortwo{Y}}{}}}+\tensorprod{\pbrac{\doubledotprod{\pullback{\tensorfour{M}}{}}{\tensortwo{Y}}}}{\tensortwo{B}}}
  \\
  +J^{-\frac{4}{3}}\doubledotprodthree{\pullback{\tensorfour{M}}{}}{\bar{\tensorfour{C}}}{\tensorfour{M}^{*T}}+\dfrac{2}{3}J^{-\frac{2}{3}}\pullback{\Trop}{}\bar{\tensortwo{S}}\tensorfour{M}^{\sharp
    *}\\
  -\dfrac{2}{3}\sqbrac{\tensorprod{\tensortwo{B}}{\tensor{S}_{\pullback{\Devop}{}}}+\tensorprod{\tensortwo{S}_{\pullback{\Devop}{}}}{\tensortwo{B}}}
\end{multline}

Now using the inverse Piola transforms which push forward we have
$\bar{\tensorfour{c}}=\inverse{J}\pushforward{\chi}{\bar{\tensorfour{C}}}$,
$\bar{\tensortwo{\sigma}}=\inverse{J}\pushforward{\chi}{\bar{\tensortwo{S}}}$,
$\tensortwo{\sigma}_{\devop}=\inverse{J}\pushforward{\chi}{\tensortwo{S}_{\pullback{\Devop}{}}}$,
$\tensor{y}=\inverse{J}\pushforward{\chi}{\tensortwo{Y}}$ and
$\trop\bar{\tensortwo{\sigma}}=\pullback{\Trop}{}\bar{\tensortwo{S}}$ where
$\bar{\tensorfour{c}}$ is the \emph{spatial second psuedo elasticity
  tensor}\symbolat{$\bar{\tensorfour{c}}$}{spatial second psuedo elasticity
  tensor}, we can derive the second spatial elasticity tensor \ie
\begin{multline}
  \tensorfour{c}=3\pbrac{JK-p}\tensorfour{k}^{\sharp}+2p\tensorfour{i}^{\sharp}\\
  +J^{\frac{1}{3}}\sqbrac{\tensorprod{\inverse{\tensortwo{g}}}{\pbrac{\doubledotprod{\tensorfour{m}}{\tensortwo{y}}}}+\tensorprod{\pbrac{\doubledotprod{\tensorfour{m}}{\tensortwo{y}}}}{\inverse{\tensortwo{g}}}}
  \\
  +J^{-\frac{4}{3}}\doubledotprodthree{\tensorfour{m}}{\bar{\tensorfour{c}}}{\transpose{\tensorfour{m}}}+\dfrac{2}{3}J^{-\frac{2}{3}}\trop\bar{\tensortwo{\sigma}}\tensorfour{m}^{\sharp}\\
  -\dfrac{2}{3}\sqbrac{\tensorprod{\inverse{\tensortwo{g}}}{\tensor{\sigma}_{\devop}}+\tensorprod{\tensortwo{\sigma}_{\devop}}{\inverse{\tensortwo{g}}}}
\end{multline}

Note that if we do not allow for any coupling between the deviatoric and
spherical parts of stress or for any change in pressure with volume then we
can make the following simplicifications
\begin{equation}
  \tensor{Y}=2\deltwoby{\fnof{\bar{W}}{\bar{\tensor{C}},J}}{J}{\bar{\tensor{C}}}=\deltwoby{\fnof{\bar{W}}{\bar{\tensor{E}},J}}{J}{\bar{\tensor{E}}}=\tensor{0}
\end{equation}
and
\begin{equation}
  K=\deltwosqby{\fnof{\bar{W}}{\bar{\tensor{C}},J}}{J}=\deltwosqby{\fnof{\bar{W}}{\bar{\tensor{E}},J}}{J}=0
\end{equation}

We thus have
\begin{equation}
  \fnof{\tensorfour{C}}{\tensor{C}}=\doubledotprodthree{\transpose{\pbrac{\delby{\bar{\tensor{C}}}{\tensor{C}}}}}{\bar{\tensorfour{C}}}{\delby{\bar{\tensor{C}}}{\tensor{C}}}+\doubledotprod{\bar{\tensor{S}}}{\deltwosqby{\bar{\tensor{C}}}{\tensor{C}}}-2p\deltwosqby{J}{\tensor{C}}
\end{equation}
or
\begin{equation}
  \fnof{\tensorfour{C}}{\tensor{E}}=\doubledotprodthree{\transpose{\pbrac{\delby{\bar{\tensor{E}}}{\tensor{E}}}}}{\bar{\tensorfour{C}}}{\delby{\bar{\tensor{E}}}{\tensor{E}}}+\doubledotprod{\bar{\tensor{S}}}{\deltwosqby{\bar{\tensor{E}}}{\tensor{E}}}-p\deltwosqby{J}{\tensor{E}}
\end{equation}

The second material elasticity tensor is thus
\begin{multline}
  \tensorfour{C}=-pJ\pbrac{3\tensorfour{K}^{\sharp *}-2\tensorfour{I}^{\sharp
      *}}\\
  +J^{-\frac{4}{3}}\doubledotprodthree{\pullback{\tensorfour{M}}{}}{\bar{\tensorfour{C}}}{\tensorfour{M}^{*T}}+\dfrac{2}{3}J^{-\frac{2}{3}}\pullback{\Trop}{}\bar{\tensortwo{S}}\tensorfour{M}^{\sharp
    *}\\
  -\dfrac{2}{3}\sqbrac{\tensorprod{\tensortwo{B}}{\tensor{S}_{\pullback{\Devop}{}}}+\tensorprod{\tensortwo{S}_{\pullback{\Devop}{}}}{\tensortwo{B}}}
\end{multline}
and the second spatial elasticity tensor is thus
\begin{multline}
  \tensorfour{c}=-p\pbrac{3\tensorfour{k}^{\sharp}-2\tensorfour{i}^{\sharp}}\\
  \\
  +J^{-\frac{4}{3}}\doubledotprodthree{\tensorfour{m}}{\bar{\tensorfour{c}}}{\transpose{\tensorfour{m}}}+\dfrac{2}{3}J^{-\frac{2}{3}}\trop\bar{\tensortwo{\sigma}}\tensorfour{m}^{\sharp}\\
  -\dfrac{2}{3}\sqbrac{\tensorprod{\inverse{\tensortwo{g}}}{\tensor{\sigma}_{\devop}}+\tensorprod{\tensortwo{\sigma}_{\devop}}{\inverse{\tensortwo{g}}}}
\end{multline}


\clearpage

Because $\tensor{C}$ is symmetric then we can deal with the invariants \ie
We thus have
$\fnof{W}{\fnof{\tensor{C}}{\vectr{N}}}=\fnof{W}{\fnof{I_{1}}{\tensor{C}},\fnof{I_{2}}{\tensor{C}},\fnof{I_{3}}{\tensor{C}}}$ and
thus
\begin{equation}
  S^{AB}=2\pbrac{\delby{W}{I_{1}}\delby{I_{1}}{C_{AB}}+\delby{W}{I_{2}}\delby{I_{2}}{C_{AB}}+\delby{W}{I_{3}}\delby{I_{3}}{C_{AB}}}
\end{equation}
or if we have
$\fnof{W}{\fnof{\tensor{E}}{\vectr{N}}}=\fnof{W}{\fnof{I_{1}}{\tensor{E}},\fnof{I_{2}}{\tensor{E}},\fnof{I_{3}}{\tensor{E}}}$ and
thus
\begin{equation}
  S^{AB}=\pbrac{\delby{W}{I_{1}}\delby{I_{1}}{E_{AB}}+\delby{W}{I_{2}}\delby{I_{2}}{E_{AB}}+\delby{W}{I_{3}}\delby{I_{3}}{E_{AB}}}
\end{equation}

The three invariants are given by
\begin{equation}
  \begin{split}
    \fnof{I_{1}}{\tensor{C}} &= \trop\tensor{C} \\
    &= C_{11} + C_{22} + C_{33} \\
    \fnof{I_{2}}{\tensor{C}} &=
    \dfrac{1}{2}\pbrac{\pbrac{\trop\tensor{C}}^{2}-\trop\tensor{C}^{2}}
    = \determinant{\tensor{C}}\trop\inverse{\tensor{C}}\\
    &=
    \dfrac{1}{2}\left(\pbrac{C_{11}+C_{22}+C_{33}}^{2}\right. \\
      & \quad\left.-\pbrac{C_{11}^{2}+C_{12}C_{21}+C_{13}C_{31}+
        C_{21}C_{12}+C_{22}^{2}+C_{23}C_{32}+C_{31}C_{13}+C_{32}C_{23}+C_{33}^{2}}\right) \\
    \fnof{I_{3}}{\tensor{C}} &= \determinant{\tensor{C}} \\
    &=C_{11}C_{22}C_{33}+C_{12}C_{23}C_{31}+C_{13}C_{21}C_{32}\\
    &\quad-C_{13}C_{22}C_{31}-C_{12}C_{21}C_{33}-C_{11}C_{23}C_{32}
  \end{split}
\end{equation}

Now we have
\begin{align}
  \delby{I_{1}}{\tensor{C}}&=\sharptensor{\tensor{G}}\\
  \delby{I_{2}}{\tensor{C}}&=\inverse{\tensor{C}}\determinant{\tensor{C}}\trop\inverse{\tensor{C}}-\determinant{\tensor{C}}\trop\delby{\inverse{\tensor{C}}}{\tensor{C}}\\
  &=I_{2}\inverse{\tensor{C}}-I_{3}\invsquared{\tensor{C}} \\
  \delby{I_{3}}{\tensor{C}}&=\determinant{\tensor{C}}\inverse{\tensor{C}} \\
  &= I_{3}\inverse{\tensor{C}} 
\end{align}
and thus
\begin{equation}
  \tensor{S}=2\pbrac{\delby{W}{I_{1}}\sharptensor{\tensor{G}}+\pbrac{\delby{W}{I_{2}}I_{2}+\delby{W}{I_{3}}I_{3}}\inverse{\tensor{C}}+\delby{W}{I_{2}}I_{3}\invsquared{\tensor{C}}}
\end{equation}
or equivalently
\begin{equation}
  \tensor{S}=2\pbrac{\delby{W}{I_{1}}\sharptensor{\tensor{G}}+\pbrac{\delby{W}{I_{2}}I_{2}+\delby{W}{I_{3}}I_{3}}\tensor{B}+\delby{W}{I_{2}}I_{3}\tensor{B}^{2}}
\end{equation}
as $\tensor{B}=\inverse{\tensor{C}}$.

Note that
\begin{equation}
  \inverse{\tensor{C}}=\dfrac{1}{\determinant{\tensor{C}}}\adjop\tensor{C}=\dfrac{1}{I_{3}}\adjop\tensor{C}=\dfrac{1}{I_{3}}\delby{I_{3}}{\tensor{C}}
\end{equation}

Now we have
\begin{equation}
  \delby{I_{1}}{C_{AB}}=\begin{bmatrix}
    1 & 0 & 0 \\
    0 & 1 & 0 \\
    0 & 0 & 1
  \end{bmatrix}
\end{equation}
and
\begin{equation}
  \delby{I_{2}}{C_{AB}}=\begin{bmatrix}
    C_{22}+C_{33} & -C_{21} & -C_{31} \\
    -C_{12} & C_{11}+C_{33} & -C_{32} \\
    -C_{13} & -C_{23} & C_{11}+C_{22}
  \end{bmatrix}
\end{equation}
and
\begin{equation}
  \delby{I_{3}}{C_{AB}}=\begin{bmatrix}
    C_{22}C_{33}-C_{23}C_{32} & C_{23}C_{31}-C_{21}C_{33} & C_{23}C_{32}-C_{22}C_{31} \\
    C_{13}C_{32}-C_{12}C_{33} & C_{11}C_{33}-C_{13}C_{31} & C_{12}C_{31}-C_{11}C_{32} \\
    C_{12}C_{32}-C_{22}C_{31} & C_{13}C_{23}-C_{11}C_{23} & C_{11}C_{22}-C_{12}C_{21}
  \end{bmatrix}
\end{equation}

\subsubsubsection{Viscoelasticity}

\subsubsubsection{Plasticity}


\subsubsection{Anisotropy}

In order to deal with anisotropy we wish to base our stress and strain
calculation on fibre coordinates in the reference configuration where they are
orthogonal \ie $\vectr{N}$ coordinates. To change our reference coordinate
system from $\vectr{X}$ to $\vectr{N}$ we need to transform
$\fnof{\tensor{F}}{\vectr{X}}$. As $\fnof{\tensor{F}}{\vectr{X}}$ is a two
point tensor the transformation rule for transforming just the reference
coordinates is given by
\begin{equation}
  \fnof{\tensor{F}}{\vectr{N}}=\fnof{\tensor{F}}{\vectr{X}}\transpose{\tensor{Q}}
\end{equation}
where $\tensor{Q}$ is the rotation matrix from $\vectr{X}$ to $\vectr{N}$ in
the reference configuration.

Consider $\vectr{X}=X^{I}\vectr{G}_{I}$ being the underformed geometric
coordinate vector, $\vectr{N}=N^{I}\vectr{G}_{I}$ the undeformed fibre
coordinate vector with respect to underformed geometric coordinates,
$\tilde{\vectr{N}}=\tilde{N}^{A}\tilde{\vectr{N}}_{A}$ the undeformed fibre
coordinate vector with respect to undeformed fibre coordinates,
$\vectr{x}=x^{i}\vectr{g}_{i}$ the deformed geometric coordinate vector,
$\vectr{\nu}=\nu^{i}\vectr{g}_{i}$ the deformed fibres with respect to the
deformed geometric coordinates,
$\tilde{\vectr{\nu}}=\tilde{\nu}^{a}\tilde{\vectr{\nu}}_{a}$ the deformed
fibre coordinate vector with respect to deformed fibre coordinates,
$\tensor{F}=F^{i}_{I}\tensorprod{\vectr{g}_{i}}{\vectr{G}^{I}}$ the
deformation gradient tensor with respect to the undeformed and deformed
geometric coordinates, and
$\tilde{\tensor{F}}=\tilde{F}^{i}_{A}\tensorprod{\vectr{g}_{i}}{\tilde{\vectr{N}}^{A}}$
the deformation gradient tensor with respect to the undeformed fibre
coordinates and the deformed geometric coordinates,
$\tensor{Q}=Q^{A}_{I}\tensorprod{\tilde{\vectr{N}}_{A}}{\vectr{G}_{I}}$ the
orthogonal rotation matrix from undeformed geometric coordinates to undeformed
fibre coordinates,
$\tensor{\sigma}=\sigma^{ij}\tensorprod{\vectr{g}_{i}}{\vectr{g}_{j}}$ the
Cauchy stress tensor with respect to deformed geometric coordinates, and
$\tilde{\tensor{\sigma}}=\sigma^{ab}\tensorprod{\tilde{\vectr{\nu}}_{a}}{\tilde{\vectr{\nu}}_{b}}$
the Cauchy stress tensor with respect to the deformed fibre coordinates. Note
that $\tilde{\cdot}$ indicates a quantity with respect to fibre coordinates.

We have
\begin{align}
  \vectr{x}=x^{i}\vectr{g}_{i}&=F^{i}_{I}X^{I}\dotprod{\tensorprod{\vectr{g}_{i}}{\vectr{G}^{I}}}{\vectr{G}_{I}}=\tensor{F}\vectr{X}\\
  \tensor{Q}=Q^{A}_{I}\tensorprod{\tilde{\vectr{N}}_{A}}{\vectr{G}^{I}}&=\delby{\tilde{N}^{A}}{X^{I}}\tensorprod{\tilde{\vectr{N}}_{A}}{\vectr{G}^{I}}\\
  \tilde{\tensor{F}}=\tilde{F}^{i}_{A}\tensorprod{\vectr{g}_{i}}{\tilde{\vectr{N}}^{A}}&=F^{i}_{I}\pbrac{\transpose{Q}}^{I}_{A}\dotprod{\tensorprod{\vectr{g_{i}}}{\vectr{G}^{I}}}{\tensorprod{\vectr{G}_{I}}{\tilde{\vectr{N}}^{A}}}=\tensor{F}\transpose{\tensor{Q}}\\
  \tilde{\vectr{N}}=\tilde{N}^{A}\tilde{\vectr{N}}_{A}&=Q^{A}_{I}X^{I}\dotprod{\tensorprod{\tilde{\vectr{N}}_{A}}{\vectr{G}^{I}}}{\vectr{G}_{I}}=\tensor{Q}\vectr{X} \\
  \vectr{\nu}=\nu^{i}\vectr{g}_{i}&=\tilde{F}^{i}_{A}\tilde{N}^{A}\dotprod{\tensorprod{\vectr{g}_{i}}{\tilde{\vectr{N}}^{A}}}{\tilde{\vectr{N}}_{A}}=\tilde{\tensor{F}}\tilde{\vectr{N}}
\end{align}

Now we have $\tilde{\tensor{\sigma}}$ from the active model and we need
$\tensor{\sigma}$ for the integration. We can thus use
\begin{equation}
  \tensor{q}=q^{i}_{a}\tensorprod{\vectr{g}_{i}}{\tilde{\vectr{\nu}}^{a}}=\delby{x^{i}}{\tilde{\nu}^{a}}\tensorprod{\vectr{g}_{i}}{\tilde{\vectr{\nu}}^{a}}
\end{equation}
where $\tensor{q}$ is the rotation matrix from $\tilde{\vectr{\nu}}$ to
$\vectr{x}$ to give
\begin{equation}
  \tensor{\sigma}=\sigma^{ij}\tensorprod{\vectr{g}_{i}}{\vectr{g}_{j}}=q^{i}_{a}q^{j}_{b}\tilde{\sigma}^{ab}\dotprod{\dotprod{\tensorprod{\vectr{g}_{i}}{\tilde{\vectr{\nu}}^{a}}}{\tensorprod{\tilde{\vectr{\nu}}_{a}}{\tilde{\vectr{\nu}}_{b}}}}{\tensorprod{\tilde{\vectr{\nu}}^{b}}{\vectr{g}_{j}}}=\tensor{q}\tilde{\tensor{\sigma}}\transpose{\tensor{q}}
\end{equation}
NOTE: this might not be strictly correct??? The above assumes that
$\tensor{q}=\transpose{\tensor{q}}$??? $\vectr{x}$ is skew but
$\tilde{\vectr{\nu}}$ is not???

However, what is $\tilde{\vectr{\nu}}$? We have $\vectr{\nu}$ and so we can
find $\tilde{\vectr{\nu}}$ from
\begin{equation}
  \tilde{\vectr{\nu}}=\tilde{\nu}^{a}\tilde{\vectr{\nu}}_{a}=s^{a}_{i}\nu^{i}\dotprod{\tensorprod{\tilde{\vectr{\nu}}_{a}}{\vectr{g}^{i}}}{\vectr{g}_{i}}=\tensor{s}\vectr{\nu}
\end{equation}
where
\begin{equation}
  \tensor{s}=s^{a}_{i}\tensorprod{\tilde{\vectr{\nu}}_{a}}{\vectr{g}^{i}}=\delby{\tilde{\nu}^{a}}{\nu^{i}}\tensorprod{\tilde{\vectr{\nu}}_{a}}{\vectr{g}^{i}}
\end{equation}
which is the structure tensor which constructs the orthogonal
$\tilde{\vectr{\nu}}$ from the skewed $\vectr{\nu}$.

Note that in \OpenCMISS we have
\begin{equation}
  F^{i}_{A}=\delby{X^{M}}{\nu^{A}}F^{i}_{M}=\delby{X^{M}}{\nu^{A}}\delby{z^{i}}{\xi^{k}}\delby{\xi^{k}}{X^{M}}
\end{equation}

\subsubsection{Growth}

To allow for growth we use a multiplicative decomposition approach \ie
\begin{equation}
  \fnof{\tensor{F}}{\vectr{N}}=\fnof{\tensor{F}_{e}}{\vectr{N}}\fnof{\tensor{F}_{g}}{\vectr{N}}
\end{equation}
where $\fnof{\tensor{F}_{g}}{\vectr{N}}$ is the growth tensor with
respect to undeformed fibre coordinates and $\fnof{\tensor{F}_{e}}{\vectr{N}}$ is the
elastic component of the deformation gradient tensor in undeformed fibre coordinates.

The elastic component of the deformation gradient tensor can be calculated
from
\begin{equation}
  \fnof{\tensor{F}_{e}}{\vectr{N}}=\fnof{\tensor{F}}{\vectr{N}}\fnof{\inverse{\tensor{F}_{g}}}{\vectr{N}}
\end{equation}

In component form we have
\begin{equation}
  F^{i}_{A}=\pbrac{F_{e}}^{i}_{B}\pbrac{F_{g}}^{B}_{A}
\end{equation}
and
\begin{equation}
  \pbrac{F_{e}}^{i}_{B}=F^{i}_{A}\pbrac{\inverse{F_{g}}}^{A}_{B}
\end{equation}

The Jacobian of the growth component of the deformation is given by
$J_{g}=\determinant{\fnof{\tensor{F}_{g}}{\vectr{N}}}$ and the Jacobian of the
elastic component of the deformation is given by
$J_{e}=\determinant{\fnof{\tensor{F}_{e}}{\vectr{N}}}$.

The right Cauchy Green deformation tensor in fibre coordinates is now given by
the pullback of the current configuration metric tensor, $\tensor{g}$,
\begin{equation}
  \fnof{\tensor{C}}{\vectr{N}}=\fnof{\transpose{\tensor{F}_{e}}}{\vectr{N}}\tensor{g}\fnof{\tensor{F}_{e}}{\vectr{N}}
\end{equation}

In component form we have
\begin{equation}
  C_{AB}=g_{ij}\pbrac{F_{e}}^{i}_{A}\pbrac{F_{e}}^{j}_{B}
\end{equation}
and
\begin{equation}
  E_{AB}=\frac{1}{2}\pbrac{C_{AB}-G_{AB}}
\end{equation}

To find the stress tensors in deformed coordinates we need to push the second
Piola Kirchhoff tensor in the reference coordinates forward to the deformed
coordinates, $\vectr{x}$, to give the Kirchhoff stress tensor,
$\fnof{\tensor{\tau}}{\vectr{x}}$. The push foward is given by
\begin{equation}
  \fnof{\tensor{\tau}}{\vectr{x}}=\fnof{\tensor{F}_{e}}{\vectr{N}}\fnof{\tensor{S}}{\vectr{N}}
  \fnof{\transpose{\tensor{F}_{e}}}{\vectr{N}}
\end{equation}

The Cauchy stress tensor, $\fnof{\tensor{\sigma}}{\vectr{x}}$, can then be calculated from the Kirchhoff stress
tensor using the Jacobian of the deformation \ie
\begin{equation}
  \fnof{\tensor{\sigma}}{\vectr{x}}=\inverse{J_{e}}\fnof{\tensor{\tau}}{\vectr{x}}=\inverse{J_{e}}
  \fnof{\tensor{F}_{e}}{\vectr{N}}\fnof{\tensor{S}}{\vectr{N}}\fnof{\transpose{\tensor{F}_{e}}}{\vectr{N}}
\end{equation}7

In component form we have
\begin{equation}
  \tau^{ij}=\pbrac{F_{e}}^{i}_{B}S^{BC}\pbrac{\transpose{F_{e}}}^{j}_{C}
\end{equation}
and
\begin{equation}
  \sigma^{ij}=\inverse{J_{e}}\pbrac{F_{e}}^{i}_{B}S^{BC}\pbrac{\transpose{F_{e}}}^{j}_{C}
\end{equation}


\subsubsection{Elasticity Tensors}

If $\fnof{\tensor{P}}{\vectr{X},\tensor{F}}$ is the first Piola-Kirchoff
constituative function that depends on $\vectr{X}$ and the deformation
gradient tensor $\tensor{F}$. The \emph{first elasticity tensor}\symbolat{$\tensorfour{A}$}{First elasticity
  tensor},
$\tensorfour{A}$, is given by 
\begin{equation}
  \tensorfour{A}=\delby{\tensor{P}}{\tensor{F}}
\end{equation}
or
\begin{equation}
  A^{iAB}_{j}=\delby{P^{iA}}{F^{j}_{B}}
\end{equation}

Note that $\tensorfour{A}$ is a two-point tensor. 

If $\fnof{\tensor{S}}{\vectr{X},\tensor{C}}$ is the second Piola-Kirchoff
constitutive function that depends on $\vectr{X}$ and the right Cauchy-Green
strain tensor $\tensor{C}$. The \emph{second elasticity tensor}\symbolat{$\tensorfour{C}$}{Second elasticity
  tensor},
$\tensorfour{C}$, is given by 
\begin{equation}
  \tensorfour{C}=\delby{\tensor{S}}{\tensor{C}}
\end{equation}
or
\begin{equation}
  C^{ABCD}=\delby{S^{AB}}{C_{CD}}
\end{equation}

We can derive a relationship between $\tensorfour{A}$ and $\tensorfour{C}$. Differentiating
\begin{equation}
  \tensor{P}=\tensor{F}\tensor{S}
\end{equation}
or
\begin{equation}
  P^{iA}=F^{i}_{B}S^{BA}
\end{equation}
with respect to the deformation gradient tensor gives
\begin{equation}
  \delby{P^{iA}}{F^{j}_{B}}=F^{i}_{C}\delby{S^{CA}}{C_{DE}}\delby{C_{DE}}{F^{i}_{C}}+\delby{F^{i}_{C}}{F^{j}_{B}}T^{CA}
\end{equation}

Now
\begin{equation}
  C_{DE}=F^{l}_{D}F^{k}_{E}g_{lk}
\end{equation}
and so
\begin{equation}
  \delby{C_{DE}}{F^{j}_{B}}=\mixedkronecker{D}{B}F^{k}_{E}g_{jk}+F^{l}_{D}\mixedkronecker{E}{B}g_{lj}
\end{equation}

Substituting this into the above equation gives
\begin{equation}
  \begin{split}
    \delby{P^{iA}}{F^{j}_{B}}&=C^{CADE}\pbrac{\mixedkronecker{D}{B}F^{k}_{E}g_{jk}+
      F^{l}_{D}\mixedkronecker{E}{B}g_{lj}}F^{i}_{C}+T^{CA}\mixedkronecker{i}{j}\mixedkronecker{C}{B}\\
    &=C^{CABE}F^{k}_{E}F^{i}_{C}g_{jk}+C^{CADE}F^{l}_{D}F^{i}_{C}g_{lj}+T^{BA}\mixedkronecker{j}{i}
  \end{split}
\end{equation}

Now, using the symmetries $C^{CABE}=C^{CAEB}$ and $T^{AB}=T^{BA}$ we have
\begin{equation}
  \tensorfour{A}=2\dotprod{\tensorfour{C}}{\dotprod{\tensortwo{F}}{\dotprod{\tensortwo{F}}{\tensortwo{g}}}}+
  \tensorprod{\tensortwo{S}}{\tensortwo{I}}
\end{equation}
or
\begin{equation}
  A^{iAB}_{j}=2C^{CADB}F^{k}_{D}F^{i}_{C}g_{kj}+T^{AB}\mixedkronecker{j}{i}
\end{equation}

Now, we can also define the \emph{first spatial elasticity tensor}\symbolat{$\tensorfour{a}$}{First spatial
elasticity tensor},
$\tensorfour{a}$, and the \emph{second spatial elasticity tensor}\symbolat{$\tensorfour{c}$}{Second spatial elasticity
tensor},
$\tensorfour{c}$, using push forwards and Piola transforms of $\tensorfour{A}$
and $\tensorfour{C}$ respectively \ie
\begin{equation}
  \begin{split}
    \tensorfour{a}&=\frac{1}{J}\pushforward{\chi}\tensorfour{A} \\
    \tensorfour{c}&=\frac{2}{J}\pushforward{\chi}\tensorfour{C} \\
  \end{split}
\end{equation}
or
\begin{equation}
  \begin{split}
    a^{ikl}_{j}&=\frac{1}{J}F^{k}_{A}F^{l}_{B}A^{iAB}_{j} \\
    c^{ijkl}&=\frac{2}{J}F^{i}_{A}F^{j}_{B}F^{k}_{C}F^{l}_{D}C^{ABCD} \\
  \end{split}
\end{equation}

The relationship between $\tensorfour{a}$ and $\tensorfour{c}$ is given by
\begin{equation}
  \tensorfour{a}=2\dotprod{\tensorfour{c}}{\tensortwo{g}}+\tensorprod{\tensortwo{\sigma}}{\tensortwo{I}}
\end{equation}
or
\begin{equation}
  a^{ikl}_{j}=c^{ikml}g_{mj}+\sigma^{kl}\mixedkronecker{j}{i}
\end{equation}

There is also a relationship between $\tensorfour{A}$ and $\tensorfour{a}$
known as the \emph{Piola identity} \ie
\begin{equation}
  \divergence{\vectr{X}}{\pbrac{\dotprod{\tensorfour{A}}{\vectr{U}}}}=
  J\divergence{\vectr{x}}{\pbrac{\dotprod{\tensorfour{a}}{\vectr{u}}}}
\end{equation}
where
\begin{equation}
  \vectr{u}=\pushforward{\chi}{\vectr{U}}
\end{equation}

Note that the tensor $\tensorfour{c}$ is different to $\tensorfour{C}$ in that
it is \emph{not} given by $\delby{\sigma^{ij}}{c_{kl}}$ and $\sigma^{ij}\neq
2\delby{W}{c_{ij}}$. They are instead given by
\begin{equation}
  c^{ijkl}=\delby{\sigma^{ij}}{g_{kl}}
\end{equation}
and
\begin{equation}
  \sigma^{ij}=2\delby{W}{g_{ij}}
\end{equation}

\subsection{Principle of Virtual Work}

Consider a configuration, defined by a displacement field $\vectr{u}$, of some
deformable body, $\embedmanifold{B}$, subject to some \emph{displacement
  boundary conditions} $\vectr{u}=\bar{\vectr{u}}$ over some part of the
boundary $\boundary{\embedmanifold{B}}_{u}$ and some \emph{traction boundary
  conditions} $\vectr{t}=\dotprod{\tensor{\sigma}}{\vectr{n}}=\bar{\vectr{t}}$
over some part of the boundary $\boundary{\embedmanifold{B}}_{t}$. Note that
$\intersection{\boundary{\embedmanifold{B}}_{u}}{\boundary{\embedmanifold{B}}_{t}}=\emptyset$
and
$\union{\boundary{\embedmanifold{B}}_{u}}{\boundary{\embedmanifold{B}}_{t}}\subseteq\boundary{\embedmanifold{B}}$.

We wish to find the displacement field $\vectr{u}$ which satisfies the
deformation boundary conditions where they are applied and the equations of
motion and the traction boundary conditions where they are applied. A
displacement field $\vectr{u}$ which satisfies the displacemnet boundary
conditions where they are applied but whose resulting stress field does not
necessarily satisfy the equations of motion or the traction boundary
conditions where they are applied is known as a \emph{kinematically admissable
  displacement field}. Similarily, a stress field $\tensor{\sigma}$ which
satisfies the equations of motion and the traction boundary conditions where
they are applied but whost resulting displacement field does not necessarily
satisfy the displacement boundary conditions where they are applied is known
as a \emph{statically admissable stress field}.

For a statically admissable stress field $\tensor{\sigma}$ conservation of
momentum gives us the equations of motion \ie
\begin{equation}
  \gint{\embedmanifold{B}}{}{\rho\vectr{a}}{v}=\gint{\embedmanifold{B}}{}{\pbrac{\divergence{}{\tensor{\sigma}}+\vectr{b}}}{v}
\end{equation}
where $\vectr{a}$ is the acceleration of the body and $\vectr{b}$ are the body
forces. If we now multiply by the equations of motion by a kinematically
admissable displacement field $\vectr{u}$ we obtain
\begin{equation}
  \begin{split}
    \gint{\embedmanifold{B}}{}{\dotprod{\rho\vectr{a}}{\vectr{u}}}{v} &=
    \gint{\embedmanifold{B}}{}{\dotprod{\pbrac{\divergence{}{\tensor{\sigma}}+\vectr{b}}}{\vectr{u}}}{v}\\
    &=\gint{\embedmanifold{B}}{}{\dotprod{\pbrac{\divergence{}{\tensor{\sigma}}}}{\vectr{u}}}{v}+
    \gint{\embedmanifold{B}}{}{\dotprod{\vectr{b}}{\vectr{u}}}{v}
  \end{split}
  \label{eqn:equationsofwork}
\end{equation}

Now, by the vector identity
\begin{equation}
  \divergence{}{\pbrac{\dotprod{\tensortwo{\sigma}}{\vectr{u}}}}=\dotprod{\pbrac{\divergence{}{\tensortwo{\sigma}}}}{\vectr{u}}+
  \doubledotprod{\transpose{\tensortwo{\sigma}}}{\gradient{}{\vectr{u}}}
\end{equation}
we have
\begin{equation}
  \dotprod{\pbrac{\divergence{}{\tensortwo{\sigma}}}}{\vectr{u}}=\divergence{}{\pbrac{\dotprod{\tensortwo{\sigma}}{\vectr{u}}}}-
  \doubledotprod{\tensortwo{\sigma}}{\gradient{}{\vectr{u}}}
\end{equation}
as the stress tensor, $\tensortwo{\sigma}$, symmetric and so
$\transpose{\tensortwo{\sigma}}=\tensortwo{\sigma}$. \Eqnref{eqn:equationsofwork} becomes
\begin{equation}
  \begin{split}
    \gint{\embedmanifold{B}}{}{\dotprod{\rho\vectr{a}}{\vectr{u}}}{v}&=
    \gint{\embedmanifold{B}}{}{\pbrac{\divergence{}{\pbrac{\dotprod{\tensor{\sigma}}{\vectr{u}}}}-
        \doubledotprod{\tensor{\sigma}}{\gradient{}{\vectr{u}}}}}{v}+
    \gint{\embedmanifold{B}}{}{\dotprod{\vectr{b}}{\vectr{u}}}{v} \\
    &=\gint{\embedmanifold{B}}{}{\divergence{}{\pbrac{\dotprod{\tensor{\sigma}}{\vectr{u}}}}}{v}-
    \gint{\embedmanifold{B}}{}{\doubledotprod{\tensor{\sigma}}{\gradient{}{\vectr{u}}}}{v}+
    \gint{\embedmanifold{B}}{}{\dotprod{\vectr{b}}{\vectr{u}}}{v}
  \end{split}
\end{equation}

Applying the divergence theorem to the first integral on the right hand side
gives
\begin{equation}
  \begin{split}
    \gint{\embedmanifold{B}}{}{\divergence{}{\pbrac{\dotprod{\tensor{\sigma}}{\vectr{u}}}}}{v}
    &=\gint{\boundary{\embedmanifold{B}}}{}{\dotprod{\pbrac{\dotprod{\tensor{\sigma}}{\vectr{u}}}}{\vectr{n}}}{a}\\
    &=\gint{\boundary{\embedmanifold{B}}}{}{\dotprod{\pbrac{\dotprod{\tensor{\sigma}}{\vectr{n}}}}{\vectr{u}}}{a}
  \end{split}
\end{equation}
or, by Cauchy's law, $\vectr{t}=\dotprod{\tensor{\sigma}}{\vectr{n}}$, we have
\begin{equation}
  \gint{\embedmanifold{B}}{}{\divergence{}{\pbrac{\dotprod{\tensor{\sigma}}{\vectr{u}}}}}{v}=
  \gint{\boundary{\embedmanifold{B}}}{}{\dotprod{\vectr{t}}{\vectr{u}}}{a}
\end{equation}

\Eqnref{eqn:equationsofwork} thus becomes
\begin{equation}
  \gint{\embedmanifold{B}}{}{\dotprod{\rho\vectr{a}}{\vectr{u}}}{v}=
  \gint{\boundary{\embedmanifold{B}}}{}{\dotprod{\vectr{t}}{\vectr{u}}}{a}-
  \gint{\embedmanifold{B}}{}{\doubledotprod{\tensor{\sigma}}{\gradient{}{\vectr{u}}}}{v}+
  \gint{\embedmanifold{B}}{}{\dotprod{\vectr{b}}{\vectr{u}}}{v}
\end{equation}
or
\begin{equation}
  \gint{\embedmanifold{B}}{}{\dotprod{\rho\vectr{a}}{\vectr{u}}}{v}+
  \gint{\embedmanifold{B}}{}{\doubledotprod{\tensor{\sigma}}{\gradient{}{\vectr{u}}}}{v}=
  \gint{\boundary{\embedmanifold{B}}}{}{\dotprod{\vectr{t}}{\vectr{u}}}{a}+
  \gint{\embedmanifold{B}}{}{\dotprod{\vectr{b}}{\vectr{u}}}{v}
\end{equation}

Taking the boundary conditions into account we have
\begin{equation}
  \gint{\embedmanifold{B}}{}{\dotprod{\rho\vectr{a}}{\vectr{u}}}{v}+
  \gint{\embedmanifold{B}}{}{\doubledotprod{\tensor{\sigma}}{\gradient{}{\vectr{u}}}}{v}=
  \gint{\boundary{\embedmanifold{B}}_{u}}{}{\dotprod{\vectr{t}}{\bar{\vectr{u}}}}{a}+
  \gint{\boundary{\embedmanifold{B}}_{t}}{}{\dotprod{\bar{\vectr{t}}}{\vectr{u}}}{a}+
  \gint{\embedmanifold{B}}{}{\dotprod{\vectr{b}}{\vectr{u}}}{v}
  \label{eqn:virtualworkdisp1}
\end{equation}

If we now consider a second kinematically admissable displacement field
$\vectr{u}^{*}$ then the above equation will also hold \ie
\begin{equation}
  \gint{\embedmanifold{B}}{}{\dotprod{\rho\vectr{a}}{\vectr{u}^{*}}}{v}+
  \gint{\embedmanifold{B}}{}{\doubledotprod{\tensor{\sigma}}{\gradient{}{\vectr{u}^{*}}}}{v}=
  \gint{\boundary{\embedmanifold{B}}_{u}}{}{\dotprod{\vectr{t}}{\bar{\vectr{u}}}}{a}+
  \gint{\boundary{\embedmanifold{B}}_{t}}{}{\dotprod{\bar{\vectr{t}}}{\vectr{u}^{*}}}{a}+
  \gint{\embedmanifold{B}}{}{\dotprod{\vectr{b}}{\vectr{u}^{*}}}{v}
  \label{eqn:virtualworkdisp2}
\end{equation}

Now, subtracting \Eqnref{eqn:virtualworkdisp2} from
\Eqnref{eqn:virtualworkdisp1} gives us
\begin{equation}
  \gint{\embedmanifold{B}}{}{\dotprod{\rho\vectr{a}}{\pbrac{\vectr{u}-\vectr{u}^{*}}}}{v}+
  \gint{\embedmanifold{B}}{}{\doubledotprod{\tensor{\sigma}}{\gradient{}{\pbrac{\vectr{u}-\vectr{u}^{*}}}}}{v}=
  \gint{\boundary{\embedmanifold{B}}_{u}}{}{\dotprod{\vectr{t}}{\bar{\vectr{u}}}}{a}+
  \gint{\boundary{\embedmanifold{B}}_{t}}{}{\dotprod{\bar{\vectr{t}}}{\pbrac{\vectr{u}-\vectr{u}^{*}}}}{a}+
  \gint{\embedmanifold{B}}{}{\dotprod{\vectr{b}}{\pbrac{\vectr{u}-\vectr{u}^{*}}}}{v}
  \label{eqn:virtualworkdispdiff}
\end{equation}

If we now define the \emph{virtual displacements} as
$\delta\vectr{u}=\vectr{u}-\vectr{u}^{*}$ and note that
$\delta\vectr{u}=\vectr{u}-\vectr{u}^{*}=\bar{\vectr{u}}-\bar{\vectr{u}}=\vectr{0}$
on $\boundary{\embedmanifold{B}}_{u}$ then we have
\begin{equation}
  \gint{\embedmanifold{B}}{}{\dotprod{\rho\vectr{a}}{\delta\vectr{u}}}{v}+
  \gint{\embedmanifold{B}}{}{\doubledotprod{\tensor{\sigma}}{\gradient{}{\delta\vectr{u}}}}{v}=
  \gint{\boundary{\embedmanifold{B}}_{t}}{}{\dotprod{\bar{\vectr{t}}}{\delta\vectr{u}}}{a}+
  \gint{\embedmanifold{B}}{}{\dotprod{\vectr{b}}{\delta\vectr{u}}}{v}
  \label{eqn:virtualworkstatment}
\end{equation}

This is known as the \emph{Principle of Virtual Work} \ie when a deformable
body undergoes some virtual displacement, $\delta\vectr{u}$, the \emph{kinetic
  work}, $\fnof{W_{kin}}{\delta\vectr{u}}$, plus the \emph{internal work},
$\fnof{W_{int}}{\delta\vectr{u}}$, is balanced by the \emph{external work},
$\fnof{W_{ext}}{\delta\vectr{u}}$, where
\begin{equation}
  \begin{split}
    \fnof{W_{kin}}{\delta\vectr{u}}&=
    \gint{\embedmanifold{B}}{}{\rho\dotprod{\vectr{a}}{\delta\vectr{u}}}{v}\\
    \fnof{W_{int}}{\delta\vectr{u}}&=
    \gint{\embedmanifold{B}}{}{\doubledotprod{\tensor{\sigma}}{\gradient{}{\delta\vectr{u}}}}{v}\\
    \fnof{W_{ext}}{\delta\vectr{u}}&=\fnof{W_{surf}}{\delta\vectr{u}}+\fnof{W_{body}}{\delta\vectr{u}}\\
    &=\gint{\boundary{\embedmanifold{B}}_{t}}{}{\dotprod{\bar{\vectr{t}}}{\delta\vectr{u}}}{a}+
    \gint{\boundary{\embedmanifold{B}}_{p}}{}{\dotprod{\bar{\vectr{t}}}{\delta\vectr{u}}}{a}+
    \gint{\embedmanifold{B}}{}{\dotprod{\fnof{\vectr{b}}{\vectr{u}}}{\delta\vectr{u}}}{v}
  \end{split}
\end{equation}
where $\fnof{W_{surf}}{\vectr{u},\delta\vectr{u}}$ is the external work due to surface
forces and $\fnof{W_{body}}{\vectr{u},\delta\vectr{u}}$ is the external work due to
body forces. 

Note that the internal work is often writen in terms of \emph{virtual strain}
instead of virtual displacement. Consider
\begin{equation}
  \begin{split}
    \gradient{}{\delta\vectr{u}} &= 
    \dfrac{1}{2}\pbrac{\gradient{}{\delta\vectr{u}}+\transpose{\pbrac{\gradient{}{\delta\vectr{u}}}}}+
    \dfrac{1}{2}\pbrac{\gradient{}{\delta\vectr{u}}-\transpose{\pbrac{\gradient{}{\delta\vectr{u}}}}}\\
    &=\delta\sqbrac{\dfrac{1}{2}\pbrac{\gradient{}{\vectr{u}}+\transpose{\pbrac{\gradient{}{\vectr{u}}}}}}+
    \delta\sqbrac{\dfrac{1}{2}\pbrac{\gradient{}{\vectr{u}}-\transpose{\pbrac{\gradient{}{\vectr{u}}}}}} \\
    &=\delta\tensor{\epsilon} +\delta\tensor{\omega}
  \end{split}
\end{equation}
where $\tensor{\epsilon}$ is the \emph{small strain tensor} and
$\tensor{\omega}$ is the \emph{small rotation tensor}. The internal work is now
given by
\begin{equation}
  \begin{split}
    \fnof{W_{int}}{\vectr{u},\delta\vectr{u}}&=
    \gint{\embedmanifold{B}}{}{\doubledotprod{\fnof{\tensor{\sigma}}{\vectr{u}}}{\gradient{}{\delta\vectr{u}}}}{v}\\
    &=\gint{\embedmanifold{B}}{}{\doubledotprod{\fnof{\tensor{\sigma}}{\vectr{u}}}{
        \pbrac{\fnof{\delta\tensor{\epsilon}}{\delta\vectr{u}}+\fnof{\delta\tensor{\omega}}{\delta\vectr{u}}}}}{v} \\
    &=\gint{\embedmanifold{B}}{}{\doubledotprod{\fnof{\tensor{\sigma}}{\vectr{u}}}{\fnof{\delta\tensor{\epsilon}}{\delta\vectr{u}}}}{v}+
    \gint{\embedmanifold{B}}{}{\doubledotprod{\fnof{\tensor{\sigma}}{\vectr{u}}}{\fnof{\delta\tensor{\omega}}{\delta\vectr{u}}}}{v}
  \end{split}
\end{equation}
Therefore
\begin{equation}
  \fnof{W_{int}}{\vectr{u},\delta\vectr{u}}=
  \gint{\embedmanifold{B}}{}{\doubledotprod{\fnof{\tensor{\sigma}}{\vectr{u}}}{\fnof{\delta\tensor{\epsilon}}{\delta\vectr{u}}}}{v}
\end{equation}
as the stress tensor is a symmetric tensor and the small rotation tensor is a
skew-symmetric tensor and the double dot product between a symmetric and
skew-symmetric tensor is always zero.

In terms of strain the principle of virtual work can thus be stated as
\begin{equation}
  \gint{\embedmanifold{B}}{}{\dotprod{\rho\vectr{a}}{\delta\vectr{u}}}{v}+
  \gint{\embedmanifold{B}}{}{\doubledotprod{\tensor{\sigma}}{\fnof{\delta\tensor{\epsilon}}{\delta\vectr{u}}}}{v}=
  \gint{\boundary{\embedmanifold{B}}_{t}}{}{\dotprod{\bar{\vectr{t}}}{\delta\vectr{u}}}{a}+
  \gint{\embedmanifold{B}}{}{\dotprod{\vectr{b}}{\delta\vectr{u}}}{v}
  \label{eqn:virtualworkstrainstatment}
\end{equation}

TALK ABOUT WORK CONJUGACY AND SHOW OTHER CONJUGATE FORMS

The internal work can also be expressed in the undeformed manifold in terms of
the $2^{nd}$ Piola-Kirchoff stress, $\tensor{S}$ and the Green-Lagrange strain
tensor, $\tensor{E}$, \ie
\begin{equation}
  \fnof{W_{int}}{\vectr{u},\delta\vectr{u}}=\gint{\embedmanifold{B}_{0}}{}{\doubledotprod{
      \fnof{\tensor{S}}{\vectr{u}}}{\fnof{\delta\tensor{E}}{\delta\vectr{u}}}}{V}
\end{equation}

\subsection{Variational principles}

The principle of virtual work is basically a statement regarding the energy of
a system. In order to solve problems of mechanics we can use the principle of
energy minimisation in order to determine the deformed shape of a body \ie we
can use the principle that a body will adopt a deformed shape that minimises
the energy of the body. Put another way, if we could quatify the total energy
in a body with a function of some quantity (or quantities) that describes the
body we could determine the deformed shape by finding the value of that
quantity that minimises (or rather more strictly that finds the extremum of)
the function. Such a problem is known as a \emph{variational principle}.

There are a number of variational principles that are used in mechanics. The
differences in the principles arise primarily from the types and numbers of
quantities that are allowed to vary. The simplest variation principle is the
\emph{potential energy variational principle} in which just displacement is
allowed to vary. This principle is the basis of \emph{displacement
  methods}. Displacement methods, whilst simple, suffer from a number of
problems included volume locking and are not so good for nearly incompressible
and compressible materials. In order to handle material (in)compressibility
\emph{mixed methods} are used in which additional quantities are allowed in
the variational principle. The \emph{Hellinger-Reissner variational principle}
is allows for both displacement and stress fields which are allowed to
independently vary. The \emph{Hu-Washizu variational principle} allows for
displacement, stress and strain fields to independently vary. 

\subsubsection{Potential Energy}

The potential energy variational principle can be stated as the difference
between the internal and external energies \ie
\begin{equation}
  \fnof{\Pi_{PE}}{\vectr{u}}=\fnof{\Pi_{int}}{\vectr{u}}-\fnof{\Pi_{ext}}{\vectr{u}}
\end{equation}
where
\begin{equation}
  \fnof{\Pi_{int}}{\vectr{u}}=\gint{\embedmanifold{B}_{0}}{}{\fnof{W}{\vectr{u}}}{V}
\end{equation}
with $\fnof{W}{\vectr{u}}$ is the potential energy in the deformation due to
the displacement $\vectr{u}$ and
\begin{equation}
  \fnof{\Pi_{ext}}{\vectr{u}}=\gint{\embedmanifold{B}_{0}}{}{\dotprod{\vectr{b}}{\vectr{u}}}{V}+\gint{\boundary{\embedmanifold{B}}_{0_{t}}}{}{\dotprod{\vectr{\bar{\vectr{t}}}}{\vectr{u}}}{\covectr{A}}  
\end{equation}

The variation is that the potential energy is stationary \ie
\begin{equation}
  \variation{\fnof{\Pi_{PE}}{\vectr{u}}}{\vectr{u}}=\variation{\fnof{\Pi_{int}}{\vectr{u}}}{\vectr{u}}-\variation{\fnof{\Pi_{int}}{\vectr{u}}}{\vectr{u}}=0
\end{equation}

The variations are given by 
\begin{equation}
  \begin{split}
    \variation{\fnof{\Pi_{int}}{\vectr{u}}}{\vectr{u}}&=\gint{\embedmanifold{B}_{0}}{}{\variation{\fnof{W}{\vectr{u}}}{\vectr{u}}}{V}
    \\
    &=\gint{\embedmanifold{B}_{0}}{}{\doubledotprod{\fnof{\tensor{S}}{\fnof{\tensor{E}}{\vectr{u}}}}{\variation{\fnof{\tensor{E}}{\vectr{u}}}{\vectr{u}}}}{V}
  \end{split}
\end{equation}
via the virtual work theorem and
\begin{equation}
  \variation{\fnof{\Pi_{ext}}{\vectr{u}}}{\vectr{u}}=\gint{\embedmanifold{B}_{0}}{}{\dotprod{\vectr{b}}{\variationdir{\vectr{u}}}}{v}+\gint{\boundary{\embedmanifold{B}}_{0_{t}}}{}{\dotprod{\bar{\vectr{t}}}{\variationdir{\vectr{u}}}}{\covectr{A}}
\end{equation}

This variation is a nonlinear equation in $\variationdir{\vectr{u}}$. In order to
solve the nonlinear system of equations a Newton scheme can be used which
requires a Jacobian. The Jacobian is given by the derivative of the
variational statement in the direction of a change in $\vectr{u}$,
$\linearisationdir{\vectr{u}}$ \ie the Lie derivative of the variational statement in the
direction $\linearisationdir{\vectr{u}}$.

This requires a linerization. A linearisation of a function
$\fnof{f}{\vectr{x}}$ in the direction of $\linearisationdir{\vectr{x}}$ is
\begin{equation}
  \linearisation{f}{\vectr{x}}{\vectr{x}}=\dby{}{\epsilon}\evalat{\fnof{f}{\vectr{x}+\epsilon\linearisationdir{\vectr{x}}}}{\epsilon=0}=\fnof{f}{\vectr{x}}+\directionalderiv{\vectr{x}}{\fnof{f}{\vectr{x}}}{\linearisationdir{\vectr{x}}}
\end{equation}

The linearization of the variational statement is given by 
\begin{equation}
  \directionalderiv{}{\variation{\fnof{\Pi_{PE}}{\vectr{u}}}{\vectr{u}}}{\linearisationdir{\vectr{u}}}=
  \directionalderiv{}{\variation{\fnof{\Pi_{int}}{\vectr{u}}}{\vectr{u}}}{\linearisationdir{\vectr{u}}}
  -\directionalderiv{}{\variation{\fnof{\Pi_{ext}}{\vectr{u}}}{\vectr{u}}}{\linearisationdir{\vectr{u}}}
\end{equation}

For the directional derivative of the internal work variation (first term on
the right hand side) it is useful to consider the internal work in terms of
second Piola-Kirchoff stress and the Green-Lagrange strain \ie
\begin{equation}
  \begin{split}
    \directionalderiv{}{\delta\fnof{\Pi_{int}}{\vectr{u},\delta\vectr{u}}}{\linearisationdir{\vectr{u}}}
    &=\directionalderiv{}{\gint{\embedmanifold{B}_{0}}{}{\doubledotprod{\fnof{\tensor{S}}{
            \fnof{\tensor{E}}{\vectr{u}}}}{\variation{\fnof{\tensor{E}}{\vectr{u}}}{\vectr{u}}}}{V}}{\linearisationdir{\vectr{u}}}\\ &=\gint{\embedmanifold{B}_{0}}{}{\left(\doubledotprod{\fnof{\tensor{S}}{
          \fnof{\tensor{E}}{\vectr{u}}}}{\directionalderiv{}{\variation{\fnof{\tensor{E}}{\vectr{u}}}{\vectr{u}}}{\linearisationdir{\vectr{u}}}}\right.\\ &\qquad
      \qquad + \left.\doubledotprod{\directionalderiv{}{\fnof{\tensor{S}}{
            \fnof{\tensor{E}}{\vectr{u}}}}{\linearisationdir{\vectr{u}}}}{\variation{\fnof{\tensor{E}}{\vectr{u}}}{\vectr{u}}}\right)}{V}
  \end{split}
\end{equation}

Now, $\variation{\fnof{\tensortwo{E}}{\vectr{u}}}{\vectr{u}}$ can be found by pulling
back $\variation{\fnof{\tensortwo{\epsilon}}{\vectr{u}}}{\vectr{u}}$ \ie
\begin{equation}
  \begin{split}
    \variation{\fnof{\tensortwo{E}}{\vectr{u}}}{\vectr{u}}&=\pullback{\chi}{\pbrac{\variation{\fnof{\tensortwo{\epsilon}}{\vectr{u}}}{\vectr{u}}}}\\
    &=\transpose{\tensortwo{F}}\variation{\fnof{\tensortwo{\epsilon}}{\vectr{u}}}{\vectr{u}}\tensortwo{F}\\
    &=\transpose{\tensortwo{F}}\frac{1}{2}\pbrac{\transpose{\pbrac{\gradient{\vectr{x}}{\variationdir{\vectr{u}}}}}
      +\gradient{\vectr{x}}{\variationdir{\vectr{u}}}}\tensortwo{F}\\
    &=\frac{1}{2}\pbrac{\transpose{\tensortwo{F}}\transpose{\pbrac{\gradient{\vectr{x}}{\variationdir{\vectr{u}}}}}\tensortwo{F}+
      \transpose{\tensortwo{F}}\gradient{\vectr{x}}{\variationdir{\vectr{u}}}\tensortwo{F}}\\
    &=\frac{1}{2}\pbrac{\transpose{\pbrac{\gradient{\vectr{X}}{\variationdir{\vectr{u}}}}}\tensortwo{F}+
      \transpose{\tensortwo{F}}\gradient{\vectr{X}}{\variationdir{\vectr{u}}}} \\
    &= \symop\pbrac{\transpose{\tensortwo{F}}\gradient{\vectr{X}}{\variationdir{\vectr{u}}}}
  \end{split}
\end{equation}

To find the directional derivative of the deformation gradient consider
\begin{equation}
  \begin{split}
    \directionalderiv{\vectr{X}}{\fnof{\tensor{F}}{\vectr{X}}}{\linearisationdir{\vectr{U}}}&=\evalat{\dby{}{\epsilon}\fnof{\tensor{F}}{\vectr{X}+\epsilon\linearisationdir{\vectr{U}}}}{\epsilon=0}\\    
    &=\evalat{\dby{}{\epsilon}\delby{\fnof{\vectr{x}}{\vectr{X}+\epsilon\linearisationdir{\vectr{U}}}}{\vectr{X}}}{\epsilon=0}\\
    &=\evalat{\dby{}{\epsilon}\delby{\pbrac{\vectr{x}+\tensor{F}\epsilon\linearisationdir{\vectr{U}}}}{\vectr{X}}}{\epsilon=0}\\
    &=\gradient{\vectr{X}}{\pbrac{\tensor{F}\linearisationdir{\vectr{U}}}}\\
    &=\gradient{\vectr{X}}{\linearisationdir{\vectr{u}}}
  \end{split}
\end{equation}
where $\linearisationdir{\vectr{U}}$ is a direction relative to the position in the
reference configuration, $\vectr{X}$, and $\linearisationdir{\vectr{u}}=\tensor{F}\linearisationdir{\vectr{U}}$ is a direction
relative to the position in the current configuration, $\vectr{x}$. The
directional derivative with respect to the current configuation is given by
\begin{equation}
  \directionalderiv{\vectr{x}}{\fnof{\tensor{F}}{\vectr{x}}}{\linearisationdir{\vectr{u}}}=\pbrac{\gradient{\vectr{x}}{\linearisationdir{\vectr{u}}}}\tensor{F}
\end{equation}

We thus have
\begin{equation}
  \begin{split}
    \directionalderiv{}{\variation{\fnof{\tensortwo{E}}{\vectr{u}}}{\vectr{u}}}{\linearisationdir{\vectr{u}}}&=
    \directionalderiv{}{\symop\pbrac{\transpose{\tensortwo{F}}\gradient{\vectr{X}}{\variationdir{\vectr{u}}}}}{\linearisationdir{\vectr{u}}}\\
    &=\symop\pbrac{\transpose{\tensortwo{F}}\directionalderiv{}{\gradient{\vectr{X}}{\variationdir{\vectr{u}}}}{\linearisationdir{\vectr{u}}}+
      \transpose{\pbrac{\directionalderiv{}{\tensortwo{F}}{\linearisationdir{\vectr{u}}}}}\gradient{\vectr{X}}{\variationdir{\vectr{u}}}}\\
    &=\symop\pbrac{0+\transpose{\pbrac{\gradient{\vectr{X}}{\linearisationdir{\vectr{u}}}}}\gradient{\vectr{X}}{\variationdir{\vectr{u}}}}\\
    &=\symop\pbrac{\transpose{\pbrac{\gradient{\vectr{X}}{\linearisationdir{\vectr{u}}}}}\gradient{\vectr{X}}{\variationdir{\vectr{u}}}}
  \end{split}
\end{equation}
as $\variationdir{\vectr{u}}$ is independent of $\vectr{u}$ and is this unaffected by
the directional derivative (CHECK).

We also have
\begin{equation}
  \directionalderiv{}{\fnof{\tensortwo{E}}{\variationdir{\vectr{u}}}}{\linearisationdir{\vectr{u}}}=
  \symop\pbrac{\transpose{\pbrac{\gradient{\vectr{X}}{\linearisationdir{\vectr{u}}}}}\tensortwo{F}}
\end{equation}
(DERIVE).

Now
\begin{equation}
  \directionalderiv{}{\fnof{\tensor{S}}{\fnof{\tensor{E}}{\vectr{u}}}}{\linearisationdir{\vectr{u}}}=
  \doubledotprod{\delby{\fnof{\tensor{S}}{\fnof{\tensor{E}}{\vectr{u}}}}{\fnof{\tensor{E}}{\vectr{u}}}}{
    \directionalderiv{}{\variation{\fnof{\tensor{E}}{\vectr{u}}}{\vectr{u}}}{\linearisationdir{\vectr{u}}}}
\end{equation}

The derivative of the stress tensor with respect to the strain tensor is the
fourth order \emph{elasticity tensor} (sometimes called the \emph{stiffness
  tensor}) \ie
\begin{equation}
  \tensorfour{C}=\delby{\fnof{\tensortwo{S}}{\tensortwo{E}}}{\tensortwo{E}}
\end{equation}
or in component form
\begin{equation}
  C^{ABCD}=\delby{S^{AB}}{E_{CD}}
\end{equation}

Thus we have
\begin{equation}
  \begin{split}
    \doubledotprod{\fnof{\tensor{S}}{\fnof{\tensor{E}}{\vectr{u}}}}{\directionalderiv{}{\variation{\fnof{\tensor{E}}{\vectr{u}}}{\vectr{u}}}{\linearisationdir{\vectr{u}}}}&=
    \doubledotprod{\fnof{\tensor{S}}{\fnof{\tensor{E}}{\vectr{u}}}}{\symop\pbrac{\transpose{\pbrac{\gradient{\vectr{X}}{\linearisationdir{\vectr{u}}}}}\gradient{\vectr{X}}{\variationdir{\vectr{u}}}}}\\
    &=\doubledotprod{\fnof{\tensor{S}}{\fnof{\tensor{E}}{\vectr{u}}}}{\transpose{\pbrac{\gradient{\vectr{X}}{\linearisationdir{\vectr{u}}}}}\gradient{\vectr{X}}{\variationdir{\vectr{u}}}}\\
    &=\doubledotprod{\gradient{\vectr{X}}{\variationdir{\vectr{u}}}}{\gradient{\vectr{X}}{\linearisationdir{\vectr{u}}}\fnof{\tensor{S}}{\fnof{\tensor{E}}{\vectr{u}}}}
  \end{split}
\end{equation}
as
$\doubledotprod{\tensor{A}}{\transpose{\tensor{B}}\tensor{C}}=\doubledotprod{\tensor{C}}{\tensor{B}\tensor{A}}$, and
\begin{equation}
  \begin{split}
    \doubledotprod{\directionalderiv{}{\fnof{\tensor{S}}{\fnof{\tensor{E}}{\vectr{u}}}}{\linearisationdir{\vectr{u}}}}{\variation{\fnof{\tensortwo{E}}{\vectr{u}}}{\vectr{u}}}
    &=\doubledotprod{\doubledotprod{\delby{\fnof{\tensor{S}}{\fnof{\tensor{E}}{\vectr{u}}}}{\fnof{\tensor{E}}{\vectr{u}}}}{\directionalderiv{}{\variation{\fnof{\tensor{E}}{\vectr{u}}}{\vectr{u}}}{\linearisationdir{\vectr{u}}}}}{\variation{\fnof{\tensor{E}}{\vectr{u}}}{\vectr{u}}}\\
    &=\doubledotprod{\doubledotprod{\tensorfour{C}}{\symop\pbrac{\transpose{\tensortwo{F}}\gradient{\vectr{X}}{\linearisationdir{\vectr{u}}}}}}{\symop\pbrac{\transpose{\tensortwo{F}}\gradient{\vectr{X}}{\variationdir{\vectr{u}}}}}\\
    &=\doubledotprod{\transpose{\tensortwo{F}}\gradient{\vectr{X}}{\variationdir{\vectr{u}}}}{\doubledotprod{\tensorfour{C}}{\transpose{\tensortwo{F}}\gradient{\vectr{X}}{\linearisationdir{\vectr{u}}}}}
  \end{split}
\end{equation}
as
$\doubledotprod{\tensor{A}}{\tensor{B}}=\doubledotprod{\tensor{B}}{\tensor{A}}$.

Putting this together gives
\begin{multline}
  \directionalderiv{}{\variation{\fnof{\Pi_{int}}{\vectr{u}}}{\vectr{u}}}{\linearisationdir{\vectr{u}}}=\\
  \gint{\embedmanifold{B}_{0}}{}{\pbrac{\doubledotprod{\gradient{\vectr{X}}{\variationdir{\vectr{u}}}}{\gradient{\vectr{X}}{\linearisationdir{\vectr{u}}}\fnof{\tensor{S}}{\fnof{\tensor{E}}{\vectr{u}}}}+\doubledotprod{\transpose{\tensortwo{F}}\gradient{\vectr{X}}{\variationdir{\vectr{u}}}}{\doubledotprod{\tensorfour{C}}{\transpose{\tensortwo{F}}\gradient{\vectr{X}}{\linearisationdir{\vectr{u}}}}}}}{V}
\end{multline}
or, in component form,
\begin{equation}
  \begin{split}
    \directionalderiv{}{\variation{\fnof{\Pi_{int}}{\vectr{u}}}{u_{i}}}{\linearisationdir{u_{j}}}
    &=\gint{\embedmanifold{B}_{0}}{}{\pbrac{\delby{\variationdir{
            u_{i}}}{X^{B}}\delby{\linearisationdir{u_{j}}}{X^{D}}\contrakronecker{i}{j}S^{BD}+F^{i}_{A}\delby{\variationdir{
            u_{i}}}{X^{B}}C^{ABCD}F^{j}_{C}\delby{\linearisationdir{u_{j}}}{X^{D}}}}{V}\\
    &=\gint{\embedmanifold{B}_{0}}{}{\pbrac{\delby{\variationdir{u_{i}}}{X^{B}}\sqbrac{\contrakronecker{i}{j}S^{BD}+F^{i}_{A}F^{j}_{C}C^{ABCD}}\delby{\linearisationdir{u_{j}}}{X^{D}}}}{V}
  \end{split}
\end{equation}

Now, consider the directional derivative of the internal work variation in the current
configuration \ie with respect to cauchy stress and ??? strain
\begin{equation}
  \begin{split}
    \directionalderiv{}{\variation{\fnof{\Pi_{int}}{\vectr{u}}}{\vectr{u}}}{\linearisationdir{\vectr{u}}}
    &=\directionalderiv{}{\gint{\embedmanifold{B}}{}{\doubledotprod{\fnof{\tensor{\sigma}}{
            \fnof{\tensor{e}}{\vectr{u}}}}{\variation{\fnof{\tensor{\epsilon}}{\vectr{u}}}{\vectr{u}}}}{v}}{\linearisationdir{\vectr{u}}}\\
    &=\gint{\embedmanifold{B}}{}{\left(\doubledotprod{\fnof{\tensor{\sigma}}{
          \fnof{\tensor{e}}{\vectr{u}}}}{\directionalderiv{}{\variation{\fnof{\tensor{\epsilon}}{\vectr{u}}}{\vectr{u}}}{\linearisationdir{\vectr{u}}}}\right.\\
     & \qquad\qquad + \left.\doubledotprod{\directionalderiv{}{\fnof{\tensor{\sigma}}{
            \fnof{\tensor{e}}{\vectr{u}}}}{\linearisationdir{\vectr{u}}}}{\variation{\fnof{\tensor{\epsilon}}{\vectr{u}}}{\vectr{u}}}\right)}{v}
  \end{split}
\end{equation}

We can now use a push forward operation to find the directional derivative of the
Cauchy stress in terms of the results from the directional derivative of the second
Piola-Kirchoff stress \ie
\begin{equation}
  \begin{split}
    \fnof{\tensor{\sigma}}{\fnof{\tensor{e}}{\vectr{u}}}&=
    \inverse{J}\pushforward{\chi}{\pbrac{\fnof{\tensor{S}}{\fnof{\tensor{E}}{\vectr{u}}}}}\\
    \directionalderiv{}{\fnof{\tensor{\sigma}}{\fnof{\tensor{e}}{\vectr{u}}}}{\linearisationdir{\vectr{u}}}&=
    \inverse{J}\pushforward{\chi}{\pbrac{\directionalderiv{}{\fnof{\tensor{S}}{\fnof{\tensor{E}}{\vectr{u}}}}{\linearisationdir{\vectr{u}}}}}\\
    \variation{\fnof{\tensor{\epsilon}}{\vectr{u}}}{\vectr{u}}&=\pushforward{\chi}{\pbrac{\variation{\fnof{\tensor{E}}{\vectr{u}}}{\vectr{u}}}} \\
    \directionalderiv{}{\variation{\fnof{\tensor{\epsilon}}{\vectr{u}}}{\vectr{u}}}{\linearisationdir{\vectr{u}}}&=
    \pushforward{\chi}{\pbrac{\directionalderiv{}{\variation{\fnof{\tensor{E}}{\vectr{u}}}{\vectr{u}}}{\linearisationdir{\vectr{u}}}}}
  \end{split}
\end{equation}

Now,
\begin{equation}
  \begin{split}
    \directionalderiv{}{\fnof{\tensor{\sigma}}{\fnof{\tensor{e}}{\vectr{u}}}}{\linearisationdir{\vectr{u}}}&=
    \inverse{J}\pushforward{\chi}{\pbrac{\directionalderiv{}{\fnof{\tensor{S}}{\fnof{\tensor{E}}{\vectr{u}}}}{\linearisationdir{\vectr{u}}}}}\\
    &=\inverse{J}\pushforward{\chi}{\pbrac{\doubledotprod{\tensorfour{C}}{\transpose{\tensortwo{F}}\gradient{\vectr{X}}{\linearisationdir{\vectr{u}}}}}}\\
    &=\inverse{J}\tensor{F}\pbrac{\doubledotprod{\tensorfour{C}}{\transpose{\tensortwo{F}}\gradient{\vectr{X}}{\linearisationdir{\vectr{u}}}}}\transpose{\tensor{F}}
  \end{split}
\end{equation}

The push forward of the fourth order elasticity tensor is
\begin{equation}
  \tensorfour{c}=\inverse{J}\pushforward{\chi}{\pbrac{\tensorfour{C}}}
\end{equation}

In component form this is
\begin{equation}
  c^{ijkl}=\inverse{J}F^{i}_{A}F^{j}_{B}F^{k}_{C}F^{l}_{D}C^{ABCD}
\end{equation}

Putting this together gives
\begin{equation}
  \directionalderiv{}{\variation{\fnof{\Pi_{int}}{\vectr{u}}}{\vectr{u}}}{\linearisationdir{\vectr{u}}}=
  \gint{\embedmanifold{B}}{}{\pbrac{\doubledotprod{\gradient{\vectr{x}}{\variationdir{\vectr{u}}}}{\gradient{\vectr{x}}{\linearisationdir{\vectr{u}}}\fnof{\tensor{\sigma}}{\fnof{\tensor{e}}{\vectr{u}}}}+
            \doubledotprod{\gradient{\vectr{x}}{\variationdir{\vectr{u}}}}{\doubledotprod{\tensorfour{c}}{\gradient{\vectr{x}}{\linearisationdir{\vectr{u}}}}}}}{v}
\end{equation}
or, in component form,
\begin{equation}
  \begin{split}
    \directionalderiv{}{\variation{\fnof{\Pi_{int}}{\vectr{u}}}{u_{i}}}{\linearisationdir{u_{j}}}
    &=\gint{\embedmanifold{B}}{}{\pbrac{\delby{\variationdir{u_{i}}}{x^{k}}\delby{\linearisationdir{u_{j}}}{x^{l}}\contrakronecker{i}{j}\sigma^{kl}+
        \delby{\variationdir{u_{i}}}{x^{k}}c^{ikjl}\delby{\linearisationdir{u_{j}}}{x^{l}}}}{v}\\
    &=\gint{\embedmanifold{B}}{}{\pbrac{\delby{\variationdir{u_{i}}}{x^{k}}\sqbrac{\contrakronecker{i}{j}\sigma^{kl}+c^{ikjl}}\delby{\linearisationdir{u_{j}}}{x^{l}}}}{v}
  \end{split}
\end{equation}

\subsubsection{External work}

The external work potential energy is given by
\begin{equation}
  \begin{split}
    \fnof{\Pi_{ext}}{\vectr{u},\vect{b},\bar{\vect{t}},P_{ext}}&=\fnof{\Pi_{body}}{\vectr{u},\vectr{b}}+\fnof{\Pi_{surf}}{\vectr{u},\bar{\vectr{t}}}+\fnof{\Pi_{press}}{\vectr{u},P_{ext}}\\
    &=a
  \end{split}
\end{equation}

Considering the external surface work it is often the case that this is due to
an external pressure. Thus the traction is given by
\begin{equation}
  \bar{\vectr{t}}=p_{ext}\hat{\vectr{n}}
\end{equation}
where $p_{ext}$ is the applied external pressure and $\hat{\vectr{n}}$ is the
unit normal vector to the surface element $\exteriorderiv{\covectr{a}}$. Now
if the surface is parameterised by the coordinates $\xi$ and $\eta$ then
\begin{equation}
  \exteriorderiv{\covectr{a}}=
  \norm{\crossprod{\delby{\vectr{x}}{\xi}}{\delby{\vectr{x}}{\eta}}}
  \wedgeprod{\exteriorderiv{\xi}}{\exteriorderiv{\eta}}
\end{equation}
and the unit normal is given by
\begin{equation}
  \hat{\vectr{n}}=\dfrac{\crossprod{\delby{\vectr{x}}{\xi}}{\delby{\vectr{x}}{\eta}}}{
    \norm{\crossprod{\delby{\vectr{x}}{\xi}}{\delby{\vectr{x}}{\eta}}}}
\end{equation}
where $\vectr{x}$ are the coordinates in the current configuration \ie $\vectr{x}=\vectr{X}+\vectr{u}$.

The external surface work is thus given by
\begin{equation}
  \begin{split}
    \fnof{W_{surf}}{\vectr{u},p_{ext},\delta\vectr{u}}&=
    \gint{\boundary{\embedmanifold{B}}_{t}}{}{\dotprod{p_{ext}\fnof{\hat{\vectr{n}}}{\vectr{u}}}{
        \delta\vectr{u}}}{\covectr{a}}\\
    &=\giint{\xi}{}{\eta}{}{\dotprod{p_{ext}\dfrac{\crossprod{\delby{\vectr{x}}{\xi}}{\delby{\vectr{x}}{\eta}}}{\norm{\crossprod{\delby{\vectr{x}}{\xi}}{\delby{\vectr{x}}{\eta}}}}}{\delta\vectr{u}}\norm{\crossprod{\delby{\vectr{x}}{\xi}}{\delby{\vectr{x}}{\eta}}}}{\xi}{\eta} \\
    &=\giint{\xi}{}{\eta}{}{\dotprod{p_{ext}\pbrac{\crossprod{\delby{\vectr{x}}{\xi}}{\delby{\vectr{x}}{\eta}}}}{\delta\vectr{u}}}{\xi}{\eta}\\
    &=\giint{\xi}{}{\eta}{}{\dotprod{p_{ext}\fnof{\vectr{n}}{\vectr{u}}}{\delta\vectr{u}}}{\xi}{\eta}
  \end{split}
\end{equation}
where $\vectr{n}$ is the normal direction (non-normalised) \ie
\begin{equation}
  \vectr{n}=\crossprod{\delby{\vectr{x}}{\xi}}{\delby{\vectr{x}}{\eta}}
\end{equation}
 
The body force is due to the effect of gravity on the body \ie
\begin{equation}
  \vectr{b}=\rho\vectr{g}
\end{equation}
where $\rho$ is the density of the current configuration and $\vectr{g}$ is
the acceleration vector due to gravity.

The external body work is thus given by
\begin{equation}
  \begin{split}
    \fnof{W_{body}}{\vectr{u},\delta\vectr{u}}
    &=\gint{\embedmanifold{B}}{}{\dotprod{\fnof{\vectr{b}}{\vectr{u}}}{\delta\vectr{u}}}{v}\\
    &=\gint{\embedmanifold{B}}{}{\dotprod{\rho\vectr{g}}{\delta\vectr{u}}}{v}\\
    &=\gint{\embedmanifold{B}_{0}}{}{\dotprod{J\dfrac{\rho_{0}}{J}\vectr{g}}{\delta\vectr{u}}}{V}\\
    &=\gint{\embedmanifold{B}_{0}}{}{\dotprod{\rho_{0}\vectr{g}}{\delta\vectr{u}}}{V}
  \end{split}
\end{equation}
where $\rho_{0}$ is the density in the reference configuration.

\begin{equation}
  c^{ijkl}=\inverse{J}F^{i}_{A}F^{j}_{B}F^{k}_{C}F^{l}_{D}C^{ABCD}
\end{equation}


If we now just consider the external traction part. For the case where the
external traction is given by a fixed external pressure we have
\begin{equation}
  \gint{\boundary{\embedmanifold{B}}_{t}}{}{\dotprod{\bar{\vectr{t}}}{\delta\vectr{u}}}{\covectr{a}}=\gint{\boundary{\embedmanifold{B}}_{t}}{}{\dotprod{p_{ext}\hat{\vectr{n}}}{\delta\vectr{u}}}{\covectr{a}}
\end{equation}
where $\hat{\vectr{n}}$ is the unit normal to the surface element
$\exteriorderiv{\covectr{a}}$. Now if the surface is parameterised by the
coordinates $\xi$ and $\eta$ then
\begin{equation}
  \exteriorderiv{\covectr{a}}=\norm{\crossprod{\delby{\vectr{x}}{\xi}}{\delby{\vectr{x}}{\eta}}}\wedgeprod{\exteriorderiv{\xi}}{\exteriorderiv{\eta}}
\end{equation}
and the unit normal is given by
\begin{equation}
  \hat{\vectr{n}}=\dfrac{\crossprod{\delby{\vectr{x}}{\xi}}{\delby{\vectr{x}}{\eta}}}{\norm{\crossprod{\delby{\vectr{x}}{\xi}}{\delby{\vectr{x}}{\eta}}}}
\end{equation}

We thus have

The directional derivative of $\fnof{\delta W_{ext}}{\vectr{u},p_{ext},\delta\vectr{u}}$ is given by
\begin{equation}
  \directionalderiv{}{\fnof{\delta
      W_{ext}}{\vectr{u},p_{ext},\delta\vectr{u}}}{\Delta\vectr{u},\Delta p_{ext}}=
  \giint{\xi}{}{\eta}{}{\pbrac{\dotprod{p_{ext}\vectr{n}}{\delta\vectr{u}}+\dotprod{\Delta
  p_{ext}\vectr{n}}{\delta\vectr{u}}+\dotprod{p_{ext}\Delta\vectr{n}}{\delta\vectr{u}}}}{\xi}{\eta}
\end{equation}

For problems where $p_{ext}$ is not part of the solution procedure then
$\Delta p_{ext}=0$ and thus
\begin{equation}
  \directionalderiv{}{\fnof{\delta
      W_{ext}}{\vectr{u},p_{ext},\delta\vectr{u}}}{\Delta\vectr{u}}=
  \giint{\xi}{}{\eta}{}{\pbrac{\dotprod{p_{ext}\vectr{n}}{\delta\vectr{u}}+
      \dotprod{p_{ext}\Delta\vectr{n}}{\delta\vectr{u}}}}{\xi}{\eta}
\end{equation}

The directional derivative of the normal vector is given by
\begin{equation}
  \directionalderiv{}{\vectr{n}}{\Delta\vectr{u}}=\crossprod{\delby{\Delta\vectr{u}}{\xi}}{\delby{\vectr{x}}{\eta}}-
  \crossprod{\delby{\Delta\vectr{u}}{\eta}}{\delby{\vectr{x}}{\xi}}
\end{equation}
and thus we have
\begin{equation}
  \directionalderiv{}{\fnof{\delta
      W_{ext}}{\vectr{u},p_{ext},\delta\vectr{u}}}{\Delta\vectr{u}}=
  \giint{\xi}{}{\eta}{}{\pbrac{\dotprod{p_{ext}\vectr{n}}{\delta\vectr{u}}+
      \dotprod{p_{ext}\pbrac{\crossprod{\delby{\Delta\vectr{u}}{\xi}}{\delby{\vectr{x}}{\eta}}-
  \crossprod{\delby{\Delta\vectr{u}}{\eta}}{\delby{\vectr{x}}{\xi}}}}{\delta\vectr{u}}}}{\xi}{\eta}
\end{equation}

The transformation rule for a triple product is
\begin{equation}
  \dotprod{\vectr{a}}{\pbrac{\crossprod{\vectr{b}}{\vectr{c}}}}=
  \dotprod{\vectr{b}}{\pbrac{\crossprod{\vectr{c}}{\vectr{a}}}}=
  \dotprod{\vectr{c}}{\pbrac{\crossprod{\vectr{a}}{\vectr{b}}}}
\end{equation}
and so we have
\begin{equation}
  \begin{split}
    \directionalderiv{}{\fnof{\delta
        W_{ext}}{\vectr{u},p_{ext},\delta\vectr{u}}}{\Delta\vectr{u}}&=\giint{\xi}{}{\eta}{}{\pbrac{\dotprod{p_{ext}\vectr{n}}{\delta\vectr{u}}+
        \dotprod{p_{ext}\pbrac{\crossprod{\delby{\Delta\vectr{u}}{\xi}}{\delby{\vectr{x}}{\eta}}-
            \crossprod{\delby{\Delta\vectr{u}}{\eta}}{\delby{\vectr{x}}{\xi}}}}{\delta\vectr{u}}}}{\xi}{\eta}\\
    &=p_{ext}\giint{\xi}{}{\eta}{}{\pbrac{\dotprod{\vectr{n}}{\delta\vectr{u}}+
        \dotprod{\pbrac{\crossprod{\delby{\vectr{x}}{\eta}}{\delta\vectr{u}}}}{\delby{\Delta\vectr{u}}{\xi}}-
        \dotprod{\pbrac{\crossprod{\delby{\vectr{x}}{\xi}}{\delta\vectr{u}}}}{\delby{\Delta\vectr{u}}{\eta}}}}{\xi}{\eta}
  \end{split}
  \label{eqn:nonsympressuredirectderiv}
\end{equation}

Now \eqnref{eqn:nonsympressuredirectderiv} is, in general, non-symmetric in terms of
$\Delta \vectr{u}$ and $\delta\vectr{u}$. This will equate to a
non-conservative force.

\subsubsection{Hellinger-Reissner}

The function is extended by the addition of a pressure constraint
\begin{equation}
  \fnof{\Pi_{HR}}{\vectr{u},p}=\fnof{\bar{\Pi}_{PE}}{\vectr{u}}+\fnof{\Pi_{P}}{\vectr{u},p}
\end{equation}
where
\begin{equation}
  \fnof{\bar{\Pi}_{PE}}{\vectr{u}}=\fnof{\bar{\Pi}_{int}}{\vectr{u}}-\fnof{\Pi_{ext}}{\vectr{u}}
\end{equation}
and
\begin{equation}
  \fnof{\bar{\Pi}_{int}}{\vectr{u}}=\gint{\embedmanifold{B}_{0}}{}{\fnof{\bar{W}}{\vectr{u}}}{V}
\end{equation}
and
\begin{equation}
  \fnof{\Pi_{P}}{\vectr{u},p}=\gint{\embedmanifold{B}_{0}}{}{-p\pbrac{\fnof{J}{\vectr{u}}-1}}{V}
\end{equation}

Note that $-p$ is used in order to establish the sign convention that a
pressure is compressive.

The variations with respect to $\variationdir{\vectr{u}}$ are
\begin{equation}
  \variation{\fnof{\bar{\Pi}_{int}}{\vectr{u}}}{\vectr{u}}=\gint{\embedmanifold{B}_{0}}{}{\variation{\fnof{\bar{W}}{\vectr{u}}}{\vectr{u}}}{V}
\end{equation}
and
\begin{equation}
  \variation{\fnof{\Pi_{P}}{\vectr{u},p}}{\vectr{u}}=\gint{\embedmanifold{B}_{0}}{}{-p\variation{\fnof{J}{\vect{u}}}{\vectr{u}}}{V}
\end{equation}

Now
\begin{equation}
  \begin{split}
    \variation{\fnof{J}{\vectr{u}}}{\vectr{u}}&=\variation{\determinant{\fnof{\tensor{F}}{\vectr{u}}}}{\vectr{u}}
    \\
    &=\directionalderiv{\vectr{u}}{\determinant{\fnof{\tensor{F}}{\vectr{u}}}}{\variationdir{\vectr{u}}}\\
    &=\directionalderiv{\tensor{F}}{\determinant{\fnof{\tensor{F}}{\vectr{u}}}}{\directionalderiv{}{\fnof{\tensor{F}}{\vectr{u}}}{\variationdir{\vectr{u}}}}
    \\
    &=\directionalderiv{\tensor{F}}{\determinant{\fnof{\tensor{F}}{\vectr{u}}}}{\gradient{\vectr{X}}{\variationdir{\vectr{u}}}}
    \\
    &=
    \determinant{\fnof{\tensor{F}}{\vectr{u}}}\doubledotprod{\invtranspose{\tensor{F}}}{\gradient{\vectr{X}}{\variationdir{\vectr{u}}}}\\
    &=J\trop\pbrac{\gradient{\vectr{X}}{\variationdir{\vectr{u}}}\inverse{\tensor{F}}}
    \\
    &= J\trop\pbrac{\gradient{\vectr{x}}{\variationdir{\vectr{u}}}}\\
    &=J\divergence{\vectr{x}}{\variationdir{\vectr{u}}}
  \end{split}
\end{equation}

And so
\begin{equation}
  \begin{split}
    \variation{\fnof{\Pi_{P}}{\vectr{u},p}}{\vectr{u}}&=\gint{\embedmanifold{B}_{0}}{}{-p\fnof{J}{\vectr{u}}\divergence{\vectr{x}}{\variationdir{\vectr{u}}}}{V}\\
    &=\gint{\embedmanifold{B}}{}{-p\divergence{\vectr{x}}{\variationdir{\vectr{u}}}}{v}\\
  \end{split}
\end{equation}

Now
\begin{equation}
  \divergence{\vectr{x}}{\covectr{a}}=\doubledotprod{\sharptensor{\tensor{g}}}{\gradient{\vectr{x}}{\covectr{a}}}=\doubledotprod{\inverse{\tensor{g}}}{\gradient{\vectr{x}}{\covectr{a}}}
\end{equation}
for a covector $\covectr{a}$ with
$\sharptensor{\tensor{g}}=\inverse{\tensor{g}}$ and so we obtain
\begin{equation}
  \begin{split}
    \variation{\fnof{\Pi_{P}}{\vectr{u},p}}{\vectr{u}}&=\gint{\embedmanifold{B}}{}{\doubledotprod{-p\inverse{\tensor{g}}}{\gradient{\vectr{x}}{\variationdir{\vectr{u}}}}}{v}\\
    &=\gint{\embedmanifold{B}}{}{\doubledotprod{\fnof{\tensor{\sigma}_{sph}}{p}}{\gradient{\vectr{x}}{\variationdir{\vectr{u}}}}}{v} \\
    &=\gint{\embedmanifold{B}}{}{\doubledotprod{\fnof{\tensor{\sigma}_{sph}}{p}}{\variation{\fnof{\tensor{\epsilon}}{\vectr{u}}}{\vectr{u}}}}{v}
  \end{split}
\end{equation}
as the spherical stress tensor due to pressure is symmetric. Note that we have
\begin{equation}
  \fnof{\tensor{\sigma}_{sph}}{p}=-p\inverse{\tensor{g}}
\end{equation}

Note that this variation is an expression in the current configuration. We can
obtain an equivalent expression in the reference configuration via a pull back
\ie
\begin{equation}
  \begin{split}
    \variation{\fnof{\Pi_{P}}{\vectr{u},p}}{\vectr{u}}&=\gint{\embedmanifold{B}_{0}}{}{\doubledotprod{-p\fnof{J}{\vectr{u}}\fnof{\inverse{\tensor{C}}}{\vectr{u}}}{\variation{\fnof{\tensor{E}}{\vectr{u}}}{\vectr{u}}}}{V}\\
    &=\gint{\embedmanifold{B}_{0}}{}{\doubledotprod{\fnof{\tensor{S}_{sph}}{\vectr{u},p}}{\variation{\fnof{\tensor{E}}{\vectr{u}}}{\vectr{u}}}}{V}\\
  \end{split}
\end{equation}
as $\exteriorderiv{v}=J\exteriorderiv{V}$,
$\tensor{E}=\pullback{\chi}{\tensor{\epsilon}}$, $\tensor{S}_{sph}=J\pullback{\chi}{\tensor{\sigma}_{sph}}$,
$\tensor{C}=\pullback{\chi}{\tensor{g}}$ and so
$\inverse{\tensor{C}}=\pullback{\chi}{\inverse{\tensor{g}}}$. Thus
\begin{equation}
  \begin{split}
    \variation{\fnof{\bar{\Pi}_{int}}{\vectr{u}}}{\vectr{u}}+\variation{\fnof{\Pi_{P}}{\vectr{u},p}}{\vectr{u}}&=\gint{\embedmanifold{B}_{0}}{}{\variation{\fnof{\bar{W}}{\vectr{u}}}{\vectr{u}}}{V}+\gint{\embedmanifold{B}_{0}}{}{\doubledotprod{\fnof{\tensor{S}_{sph}}{\vectr{u},p}}{\variation{\fnof{\tensor{E}}{\vectr{u}}}{\vectr{u}}}}{v}\\
    &=\gint{\embedmanifold{B}_{0}}{}{\doubledotprod{\fnof{\tensor{S}_{dev}}{\vectr{u}}}{\variation{\fnof{\tensor{E}}{\vectr{u}}}{\vectr{u}}}}{V}+\gint{\embedmanifold{B}_{0}}{}{\doubledotprod{\fnof{\tensor{S}_{sph}}{\vectr{u},p}}{\variation{\fnof{\tensor{E}}{\vectr{u}}}{\vectr{u}}}}{V}\\
    &=\gint{\embedmanifold{B}_{0}}{}{\doubledotprod{\fnof{\tensor{S}}{\vectr{u},p}}{\variation{\fnof{\tensor{E}}{\vectr{u}}}{\vectr{u}}}}{V}\\
  \end{split}
\end{equation}
where
\begin{equation}
  \begin{split}
    \tensor{S}&=\tensor{S}_{dev}+\tensor{S}_{sph}\\
    &= \bar{\tensor{S}}-pJ\inverse{\tensor{C}}
  \end{split}
\end{equation}

The variations with respect to $\variationdir{p}$ are
\begin{equation}
  \variation{\fnof{\bar{\Pi}_{int}}{\vectr{u}}}{p}=0
\end{equation}
and
\begin{equation}
  \variation{\fnof{\Pi_{P}}{\vectr{u},p}}{p}=\gint{\embedmanifold{B}_{0}}{}{-\variationdir{p}\pbrac{\fnof{J}{\vectr{u}}-1}}{V}
\end{equation}

The variation statement is the same as the variation statement for the
potential energy case with the exception of the addition of a hydrostatic
pressure term. The linearisation of this term in the reference configuration is
\begin{equation}
  \directionalderiv{}{\variation{\fnof{\bar{\Pi}_{int}}{\vectr{u}}}{\vectr{u}}+\variation{\fnof{\Pi_{P}}{\vectr{u},p}}{\vectr{u}}}{\linearisationdir{\vectr{u}}}=\directionalderiv{}{\gint{\embedmanifold{B}_{0}}{}{\doubledotprod{\fnof{\tensor{S}}{
            \fnof{\tensor{E}}{\vectr{u}}}}{\variation{\fnof{\tensor{E}}{\vectr{u}}}{\vectr{u}}}}{V}}{\linearisationdir{\vectr{u}}}\\
\end{equation}

Following the same procedure as for the potential energy case we obtain
\begin{multline}
  \directionalderiv{}{\variation{\fnof{\bar{\Pi}_{int}}{\vectr{u}}}{\vectr{u}}+\variation{\fnof{\Pi_{P}}{\vectr{u},p}}{\vectr{u}}}{\linearisationdir{\vectr{u}}}=\\
  \gint{\embedmanifold{B}_{0}}{}{\pbrac{\doubledotprod{\gradient{\vectr{X}}{\variationdir{\vectr{u}}}}{\gradient{\vectr{X}}{\linearisationdir{\vectr{u}}}\fnof{\tensor{S}}{\fnof{\tensor{E}}{\vectr{u}}}}+\doubledotprod{\transpose{\tensortwo{F}}\gradient{\vectr{X}}{\variationdir{\vectr{u}}}}{\doubledotprod{\tensorfour{C}}{\transpose{\tensortwo{F}}\gradient{\vectr{X}}{\linearisationdir{\vectr{u}}}}}}}{V}
\end{multline}
where
\begin{equation}
  \tensorfour{C}=\bar{\tensorfour{C}}+\tensorfour{C}_{p}
\end{equation}
and
\begin{equation}
  \bar{\tensorfour{C}}=2\delby{\bar{\tensortwo{S}}}{\tensortwo{C}}=\delby{\bar{\tensortwo{S}}}{\tensortwo{E}}
\end{equation}
and
\begin{equation}
  \tensorfour{C}_{p}=-pJ\pbrac{\tensorprod{\inverse{\tensor{C}}}{\inverse{\tensor{C}}}-2\tensorfour{I}}
\end{equation}

In the current configuration we have
\begin{equation}
  \begin{split}
    \directionalderiv{}{\variation{\fnof{\bar{\Pi}_{int}}{\vectr{u}}}{\vectr{u}}+\variation{\fnof{\Pi_{P}}{\vectr{u},p}}{\vectr{u}}}{\linearisationdir{\vectr{u}}}&= \\
  &\gint{\embedmanifold{B}}{}{\pbrac{\doubledotprod{\gradient{\vectr{x}}{\variationdir{\vectr{u}}}}{\gradient{\vectr{x}}{\linearisationdir{\vectr{u}}}\fnof{\tensor{\sigma}}{\fnof{\tensor{e}}{\vectr{u}}}}+
            \doubledotprod{\gradient{\vectr{x}}{\variationdir{\vectr{u}}}}{\doubledotprod{\tensorfour{c}}{\gradient{\vectr{x}}{\linearisationdir{\vectr{u}}}}}}}{v}
  \end{split}
\end{equation}
where
\begin{equation}
  \tensorfour{c}=\bar{\tensorfour{c}}+\tensorfour{c}_{p}
\end{equation}
and
\begin{equation}
  \bar{\tensorfour{c}}=\inverse{J}\pushforward{\chi}{\pbrac{\bar{\tensorfour{C}}}}
\end{equation}
and
\begin{equation}
  \tensorfour{c}_{p}=\inverse{J}\pushforward{\chi}{\pbrac{\tensorfour{C}_{p}}}=-p\pbrac{\tensorprod{\tensortwo{i}}{\tensortwo{i}}-2\tensorfour{i}}
\end{equation}

The linearisation of the other terms are
\begin{equation}
  \directionalderiv{}{\variation{\fnof{\bar{\Pi}_{int}}{\vectr{u}}}{\vectr{u}}+\variation{\fnof{\Pi_{P}}{\vectr{u},p}}{\vectr{u}}}{\linearisationdir{p}}=\gint{\embedmanifold{B}_{0}}{}{-\linearisationdir{p}J\inverse{\tensor{C}}}{V}
\end{equation}
or
\begin{equation}
  \directionalderiv{}{\variation{\fnof{\bar{\Pi}_{int}}{\vectr{u}}}{\vectr{u}}+\variation{\fnof{\Pi_{P}}{\vectr{u},p}}{\vectr{u}}}{\linearisationdir{p}}=\gint{\embedmanifold{B}}{}{-\linearisationdir{p}\divergence{\vectr{x}}{\variationdir{\vectr{u}}}}{v}
\end{equation}
WHERE IS THE DELTA U TERM IN THE ABOVE???

Now, the linearisation of the pressure variation with respect to pressure is
given by
\begin{equation}
  \begin{split}
    \directionalderiv{}{\variation{\fnof{\Pi_{P}}{\vectr{u},p}}{p}}{\linearisationdir{\vectr{u}}}&=\directionalderiv{}{\gint{\embedmanifold{B}_{0}}{}{-\variationdir{p}\pbrac{\fnof{J}{\vectr{u}}-1}}{V}}{\linearisationdir{\vectr{u}}}\\
    &=\gint{\embedmanifold{B}_{0}}{}{-\variationdir{p}\fnof{J}{\vectr{u}}\divergence{\vectr{x}}{\linearisationdir{\vectr{u}}}}{V} \\
    &=\gint{\embedmanifold{B}}{}{-\variationdir{p}\divergence{\vectr{x}}{\linearisationdir{\vectr{u}}}}{v}
  \end{split}
\end{equation}
and
\begin{equation}
  \begin{split}
    \directionalderiv{}{\variation{\fnof{\Pi_{P}}{\vectr{u},p}}{p}}{\linearisationdir{p}}&=\directionalderiv{}{\gint{\embedmanifold{B}_{0}}{}{-\variationdir{p}\pbrac{\fnof{J}{\vectr{u}}-1}}{V}}{\linearisationdir{p}}\\
    &=0
  \end{split}
\end{equation}

\subsubsection{Hu-Washizu}


\subsection{Finite Element Formulation}

\subsubsection{Residual}

The residual statement is given by
\begin{equation}
  R=\delta\fnof{\Pi}{\vectr{u},p,\delta\vectr{u},\delta p}=0
\end{equation}

Firstly, consider the displacement, $\vectr{u}$, between the
reference/material coordinates, $\vectr{x}$, and the current/spatial
coordinates, $\vectr{z}$. The displacement is defined by
\begin{equation}
  \vectr{u}=\vectr{z}-\vectr{x}
\end{equation}
and thus
\begin{equation}
  \begin{split}
    \delta\vectr{u} &=\delta\pbrac{\vectr{z} -\vectr{X}} \\
    &=\delta\vectr{z}-\delta\vectr{X} \\
    &=\delta\vectr{z}
  \end{split}
\end{equation}
as there is no variation in the original reference configuration $\vectr{X}$. 


In component form, we have
\begin{equation}
  \gint{\embedmanifold{B}}{}{\sigma^{ij}\covarderiv{\delta u_{j}}{i}}{v}+
  \gint{\embedmanifold{B}}{}{\sigma^{ij}_{p}\covarderiv{\delta u_{j}}{i}}{v}=
  \gint{\embedmanifold{B}}{}{b^{j}\delta u_{j}}{v}+
  \gint{\boundary{\embedmanifold{B}}}{}{t^{j}\delta u_{j}}{a}
\end{equation}

If we now substitute $\delta\vectr{u}=\delta\vectr{z}$ and convert the left
hand side of the virtual work statement from an integral with respect to
spatial coordinates to an integral with respect to $\vectr{\xi}$ coordinates we obtain

\begin{equation}
  \begin{split}
    \gint{\embedmanifold{B}}{}{\sigma^{ij}\pbrac{\delby{\delta
          u_{j}}{x^{i}}-\christoffel{k}{j}{i}\delta u_{k}}}{v}
    &= \gint{\embedmanifold{B}}{}{\sigma^{ij}\pbrac{\delby{\delta
          z_{j}}{x^{i}}-\christoffel{k}{j}{i}\delta z_{k}}}{v} \\
    &= \gint{\vectr{0}}{\vectr{1}}{\fnof{\sigma^{ij}}{\vectr{\xi}}\pbrac{\delby{\xi_{l}}{x^{i}}\delby{\delta
          \fnof{z_{j}}{\vectr{\xi}}}{\xi^{l}}-\christoffel{k}{j}{i}\delta\fnof{z_{k}}{\vect{\xi}}}\fnof{J_{\embedmanifold{B}}}{\vectr{\xi}}}{\vectr{\xi}}
  \end{split}
\end{equation}

Note that in rectangular cartesian coordinates $\christoffel{k}{j}{i}=0$ 
$\forall i,j,k$. In addition it is not necessary to transform either the
Cauchy stress tensor or gradient of the virtual displacements so that the
components are with respect to $\vectr{\xi}$ coordinates. What is important is that
the stress and displacement are with respect to the same coordinate
system. Because the gradient of $\delta \vectr{z}$ is with respect to $\vectr{x}$ coordinates
then $\tensor{\sigma}$ needs to be with respect to $\vectr{x}$ coordinates. As there
is no coordinate transformations the Christoffel symbols are all zero and can
be dropped.

The first integral on the left hand side of the virtual work statement is
\begin{equation}
  \begin{split}
    \gint{\embedmanifold{B}}{}{\sigma^{ij}\covarderiv{\delta u_{j}}{i}}{v}
    &= \gint{\embedmanifold{B}}{}{\sigma^{ij}\pbrac{\partialderiv{\delta
          u_{j}}{i}-\christoffel{k}{j}{i}\delta u_{k}}}{v} \\
    &= \gint{\embedmanifold{B}}{}{\sigma^{ij}\pbrac{\delby{\delta
          u_{j}}{x^{i}}-\christoffel{k}{j}{i}\delta u_{k}}}{v}
  \end{split}
\end{equation}
The second integral on the left hand side is thus
\begin{equation}
  \gint{\embedmanifold{B}}{}{\sigma^{ij}_{p}\pbrac{\delby{\delta
        u_{j}}{x^{i}}-\christoffel{k}{j}{i}\delta u_{k}}}{v}=
  \gint{\vectr{0}}{\vectr{1}}{\fnof{\sigma^{ij}_{p}}{\vectr{\xi}}
    \delby{\xi_{l}}{x^{i}}\delby{\delta\fnof{z_{j}}{\vectr{\xi}}}{\xi^{l}}
    \fnof{J_{\embedmanifold{B}}}{\vectr{\xi}}}{\vectr{\xi}}
\end{equation}

The right hand side of the virtual work statement is
\begin{equation}
  \begin{split}
    \gint{\embedmanifold{B}}{}{b^{j}\delta u_{j}}{v}+
    \gint{\boundary{\embedmanifold{B}_{t}}}{}{\bar{t}^{j}\delta u_{j}}{a}
    &= \gint{\embedmanifold{B}}{}{b^{j}\delta z_{j}}{v}+
    \gint{\boundary{\embedmanifold{B}_{t}}}{}{t^{j}\delta z_{j}}{a} +
    \gint{\boundary{\embedmanifold{B}_{P}}}{}{Pn^{j}\delta z_{j}}{a} \\
    &= \gint{\vectr{0}}{\vectr{1}}{\fnof{b^{j}}{\vectr{\xi}}\delta
      \fnof{z_{j}}{\vectr{\xi}}\fnof{J_{\embedmanifold{B}}}{\vectr{\xi}}}{\vectr{\xi}}\\
    &\quad+\gint{\vectr{0}}{\vectr{1}}{\fnof{t^{j}}{\vectr{\xi}}\delta
      \fnof{z_{j}}{\vectr{\xi}}\fnof{J_{\embedmanifold{B}}}{\vectr{\xi}}}{\vectr{\xi}}\\ 
    &\quad+\gint{\vectr{0}}{\vectr{1}}{\fnof{P}{\vectr{\xi}}\fnof{n^{j}}{\vectr{\xi}}\delta
      \fnof{z_{j}}{\vectr{\xi}}\fnof{J_{\embedmanifold{B}}}{\vectr{\xi}}}{\vectr{\xi}}    
   \end{split}
\end{equation}
where $P$ is the applied surface pressure.

If we now use basis functions to interpolate the virtual displacements \ie
\begin{equation}
  \delta \fnof{z_{j}}{\vectr{\xi}} = \idxgbfn{j}{m}{\alpha}{\vectr{\xi}}\delta z_{j,\alpha}^{m}\gsf{m}{\alpha}
\end{equation}
which, assuming rectangular cartesian coordinates, gives for the first left hand side integral
\begin{equation}
  \begin{split}
    \gint{\vectr{0}}{\vectr{1}}{\fnof{\sigma^{ij}}{\vectr{\xi}}\delby{\xi_{l}}{x^{i}}\delby{\delta
          \fnof{z_{j}}{\vectr{\xi}}}{\xi^{l}}\fnof{J_{\embedmanifold{B}}}{\vectr{\xi}}}{\vectr{\xi}}
    &= \gint{\vectr{0}}{\vectr{1}}{\fnof{\sigma^{ij}}{\vectr{\xi}}\delby{\xi_{l}}{x^{i}}\delby{
          \pbrac{\idxgbfn{j}{m}{\alpha}{\vectr{\xi}}\delta z_{j,\alpha}^{m}\gsf{m}{\alpha}}}{\xi^{l}}
      \fnof{J_{\embedmanifold{B}}}{\vectr{\xi}}}{\vectr{\xi}} \\
    &= \gint{\vectr{0}}{\vectr{1}}{\fnof{\sigma^{ij}}{\vectr{\xi}}\delby{\xi_{l}}{x^{i}}\delby{
          \idxgbfn{j}{m}{\alpha}{\vectr{\xi}}}{\xi^{l}}\delta z_{j,\alpha}^{m}\gsf{m}{\alpha}
      \fnof{J_{\embedmanifold{B}}}{\vectr{\xi}}}{\vectr{\xi}} \\
    &= \delta z_{j,\alpha}^{m}\gsf{m}{\alpha}
    \gint{\vectr{0}}{\vectr{1}}{\fnof{\sigma^{ij}}{\vectr{\xi}}\delby{\xi_{l}}{x^{i}}\delby{
          \gbfn{m}{j\alpha}{\vectr{\xi}}}{\xi^{l}}
      \fnof{J_{\embedmanifold{B}}}{\vectr{\xi}}}{\vectr{\xi}} 
  \end{split}
\end{equation}
and for the second left hand side integral
\begin{equation}
  \gint{\vectr{0}}{\vectr{1}}{\fnof{\sigma^{ij}_{p}}{\vectr{\xi}}\delby{\xi_{l}}{x^{i}}\delby{\delta
      \fnof{z_{j}}{\vectr{\xi}}}{\xi^{l}}\fnof{J_{\embedmanifold{B}}}{\vectr{\xi}}}{\vectr{\xi}}=
  \delta z_{j,\alpha}^{m}\gsf{m}{\alpha}
  \gint{\vectr{0}}{\vectr{1}}{\fnof{\sigma^{ij}_{p}}{\vectr{\xi}}\delby{\xi_{l}}{x^{i}}\delby{
      \gbfn{m}{j\alpha}{\vectr{\xi}}}{\xi^{l}}
    \fnof{J_{\embedmanifold{B}}}{\vectr{\xi}}}{\vectr{\xi}} 
\end{equation}
and for the first integral on the right hand side integral we have
\begin{equation}
  \begin{split}
    \gint{\vectr{0}}{\vectr{1}}{\fnof{b^{j}}{\vectr{\xi}}\delta
      \fnof{z_{j}}{\vectr{\xi}}\fnof{J_{\embedmanifold{B}}}{\vectr{\xi}}}{\vectr{\xi}}
    &= \gint{\vectr{0}}{\vectr{1}}{\fnof{b^{j}}{\vectr{\xi}}
      \idxgbfn{j}{m}{\alpha}{\vectr{\xi}}\delta
      z_{j,\alpha}^{m}\gsf{m}{\alpha}
      \fnof{J_{\embedmanifold{B}}}{\vectr{\xi}}}{\vectr{\xi}} \\
    &= \delta
    z_{j,\alpha}^{m}\gsf{m}{\alpha}\gint{\vectr{0}}{\vectr{1}}{\fnof{b^{j}}{\vectr{\xi}}\gbfn{m}{j\alpha}{\vectr{\xi}}
      \fnof{J_{\embedmanifold{B}}}{\vectr{\xi}}}{\vectr{\xi}}
  \end{split}
\end{equation}
and for the second integral on the right hand side we have
\begin{equation}
  \begin{split}
    \gint{\vectr{0}}{\vectr{1}}{\fnof{t^{j}}{\vectr{\xi}}\delta
      \fnof{z_{j}}{\vectr{\xi}}\fnof{J_{\embedmanifold{B}}}{\vectr{\xi}}}{\vectr{\xi}}
    &= \gint{\vectr{0}}{\vectr{1}}{\fnof{t^{j}}{\vectr{\xi}}
      \idxgbfn{j}{m}{\alpha}{\vectr{\xi}}\delta
      z_{j,\alpha}^{m}\gsf{m}{\alpha}
      \fnof{J_{\embedmanifold{B}}}{\vectr{\xi}}}{\vectr{\xi}} \\
    &= \delta
    z_{j,\alpha}^{m}\gsf{m}{\alpha}\gint{\vectr{0}}{\vectr{1}}{\fnof{t^{j}}{\vectr{\xi}}\gbfn{m}{j\alpha}{\vectr{\xi}}
      \fnof{J_{\embedmanifold{B}}}{\vectr{\xi}}}{\vectr{\xi}}
  \end{split}
\end{equation}
and for the third integral on the right hand side we have
\begin{equation}
  \begin{split}
    \gint{\vectr{0}}{\vectr{1}}{\fnof{P}{\vectr{\xi}}\fnof{n^{j}}{\vectr{\xi}}\delta
      \fnof{z_{j}}{\vectr{\xi}}\fnof{J_{\embedmanifold{B}}}{\vectr{\xi}}}{\vectr{\xi}}
    &= \gint{\vectr{0}}{\vectr{1}}{\fnof{P}{\vectr{\xi}}\fnof{n^{j}}{\vectr{\xi}}
      \idxgbfn{j}{m}{\alpha}{\vectr{\xi}}\delta
      z_{j,\alpha}^{m}\gsf{m}{\alpha}
      \fnof{J_{\embedmanifold{B}}}{\vectr{\xi}}}{\vectr{\xi}}
    \\
    &= \delta z_{j,\alpha}^{m}\gsf{m}{\alpha}\gint{\vectr{0}}{\vectr{1}}{\fnof{P}{\vectr{\xi}}\fnof{n^{j}}{\vectr{\xi}}
      \gbfn{m}{j\alpha}{\vectr{\xi}}
      \fnof{J_{\embedmanifold{B}}}{\vectr{\xi}}}{\vectr{\xi}}
  \end{split}
\end{equation}

This can be formulated as
\begin{equation}
  r_{m}^{j\alpha}\delta z_{j,\alpha}^{m}=0
\end{equation}
where the residual vector is thus given by
\begin{multline}
  r_{m}^{j\alpha}=\gsf{m}{\alpha}\left(
    \gint{\vectr{0}}{\vectr{1}}{\pbrac{\fnof{\sigma^{ij}}{\vectr{\xi}}\delby{\xi_{l}}{x^{i}}\delby{
          \gbfn{m}{j\alpha}{\vectr{\xi}}}{\xi^{l}}+\rho\pbrac{
        \fnof{a^{j}}{\vectr{\xi}}-\fnof{b^{j}}{\vectr{\xi}}}\gbfn{m}{j\alpha}{\vectr{\xi}}}
      \fnof{J_{\embedmanifold{B}}}{\vectr{\xi}}}{\vectr{\xi}}\right. \\
    \left.-\gint{\vectr{0}}{\vectr{1}}{\fnof{P}{\vectr{\xi}}\fnof{n^{j}}{\vectr{\xi}}
      \gbfn{m}{j\alpha}{\vectr{\xi}}
      \fnof{J_{\embedmanifold{B}}}{\vectr{\xi}}}{\vectr{\xi}}\right)
\end{multline}

Now, as the virtual displacements are arbitrary we have the residual statement
\begin{equation}
  r_{m}^{j\alpha}=0
\end{equation}

In order to handle incompressible materials we need an additional constraint
which penalises change in volume. The change in volume is given by
\begin{equation}
  \Delta V = \dfrac{\sqrt{\determinant{\tensor{g}}}}{J_{g}\sqrt{\determinant{\tensor{G}}}}
\end{equation}
and the residual equation is
\begin{equation}
  \begin{split}
    r_{m}^{\pbrac{N+1}\alpha}&=\gint{\vectr{0}}{\vectr{1}}{\pbrac{\fnof{\Delta V}{\vectr{\xi}} -
        1}\gbfn{m}{\pbrac{N+1}\alpha}{\vectr{\xi}}\gsf{m}{\alpha}\fnof{J_{\embedmanifold{B}}}{\vectr{\xi}}}{\vectr{\xi}}
    \\
    &=\gsf{m}{\alpha}\gint{\vectr{0}}{\vectr{1}}{\pbrac{\dfrac{\sqrt{\determinant{
            \fnof{\tensor{g}}{\vectr{\xi}}}}}{\fnof{J_{g}}{\vectr{\xi}}\sqrt{\determinant{
            \fnof{\tensor{G}}{\vectr{\xi}}}}} -
        1}\gbfn{m}{\pbrac{N+1}\alpha}{\vectr{\xi}}\fnof{J_{\embedmanifold{B}}}{\vectr{\xi}}}{\vectr{\xi}}
  \end{split}
\end{equation}
where $N$ is the number of dimensions.

\subsubsection{Jacobian}

\subsection{Constitutive Laws}

\subsubsection{St Venant-Kirchoff}

The relationships between the elastic constants are given in \Tabref{tab:RelationshipBetweenElasticConstants}.

TODO: Eliminate either $\mu$ or $G$ (they are the same) to make the table
fit. Then fill in.

\begin{table}[htb] \centering
  \begin{tabular}{|c|c|c|c|c|c|c|} \hline
    & $\lambda$ & $\mu$ & $E$ & $\nu$ & $G$ & $K$ \\ \hline \hline
    $\pbrac{\lambda,\mu}$ & - & - &
    $\dfrac{\mu\pbrac{3\lambda+2\mu}}{\lambda+\mu}$ & $\dfrac{\lambda}{2\pbrac{\lambda+\mu}}$ & $\mu$ & $\dfrac{3\lambda+2\mu}{3}$ \\ \hline
    $\pbrac{E,\nu}$ & $\dfrac{\nu E}{\pbrac{1+\nu}\pbrac{1-2\nu}}$ & $\dfrac{E}{2\pbrac{1-\nu}}$ &
    - & - & $\dfrac{E}{2\pbrac{1+\nu}}$ & $\dfrac{E}{3\pbrac{1-2\nu}}$
    \\ \hline
    $\pbrac{G,K}$ & $\dfrac{3K-2G}{3}$ & $G$ & $\dfrac{9KG}{3K+G}$ &
    $\dfrac{3K-2G}{2\pbrac{3K+G}}$ & - & - \\ \hline
    $\pbrac{\lambda,E}$ & - & $\dfrac{E-3\lambda+c}{4}$ & - &
    $\dfrac{2\lambda}{E+\lambda+c}$ & $\dfrac{E-3\lambda+c}{4}$ &
    $\dfrac{E+3\lambda+c}{6}$ \\ \hline
    $\pbrac{\lambda,\nu}$ & - & $\dfrac{\lambda\pbrac{1-2\nu}}{2\nu}$ & $\dfrac{\lambda\pbrac{1+\nu}\pbrac{1-2\nu}}{\nu}$ & - & $$ & $\dfrac{\lambda\pbrac{1+\nu}}{3\nu}$ \\ \hline
  \end{tabular}
  \caption{Reltionships between elastic constants. $\lambda$ is the first
    Lam\'e constant, $\mu$ is the second Lam\'e constant, $E$ is Young's
    modulus, $\nu$ is Poisson's ratio, $G$ is the shear modulus and $K$ is the
    bulk modulus. $c=\sqrt{E^{2}+9\lambda^{2}+2E\lambda}$}
  \label{tab:RelationshipBetweenElasticConstants}
\end{table}


\subsubsection{Mooney-Rivlin}

As an example consider a Mooney-Rivlin material. The strain energy function is
given by
\begin{equation}
  \fnof{W}{I_{1},I_{2}}=c_{1}\pbrac{I_{1}-3}+c_{2}\pbrac{I_{2}-3}
\end{equation}

The second Piola Kirchhoff tensor is thus
\begin{equation}
  S^{AB}=\begin{bmatrix}
    2c_{1}+2c_{2}\pbrac{C_{22}+C_{33}} & -2c_{2}C_{21} & -2c_{2}C_{31} \\
    -2c_{2}C_{12} & 2c_{1}+2c_{2}\pbrac{C_{11}+C_{33}} & -2c_{2}C_{32} \\
    -2c_{2}C_{13} & -2c_{2}C_{23} & 2c_{1}+2c_{2}\pbrac{C_{11}+C_{22}}
  \end{bmatrix}
\end{equation}
or
\begin{equation}
  S^{AB}=\begin{bmatrix}
    c_{1}+c_{2}\pbrac{E_{22}+E_{33}} & -c_{2}E_{21} & -c_{2}E_{31} \\
    -c_{2}E_{12} & c_{1}+c_{2}\pbrac{E_{11}+E_{33}} & -c_{2}E_{32} \\
    -c_{2}E_{13} & -c_{2}E_{23} & c_{1}+c_{2}\pbrac{E_{11}+E_{22}}
  \end{bmatrix}
\end{equation}

\subsubsection{Humpfrey Model}

The Humpfrey strain energy function for an incompressible material can be
stated as
\begin{equation}
  \fnof{W}{\bar{I}_{1},J}=\dfrac{c_{1}}{c_{2}}\pbrac{e^{c_{2}\pbrac{\bar{I}_{1}-3}}-1}+\dfrac{p}{2}\pbrac{J-1}^{2}
\end{equation}
where
\begin{equation}
  \bar{I}_{1}=J^{-\frac{1}{3}}I_{1}
\end{equation}


\section{Uniaxial extension of a unit cube}

Consider a uniaxial stretch of $\alpha$ of a unit cube in the $x$
direction. The resulting deformed configuration will be a cube of dimensions
$X$, $Y$ and $Z$.

Here the deformation gradient tensor will be
\begin{equation}
  \tensor{F}= \begin{bmatrix}
    1+\alpha & 0 & 0 \\
    0 & 1 & 0 \\
    0 & 0 & 1
  \end{bmatrix}
\end{equation}
everywhere, and the Jacobian of transformation will be 
\begin{equation}
  J = \determinant{\tensor{F}}=1+\alpha
\end{equation}

The right Cauchy-Green tensor will be
\begin{equation}
  \tensor{C} = \transpose{\tensor{F}}\tensor{F} = \begin{bmatrix}
    1+\alpha & 0 & 0 \\
    0 & 1 & 0 \\
    0 & 0 & 1
  \end{bmatrix} \begin{bmatrix}
    1+\alpha & 0 & 0 \\
    0 & 1 & 0 \\
    0 & 0 & 1
  \end{bmatrix} = \begin{bmatrix}
    \pbrac{1+\alpha}^2 & 0 & 0 \\
    0 & 1 & 0 \\
    0 & 0 & 1
  \end{bmatrix}
\end{equation}

And the Lagrange strain tensor will be
\begin{equation}
  \tensor{E} = \frac{1}{2}\pbrac{\tensor{C}-\tensor{I}} = \frac{1}{2}\pbrac{\begin{bmatrix}
    \pbrac{1+\alpha}^2 & 0 & 0 \\
    0 & 1 & 0 \\
    0 & 0 & 1
  \end{bmatrix} - \begin{bmatrix}
    1 & 0 & 0 \\
    0 & 1 & 0 \\
    0 & 0 & 1
  \end{bmatrix}} = \begin{bmatrix}
    \alpha+\frac{\alpha^2}{2} & 0 & 0 \\
    0 & 1 & 0 \\
    0 & 0 & 1
  \end{bmatrix}
\end{equation}

The Piola deformation tensor is thus
\begin{equation}
  \tensor{B} = \inverse{\tensor{C}} =  \begin{bmatrix}
    \frac{1}{\pbrac{1+\alpha}^2} & 0 & 0 \\
    0 & 1 & 0 \\
    0 & 0 & 1
  \end{bmatrix}
\end{equation}

The second Piola-Kirchoff stress tensor will be
\begin{equation}
  \begin{split}
    \tensor{S} &= \begin{bmatrix}
      2c_{1}+2c_{2}\pbrac{C_{22}+C_{33}} & -2c_{2}C_{21} & -2c_{2}C_{31} \\
      -2c_{2}C_{12} & 2c_{1}+2c_{2}\pbrac{C_{11}+C_{33}} & -2c_{2}C_{32} \\
      -2c_{2}C_{13} & -2c_{2}C_{23} & 2c_{1}+2c_{2}\pbrac{C_{11}+C_{22}}
    \end{bmatrix} \\
    &= \begin{bmatrix}
      2c_{1}+4c_{2} & 0 & 0 \\
      0 & 2c_{1}+2c_{2}\pbrac{\pbrac{1+\alpha}^2+1} & 0 \\
      0 & 0 & 2c_{1}+2c_{2}\pbrac{\pbrac{1+\alpha}^2+1}
    \end{bmatrix}
  \end{split}
\end{equation}
and the Kirchoff stress tensor will be
\begin{equation}
  \begin{split}
    \tensor{\tau}&=\tensor{F}\tensor{S}\transpose{\tensor{F}} \\
    &= \begin{bmatrix}
      1+\alpha & 0 & 0 \\
      0 & 1 & 0 \\
      0 & 0 & 1
    \end{bmatrix} \begin{bmatrix}
      2c_{1}+4c_{2} & 0 & 0 \\
      0 & 2c_{1}+2c_{2}\pbrac{\pbrac{1+\alpha}^2+1} & 0 \\
      0 & 0 & 2c_{1}+2c_{2}\pbrac{\pbrac{1+\alpha}^2+1}
    \end{bmatrix}\begin{bmatrix}
      1+\alpha & 0 & 0 \\
      0 & 1 & 0 \\
      0 & 0 & 1
    \end{bmatrix} \\
    &= \begin{bmatrix}
      \pbrac{2c_{1}+4c_{2}}\pbrac{1+\alpha}^2 & 0 & 0 \\
      0 & 2c_{1}+2c_{2}\pbrac{\pbrac{1+\alpha}^2+1} & 0 \\
      0 & 0 & 2c_{1}+2c_{2}\pbrac{\pbrac{1+\alpha}^2+1}
    \end{bmatrix}
  \end{split}  
\end{equation}

The Cauchy stress tensor is thus
\begin{equation}
  \tensor{\sigma} = \frac{\tensor{\tau}}{J}= \begin{bmatrix}
    \pbrac{2c_{1}+4c_{2}}\pbrac{1+\alpha} & 0 & 0 \\
    0 & \frac{2c_{1}+2c_{2}\pbrac{\pbrac{1+\alpha}^2+1}}{1+\alpha} & 0 \\
    0 & 0 & \frac{2c_{1}+2c_{2}\pbrac{\pbrac{1+\alpha}^2+1}}{1+\alpha}
  \end{bmatrix}
\end{equation}

The Cauchy tensor resulting from hydrostatic pressure is 
\begin{equation}
  \tensor{\sigma}_{p}= p\tensor{I}=\begin{bmatrix}
    p & 0 & 0 \\
    0 & p & 0 \\
    0 & 0 & p
  \end{bmatrix}
\end{equation}

\subsection{Old Stuff}

Formulation of finite element equations for finite elasticity (large
deformation mechanics) implemented in OpenCMISS is based on the
\textit{\textbf{principle of virtual work}}. The finite element model consists
of a set of non-linear algebraic equations. Non-linearity of equations stems
from non-linear stress-strain relationship and quadratic terms present in the
strain tensor. A typical problem in large deformation mechanics involves
determination of the deformed geometry or mesh nodal parameters, from the
finite element point of view, of the continuum from a known undeformed
geometry, subject to boundary conditions and satisfying stress-strain
(constitutive) relationship.
  
The boundary conditions can be either \textit{\textbf{Dirichlet}}
(displacement), \textit{\textbf{Neumann}} (force) or a combination of them,
known as the mixed boundary conditions. Displacement boundary conditions are
generally nodal based. However, force boundary conditions can take any of the
following forms or a combination of them - nodal-based, distributed load
(e.g. pressure) or force acting at a discrete point on the boundary. In the
latter two forms, the equivalent nodal forces are determined using the
\textit{\textbf{method of work equivalence}} \cite{hutton:2004} and the forces
so obtained will then be added to the right hand side or the residual vector
of the linear equation system.

There are a numerous ways of describing the mechanical characteristics of
deformable materials in large deformation mechanics or finite elasticity
analyses. A predominantly used form for representing constitutive properties
is a strain energy density function. This model gives the energy required to
deform a unit volume (hence energy density) of the deformable continuum as a
function of Green-Lagrange strain tensor components or its derived variables
such as invariants or principal stretches. A material that has a strain energy
density function is known as a \textit{\textbf{hyperelastic}} or
\textit{\textbf{Green-elastic material}}.

The deformed equilibrium state should also give the minimum total elastic
potential energy. One can therefore formulate finite element equations using
the \textit{\textbf{Variational method}} approach where an extremum of a
functional (in this case total strain energy) is determined to obtain mesh
nodal parameters of the deformed continuum. It is also possible to derive the
finite element equations starting from the governing equilibrium equations
known as Cauchy equation of motion. The weak form of the governing equations
is obtained by multiplying them with suitable weighting functions and
integrating over the domain (method of weighted residuals). If interpolation
or shape functions are used as weighting functions, then the method is called
the Galerkin finite element method. All three approaches (virtual work,
variational method and Galerkin formulation) result in the same finite element
equations.

In the following sections the derivation of kinematic relationships of
deformation, energy conjugacy, constitutive relationships and final form the
finite element equations using the virtual work approach will be discussed in
detail.

\subsubsection{Kinematics of Deformation}
In order to track the deformation of an infinitesimal length at a particle of
the continuum, two coordinates systems are defined. An arbitrary orthogonal
spatial coordinate system, which is fixed in space and a material coordinate
system which is attached to the continuum and deforms with the continuum. The
material coordinate system, in general, is a curvi-linear coordinate system
but must have mutually orthogonal axes at the undeformed state. However, in
the deformed state, these axes are no longer orthogonal as they deform with
the continuum (fig 1). In addition to these coordinate systems, there exist
finite element coordinate systems (one for each element) as well. These
coordinates are normalised and vary from 0.0 to 1.0. The following notations are used to represent various coordinate systems and coordinates of a particle of the continuum.\\

\noindent $Y_{1}$-$Y_{2}$-$Y_{3}$ - fixed spatial coordinate system axes - orthogonal\\
$N_{1}$-$N_{2}$-$N_{3}$ - deforming material coordinate system axes  - orthogonal in the undeformed state\\
$\Xi_{1}$-$\Xi_{2}$-$\Xi_{3}$ - element coordinate system - non-orthogonal in general and deforms with continuum\\

\noindent $x_{1}$-$x_{2}$-$x_{3}$ [$\vect{x}$] - spatial coordinates of a particle in the undeformed state wrt $Y_{1}$-$Y_{2}$-$Y_{3}$ CS \\
$z_{1}$-$z_{2}$-$z_{3}$ [$\vect{z}$] - spatial coordinates of the same particle in the deformed state wrt $Y_{1}$-$Y_{2}$-$Y_{3}$ CS \\
$\nu_{1}$-$\nu_{2}$-$\nu_{3}$ [$\vect{\nu}$] - material coordinates of the particle wrt $N_{1}$-$N_{2}$-$N_{3}$ CS (these do not change) \\
$\xi_{1}$-$\xi_{2}$-$\xi_{3}$ [$\vect{\xi}$] - element coordinates of the particle wrt $\Xi_{1}$-$\Xi_{2}$-$\Xi_{3}$ CS (these too do not change)\\

Since the directional vectors of the material coordinate system at any given
point in the undeformed state is mutually orthogonal, the relationship between
spatial $\vect{x}$ and material $\vect{\nu}$ coordinates is simply a
rotation. The user must define the undeformed material coordinate
system. Typically a nodal based interpolatable field known as fibre
information (fibre, imbrication and sheet angles) is input to OpenCMISS. These
angles define how much the \textit{\textbf{reference or default material
    coordinate system}} must be rotated about the reference material axes. The
reference material coordinate system at a given point is defined as
follows. The first direction $\nu_{1}$ is in the $\xi_{1}$ direction. The
second direction, $\nu_{2}$ is in the $\xi_{1}-\xi_{2}$ plane but orthogonal
to $\nu_{1}$. Finally the third direction $\nu_{3}$ is determined to be normal
to both $\nu_{1}$ and $\nu_{2}$. Once the reference coordinate system is
defined, it is then rotated about $\nu_{3}$ by an angle equal to the
interpolated fibre value at the point in counter-clock wise direction. This
will be followed by a rotation about new $\nu_{2}$ axis again in the
counter-clock wise direction by an angle equal to the sheet value. The final
rotation is performed about the current $\nu_{1}$ by an angle defined by
interpolated sheet value. Note that before a rotation is carried out about an
arbitrary axis one must first align(transform) the axis of rotation with one
of the spatial coordinate system axes. Once the rotation is done, the rotated
coordinate system (material) must be inverse-transformed.

Having defined the undeformed orthogonal material coordinate system, the
metric tensor $\delby{\vect{x}}{\vect{\nu}}$ can be determined. As mentioned,
the tensor $\delby{\vect{x}}{\vect{\nu}}$ contains rotation required to align
material coordinate system with spatial coordinate system. This tensor is
therefore orthogonal. A similar metric tensor can be defined to relate the
deformed coordinates $\vect{z}$ of the point to its material coordinates
$\vect{\nu}$. Note that the latter coordinates do not change as the continuum
deforms and more importantly this tensor is not orthogonal as well. The metric
tensor, $\delby{\vect{z}}{\vect{\nu}}$ is called the
\textit{\textbf{deformation gradient tensor}} and denoted as $\matr{F}$.

\begin{equation}
  \matr{F}=\delby{\vect{z}}{\vect{\nu}}
  \label{eqn:deformationgradienttensor}
\end{equation}
 
It can be shown that the deformation gradient tensor contains rotation when an
infinitesimal length $\vect{dr_{0}}$ in the undeformed state undergoes
deformation. Since rotation does not contribute to any strain, it must be
removed from the deformation gradient tensor. Any tensor can be decomposed
into an orthogonal tensor and a symmetric tensor (known as polar
decomposition). In other words, the same deformation can be achieved by first
rotating $\vect{dr}$ and then stretching (shearing and scaling) or
vice-verse. Thus, the deformation gradient tensor can be given by,

\begin{equation}
  \matr{F}=\delby{\vect{z}}{\vect{\nu}}=\matr{R}\matr{U}=\matr{V}\matr{R_{1}}
  \label{eqn:polardecomposition}
\end{equation}
 
The rotation present in the deformation gradient tensor can be removed either
by right or left multiplication of $\matr{F}$. The resulting tensors lead to
different strain measures. The right Cauchy deformation tensor $\matr{C}$ is
obtained from,

\begin{equation}
  \matr{C}=\transpose{[\matr{R}\matr{U}]}[\matr{R}\matr{U}]=\transpose{\matr{U}}\transpose{\matr{R}}\matr{R}\matr{U}=\transpose{\matr{U}}\matr{U}
  \label{eqn:rightCauchy}
\end{equation}

Similarly the left Cauchy deformation tensor or the Finger tensor \matr{B} is
obtained from the left multiplication of \matr{F},

\begin{equation}
  \matr{B}=[\matr{V}\matr{R_{1}}]\transpose{[\matr{V}\matr{R_{1}}]}=\matr{V}\matr{R_{1}}\transpose{\matr{R_{1}}}\transpose{\matr{V}}=\matr{V}\transpose{\matr{V}}
  \label{eqn:leftCauchy}
\end{equation}

\noindent Note that both $\matr{R}$ and $\matr{R_{1}}$ are orthogonal tensors
and therefore satisfy the following condition,

\begin{equation}
  \transpose{\matr{R}}\matr{R}=\matr{R_{1}}\transpose{\matr{R_{1}}}=\matr{I}
  \label{eqn:orthoganality}
\end{equation}

Since there is no rotation present in both $\matr{C}$ and $\matr{B}$, they can
be used to define suitable strain measures as follows,

\begin{equation}
  \matr{E}=\frac{1}{2}\pbrac{\transpose{\delby{\vect{z}}{\vect{\nu}}}\delby{\vect{z}}{\vect{\nu}}-
                       \transpose{\delby{\vect{x}}{\vect{\nu}}}\delby{\vect{x}}{\vect{\nu}}}=
	    \frac{1}{2}(\matr{C}-\matr{I})	       
  \label{eqn:greenstrain}
\end{equation}

\noindent and

\begin{equation}
  \vect{e}=\frac{1}{2}\bbrac{\pbrac{\delby{\vect{x}}{\vect{\nu}}\transpose{\delby{\vect{x}}{\vect{\nu}}}}^{-1}-
                             \pbrac{\delby{\vect{z}}{\vect{\nu}}\transpose{\delby{\vect{z}}{\vect{\nu}}}}^{-1}}=
			     \frac{1}{2}\pbrac{\matr{I}-\matr{B}^{-1}}  
  \label{eqn:almansistrain}
\end{equation}

\noindent where $\matr{E}$ and $\vect{e}$ are called Green and Almansi strain tensors respectively. 
Also note that $\delby{\vect{x}}{\vect{\nu}}$ is an orthogonal tensor. \\

It is now necessary to establish a relationship between strain and displacement. Referring to figure 1, 

\begin{equation}
  \vect{z}=\vect{x}+\vect{u}
  \label{eqn:displacement}
\end{equation}

\noindent where \vect{u} is the displacement vector. \\

\noindent Differentiating \eqnref{eqn:displacement} using the chain rule,

\begin{equation}
  \delby{\vect{z}}{\vect{\nu}}=\delby{\vect{x}}{\vect{\nu}}+\delby{\vect{u}}{\vect{x}}\delby{\vect{x}}{\vect{\nu}}=
                               \pbrac{\matr{I}+\delby{\vect{u}}{\vect{x}}}\delby{\vect{x}}{\vect{\nu}}  
  \label{eqn:displacementgradient}
\end{equation}

\noindent Substituting \eqnref{eqn:displacementgradient} into \eqnref{eqn:greenstrain},

\begin{equation}
  \matr{E}=\frac{1}{2}\bbrac{\transpose{\delby{\vect{x}}{\vect{\nu}}}\transpose{\pbrac{\matr{I}+\delby{\vect{u}}{\vect{x}}}}
                  \pbrac{\matr{I}+\delby{\vect{u}}{\vect{x}}}\delby{\vect{x}}{\vect{\nu}}-\matr{I}}
  \label{eqn:greendisplacement1}
\end{equation}

\noindent Simplifying,

\begin{equation}
  \matr{E}=\frac{1}{2}\transpose{\delby{\vect{x}}{\vect{\nu}}}
           \pbrac{\delby{\vect{u}}{\vect{x}}+\transpose{\delby{\vect{u}}{\vect{x}}}+
	   \transpose{\delby{\vect{u}}{\vect{x}}}\delby{\vect{u}}{\vect{x}}}
	   \delby{\vect{x}}{\vect{\nu}}
  \label{eqn:greendisplacement2}
\end{equation}
 
As can be seen from \eqnref{eqn:greendisplacement2} the displacement gradient
tensor $\delby{\vect{u}}{\vect{x}}$ is defined with respect to undeformed
coordinates $\vect{x}$. This means that the strain tensor $\matr{E}$ has
Lagrangian description and hence it is also also called the Green-Lagrange
strain tensor.
 
A similar derivation can be employed to establish a relationship between the
Almansi and displacement gradient tensors and the final form is given by,

\begin{equation}
  \vect{e}=\frac{1}{2}\delby{\vect{u}}{\vect{z}}+\transpose{\delby{\vect{u}}{\vect{z}}}-
	   \transpose{\delby{\vect{u}}{\vect{z}}}\delby{\vect{u}}{\vect{z}}
  \label{eqn:almansidisplacement}
\end{equation}
 
The displacement gradient tensor terms in \eqnref{eqn:almansidisplacement} are defined with respect to deformed coordinates $\vect{z}$ and
therefore the strain tensor has Eulerian description. Thus it is also known as the Almansi-Euler strain tensor.

\subsubsection{Energy Conjugacy}



\subsubsection{Constitutive models}



\subsubsection{Principle of Virtual Work}
Elastic potential energy or simply elastic energy associated with the
deformation can be given by strain and its energetically conjugate stress.
Note that the Cauchy stress and Almansi-Euler strain tensors and Second
Piola-Kirchhoff (2PK) and Green-Lagrange tensors are energetically
conjugate. Thus, the \textit{\textbf{total internal energy}} due to strain in
the body at the deformed state (fig. 3.1) can be given by,
 
\begin{equation}
  W_{int}=\gint{0}{v}{(\vect{e}:\vect{\sigma})}v
  \label{eqn:totalenergy}
\end{equation}

where \vect{e} and \vect{\sigma} are Almansi strain tensor and Cauchy stress
tensor respectively.

If the deformed body is further deformed by introducing virtual displacements,
then the new internal elastic energy can be given by,

\begin{equation}
  {W_{int}+\delta W_{int}}=\gint{0}{v}{[\vect{(e+\delta{e})}:\vect{\sigma}]}v
  \label{eqn:virtualtotalenergy}
\end{equation}

Deducting \eqnref{eqn:totalenergy} from \eqnref{eqn:virtualtotalenergy},

\begin{equation}
  \delta W_{int}=\gint{0}{v}{\pbrac{\vect{\delta \epsilon} : \vect{\sigma}}}v
  \label{eqn:virtualenergy}
\end{equation}

Using \eqnref{eqn:almansidisplacement} for virtual strain,

\begin{equation}
  \vect{\delta e}=\delby{\vect{\delta u}}{\vect{z}} + \transpose{\delby{\vect{\delta u}}{\vect{z}}} + 
                  \transpose{\delby{\vect{\delta u}}{\vect{z}}}\delby{\vect{\delta u}}{\vect{z}}
  \label{eqn:virtualalmansidisplacement}
\end{equation}

Since virtual displacements are infinitesimally small, quadratic terms in
\eqnref{eqn:virtualalmansidisplacement} can be neglected.  The resulting
strain tensor, known as small strain tensor \vect{\epsilon}, can be given as,

\begin{equation}
  \vect{\delta \epsilon}=\delby{\vect{\delta u}}{\vect{z}} + \transpose{\delby{\vect{\delta u}}{\vect{z}}} 
  \label{eqn:virtualsmalldisplacement}
\end{equation}
 
Since both $\vect{\sigma}$ and $\vect{\delta \epsilon}$ are symmetric, new
vectors are defined by inserting tensor components as follows,

\begin{equation}
  \vect{\delta \epsilon}=\transpose{\sqbrac{\delta \epsilon_{11} \hspace{4 pt} \delta \epsilon_{22} \hspace{4 pt} \delta \epsilon_{33} 
      \hspace{4 pt} 2\delta \epsilon_{12} \hspace{4 pt} 2\delta \epsilon_{23} \hspace{4 pt} 2\delta \epsilon_{13}}} :
  \vect{\sigma}=\transpose{\sqbrac{\delta \sigma_{11} \hspace{4 pt} \delta \sigma_{22} \hspace{4 pt} \delta \sigma_{33} 
      \hspace{4 pt} 2\delta \sigma_{12} \hspace{4 pt} 2\delta \sigma_{23} \hspace{4 pt} 2\delta \sigma_{13} }}	  		  
  \label{eqn:newvectors}
\end{equation} 

Substituting \eqnref{eqn:newvectors} into \eqnref{eqn:virtualenergy},

\begin{equation}
  \delta W_{int}=\gint{0}{v}{\pbrac{\transpose{\vect{\delta \epsilon}} \vect{\sigma}}}v
  \label{eqn:virtualenergy1}
\end{equation}

The strain vector $\vect{\delta \epsilon}$ can be related to displacement
vector using the following equation,

\begin{equation}
  \vect{\delta \epsilon}=\matr{D} \vect{\delta u} 
  \label{eqn:virtualsmalldisplacement1}
\end{equation}

\noindent where $\matr{D}$ and $\vect{u}$ are linear differential operator and
displacement vector respectively and given by,

\begin{equation}
  \begin{array}{c} \matr{D} \end{array} =
  \pbrac{ \begin{array}{ccc} \delby{}{z_{1}} & 0 & 0 \\ 
      0 & \delby{}{z_{2}} & 0 \\
      0 & 0 & \delby{}{z_{3}} \\
      \delby{}{z_{2}} & \delby{}{z_{1}} & 0 \\ 
      0 & \delby{}{z_{3}} & \delby{}{z_{2}} \\ 
      \delby{}{z_{3}} & 0 & \delby{}{z_{1}} \\ \end{array} }
  \label{eqn:differentialoperator}
\end{equation}

\begin{equation}
  \vect{\delta u}=\transpose{\pbrac{\delta u_{1} \hspace{4 pt} \delta u_{2} \hspace{4 pt} \delta u_{3}}}
  \label{eqn:displacementvector}
\end{equation}

The virtual displacement is a finite element field and hence the value at any
point can be obtained by interpolating nodal virtual displacements.

\begin{equation}
  \vect{\delta u}=\matr{\Phi}\matr{\Delta}
  \label{eqn:interpolation}
\end{equation}

