\chapter{Coordinate Systems}
\label{app:CoordinateSystems}

\section{Rectangular Cartesian}
\label{sec:CoordinateSystemsRectangularCartesian}

The base vectors with respect to the global coordinate system are
\begin{equation}
  \generalbasevector_{i}=\begin{bmatrix} 
    \worldbasevector_{1} \\ 
    \worldbasevector_{2} \\
    \worldbasevector_{3} 
  \end{bmatrix}
\end{equation}

The covariant metric tensor is
\begin{equation}
  \generalmetrictensorsymbol_{ij}=\begin{bmatrix}
    1 & 0 & 0 \\
    0 & 1 & 0 \\
    0 & 0 & 1
  \end{bmatrix}
\end{equation}
and the contravariant metric tensor is
\begin{equation}
  \generalmetrictensorsymbol^{ij}=\begin{bmatrix}
    1 & 0 & 0 \\
    0 & 1 & 0 \\
    0 & 0 & 1
  \end{bmatrix}
\end{equation}

The Christoffel symbols of the second kind are all zero.

\section{Cylindrical Polar}
\label{sec:CoordinateSystemsCylindricalPolar}

The global coordinates  $\pbrac{x,y,z}$ with respect to the cylindrical polar
coordinates $\pbrac{r,\theta,z}$ are defined by
\begin{equation}
  \begin{aligned}
    x = r\cos\theta  & \qquad r \ge0 \\
    y = r\sin\theta & \qquad 0 \le\theta\le2\pi \\
    z = z          & \qquad -\infty < z < \infty
  \end{aligned}
\end{equation}

The base vectors with respect to the global coordinate system are
\begin{equation}
  \generalbasevector_{i}=\begin{bmatrix} 
    \cos\theta\worldbasevector_{1} + \sin\theta\worldbasevector_{2} \\ 
    -r\sin\theta\worldbasevector_{1}+ r\cos\theta\worldbasevector_{2} \\
    \worldbasevector_{3} 
  \end{bmatrix}
\end{equation}

The covariant metric tensor is
\begin{equation}
  \generalmetrictensorsymbol_{ij}=\begin{bmatrix}
    1 & 0 & 0 \\
    0 & r^{2} & 0 \\
    0 & 0 & 1
  \end{bmatrix}
\end{equation}
and the contravariant metric tensor is
\begin{equation}
  \generalmetrictensorsymbol^{ij}=\begin{bmatrix}
    1 & 0 & 0 \\
    0 & \frac{1}{r^{2}} & 0 \\
    0 & 0 & 1
  \end{bmatrix}
\end{equation}

The Christoffell symbols of the second kind are
\begin{align}
  \christoffelsecond{r}{\theta}{\theta}&=-r \\
  \christoffelsecond{\theta}{r}{\theta}=\christoffelsecond{\theta}{\theta}{r}&=\frac{1}{r}
\end{align}
with all other Christoffell symbols zero.

\section{Spherical Polar}
\label{sec:CoordinateSystemsSphericalPolar}

The global coordinates $\pbrac{x,y,z}$ with respect to the cylindrical polar
coordinates $\pbrac{r,\theta,\phi}$ are defined by
\begin{equation}
  \begin{aligned}
    x = r\cos\theta\sin\phi & \qquad r \ge 0 \\
    y = r\sin\theta\sin\phi & \qquad 0 \le \theta \le 2\pi \\
    z = r\cos\phi & \qquad 0 \le \phi \le \pi
  \end{aligned}
\end{equation}

The base vectors with respect to the spherical polar coordinate system are
\begin{equation}
  \generalbasevector_{i}=\begin{bmatrix} 
    \cos\theta\sin\phi\worldbasevector_{1}+\sin\theta\sin\phi\worldbasevector_{2}+\cos\phi\worldbasevector_{3} \\ 
    -r\sin\theta\sin\phi\worldbasevector_{1}+r\cos\theta\sin\phi\worldbasevector_{2} \\
    r\cos\theta\cos\phi\worldbasevector_{1}+r\sin\theta\cos\phi\worldbasevector_{2}-r\sin\phi\worldbasevector_{3}
  \end{bmatrix}
\end{equation}

The covariant metric tensor is
\begin{equation}
  \generalmetrictensorsymbol_{ij}=\begin{bmatrix}
    1 & 0 & 0 \\
    0 & r^{2}\sin^{2}\phi & 0 \\
    0 & 0 & r^{2} 
  \end{bmatrix}
\end{equation}
and the contravariant metric tensor is
\begin{equation}
  \generalmetrictensorsymbol^{ij}=\begin{bmatrix}
    1 & 0 & 0 \\
    0 &  \frac{1}{r^{2}\sin^{2}\phi} & 0 \\
    0 & 0 & \frac{1}{r^{2}} 
  \end{bmatrix}
\end{equation}

The Christoffell symbols of the second kind are
\begin{align}
  \christoffelsecond{r}{\theta}{\theta}&=-r\sin^{2}\phi \\
  \christoffelsecond{r}{\phi}{\phi}&=-r \\
  \christoffelsecond{\phi}{\theta}{\theta}&=-\sin\phi\cos\phi \\
  \christoffelsecond{\theta}{r}{\theta}=\christoffelsecond{\theta}{\theta}{r}&=\frac{1}{r} \\
  \christoffelsecond{\phi}{r}{\phi}=\christoffelsecond{\phi}{\phi}{r}&=\frac{1}{r} \\
  \christoffelsecond{\theta}{\theta}{\phi}=\christoffelsecond{\theta}{\phi}{\theta}&=\cot\phi
\end{align}
with all other Christofell symbols zero.

\section{Prolate Spheroidal}
\label{sec:CoordinateSystemsProlateSpheroidal}

The global coordinates $\pbrac{x,y,z}$ with respect to the prolate spheroidal
coordinates $\pbrac{\lambda,\mu,\theta}$ are defined by
\begin{equation}
  \begin{aligned}
    x = a\sinh\lambda\sin\mu\cos\theta & \qquad \lambda \ge 0 \\
    y = a\sinh\lambda\sin\mu\sin\theta & \qquad 0 \le \mu \le \pi \\
    z = a\cosh\lambda\cos\mu & \qquad 0 \le \theta \le 2\pi 
  \end{aligned}
\end{equation}
where $a\ge0$ is the focus.

The base vectors with respect to the global coordinate system are
\begin{equation}
  \generalbasevector_{i}=\begin{bmatrix} 
  a\cosh\lambda\sin\mu\cos\theta\worldbasevector_{1}+a\cosh\lambda\sin\mu\sin\theta\worldbasevector_{2}+
  a\sinh\lambda\cos\mu\worldbasevector_{3}\\ 
  a\sinh\lambda\cos\mu\cos\theta\worldbasevector_{1}+a\sinh\lambda\cos\mu\sin\theta\worldbasevector_{2}-
  a\cosh\lambda\sin\mu\worldbasevector_{3}\\
  -a\sinh\lambda\sin\mu\sin\theta\worldbasevector_{1}+a\sinh\lambda\sin\mu\cos\theta\worldbasevector_{2}
  \end{bmatrix}
\end{equation}

The covariant metric tensor is
\begin{equation}
  \generalmetrictensorsymbol_{ij}=\begin{bmatrix}
    a^{2}\pbrac{\sinh^{2}\lambda+\sin^{2}\mu} & 0 & 0 \\
    0 & a^{2}\pbrac{\sinh^{2}\lambda+\sin^{2}\mu} & 0 \\
    0 & 0 & a^{2}\sinh^{2}\lambda\sin^{2}\mu 
  \end{bmatrix}
\end{equation}
and the contravariant metric tensor is
\begin{equation}
  \generalmetrictensorsymbol^{ij}=\begin{bmatrix}
    \frac{1}{a^{2}\pbrac{\sinh^{2}\lambda+\sin^{2}\mu}}& 0 & 0 \\
    0 & \frac{1}{a^{2}\pbrac{\sinh^{2}\lambda+\sin^{2}\mu}} & 0 \\
    0 & 0 & \frac{1}{a^{2}\sinh^{2}\lambda\sin^{2}\mu} 
  \end{bmatrix}
\end{equation}

The Christoffell symbols of the second kind are
\begin{align}
  \christoffelsecond{\lambda}{\lambda}{\lambda}&=\frac{\sinh\lambda\cosh\lambda}{\sinh^{2}\lambda+\sin^{2}\mu} \\
  \christoffelsecond{\lambda}{\mu}{\mu}&=\frac{-\sinh\lambda\cosh\lambda}{\sinh^{2}\lambda+\sin^{2}\mu} \\
  \christoffelsecond{\lambda}{\theta}{\theta}&=\frac{-\sinh\lambda\cosh\lambda\sin^{2}\mu}{\sinh^{2}\lambda+\sin^{2}\mu} \\
  \christoffelsecond{\lambda}{\lambda}{\mu}&=\frac{\sin\mu\cos\mu}{\sinh^{2}\lambda+\sin^{2}\mu} \\
  \christoffelsecond{\mu}{\mu}{\mu}&=\frac{\sin\mu\cos\mu}{\sinh^{2}\lambda+\sin^{2}\mu} \\
  \christoffelsecond{\mu}{\lambda}{\lambda}&=\frac{-\sin\mu\cos\mu}{\sinh^{2}\lambda+\sin^{2}\mu} \\
  \christoffelsecond{\mu}{\theta}{\theta}&=\frac{-\sinh^{2}\lambda\sin\mu\cos\mu}{\sinh^{2}\lambda+\sin^{2}\mu} \\
  \christoffelsecond{\mu}{\mu}{\lambda}&=\frac{\sinh\lambda\cosh\lambda}{\sinh^{2}\lambda+\sin^{2}\mu} \\
  \christoffelsecond{\theta}{\theta}{\lambda}&=\frac{\cosh\lambda}{\sinh\lambda} \\
  \christoffelsecond{\theta}{\theta}{\mu}&=\frac{\cos\mu}{\sin\mu} \\
 \end{align}
with all other Christofell symbols zero.

\section{Oblate Spheroidal}
\label{sec:CoordinateSystemsOblateSpheroidal}

The global coordinates $\pbrac{x,y,z}$ with respect to the oblate spheroidal
coordinates $\pbrac{\lambda,\mu,\theta}$  are defined by
\begin{equation}
  \begin{aligned}
    x = a\cosh\lambda\cos\mu\cos\theta & \qquad \lambda \ge 0 \\
    y = a\cosh\lambda\cos\mu\sin\theta & \qquad \frac{-\pi}{2} \le \mu \le \frac{\pi}{2} \\
    z = a\sinh\lambda\sin\mu & \qquad 0 \le \theta \le 2\pi 
  \end{aligned}
\end{equation}
where $a\ge0$ is the focus.

The base vectors with respect to the global coordinate system are
\begin{equation}
  \generalbasevector_{i}=\begin{bmatrix} 
  a\sinh\lambda\cos\mu\cos\theta\worldbasevector_{1}+a\sinh\lambda\cos\mu\sin\theta\worldbasevector_{2}+
  a\cosh\lambda\sin\mu\worldbasevector_{3}\\
  -a\cosh\lambda\sin\mu\cos\theta\worldbasevector_{1}-a\cosh\lambda\sin\mu\sin\theta\worldbasevector_{2}+
  a\sinh\lambda\cos\mu\worldbasevector_{3}\\    
  -a\cosh\lambda\cos\mu\sin\theta\worldbasevector_{1}+a\cosh\lambda\cos\mu\cos\theta\worldbasevector_{2}
  \end{bmatrix}
\end{equation}

The covariant metric tensor is
\begin{equation}
  \generalmetrictensorsymbol_{ij}=\begin{bmatrix}
    a^{2}\pbrac{\sinh^{2}\lambda+\sin^{2}\mu} & 0 & 0 \\
    0 & a^{2}\pbrac{\sinh^{2}\lambda+\sin^{2}\mu} & 0 \\
    0 & 0 & a^{2}\cosh^{2}\lambda\cos^{2}\mu 
  \end{bmatrix}
\end{equation}
and the contravariant metric tensor is
\begin{equation}
  \generalmetrictensorsymbol^{ij}=\begin{bmatrix}
    \frac{1}{a^{2}\pbrac{\sinh^{2}\lambda+\sin^{2}\mu}}& 0 & 0 \\
    0 & \frac{1}{a^{2}\pbrac{\sinh^{2}\lambda+\sin^{2}\mu}} & 0 \\
    0 & 0 & \frac{1}{a^{2}\cosh^{2}\lambda\cos^{2}\mu}
  \end{bmatrix}
\end{equation}

The Christoffell symbols of the second kind are
\begin{align}
  \christoffelsecond{\lambda}{\lambda}{\lambda}&=\frac{\sinh\lambda\cosh\lambda}{\sinh^{2}\lambda+\sin^{2}\mu} \\
  \christoffelsecond{\lambda}{\mu}{\mu}&=\frac{-\sinh\lambda\cosh\lambda}{\sinh^{2}\lambda+\sin^{2}\mu} \\
  \christoffelsecond{\lambda}{\theta}{\theta}&=\frac{-\sinh\lambda\cosh\lambda\cos^{2}\mu}{\sinh^{2}\lambda+\sin^{2}\mu} \\
  \christoffelsecond{\lambda}{\lambda}{\mu}&=\frac{\sin\mu\cos\mu}{\sinh^{2}\lambda+\sin^{2}\mu} \\
  \christoffelsecond{\mu}{\mu}{\mu}&=\frac{\sin\mu\cos\mu}{\sinh^{2}\lambda+\sin^{2}\mu} \\
  \christoffelsecond{\mu}{\lambda}{\lambda}&=\frac{-\sin\mu\cos\mu}{\sinh^{2}\lambda+\sin^{2}\mu} \\
  \christoffelsecond{\mu}{\theta}{\theta}&=\frac{\cosh^{2}\lambda\sin\mu\cos\mu}{\sinh^{2}\lambda+\sin^{2}\mu} \\
  \christoffelsecond{\mu}{\mu}{\lambda}&=\frac{\sinh\lambda\cosh\lambda}{\sinh^{2}\lambda+\sin^{2}\mu} \\
  \christoffelsecond{\theta}{\theta}{\lambda}&=\frac{\sinh\lambda}{\cosh\lambda} \\
  \christoffelsecond{\theta}{\theta}{\mu}&=\frac{-\sin\mu}{\cos\mu} \\
\end{align}
with all other Christofell symbols zero.

\section{Cylindrical Parabolic}
\label{sec:CoordinateSystemsCylindricalParabolic}

The global coordinates $\pbrac{x,y,z}$ with respect to the cylindrical parabolic
coordinates $\pbrac{\xi,\eta,z}$  are defined by
\begin{equation}
  \begin{aligned}
    x = \xi\eta & \qquad -\infty < \xi < \infty \\
    y = \frac{1}{2}\pbrac{\xi^{2}-\eta^{2}} & \qquad \eta \ge 0 \\
    z =  z & \qquad -\infty < z < \infty
  \end{aligned}
\end{equation}

The base vectors with respect to the global coordinate system are
\begin{equation}
  \generalbasevector_{i}=\begin{bmatrix} 
    \eta\worldbasevector_{1}+\xi\worldbasevector_{2}\\
    \xi\worldbasevector_{1}-\eta\worldbasevector_{2}\\    
    \worldbasevector_{3}
  \end{bmatrix}
\end{equation}

The covariant metric tensor is
\begin{equation}
  \generalmetrictensorsymbol_{ij}=\begin{bmatrix}
    \xi^{2}+\eta^{2} & 0 & 0 \\
    0 & \xi^{2}+\eta^{2} & 0 \\
    0 & 0 & 1
  \end{bmatrix}
\end{equation}
and the contravariant metric tensor is
\begin{equation}
  \generalmetrictensorsymbol^{ij}=\begin{bmatrix}
    \frac{1}{\xi^{2}+\eta^{2}}& 0 & 0 \\
    0 & \frac{1}{\xi^{2}+\eta^{2}} & 0 \\
    0 & 0 & 1
  \end{bmatrix}
\end{equation}

The Christoffell symbols of the second kind are
\begin{align}
  \christoffelsecond{\xi}{\xi}{\xi}&=\frac{\xi}{\xi^{2}+\eta^{2}} \\
  \christoffelsecond{\eta}{\eta}{\eta}&=\frac{\eta}{\xi^{2}+\eta^{2}} \\
  \christoffelsecond{\eta}{\xi}{\xi}&=\frac{-\eta}{\xi^{2}+\eta^{2}} \\
  \christoffelsecond{\xi}{\eta}{\eta}&=\frac{-\xi}{\xi^{2}+\eta^{2}} \\
  \christoffelsecond{\xi}{\xi}{\eta}=\christoffelsecond{\xi}{\eta}{\xi}&=\frac{\eta}{\xi^{2}+\eta^{2}} \\
  \christoffelsecond{\eta}{\xi}{\eta}=\christoffelsecond{\eta}{\eta}{\xi}&=\frac{\xi}{\xi^{2}+\eta^{2}} \\
\end{align}
with all other Christofell symbols zero.

\section{Parabolic Polar}
\label{sec:CoordinateSystemsParabolicPolar}

The global coordinates $\pbrac{x,y,z}$ with respect to the cylindrical parabolic
coordinates $\pbrac{\xi,\eta,\theta}$  are defined by
\begin{equation}
  \begin{aligned}
    x = \xi\eta\cos\theta & \qquad \xi \ge 0 \\
    y = \xi\eta\sin\theta & \qquad \eta \ge 0 \\
    z = \frac{1}{2}\pbrac{\xi^{2}-\eta^{2}} & \qquad 0 \le \theta < 2\pi
  \end{aligned}
\end{equation}

The base vectors with respect to the global coordinate system are
\begin{equation}
  \generalbasevector_{i}=\begin{bmatrix} 
    \eta\cos\theta\worldbasevector_{1}+\eta\sin\theta\worldbasevector_{3}+\xi\worldbasevector_{3}\\
    \xi\cos\theta\worldbasevector_{1}+\xi\sin\theta\worldbasevector_{3}-\eta\worldbasevector_{3}\\ 
    -\xi\eta\sin\theta\worldbasevector_{1}+\xi\eta\cos\theta\worldbasevector_{2}
  \end{bmatrix}
\end{equation}

The covariant metric tensor is
\begin{equation}
  \generalmetrictensorsymbol_{ij}=\begin{bmatrix}
    \xi^{2}+\eta^{2} & 0 & 0 \\
    0 & \xi^{2}+\eta^{2} & 0 \\
    0 & 0 & \xi\eta
  \end{bmatrix}
\end{equation}
and the contravariant metric tensor is
\begin{equation}
  \generalmetrictensorsymbol^{ij}=\begin{bmatrix}
    \frac{1}{\xi^{2}+\eta^{2}}& 0 & 0 \\
    0 & \frac{1}{\xi^{2}+\eta^{2}} & 0 \\
    0 & 0 & \frac{1}{\xi\eta}
  \end{bmatrix}
\end{equation}

The Christoffell symbols of the second kind are
\begin{align}
  \christoffelsecond{\xi}{\xi}{\xi}&=\frac{\xi}{\xi^{2}+\eta^{2}} \\
  \christoffelsecond{\eta}{\eta}{\eta}&=\frac{\eta}{\xi^{2}+\eta^{2}} \\
  \christoffelsecond{\xi}{\eta}{\eta}&=\frac{-\xi}{\xi^{2}+\eta^{2}} \\
  \christoffelsecond{\eta}{\xi}{\xi}&=\frac{-\eta}{\xi^{2}+\eta^{2}} \\
  \christoffelsecond{\eta}{\theta}{\theta}&=\frac{-\xi^{2}\eta}{\xi^{2}+\eta^{2}} \\
  \christoffelsecond{\xi}{\theta}{\theta}&=\frac{-\xi\eta^{2}}{\xi^{2}+\eta^{2}} \\
  \christoffelsecond{\xi}{\xi}{\eta}=\christoffelsecond{\xi}{\eta}{\xi}&=\frac{\eta}{\xi^{2}+\eta^{2}} \\
  \christoffelsecond{\eta}{\xi}{\eta}=\christoffelsecond{\eta}{\eta}{\xi}&=\frac{\xi}{\xi^{2}+\eta^{2}} \\
  \christoffelsecond{\theta}{\xi}{\theta}=\christoffelsecond{\theta}{\theta}{\xi}&=\frac{1}{\xi} \\
  \christoffelsecond{\theta}{\eta}{\theta}=\christoffelsecond{\theta}{\theta}{\eta}&=\frac{1}{\eta} \\
\end{align}
with all other Christofell symbols zero.

