\section{Linear Elasticity}
\label{sec:LinearElasticity}


\subsection{Linearisation of Finite Elasticity}

\subsection{Equations of Linear Elasticity}

The equations of \emph{linear (classical) elasticity} are given by
\begin{equation}
  \addtolength{\fboxsep}{5pt}
  \boxed{
    \fnof{\rho}{\vectr{x}}\fnof{\ddot{\vectr{u}}}{\vectr{x},t}=\fnof{\rho}{\vectr{x}}\fnof{\vectr{b}}{\vectr{x}}+
    \divergence{}{\pbrac{\dotprod{\fnof{\tensorfour{c}}{\vectr{x}}}{\gradient{}{\fnof{\vectr{u}}{\vectr{x},t}}}}}
  }
  \label{eqn:LinearElasticityEquation}
\end{equation}
where $\fnof{\rho}{\vectr{x}}$ is the mass density,
$\fnof{\vectr{b}}{\vectr{x},t}$ is the body force field,
$\fnof{\tensorfour{c}}{\vectr{x}}$ is the fourth-order elasticity tensor
field, and $\fnof{\vectr{u}}{\vect{x},t}$ is the is the displacement field.

In component form \eqnref{eqn:LinearElasticityEquation} can be written as
\begin{equation}
  \rho{\ddot{u}}^{i}=\rho b^{i} + \covarderiv{\pbrac{c^{ijkl}\covarderiv{u_{k}}{l}}}{j}
\end{equation}

The elasticity tensor is subject to the following symmetry conditions
\begin{equation}
  c^{ijkl}=c^{jikl}=c^{ijlk}=c^{klij}
\end{equation}

Note that for \emph{linear elastostatics} the
$\fnof{\rho}{\vectr{x}}\fnof{\ddot{\vectr{u}}}{\vectr{x},t}$ term is zero and
is dropped. 
 
The boundary conditions for linear elasticity are Dirichlet conditions on
displacement \ie
\begin{equation}
  \fnof{\vectr{u}}{\vectr{x},t} = \fnof{\vectr{g}}{\vectr{x},t} \quad \vectr{x}\in\Gamma_{D},t\in[0,\infty)
  \label{eqn:LinearElasticityDirichletBC} 
\end{equation}
or, in component form,
\begin{equation}
  u^{i}=g^{i} \quad \vectr{x}\in\Gamma_{D},t\in[0,\infty)
\end{equation}
and Neumann conditions on traction \ie
\begin{equation}
  \fnof{\vectr{t}}{\vectr{x},t} =\dotprod{\pbrac{\dotprod{\fnof{\tensorfour{c}}{\vectr{x}}}
        {\gradient{}{\fnof{\vectr{u}}{\vectr{x},t}}}}}{\fnof{\normal}{\vectr{x}}}
  = \fnof{\vectr{h}}{\vectr{x},t} \quad \vectr{x}\in\Gamma_{N},t\in[0,\infty)
  \label{eqn:LinearElasticityNeumannBC} 
\end{equation}
or, in component form,
\begin{equation}
  t^{i}=c^{ijkl}\covarderiv{u_{k}}{l}n_{j}=h^{i} \quad \vectr{x}\in\Gamma_{N},t\in[0,\infty)
\end{equation}
where $\fnof{t}{\vectr{x},t}$ is the boundary traction, $\fnof{\normal}{\vectr{x}}$ is the normal
vector to the boundary and $\Gamma = \Gamma_D \cup \Gamma_N$.

The \emph{small strain tensor} is given by a linearisation of the Lagrangian
strain tensor $\tensortwo{E}=\frac{1}{2}\pbrac{\tensortwo{C}-\tensortwo{G}}$
and can be written as
\begin{equation}
  \tensortwo{e}=\frac{1}{2}\liederiv{\vectr{u}}{\tensortwo{g}}=\frac{1}{2}\pbrac{\gradient{}{\vectr{u}}+\transpose{\gradient{}{\vectr{u}}}}
\end{equation}
or, in component form,
\begin{equation}
  e_{ij}=\frac{1}{2}\pbrac{\covarderiv{u_{i}}{j}+\covarderiv{u_{j}}{i}}
\end{equation}

The \emph{stress tensor} is given by
\begin{equation}
  \tensortwo{s}=\dotprod{\tensorfour{c}}{\gradient{}{\vectr{u}}}
\end{equation}
or, in component form,
\begin{equation}
  s^{ij}=c^{ijkl}\covarderiv{u_{k}}{l}
\end{equation}

Note that $\tensortwo{s}$ is symmetric and so $s^{ij}=s^{ji}$. The stress
tensor, through the symmetry of the elasticity tensor, also follows the
constitutive relationship
\begin{equation}
  \tensortwo{s}=\doubledotprod{\tensorfour{c}}{\tensortwo{e}}
\end{equation}

Similarily to hyperelastic materials the stored energy function for linear
elasticity is given by
\begin{equation}
  \varepsilon=\frac{1}{2}\doubledotprodthree{\tensortwo{e}}{\tensorfour{c}}{\tensortwo{e}}=\frac{1}{2}e_{ij}c^{ijkl}e_{kl}
\end{equation}

The stress tensor is thus given by
\begin{equation}
  \tensortwo{s}=s^{ij}=\delby{\varepsilon}{\tensortwo{e}}=\delby{\varepsilon}{e_{ij}}
\end{equation}
and the elasticity tensor is given by
\begin{equation}
  \tensorfour{c}=c^{ijkl}=\delby{\tensortwo{s}}{\tensortwo{e}}=\deltwoby{\varepsilon}{\tensortwo{e}}{\tensortwo{e}}=\deltwoby{\varepsilon}{e_{ij}}{e_{kl}}
\end{equation}
