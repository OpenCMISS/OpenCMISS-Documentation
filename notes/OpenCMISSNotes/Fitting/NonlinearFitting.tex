\section{Non-linear Fitting}
\label{sec:nonlinearfitting} 

\subsection{Problem formulation}
\label{sec:problemformulation}

One problem that arises when using linear fitting with cubic Hermite elements is
that arc-length derivatives and average arc-length scaling are not maintained during
the fit. In linear fitting there is no information supplied in the linear
fitting model that enforces arc-length derivatives and the scale factors are held
constant during the fit. In this section the question of how to fit the data
whilst maintaining arc-length derivatives and an even spacing of arc-length with
$\xi$ is considered.  Because both the value of the arc-length for the element
and the relationship between the derivatives in the various spatial directions
depend upon the mesh parameters in a non-linear fashion, the only way to ensure
that arc-length derivatives are maintained during fitting is to use a non-linear
fitting procedure.

Consider the following: Let $\vectr{u}$ be the vector of mesh parameters,
$\vectr{z}^{d}$ be the vector of the location of data point $d$ in space
and $\fnof{\vectr{z}}{\vectr{u},\vectr{\xi}^{d}}$ the vector of the location
of the closest projection (at the points given by the vector
$\vectr{\xi}^{d}$) of that data point onto the mesh. Now the error in the
$d^{\text{th}}$ data point can be expressed in terms of an error vector
\begin{equation}
  \fnof{\vectr{e}^{d}}{\vectr{u},\vectr{\xi}^{d}}=\fnof{\vectr{z}}
  {\vectr{u},\vectr{\xi}^{d}}-\vectr{z}^{d}
  \label{eqn:errorvector}
\end{equation}
and a residual
\begin{equation}
  \fnof{f^{d}}{\vectr{u},\vectr{\xi}^{d}}=\gamma^{d}\norm{\fnof{
      \vectr{e}^{d}}{\vectr{u},\vectr{\xi}^{d}}}^{2}
  \label{eqn:dataresidual}
\end{equation}

A \emph{data objective} function,
$\fnof{\mathcal{F}}{\vectr{u},\vectr{\xi}^{d}}$, is formed as the sum of
squares of the individual residuals for the mesh $\vectr{u}$ and projections
$\vectr{\xi}^{d}$
\begin{equation}
  \fnof{\mathcal{F}}{\vectr{u},\vectr{\xi}^{d}}=\fntof{\vectr{f}}{\vectr{u},
    \vectr{\xi}^{d}}\fnof{\vectr{f}}{\vectr{u},\vectr{\xi}^{d}}
  \label{eqn:nldataresidualfn}
\end{equation}
The fitting problem then becomes, for constant $\vectr{\xi}^{d}$,
\begin{equation}
  \min_{\vectr{u} \in \mathbb{R}} \quad \fnof{\mathcal{F}}{\vectr{u}}=\fntof{\vectr{f}}
  {\vectr{u},\vectr{\xi}^{d}}\fnof{\vectr{f}}{\vectr{u},\vectr{\xi}^{d}}
\end{equation}

In order to maintain arc-length derivatives a non-linear constraint is needed.  This
constraint comes from the geometric properties of arc-length derivatives.  For a
global node $A$ and $\xi$-direction $l$ the magnitude of the vector of arc-length
derivatives in the various spatial directions must be of unit length as
given in \eqnref{eqn:chnormconstraint}.  Hence the constraint is
\begin{equation}
  c^{A}_{l}=\norm{\vectr{x}^{A}_{,l}}=1
  \label{eqn:normconstraint}
\end{equation}
for $l=1,2$.

This also implies simple bounds on the derivative variables:
\begin{equation}
  -1 \leq x^{A}_{j,l} \leq 1
\end{equation}

The fitting problem can thus be written in terms of the non-linearly
constrained, non-linear optimisation problem
\begin{equation}
  \begin{split}
    \min_{\vectr{u}\in\mathbb{R},~\vectr{\xi}^{d}\text{ constant}}\quad &
    \fnof{\mathcal{F}}{\vectr{u}}=\fntof{\vectr{f}}{\vectr{u},
      \vectr{\xi}^{d}}\fnof{\vectr{f}}{\vectr{u},\vectr{\xi}^{d}} \\
    \text{subject to} \quad & \vectr{a} \leq 
    \begin{pmatrix} 
      \vectr{u} \\ 
      \fnof{\vectr{c}}{\vectr{u}}
    \end{pmatrix} \leq \vectr{b}
  \end{split}
  \label{eqn:nloproblem}
\end{equation}
where $\vectr{a}$ is a vector of lower bounds, $\vectr{b}$ a vector of upper
bounds and $\fnof{\vectr{c}}{\vectr{u}}$ a vector of non-linear constraints. This
type of optimisation problem can be solved with readily available non-linear
optimisation packages.

In order to ensure that there is an approximately uniform spacing of $\xi$
with arc-length two approaches can be used. The first approach is to use an
iterative technique for the scale factors (as per the linear fitting case) and
is detailed here. The second approach is to include the scale factors in the
optimisation problem and is detailed in the next sub-section.

The constraint on the nodal scale factors (being the average of the line
lengths either side of the node) is placed upon the problem to ensure that the
arc-length to $\xi$ spacing is approximately uniform. As this only yields
approximately uniform arc-length to $\xi$ spacing the constraint on the nodal
scale factors can be relaxed \ie if the nodal scale factors are held fixed
during the fit then although there will not be average arc-length scaling
throughout the fitting process there will be a reasonable approximation to it.
With the nodal scale factors fixed the variables $\gsf{A}{l}$ can be removed
from the vector of mesh parameters $\vectr{u}$.

Hence, the non-linear fitting approach is to hold the scale factors constant, fit
the mesh with these scale factors, and then update the scale factors (based on
the new mesh) to be average arc-length. This process can be repeated iteratively
until the desired fit has been achieved. It should be noted that as
\eqnref{eqn:nloproblem} is only defined for constant data point projections,
this iterative approach means that the data point projections can also be
updated at the same time as the scale factors.

The convergence of the fitting problem can be measured in two ways. The first
is the convergence of the RMS error in the data and second is the convergence
in the magnitude of the nodal scale factors.

The algorithm is therefore:
\begin{enumerate}
\item Define an initial mesh (and calculate the initial nodal scale factors).
  \item Calculate the initial data point projections and initial error in the
    mesh.
  \item Repeat until converged or the maximum number of iterations is exceeded.
  \begin{enumerate}
    \item Fit the mesh to the data by solving \eqnref{eqn:nloproblem}.
    \item Update the scale factors to be average arc-lengths based on the new
      mesh.
    \item Recalculate the data point projections on the new mesh.
  \end{enumerate}
\end{enumerate}

\subsection{Alternative formulation}
\label{sec:alternativeformulation}

An alternative approach to ensuring average arc-length scale factors can be
formulated by including the scale factors in the optimisation problem as
optimisation variables.  The vector of mesh parameters, $\vectr{u}$, is hence
extended to include the nodal scale factors. With this a new constraint is
introduced to \eqnref{eqn:nloproblem} to ensure that a nodal scale factor is
the average of the arc-lengths on either side of that node. If $\gsf{A}{l}$ is
the nodal scale factor for global node $A$ in the $s_{l}$ direction, as given
by \eqnref{eqn:avearclenscale}, then the constraint that the nodal scale
factor equals the average arc-length is given by
\begin{equation}
  c^{A}_{l}=\dfrac{\fnof{s_{l}}{\fnof{A_{\ominus}}{l}}+
    \fnof{s_{l}}{\fnof{A_{\oplus}}{l}}}{2}-\gsf{A}{l}=0
\end{equation}
or
\begin{equation} 
  c^{A}_{l}=\dfrac{1}{2}\pbrac{
    \gint{0}{1}{\norm{\delby{\fnof{\vectr{x}}{\vectr{\xi}}}{\xi_{l}}}}{
      \fnof{\xi_{l}}{\fnof{A_{\ominus}}{l}}}+
    \gint{0}{1}{\norm{\delby{\fnof{\vectr{x}}{\vectr{\xi}}}{\xi_{l}}}}{
      \fnof{\xi_{l}}{\fnof{A_{\oplus}}{l}}}}-\gsf{A}{l}=0
  \label{eqn:avarclenconstraint}
\end{equation}
where $\fnof{A_{\ominus}}{l}$ is the element immediately preceding (in the
\nth{l} direction) node $A$, and $\fnof{A_{\oplus}}{l}$ is the element
immediately after (in the \nth{l} direction) node $A$. Note that when the
infinitesimal in the integrals above contain
$\fnof{A_{\ominus}}{l}$ ($\fnof{A_{\oplus}}{l}$) the $\xi_{l}$ variable is
assumed to be taken over the previous (next) element from the global node $A$
and thus the integrals are always one-dimensional.

This also generates a simple bound on the scale-factors:
\begin{equation}
  \gsf{A}{l} > 0
\end{equation}

This alternative formulation of the non-linear fitting problem does have one
practical limitation. With average arc-length scaling the interpolation within an
element depends on the arc-length in the neighbouring elements, and the arc-length
in the neighbouring elements depends on their neighbouring elements and so on.
This results in a \emph{global mesh} in that every part of the mesh is
dependent on every other part of the mesh. The implication of this is that the
entire mesh must be included in the fit otherwise average arc-length scaling
cannot be achieved. This is clearly not a desirable feature for very large
problems or for problems where only a small part of the mesh is in error and
needs to be fitted. 

This formulation still requires iteration to converge the data point
projections but does have the advantage that the number of iterations required
is reduced as the scale factors are found during the fit. This is at the
expense of having to solve a much larger non-linear optimisation problem with
more variables and, more importantly, more non-linear constraints.

The fit can be considered converged when the RMS error in the fit does not
change significantly between iterations.  The algorithm for the non-linear data
fitting procedure is as follows:
\begin{enumerate}
\item Define an initial mesh (and calculate the initial nodal scale factors).
\item Calculate the initial data point projections.
\item Repeat until converged or the maximum number of iterations is exceeded.
  \begin{enumerate}
  \item Fit the mesh to the data by solving \eqnref{eqn:nloproblem} (with
    the new constraints).
  \item Recalculate the data point projections on the new mesh.
  \end{enumerate}
\end{enumerate}

\subsection{Residual and constraint Jacobians}
\label{sec:resandcontjacs}

Solution of the non-linear problem given in \eqnref{eqn:nloproblem} will
generally require the evaluation of the objective gradient (or residual vector
Jacobian) and the constraint Jacobian with respect to the optimisation (mesh)
parameters. These optimisation parameters fall into two basic types of
variable: the nodal variables, $x^{\beta}_{j,v}$, and the nodal scale
factors, $\gsf{\beta}{v}$.

To find the residual Jacobian consider the error vector for the data point
$d$, $\vectr{e}^{d}$, defined in \eqnref{eqn:errorvector}, and its
corresponding residual, $f^{d}$, defined in \eqnref{eqn:dataresidual}. The
position of the projection of data point $d$ within the element
$\varepsilon^{d}$ is given by \eqnref{eqn:dataprojectionpos}. Hence the
Jacobian of the residual vector for the various optimisation variables within
$\varepsilon^{d}$ can be found by substituting
\eqnref{eqn:dataprojectionpos} into
\eqnrefs{eqn:errorvector}{eqn:dataresidual} and differentiating \ie
\begin{align}
  \delby{f^{d}}{x^{\beta}_{j,v}}&=2\gbfn{\beta}{v}{
    \vectr{\xi}^{d}}\gsf{\beta}{v}e^{d}_{j} 
  \label{eqn:dataresjacnodevars} \\
  \delby{f^{d}}{\gsf{\beta}{v}}&=2\gbfn{\beta}{v}{\vectr{\xi}^{d}}
  \dotprod{\vectr{x}^{\pbrac{\beta}}_{,\pbrac{v}}}{\vectr{e}^{d}}
  \label{eqn:dataresjacsfvars}
\end{align}
For the optimisation variables not contained within $\varepsilon^{d}$ the
Jacobian is zero.
  
Note that if the data residual is defined as
\begin{equation}
  \fnof{f^{d}}{\vectr{u},\vectr{\xi}^{d}}=\norm{
    \fnof{\vectr{e}^{d}}{\vectr{u},\vectr{\xi}^{d}}}
\end{equation}
then \eqnref{eqn:dataresjacnodevars} becomes
\begin{equation}
  \delby{f^{d}}{x^{\beta}_{j,v}}=\dfrac{\gbfn{\beta}{v}{
      \vectr{\xi}^{d}}\gsf{\beta}{v}
    e^{d}_{j}}{f^{d}}
\end{equation}
which is singular at the optimal solution $f^{d}=0$. To avoid numerical
problems the residual is therefore defined by \eqnref{eqn:dataresidual} as the
square of the Euclidean distance rather than the straight Euclidean distance
as per the linear fitting case.

Now consider the constraint Jacobians. Differentiating the first constraint,
\eqnref{eqn:normconstraint}, with respect to the global mesh parameters gives
the constraint gradients:
\begin{align}
  \delby{c^{A}_{l}}{x^{B}_{j,v}}&=\kronecker{A}{B}
  \kronecker{l}{v}\dfrac{x^{B}_{j,v}}{\norm{
      \vectr{x}^{B}_{,v}}} \label{eqn:normjacderiv} \\ 
  \delby{c^{A}_{l}}{\gsf{B}{v}}&=0 \label{eqn:normjacline}
\end{align}
for $l=1,2$.

To calculate the gradients for the second constraint,
\eqnref{eqn:avarclenconstraint}, the rate of change of $\vectr{x}$ with respect
to $\xi_{l}$ within an element is first computed by differentiating the
generalised form of \eqnref{eqn:chsfinterpolation} \ie
\begin{equation}
  \delby{\fnof{\vectr{x}}{\vectr{\xi}}}{\xi_{l}}=\delby{\gbfn{\alpha}{u}{
      \vectr{\xi}}}{\xi_{l}}\vectr{x}^{\alpha}_{,u}\gsf{\alpha}{u}
  \label{eqn:delxdelxiinterp}
\end{equation}

Substituting \eqnref{eqn:delxdelxiinterp} into \eqnref{eqn:avarclenconstraint}
and differentiating with respect to the optimisation variables gives the
constraint gradients: 
\begin{multline}
  \delby{c_{l}^{A}}{x^{B}_{j,v}}=\dfrac{1}{2}\left(\kronecker{B}{
      \fnof{\Delta}{\beta,\fnof{A_{\ominus}}{l}}}\gint{0}{1}{\dfrac{\delby{
          \gbfn{\beta}{v}{\vectr{\xi}}}{\xi_{l}}\gsf{\beta}{v}
        \delby{\fnof{x_{j}}{\vectr{\xi}}}{\xi_{l}}}{\norm{\delby{\fnof{
              \vectr{x}}{\vectr{\xi}}}{\xi_{l}}}}}{\fnof{\xi_{l}}{
        \fnof{A_{\ominus}}{l}}+\right.\\  
    \left.\kronecker{B}{\fnof{\Delta}{\beta,\fnof{A_{\oplus}}{l}}}\gint{0}{1}{
        \dfrac{\delby{\gbfn{\beta}{v}{\vectr{\xi}}}{\xi_{l}}
          \gsf{\beta}{v}\delby{\fnof{x_{j}}{
              \vectr{\xi}}}{\xi_{l}}}{\norm{\delby{\fnof{\vectr{x}}{
                \vectr{\xi}}}{\xi_{l}}}}}{\fnof{\xi_{l}}{
          \fnof{A_{\oplus}}{l}}}}\right) 
  \label{eqn:arclenjacnode}
\end{multline}
and
\begin{multline}  
  \delby{c_{l}^{A}}{\gsf{B}{v}}=\dfrac{1}{2}\left(\kronecker{B}{
      \fnof{\Delta}{\beta,\fnof{A_{\ominus}}{l}}}\gint{0}{1}{\dfrac{\delby{
          \gbfn{\beta}{v}{\vectr{\xi}}}{\xi_{l}}\dotprod{
            \vectr{x}^{\beta}_{,v}}{\delby{\fnof{\vectr{x}}{\vectr{\xi}}}{
            \xi_{l}}}}{\norm{\delby{\fnof{\vectr{x}}{\vectr{\xi}}}{\xi_{l}}}}}{
      \fnof{\xi_{l}}{\fnof{A_{\ominus}}{l}}}+\right. \\ 
  \left.\kronecker{B}{\fnof{\Delta}{\beta,\fnof{A_{\oplus}}{l}}}\gint{0}{1}{
      \dfrac{\delby{\gbfn{\beta}{v}{\vectr{\xi}}}{\xi_{l}}\dotprod{\vectr{x}^{
            \beta}_{,v}}{\delby{\fnof{\vectr{x}}{\vectr{\xi}}}
          {\xi_{l}}}}{\norm{\delby{\fnof{\vectr{x}}{\vectr{\xi}}}{\xi_{l}}}}}{
      \fnof{\xi_{l}}{\fnof{A_{\oplus}}{l}}}\right)-
  \kronecker{A}{B}\kronecker{l}{v} \label{eqn:arclenjacline}
\end{multline}
where $\beta$ ranges from one to the number of local nodes in the element.
Again when the infinitesimal in the integrals above contain
$\fnof{A_{\ominus}}{l}$ ($\fnof{A_{\oplus}}{l}$) the $\xi_{l}$ variable is
assumed to be taken over the previous (next) element and the integrals are
thus always one-dimensional. In the above formulae the Kronecker delta is used to
extract out the local node and derivative that corresponds to the global
degree-of-freedom which the gradient is taken with respect to.

\subsection{Sobolev smoothing for non-linear fitting}

To implement Sobolev smoothing in the form of the non-linear optimisation problem
given in \eqnref{eqn:nloproblem} the residual vector is modified by defining
an additional residual equal to the Sobolev value for the mesh being fitted \ie
\begin{equation}
  \fnof{f^{D+1}}{\vectr{u},\vectr{\xi}}=\fnof{G}{\fnof{\vectr{u}}{\vectr{\xi}}}
  \label{eqn:addSobresidual}
\end{equation}
where $\fnof{G}{\fnof{\vectr{u}}{\vectr{\xi}}}$ is defined in
\eqnref{eqn:Sobolevvalue}.  The objective function to minimise now becomes
\begin{equation}
  \fnof{F}{\vectr{u}}=\fnof{\mathcal{F}}{\vectr{u}}+\fnof{G^{2}}{\vectr{u}}=
  \fnof{\mathcal{F}}{\vectr{u}}+\fnof{\mathcal{G}}{\vectr{u}}
\end{equation}
where $\fnof{\mathcal{F}}{\vectr{u}}$ is the data objective defined in
\eqnref{eqn:nldataresidualfn} and
$\fnof{\mathcal{G}}{\vectr{u}}=\fnof{G^{2}}{\vectr{u}}$ is the \emph{Sobolev
objective}.

The gradients of this additional residual can then be found by differentiating
either \eqnref{eqn:2DptwoSobolevnorm} or \eqnref{eqn:3DptwoSobolevnorm} with
respect to the optimisation (mesh) parameters. For \twodal fitting these
gradients can be found by breaking the domain up into elements \ie
\begin{equation}
  \fnof{G}{\vectr{u}}=\dsum_{e}\gint{0}{1}{\pbrac{\tau\norm{\delby{
          \fnof{\vectr{x}}{\xi}}{\xi}}^{2}+\kappa\norm{\deltwosqby{\fnof{
            \vectr{x}}{\xi}}{\xi}}^{2}}\fnof{J}{\xi}}{\xi}
\end{equation}

Interpolating within an element and differentiating with respect to the
different optimisation parameters within that element gives the additional
residual gradients. For \twods these are
\begin{align}
  \delby{f^{D+1}}{x^{\beta}_{j,v}}&=2\gint{0}{1}{\pbrac{\tau
      \delby{\gbfn{\beta}{v}{\xi}}{\xi}\gsf{\beta}{v}
      \delby{\fnof{x_{j}}{\xi}}{\xi}+\kappa\deltwosqby{\gbfn{\beta}{v}{\xi}}{\xi}
      \gsf{\pbrac{\beta}}{\pbrac{v}}
      \deltwosqby{\fnof{x_{j}}{\xi}}{\xi}}\fnof{J}{\xi}}{\xi} 
  \label{eqn:2Dsobjacnode} \\
  \delby{f^{D+1}}{\gsf{\beta}{v}}&=2\gint{0}{1}{\pbrac{\tau\delby{
        \gbfn{\beta}{v}{\xi}}{\xi}\dotprod{\vectr{x}^{\beta}_{,v}}{
        \delby{\fnof{\vectr{x}}{\xi}}{\xi}}+\kappa\deltwosqby{
        \gbfn{\beta}{v}{\xi}}{\xi}\dotprod{\vectr{x}^{\beta}_{,v}}{
        \deltwosqby{\fnof{\vectr{x}}{\xi}}{\xi}}}\fnof{J}{\xi}}{\xi}
  \label{eqn:2Dsobjacline}
\end{align}

In \threeds the additional residual gradients become
\begin{multline}
  \delby{f^{D+1}}{x^{\beta}_{j,v}}=2\dintl{\vectr{0}}{\vectr{1}}\left(\tau
    \pbrac{\delby{\gbfn{\beta}{v}{\vectr{\xi}}}{\xione}\gsf{\beta}{
        v}\delby{\fnof{x_{j}}{\vectr{\xi}}}{\xione}+\delby{
        \gbfn{\beta}{v}{\vectr{\xi}}}{\xitwo}\gsf{\beta}{v}
      \delby{\fnof{x_{j}}{\vectr{\xi}}}{\xitwo}}+\right. \\
  \left.\kappa\pbrac{\deltwosqby{\gbfn{\beta}{v}{\vectr{\xi}}}{\xione}
      \gsf{\beta}{v}\deltwosqby{\fnof{x_{j}}{\vectr{\xi}}}{\xione}
      +2\deltwoby{\gbfn{\beta}{v}{\vectr{\xi}}}{\xione}{\xitwo}
      \gsf{\beta}{v}\deltwoby{\fnof{x_{j}}{\vectr{\xi}}}{
        \xione}{\xitwo}+\deltwosqby{\gbfn{\beta}{v}{\vectr{\xi}}}{\xitwo}
      \gsf{\beta}{v}\deltwosqby{\fnof{x_{j}}{\vectr{\xi}}}{\xitwo}}
  \right)\abs{\fnof{\matr{J}}{\vectr{\xi}}}\,d\vectr{\xi}
\end{multline}
and
\begin{multline}
  \delby{f^{D+1}}{\gsf{\beta}{v}}=2\dintl{\vectr{0}}{\vectr{1}}\left(\tau
    \pbrac{\delby{\gbfn{\beta}{v}{\vectr{\xi}}}{\xione}\dotprod{\vectr{x}^{
          \beta}_{,v}}{\delby{\fnof{\vectr{x}}{\vectr{\xi}}}{\xione}}+\delby{\gbfn{
          \beta}{v}{\vectr{\xi}}}{\xitwo}\dotprod{\vectr{x}^{\beta}_{,v}}{
        \delby{\fnof{\vectr{x}}{\vectr{\xi}}}{\xitwo}}}+\right. \\
  \left.\kappa\pbrac{\deltwosqby{\gbfn{\beta}{v}{\vectr{\xi}}}{\xione}
      \dotprod{\vectr{x}^{\beta}_{,v}}{\deltwosqby{\fnof{\vectr{x}}{
            \vectr{\xi}}}{\xione}}+2\deltwoby{\gbfn{\beta}{v}{\vectr{\xi}}}{
        \xione}{\xitwo}\dotprod{\vectr{x}^{\beta}_{,v}}{\deltwoby{
          \fnof{\vectr{x}}{\vectr{\xi}}}{\xione}{\xitwo}}+\deltwosqby{
        \gbfn{\beta}{v}{\vectr{\xi}}}{\xitwo}\dotprod{\vectr{x}^{\beta}_{,v}}{
        \deltwosqby{\fnof{\vectr{x}}{\vectr{\xi}}}{\xitwo}}}\right)\abs{\fnof{\matr{J}}{\vectr{\xi}}}
  \,d\vectr{\xi}
\end{multline}
 
Note that the above integrals are for a single element. To obtain the complete
residual gradients with respect to the optimisation parameters the
contributions from each element need to be summed.

