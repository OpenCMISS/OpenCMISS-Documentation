\section{Burgers's Equations}

\subsection{Generalised Burgers's Equation}

\subsubsection{Governing equations:}

For a given velocity $\fnof{u}{x,t}$ and a kinematic viscosity of $\nu$, Burgers's equation
in one dimension is
\begin{equation}
  \delby{u}{t}+u\delby{u}{x}=\nu\deltwosqby{u}{x}
  \label{eqn:BurgersEquation}
\end{equation}

The general form of this equation in \OpenCMISS is
\begin{equation}
  \addtolength{\fboxsep}{5pt}
  \boxed{
    \fnof{a}{\vectr{x}}\delby{\fnof{\vectr{u}}{\vectr{x},t}}{t}+\fnof{b}{\vectr{x}}\laplacian{}{\fnof{\vectr{u}}{\vectr{x},t}}+
    \fnof{c}{\vectr{x}}\dotprod{\fnof{\vectr{u}}{\vectr{x},t}}{\gradient{}{\fnof{\vectr{u}}{\vectr{x},t}}}=\vectr{0}
  }
  \label{eqn:GeneralFormBurgersEquation}
\end{equation}
where $\fnof{\vectr{u}}{\vectr{x},t}$ is the velocity vector and
$\fnof{a}{\vectr{x}}$, $\fnof{b}{\vectr{x}}$ and $\fnof{c}{\vectr{x}}$ are
material parameters. The standard form of Burgers's equation can be found with
$a=1$, $b=-\nu$ and $c=1$.

Appropriate boundary conditions conditions for Burgers's
equation are specification of Dirichlet boundary conditions on the solution,
$\vectr{d}$ \ie
\begin{equation}
  \fnof{\vectr{u}}{\vectr{x},t} = \fnof{\vectr{d}}{\vectr{x},t} \quad \vectr{x}\in\Gamma_{D},
  \label{eqn:BurgersDirichletBC} 
\end{equation}
and/or Neumann conditions in terms of the solution flux in the normal
direction, $\vectr{e}$ \ie
\begin{equation}
  \fnof{\vectr{q}}{\vectr{x},t} = \dotprod{\pbrac{\fnof{b}{\vectr{x}}
      \gradient{}{\fnof{\vectr{u}}{\vectr{x},t}}}}{\normal} =
  \fnof{\vectr{e}}{\vectr{x},t} \quad \vectr{x}\in\Gamma_{N},
  \label{eqn:BurgersNeumannBC} 
\end{equation}
where $\fnof{\vectr{q}}{\vectr{x},t}$ is the flux in the normal direction, $\normal$ is the normal
vector to the boudary and $\Gamma=\union{\Gamma_{D}}{\Gamma_{N}}$.

Appropriate initial conditions for the diffusion equation are the
specification of an initial value of the solution, $\vectr{f}$ \ie
\begin{equation}
  \fnof{\vectr{u}}{\vectr{x},0} = \fnof{\vectr{f}}{\vectr{x}} \quad \vectr{x}\in\Omega.
  \label{eqn:BurgersInitialC} 
\end{equation}

\subsubsection{Weak formulation:}

The corresponding weak form of \eqnref{eqn:GeneralFormBurgersEquation} can be
found by integrating over the domain with test functions \ie
\begin{equation}
  \gint{\Omega}{}{\pbrac{a\delby{\vectr{u}}{t}+
      b\laplacian{}{\vectr{u}}+
      c\dotprod{\vectr{u}}{\gradient{}{\vectr{u}}}}\vectr{w}}{\Omega}=\vectr{0}
  \label{eqn:Burgersweakform1}
\end{equation}
where $\vectr{w}$ are suitable spatial test functions.

Applying the divergence theorem in \eqnref{eqn:DivergenceTheormScalarVector} to \eqnref{eqn:Burgersweakform1} gives
\begin{equation}
  \gint{\Omega}{}{\pbrac{a\delby{\vectr{u}}{t}}\vectr{w}}{\Omega}-
      \gint{\Omega}{}{b\dotprod{\gradient{}{\vectr{u}}}{\gradient{}{\vectr{w}}}}{\Omega}
      +\gint{\Gamma}{}{\pbrac{b\dotprod{\gradient{}{\vectr{u}}}{\normal}}\vectr{w}}{\Gamma}+
      \gint{\Omega}{}{\pbrac{c\dotprod{\vectr{u}}{\gradient{}{\vectr{u}}}}\vectr{w}}{\Omega}
      =\vectr{0}
  \label{eqn:Burgersweakform2}
\end{equation}

\subsubsection{Tensor notation:}

\Eqnref{eqn:Burgersweakform2} can be written in tensor notation as
\begin{equation}
  \gint{\Omega}{}{a\dot{u}_{i}w_{i}}{\Omega}-
  \gint{\Omega}{}{bG^{jk}\covarderiv{u_{i}}{j}\covarderiv{w_{i}}{k}}{\Omega}+
  \gint{\Gamma}{}{bG^{jk}\covarderiv{u_{i}}{k}n_{j}w_{i}}{\Gamma} +
  \gint{\Omega}{}{cG^{jk}u_{j}\covarderiv{u_{i}}{k}w_{i}}{\Omega}=\vectr{0}
  \label{eqn:Burgerstensorform1}
\end{equation}
or
\begin{equation}
  \gint{\Omega}{}{a\dot{u}_{i}w_{i}}{\Omega}-
  \gint{\Omega}{}{bG^{jk}\covarderiv{u_{i}}{j}\covarderiv{w_{i}}{k}}{\Omega}+
  \gint{\Gamma}{}{q_{i}w_{i}}{\Gamma} +
  \gint{\Omega}{}{cG^{jk}u_{j}\covarderiv{u_{i}}{k}w_{i}}{\Omega}=\vectr{0}
  \label{eqn:Burgerstensorform2}
\end{equation}
or
\begin{multline}
  \gint{\Omega}{}{a\dot{u}_{i}w_{i}}{\Omega}-
  \gint{\Omega}{}{bG^{jk}\pbrac{\partialderiv{u_{i}}{j}-\christoffel{i}{j}{h}u_{h}}
    \pbrac{\partialderiv{w_{i}}{k}-\christoffel{i}{k}{l}w_{l}}}{\Omega}+ \\
  \gint{\Gamma}{}{q_{i}w_{i}}{\Gamma} +
  \gint{\Omega}{}{cG^{jk}u_{j}\pbrac{\partialderiv{u_{i}}{k}-
      \christoffel{i}{k}{h}u_{h}}w_{i}}{\Omega}=\vectr{0}
  \label{eqn:Burgerstensorform3}
\end{multline}
where $G^{jk}$ is the contravariant metric tensor and $\christoffel{i}{j}{k}$
is the Christoffel symbol for the spatial coordinates.

\subsubsection{Finite element formulation:}

We can now discretise the domain into finite elements \ie $\Omega=
\displaystyle{\bigcup_{e=1}^{E}}\Omega_{e}$ with
$\Gamma=\displaystyle{\bigcup_{f=1}^{F}}\Gamma_{f}$. \Eqnref{eqn:Burgerstensorform2}
now becomes
\begin{multline}
  \dsum_{e=1}^{E}\gint{\Omega_{e}}{}{a\dot{u}_{i}w_{i}}{\Omega}-
  \dsum_{e=1}^{E}\gint{\Omega_{e}}{}{bG^{jk}\pbrac{\partialderiv{u_{i}}{j}-
      \christoffel{i}{j}{h}u_{h}} \pbrac{\partialderiv{w_{i}}{k}-
      \christoffel{i}{k}{l}w_{l}}}{\Omega}+ \\
  \dsum_{f=1}^{F}\gint{\Gamma_{f}}{}{q_{i}w_{i}}{\Gamma} +
  \dsum_{e=1}^{E}\gint{\Omega_{e}}{}{cG^{jk}u_{j}\pbrac{\partialderiv{u_{i}}{k}-
      \christoffel{i}{k}{h}u_{h}}w_{i}}{\Omega}=\vectr{0}
  \label{eqn:Burgersfemform1}
\end{multline}

If we assume that we are in rectangular cartesian coordinates then the
Christoffel symbols are all zero and
$G^{jk}=\contrakronecker{j}{k}$. \Eqnref{eqn:Burgersfemform1} thus becomes
\begin{equation}
  \dsum_{e=1}^{E}\gint{\Omega_{e}}{}{a\dot{u}_{i}w_{i}}{\Omega}-
  \dsum_{e=1}^{E}\gint{\Omega_{e}}{}{b\contrapartialderiv{u_{i}}{k}
    \partialderiv{w_{i}}{k}}{\Omega}+ 
  \dsum_{f=1}^{F}\gint{\Gamma_{f}}{}{q_{i}w_{i}}{\Gamma} +
  \dsum_{e=1}^{E}\gint{\Omega_{e}}{}{cu^{k}\partialderiv{u_{i}}{k}w_{i}}{\Omega}=\vectr{0}
  \label{eqn:Burgersfemform2}
\end{equation}

If we now assume that the dependent variable $\vectr{u}$ can be interpolated
separately in space and in time we can write
\begin{equation}
  \fnof{\vectr{u}}{\vectr{x},t}=\gbfn{n}{}{\vectr{x}}\fnof{\nodept{\vectr{u}}{n}}{t}
\end{equation}
or, in standard interpolation notation within an element,
\begin{equation}
  \fnof{u_{i}}{\vectr{\xi},t}=\idxgbfn{i}{n}{\beta}{\vectr{\xi}}
  \fnof{\idxnodedof{u}{i}{n}{\beta}}{t}\idxgsf{i}{n}{\beta}
\end{equation}
and
\begin{equation}
  \fnof{u^{i}}{\vectr{\xi},t}=G^{ij}\idxgbfn{j}{n}{\beta}{\vectr{\xi}}
  \fnof{\idxnodedof{u}{j}{n}{\beta}}{t}\idxgsf{j}{n}{\beta}
\end{equation}
where $\fnof{\idxnodedof{u}{i}{n}{\beta}}{t}$ are the time varying nodal
degrees-of-freedom for velocity component $i$, node $n$, global derivative $\beta$,
$\idxgbfn{i}{n}{\beta}{\vectr{\xi}}$ are the corresponding basis functions 
and $\idxgsf{i}{n}{\beta}$ are the scale factors. 

We can also interpolate the other variables in a similar manner \ie
\begin{equation}
  \begin{split}
    \fnof{q_{i}}{\vectr{\xi},t} &= \idxgbfn{i}{o}{\gamma}{\vectr{\xi}}
    \fnof{\idxnodedof{q}{i}{o}{\gamma}}{t}\idxgsf{i}{o}{\gamma} \\
    \fnof{a}{\vectr{\xi}} &=\gbfn{p}{\delta}{\vectr{\xi}}
    \nodedof{a}{p}{\delta}\gsf{p}{\delta} \\
    \fnof{b}{\vectr{\xi}} &=\gbfn{p}{\delta}{\vectr{\xi}}\nodedof{b}{p}{\delta}
    \gsf{p}{\delta} \\
    \fnof{c}{\vectr{\xi}}
    &=\gbfn{p}{\delta}{\vectr{\xi}}\nodedof{c}{p}{\delta}\gsf{p}{\delta}
  \end{split}
\end{equation}
where $\fnof{\nodedof{q_{i}}{o}{\gamma}}{t}$, $\nodedof{a}{p}{\delta}$,
$\nodedof{b}{p}{\delta}$ and $\nodedof{c}{p}{\delta}$ are the
nodal degrees-of-freedom for the variables.

For a Galerkin finite element formulation we also choose the spatial weighting
function $\vectr{w}$ to be equal to the basis fucntions \ie
\begin{equation}
  \fnof{w_{i}}{\vectr{\xi}}=\idxgbfn{i}{m}{\alpha}{\vectr{\xi}}\idxgsf{i}{m}{\alpha}
\end{equation}

\subsubsection{Spatial integration:}

Adopting the standard integration notation we can write the spatial
integration term in \eqnref{eqn:Burgersfemform} as
% \begin{multline}
%   \dsum_{e=1}^{E}\gint{\vectr{0}}{\vectr{1}}{\fnof{a}{\vectr{\xi}}
%     \delby{\pbrac{\idxgbfn{i}{n}{\beta}{\vectr{\xi}}
%         \fnof{\idxnodedof{u}{i}{n}{\beta}}{t}\idxgsf{i}{n}{\beta}}}{t}
%     \idxgbfn{i}{m}{\alpha}{\vectr{\xi}}\idxgsf{i}{m}{\alpha}
%     \abs{\fnof{\matr{J}}{\vectr{\xi}}}}{\vectr{\xi}}- \\
%   \dsum_{e=1}^{E}\gint{\vectr{0}}{\vectr{1}}\fnof{b}{\vectr{\xi}}
%   \delby{\pbrac{\idxgbfn{i}{n}{\beta}{\vectr{\xi}}
%       \fnof{\idxnodedof{u}{i}{n}{\beta}}{t}\idxgsf{i}{n}{\beta}}}{x_{k}} \\
%   \delby{\pbrac{\idxgbfn{i}{m}{\alpha}{\vectr{\xi}}\idxgsf{i}{m}{\alpha}}}{x^{k}}
%   \abs{\fnof{\matr{J}}{\vectr{\xi}}}d\vectr{\xi} + \\
%   \dsum_{f=1}^{F}\gint{\vectr{0}}{\vectr{1}}{\idxgbfn{i}{o}{\gamma}{\vectr{\xi}}
%     \fnof{\idxnodedof{q}{i}{o}{\gamma}}{t}\idxgsf{i}{o}{\gamma}
%     \idxgbfn{i}{m}{\alpha}{\vectr{\xi}}\idxgsf{i}{m}{\alpha}
%     \abs{\fnof{\matr{J}}{\vectr{\xi}}}}{\vectr{\xi}} + \\
%   \dsum_{e=1}^{E}\gint{\vectr{0}}{\vectr{1}}{\fnof{c}{\vectr{\xi}}
%   \idxgbfn{k}{n}{\beta}{\vectr{\xi}}\fnof{\idxnodedof{u}{j}{n}{\beta}}{t}\idxgsf{i}{n}{\beta}
%   \pbracr{\delby{\idxgbfn{i}{n}{\beta}{\vectr{\xi}}\fnof{\idxnodedof{u}{i}{n}{\beta}}{t}
%       \idxgsf{i}{n}{\beta}}{x^{k}}}- \\
%   \pbracl{\christoffel{i}{k}{h}\idxgbfn{h}{m}{\alpha}{\vectr{\xi}}
%     \fnof{\idxnodedof{u}{h}{n}{\beta}}{t}\idxgsf{h}{m}{\alpha}}
%   \idxgbfn{i}{m}{\alpha}{\vectr{\xi}}\idxgsf{i}{m}{\alpha}
%   \abs{\fnof{\matr{J}}{\vectr{\xi}}}d\vectr{\xi}=\vectr{0}
%   \label{eqn:Burgersfemform1}
% \end{multline}

% Rearranging \eqnref{eqn:Burgersfemform1} gives
% \begin{multline}
%   \dsum_{e=1}^{E}\gint{\vectr{0}}{\vectr{1}}{\fnof{a}{\vectr{\xi}}
%     \delby{\idxgbfn{i}{n}{\beta}{\vectr{\xi}}\fnof{\idxnodedof{u}{i}{n}{\beta}}{t}
%       \idxgsf{i}{n}{\beta}}{t}\idxgbfn{i}{m}{\alpha}{\vectr{\xi}}\idxgsf{i}{m}{\alpha}
%     \abs{\fnof{\matr{J}}{\vectr{\xi}}}}{\vectr{\xi}}- \\
%   \dsum_{e=1}^{E}\gint{\vectr{0}}{\vectr{1}}{\fnof{b}{\vectr{\xi}}G^{jk}
%     \delby{\pbrac{\idxgbfn{i}{n}{\beta}{\vectr{\xi}}
%         \fnof{\idxnodedof{u}{i}{n}{\beta}}{t}\idxgsf{i}{n}{\beta}}}{x^{j}}
%     \delby{\pbrac{\idxgbfn{i}{m}{\alpha}{\vectr{\xi}}\idxgsf{i}{m}{\alpha}}}{x^{k}}
%     \abs{\fnof{\matr{J}}{\vectr{\xi}}}}{\vectr{\xi}}- \\
%   \dsum_{e=1}^{E}\dintl{\vectr{0}}{\vectr{1}}\fnof{b}{\vectr{\xi}}G^{jk}
%   \pbracl{\delby{\pbrac{\idxgbfn{i}{n}{\beta}{\vectr{\xi}}
%         \fnof{\idxnodedof{u}{i}{n}{\beta}}{t}\idxgsf{i}{n}{\beta}}}{x^{j}}
%     \christoffel{i}{k}{l}\idxgbfn{l}{m}{\alpha}{\vectr{\xi}}
%     \fnof{\idxnodedof{u}{h}{n}{\beta}}{t}\idxgsf{l}{m}{\alpha}}+ \\
%   \delby{\pbrac{\idxgbfn{i}{m}{\alpha}{\vectr{\xi}}\idxgsf{i}{m}{\alpha}}}{x^{k}}
%   \christoffel{i}{j}{h}\idxgbfn{h}{n}{\beta}{\vectr{\xi}}
%   \fnof{\idxnodedof{u}{h}{n}{\beta}}{t}\idxgsf{h}{n}{\beta}- \\ 
%   \pbracl{\christoffel{i}{j}{h}\idxgbfn{h}{n}{\beta}{\vectr{\xi}}
%     \fnof{\idxnodedof{u}{h}{n}{\beta}}{t}\idxgsf{h}{n}{\beta}
%     \christoffel{i}{k}{l}\idxgbfn{l}{m}{\alpha}{\vectr{\xi}}
%     \fnof{\idxnodedof{u}{h}{n}{\beta}}{t}
%     \idxgsf{l}{m}{\alpha}}\abs{\fnof{\matr{J}}{\vectr{\xi}}}d\vectr{\xi} + \\ 
%   \dsum_{f=1}^{F}\gint{\vectr{0}}{\vectr{1}}{\idxgbfn{i}{o}{\gamma}{\vectr{\xi}}
%     \fnof{\idxnodedof{q}{i}{o}{\gamma}}{t}\idxgsf{i}{o}{\gamma}
%     \idxgbfn{i}{m}{\alpha}{\vectr{\xi}}\idxgsf{i}{m}{\alpha}
%     \abs{\fnof{\matr{J}}{\vectr{\xi}}}}{\vectr{\xi}} + \\
%   \dsum_{e=1}^{E}\dintl{\vectr{0}}{\vectr{1}}\fnof{c}{\vectr{\xi}}
%   G^{jk}\idxgbfn{j}{n}{\beta}{\vectr{\xi}}
%   \fnof{\idxnodedof{u}{j}{n}{\beta}}{t}\idxgsf{i}{n}{\beta}
%   \pbracr{\delby{\idxgbfn{i}{n}{\beta}{\vectr{\xi}}\fnof{\idxnodedof{u}{i}{n}{\beta}}{t}
%       \idxgsf{i}{n}{\beta}}{x^{k}}}- \\
%   \pbracl{\christoffel{i}{k}{h}\idxgbfn{h}{m}{\alpha}{\vectr{\xi}}
%     \fnof{\idxnodedof{u}{h}{n}{\beta}}{t}\idxgsf{h}{m}{\alpha}}
%   \idxgbfn{i}{m}{\alpha}{\vectr{\xi}}\idxgsf{i}{m}{\alpha}
%   \abs{\fnof{\matr{J}}{\vectr{\xi}}}d\vectr{\xi}=\vectr{0}
%   \label{eqn:Burgersfemform1}
% \end{multline}
