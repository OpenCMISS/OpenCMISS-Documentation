\section{Burgers's Equations}

\subsection{Governing equations:}

For a given velocity $\fnof{u}{x,t}$ and a kinematic viscosity of $\nu$, Burgers's equation
in one dimension is
\begin{equation}
  \delby{u}{t}+u\delby{u}{x}=\nu\deltwosqby{u}{x}
  \label{eqn:BurgersEquation}
\end{equation}

The general form of this equation, the \emph{Generalised Burger's Equation}, in \OpenCMISS is
\begin{equation}
  \addtolength{\fboxsep}{5pt}
  \boxed{
    \fnof{a}{\vectr{x}}\delby{\fnof{\vectr{u}}{\vectr{x},t}}{t}+\fnof{b}{\vectr{x}}\laplacian{}{\fnof{\vectr{u}}{\vectr{x},t}}+
    \fnof{c}{\vectr{x}}\dotprod{\fnof{\vectr{u}}{\vectr{x},t}}{\gradient{}{\fnof{\vectr{u}}{\vectr{x},t}}}=\vectr{0}
  }
  \label{eqn:GeneralFormBurgersEquation}
\end{equation}
where $\fnof{\vectr{u}}{\vectr{x},t}$ is the velocity vector and
$\fnof{a}{\vectr{x}}$, $\fnof{b}{\vectr{x}}$ and $\fnof{c}{\vectr{x}}$ are
material parameters.

From the general form \OpenCMISS also has defined the \emph{Standard
Burger's equation} which can be found from
\eqnref{eqn:GeneralFormBurgersEquation} with $a=1$, $b=-\nu$ and
$c=1$ \ie
\begin{equation}
  \addtolength{\fboxsep}{5pt}
  \boxed{
    \delby{\fnof{\vectr{u}}{\vectr{x},t}}{t}-\fnof{\nu}{\vectr{x}}\laplacian{}{\fnof{\vectr{u}}{\vectr{x},t}}+
    \dotprod{\fnof{\vectr{u}}{\vectr{x},t}}{\gradient{}{\fnof{\vectr{u}}{\vectr{x},t}}}=\vectr{0}
  }
  \label{eqn:StandardFormBurgersEquation}
\end{equation}

Other forms are the \emph{Static Burger's Equation} which can be found
with $a=0$, $b=-\nu$ and $c=1$ \ie
\begin{equation}
  \addtolength{\fboxsep}{5pt}
  \boxed{
    -\fnof{\nu}{\vectr{x}}\laplacian{}{\fnof{\vectr{u}}{\vectr{x},t}}+
    \dotprod{\fnof{\vectr{u}}{\vectr{x},t}}{\gradient{}{\fnof{\vectr{u}}{\vectr{x},t}}}=\vectr{0}
  }
  \label{eqn:StaticFormBurgersEquation}
\end{equation}
and the \emph{Inviscid Burger's Equation} which can be found
with $a=1$, $b=0$ and $c=1$ \ie
\begin{equation}
  \addtolength{\fboxsep}{5pt}
  \boxed{
    \delby{\fnof{\vectr{u}}{\vectr{x},t}}{t}+
    \dotprod{\fnof{\vectr{u}}{\vectr{x},t}}{\gradient{}{\fnof{\vectr{u}}{\vectr{x},t}}}=\vectr{0}
  }
  \label{eqn:InviscidFormBurgersEquation}
\end{equation}

Appropriate boundary conditions conditions for Burgers's
equation are specification of Dirichlet boundary conditions on the solution,
$\vectr{d}$ \ie
\begin{equation}
  \fnof{\vectr{u}}{\vectr{x},t} = \fnof{\vectr{d}}{\vectr{x},t} \quad \vectr{x}\in\Gamma_{D},
  \label{eqn:BurgersDirichletBC} 
\end{equation}
and/or Neumann conditions in terms of the solution flux in the normal
direction, $\vectr{e}$ \ie
\begin{equation}
  \fnof{\vectr{q}}{\vectr{x},t} = \dotprod{\pbrac{\fnof{b}{\vectr{x}}
      \gradient{}{\fnof{\vectr{u}}{\vectr{x},t}}}}{\normal} =
  \fnof{\vectr{e}}{\vectr{x},t} \quad \vectr{x}\in\Gamma_{N},
  \label{eqn:BurgersNeumannBC} 
\end{equation}
where $\fnof{\vectr{q}}{\vectr{x},t}$ is the flux in the normal direction, $\normal$ is the normal
vector to the boudary and $\Gamma=\union{\Gamma_{D}}{\Gamma_{N}}$.

Appropriate initial conditions for the diffusion equation are the
specification of an initial value of the solution, $\vectr{f}$ \ie
\begin{equation}
  \fnof{\vectr{u}}{\vectr{x},0} = \fnof{\vectr{f}}{\vectr{x}} \quad \vectr{x}\in\Omega.
  \label{eqn:BurgersInitialC} 
\end{equation}

\subsection{Weak formulation:}

The corresponding weak form of \eqnref{eqn:GeneralFormBurgersEquation} can be
found by integrating over the domain with test functions \ie
\begin{equation}
  \gint{\Omega}{}{\pbrac{a\delby{\vectr{u}}{t}+
      b\laplacian{}{\vectr{u}}+
      c\dotprod{\vectr{u}}{\gradient{}{\vectr{u}}}}\vectr{w}}{\Omega}=\vectr{0}
  \label{eqn:Burgersweakform1}
\end{equation}
where $\vectr{w}$ are suitable spatial test functions.

Applying the divergence theorem in \eqnref{eqn:DivergenceTheormScalarVector} to \eqnref{eqn:Burgersweakform1} gives
\begin{equation}
  \gint{\Omega}{}{\pbrac{a\delby{\vectr{u}}{t}}\vectr{w}}{\Omega}-
      \gint{\Omega}{}{b\dotprod{\gradient{}{\vectr{u}}}{\gradient{}{\vectr{w}}}}{\Omega}
      +\gint{\Gamma}{}{\pbrac{b\dotprod{\gradient{}{\vectr{u}}}{\normal}}\vectr{w}}{\Gamma}+
      \gint{\Omega}{}{\pbrac{c\dotprod{\vectr{u}}{\gradient{}{\vectr{u}}}}\vectr{w}}{\Omega}
      =\vectr{0}
  \label{eqn:Burgersweakform2}
\end{equation}

\subsection{Tensor notation:}

\Eqnref{eqn:Burgersweakform2} can be written in tensor notation as
\begin{equation}
  \gint{\Omega}{}{a\dot{u}_{i}w_{i}}{\Omega}-
  \gint{\Omega}{}{bG^{jk}\covarderiv{u_{i}}{j}\covarderiv{w_{i}}{k}}{\Omega}+
  \gint{\Gamma}{}{bG^{jk}\covarderiv{u_{i}}{k}n_{j}w_{i}}{\Gamma} +
  \gint{\Omega}{}{cG^{jk}u_{j}\covarderiv{u_{i}}{k}w_{i}}{\Omega}=\vectr{0}
  \label{eqn:Burgerstensorform1}
\end{equation}
or
\begin{equation}
  \gint{\Omega}{}{a\dot{u}_{i}w_{i}}{\Omega}-
  \gint{\Omega}{}{bG^{jk}\covarderiv{u_{i}}{j}\covarderiv{w_{i}}{k}}{\Omega}+
  \gint{\Gamma}{}{q_{i}w_{i}}{\Gamma} +
  \gint{\Omega}{}{cG^{jk}u_{j}\covarderiv{u_{i}}{k}w_{i}}{\Omega}=\vectr{0}
  \label{eqn:Burgerstensorform2}
\end{equation}
or
\begin{multline}
  \gint{\Omega}{}{a\dot{u}_{i}w_{i}}{\Omega}-
  \gint{\Omega}{}{bG^{jk}\pbrac{\partialderiv{u_{i}}{j}-\christoffel{i}{j}{h}u_{h}}
    \pbrac{\partialderiv{w_{i}}{k}-\christoffel{i}{k}{l}w_{l}}}{\Omega}+ \\
  \gint{\Gamma}{}{q_{i}w_{i}}{\Gamma} +
  \gint{\Omega}{}{cG^{jk}u_{j}\pbrac{\partialderiv{u_{i}}{k}-
      \christoffel{i}{k}{h}u_{h}}w_{i}}{\Omega}=\vectr{0}
  \label{eqn:Burgerstensorform3}
\end{multline}
where $G^{jk}$ is the contravariant metric tensor and $\christoffel{i}{j}{k}$
is the Christoffel symbol for the spatial coordinates.

\subsection{Finite element formulation:}

We can now discretise the domain into finite elements \ie $\Omega=
\displaystyle{\bigcup_{e=1}^{E}}\Omega_{e}$ with
$\Gamma=\displaystyle{\bigcup_{f=1}^{F}}\Gamma_{f}$. \Eqnref{eqn:Burgerstensorform2}
now becomes
\begin{multline}
  \dsum_{e=1}^{E}\gint{\Omega_{e}}{}{a\dot{u}_{i}w_{i}}{\Omega}-
  \dsum_{e=1}^{E}\gint{\Omega_{e}}{}{bG^{jk}\pbrac{\partialderiv{u_{i}}{j}-
      \christoffel{i}{j}{h}u_{h}} \pbrac{\partialderiv{w_{i}}{k}-
      \christoffel{i}{k}{l}w_{l}}}{\Omega}+ \\
  \dsum_{f=1}^{F}\gint{\Gamma_{f}}{}{q_{i}w_{i}}{\Gamma} +
  \dsum_{e=1}^{E}\gint{\Omega_{e}}{}{cG^{jk}u_{j}\pbrac{\partialderiv{u_{i}}{k}-
      \christoffel{i}{k}{h}u_{h}}w_{i}}{\Omega}=\vectr{0}
  \label{eqn:Burgersfemform1}
\end{multline}

If we assume that we are in rectangular cartesian coordinates then the
Christoffel symbols are all zero and
$G^{jk}=\contrakronecker{j}{k}$. \Eqnref{eqn:Burgersfemform1} thus becomes
\begin{equation}
  \dsum_{e=1}^{E}\gint{\Omega_{e}}{}{a\dot{u}_{i}w_{i}}{\Omega}-
  \dsum_{e=1}^{E}\gint{\Omega_{e}}{}{b\contrapartialderiv{u_{i}}{k}
    \partialderiv{w_{i}}{k}}{\Omega}+ 
  \dsum_{f=1}^{F}\gint{\Gamma_{f}}{}{q_{i}w_{i}}{\Gamma} +
  \dsum_{e=1}^{E}\gint{\Omega_{e}}{}{cu^{k}\partialderiv{u_{i}}{k}w_{i}}{\Omega}=\vectr{0}
  \label{eqn:Burgersfemform2}
\end{equation}

If we now assume that the dependent variable $\vectr{u}$ can be interpolated
separately in space and in time we can write
\begin{equation}
  \fnof{\vectr{u}}{\vectr{x},t}=\gbfn{n}{}{\vectr{x}}\fnof{\nodept{\vectr{u}}{n}}{t}
\end{equation}
or, in standard interpolation notation within an element,
\begin{equation}
  \fnof{u_{i}}{\vectr{\xi},t}=\idxgbfn{i}{n}{\beta}{\vectr{\xi}}
  \fnof{\idxnodedof{u}{i}{n}{\beta}}{t}\idxgsf{i}{n}{\beta}
\end{equation}
and
\begin{equation}
  \fnof{u^{i}}{\vectr{\xi},t}=G^{ij}\idxgbfn{j}{n}{\beta}{\vectr{\xi}}
  \fnof{\idxnodedof{u}{j}{n}{\beta}}{t}\idxgsf{j}{n}{\beta}
\end{equation}
where $\fnof{\idxnodedof{u}{i}{n}{\beta}}{t}$ are the time varying nodal
degrees-of-freedom for velocity component $i$, node $n$, global derivative $\beta$,
$\idxgbfn{i}{n}{\beta}{\vectr{\xi}}$ are the corresponding basis functions 
and $\idxgsf{i}{n}{\beta}$ are the scale factors. 

We can also interpolate the other variables in a similar manner \ie
\begin{equation}
  \begin{split}
    \fnof{q_{i}}{\vectr{\xi},t} &= \idxgbfn{i}{o}{\gamma}{\vectr{\xi}}
    \fnof{\idxnodedof{q}{i}{o}{\gamma}}{t}\idxgsf{i}{o}{\gamma} \\
    \fnof{a}{\vectr{\xi}} &=\gbfn{p}{\delta}{\vectr{\xi}}
    \nodedof{a}{p}{\delta}\gsf{p}{\delta} \\
    \fnof{b}{\vectr{\xi}} &=\gbfn{p}{\delta}{\vectr{\xi}}\nodedof{b}{p}{\delta}
    \gsf{p}{\delta} \\
    \fnof{c}{\vectr{\xi}}
    &=\gbfn{p}{\delta}{\vectr{\xi}}\nodedof{c}{p}{\delta}\gsf{p}{\delta}
  \end{split}
\end{equation}
where $\fnof{\nodedof{q_{i}}{o}{\gamma}}{t}$, $\nodedof{a}{p}{\delta}$,
$\nodedof{b}{p}{\delta}$ and $\nodedof{c}{p}{\delta}$ are the
nodal degrees-of-freedom for the variables.

For a Galerkin finite element formulation we also choose the spatial weighting
function $\vectr{w}$ to be equal to the basis fucntions \ie
\begin{equation}
  \fnof{w_{i}}{\vectr{\xi}}=\idxgbfn{i}{m}{\alpha}{\vectr{\xi}}\idxgsf{i}{m}{\alpha}
\end{equation}

\subsection{Spatial integration:}

Adopting the standard integration notation we can write the spatial
integration term in \eqnref{eqn:Burgersfemform} as
% \begin{multline}
%   \dsum_{e=1}^{E}\gint{\vectr{0}}{\vectr{1}}{\fnof{a}{\vectr{\xi}}
%     \delby{\pbrac{\idxgbfn{i}{n}{\beta}{\vectr{\xi}}
%         \fnof{\idxnodedof{u}{i}{n}{\beta}}{t}\idxgsf{i}{n}{\beta}}}{t}
%     \idxgbfn{i}{m}{\alpha}{\vectr{\xi}}\idxgsf{i}{m}{\alpha}
%     \abs{\fnof{\matr{J}}{\vectr{\xi}}}}{\vectr{\xi}}- \\
%   \dsum_{e=1}^{E}\gint{\vectr{0}}{\vectr{1}}\fnof{b}{\vectr{\xi}}
%   \delby{\pbrac{\idxgbfn{i}{n}{\beta}{\vectr{\xi}}
%       \fnof{\idxnodedof{u}{i}{n}{\beta}}{t}\idxgsf{i}{n}{\beta}}}{x_{k}} \\
%   \delby{\pbrac{\idxgbfn{i}{m}{\alpha}{\vectr{\xi}}\idxgsf{i}{m}{\alpha}}}{x^{k}}
%   \abs{\fnof{\matr{J}}{\vectr{\xi}}}d\vectr{\xi} + \\
%   \dsum_{f=1}^{F}\gint{\vectr{0}}{\vectr{1}}{\idxgbfn{i}{o}{\gamma}{\vectr{\xi}}
%     \fnof{\idxnodedof{q}{i}{o}{\gamma}}{t}\idxgsf{i}{o}{\gamma}
%     \idxgbfn{i}{m}{\alpha}{\vectr{\xi}}\idxgsf{i}{m}{\alpha}
%     \abs{\fnof{\matr{J}}{\vectr{\xi}}}}{\vectr{\xi}} + \\
%   \dsum_{e=1}^{E}\gint{\vectr{0}}{\vectr{1}}{\fnof{c}{\vectr{\xi}}
%   \idxgbfn{k}{n}{\beta}{\vectr{\xi}}\fnof{\idxnodedof{u}{j}{n}{\beta}}{t}\idxgsf{i}{n}{\beta}
%   \pbracr{\delby{\idxgbfn{i}{n}{\beta}{\vectr{\xi}}\fnof{\idxnodedof{u}{i}{n}{\beta}}{t}
%       \idxgsf{i}{n}{\beta}}{x^{k}}}- \\
%   \pbracl{\christoffel{i}{k}{h}\idxgbfn{h}{m}{\alpha}{\vectr{\xi}}
%     \fnof{\idxnodedof{u}{h}{n}{\beta}}{t}\idxgsf{h}{m}{\alpha}}
%   \idxgbfn{i}{m}{\alpha}{\vectr{\xi}}\idxgsf{i}{m}{\alpha}
%   \abs{\fnof{\matr{J}}{\vectr{\xi}}}d\vectr{\xi}=\vectr{0}
%   \label{eqn:Burgersfemform1}
% \end{multline}

% Rearranging \eqnref{eqn:Burgersfemform1} gives
% \begin{multline}
%   \dsum_{e=1}^{E}\gint{\vectr{0}}{\vectr{1}}{\fnof{a}{\vectr{\xi}}
%     \delby{\idxgbfn{i}{n}{\beta}{\vectr{\xi}}\fnof{\idxnodedof{u}{i}{n}{\beta}}{t}
%       \idxgsf{i}{n}{\beta}}{t}\idxgbfn{i}{m}{\alpha}{\vectr{\xi}}\idxgsf{i}{m}{\alpha}
%     \abs{\fnof{\matr{J}}{\vectr{\xi}}}}{\vectr{\xi}}- \\
%   \dsum_{e=1}^{E}\gint{\vectr{0}}{\vectr{1}}{\fnof{b}{\vectr{\xi}}G^{jk}
%     \delby{\pbrac{\idxgbfn{i}{n}{\beta}{\vectr{\xi}}
%         \fnof{\idxnodedof{u}{i}{n}{\beta}}{t}\idxgsf{i}{n}{\beta}}}{x^{j}}
%     \delby{\pbrac{\idxgbfn{i}{m}{\alpha}{\vectr{\xi}}\idxgsf{i}{m}{\alpha}}}{x^{k}}
%     \abs{\fnof{\matr{J}}{\vectr{\xi}}}}{\vectr{\xi}}- \\
%   \dsum_{e=1}^{E}\dintl{\vectr{0}}{\vectr{1}}\fnof{b}{\vectr{\xi}}G^{jk}
%   \pbracl{\delby{\pbrac{\idxgbfn{i}{n}{\beta}{\vectr{\xi}}
%         \fnof{\idxnodedof{u}{i}{n}{\beta}}{t}\idxgsf{i}{n}{\beta}}}{x^{j}}
%     \christoffel{i}{k}{l}\idxgbfn{l}{m}{\alpha}{\vectr{\xi}}
%     \fnof{\idxnodedof{u}{h}{n}{\beta}}{t}\idxgsf{l}{m}{\alpha}}+ \\
%   \delby{\pbrac{\idxgbfn{i}{m}{\alpha}{\vectr{\xi}}\idxgsf{i}{m}{\alpha}}}{x^{k}}
%   \christoffel{i}{j}{h}\idxgbfn{h}{n}{\beta}{\vectr{\xi}}
%   \fnof{\idxnodedof{u}{h}{n}{\beta}}{t}\idxgsf{h}{n}{\beta}- \\ 
%   \pbracl{\christoffel{i}{j}{h}\idxgbfn{h}{n}{\beta}{\vectr{\xi}}
%     \fnof{\idxnodedof{u}{h}{n}{\beta}}{t}\idxgsf{h}{n}{\beta}
%     \christoffel{i}{k}{l}\idxgbfn{l}{m}{\alpha}{\vectr{\xi}}
%     \fnof{\idxnodedof{u}{h}{n}{\beta}}{t}
%     \idxgsf{l}{m}{\alpha}}\abs{\fnof{\matr{J}}{\vectr{\xi}}}d\vectr{\xi} + \\ 
%   \dsum_{f=1}^{F}\gint{\vectr{0}}{\vectr{1}}{\idxgbfn{i}{o}{\gamma}{\vectr{\xi}}
%     \fnof{\idxnodedof{q}{i}{o}{\gamma}}{t}\idxgsf{i}{o}{\gamma}
%     \idxgbfn{i}{m}{\alpha}{\vectr{\xi}}\idxgsf{i}{m}{\alpha}
%     \abs{\fnof{\matr{J}}{\vectr{\xi}}}}{\vectr{\xi}} + \\
%   \dsum_{e=1}^{E}\dintl{\vectr{0}}{\vectr{1}}\fnof{c}{\vectr{\xi}}
%   G^{jk}\idxgbfn{j}{n}{\beta}{\vectr{\xi}}
%   \fnof{\idxnodedof{u}{j}{n}{\beta}}{t}\idxgsf{i}{n}{\beta}
%   \pbracr{\delby{\idxgbfn{i}{n}{\beta}{\vectr{\xi}}\fnof{\idxnodedof{u}{i}{n}{\beta}}{t}
%       \idxgsf{i}{n}{\beta}}{x^{k}}}- \\
%   \pbracl{\christoffel{i}{k}{h}\idxgbfn{h}{m}{\alpha}{\vectr{\xi}}
%     \fnof{\idxnodedof{u}{h}{n}{\beta}}{t}\idxgsf{h}{m}{\alpha}}
%   \idxgbfn{i}{m}{\alpha}{\vectr{\xi}}\idxgsf{i}{m}{\alpha}
%   \abs{\fnof{\matr{J}}{\vectr{\xi}}}d\vectr{\xi}=\vectr{0}
%   \label{eqn:Burgersfemform1}
% \end{multline}

\subsection{Analytic Solutions}

\subsubsection{One Dimensional Analytic Function 1}

The \oned Burgers' equation in \OpenCMISS form can be written as
\begin{equation}
  a\delby{u}{t}+b\deltwosqby{u}{x}+cu\delby{u}{x}=0
  \label{eqn:OpenCMISSBurgerEquationOneDim}
\end{equation}

Adapted from \urllink{http://eqworld.ipmnet.ru/en/solutions/npde/npde1301.pdf} we
have for the \oned Burgers' equation the analytic solution
\begin{equation}
  \fnof{u}{x,t}=\dfrac{A+ax}{B+ct}
  \label{eqn:AnalyticBurgersEquationOneDim1}
\end{equation}

To prove the analytic solution we differentiate \eqnref{eqn:AnalyticBurgersEquationOneDim1} to give
\begin{align}
  \delby{u}{t} &= \dfrac{-c\pbrac{A+ax}}{\pbrac{B+ct}^{2}} \\
  \delby{u}{x} &= \dfrac{a}{\pbrac{B+ct}} \\
  \deltwosqby{u}{x} &= 0 \\
  u\delby{u}{x} &= \dfrac{a\pbrac{A+ax}}{\pbrac{B+ct}^{2}} \\
  \label{eqn:AnalyticBurgersEquationOneDim1Derivatives}
\end{align}

Substiting \eqnref{eqn:AnalyticBurgersEquationOneDim1Derivatives} into \eqnref{eqn:OpenCMISSBurgerEquationOneDim} gives
\begin{align}
  a\delby{u}{t}+b\deltwosqby{u}{x}+cu\delby{u}{x} &=
  \dfrac{-ac\pbrac{A+ax}}{\pbrac{B+ct}^{2}}+b.0+\dfrac{ac\pbrac{A+ax}}{\pbrac{B+ct}^{2}}\\
  &= 0
\end{align}

The analytic field component definitions in \OpenCMISS are shown in \tabref{tab:OpenCMISSAnalyticFieldBurgersEquationOneDim1}.

\begin{table}[htb] \centering
  \begin{tabular}{|c|c|} \hline
    \emph{Analytic constant} & \emph{Analytic field component} \\ \hline \hline
    $A$ & $1$ \\ 
    $B$ & $2$ \\  \hline
  \end{tabular}
  \caption{\OpenCMISS analytic field components for the \oned diffusion equation
    analytic function 1.}
  \label{tab:OpenCMISSAnalyticFieldBurgersEquationOneDim1}
\end{table}

\subsubsection{One Dimensional Analytic Function 2}

Adapted from \urllink{http://eqworld.ipmnet.ru/en/solutions/npde/npde1301.pdf} we
have for the \oned Burgers's equation the analytic solution
\begin{equation}
  \fnof{u}{x,t}=aA+\dfrac{2b}{c\pbrac{x-cAt+B}}
  \label{eqn:AnalyticBurgersEquationOneDim2}
\end{equation}

To prove the analytic solution we differentiate \eqnref{eqn:AnalyticBurgersEquationOneDim1} to give
\begin{align}
  \delby{u}{t} &= \dfrac{2bcA}{c\pbrac{x-cAt+B}^{2}} \\
  \delby{u}{x} &= \dfrac{-2b}{c\pbrac{x-cAt+B}^{2}} \\
  \deltwosqby{u}{x} &= \dfrac{4b}{c\pbrac{x-cAt+B}^{3}} \\
  u\delby{u}{x} &= \dfrac{-2abA}{c\pbrac{x-cAt+B}^{2}} - \dfrac{4b^{2}}{c^{2}\pbrac{x-cAt+B}^{3}}  \\
  \label{eqn:AnalyticBurgersEquationOneDim2Derivatives}
\end{align}

Substiting \eqnref{eqn:AnalyticBurgersEquationOneDim2Derivatives} into \eqnref{eqn:OpenCMISSBurgerEquationOneDim} gives
\begin{align}
  a\delby{u}{t}+b\deltwosqby{u}{x}+cu\delby{u}{x} &=
  \dfrac{2abcA}{c\pbrac{x-cAt+B}^{2}}+\dfrac{4b^{2}}{c\pbrac{x-cAt+B}^{3}}- \\
  & \qquad \dfrac{-2abcA}{c\pbrac{x-cAt+B}^{2}}-\dfrac{4b^{2}c}{c^{2}\pbrac{x-cAt+B}^{3}} \\
  &=\dfrac{2abcA-2abcA}{c\pbrac{x-cAt+B}^{2}}+\dfrac{4b^{2}-4b^{2}}{c\pbrac{x-cAt+B}^{3}}\\
  &= 0
\end{align}

The analytic field component definitions in \OpenCMISS are shown in \tabref{tab:OpenCMISSAnalyticFieldBurgersEquationOneDim2}.

\begin{table}[htb] \centering
  \begin{tabular}{|c|c|} \hline
    \emph{Analytic constant} & \emph{Analytic field component} \\ \hline \hline
    $A$ & $1$ \\ 
    $B$ & $2$ \\  \hline
  \end{tabular}
  \caption{\OpenCMISS analytic field components for the \oned Burgers' equation
    analytic function 2.}
  \label{tab:OpenCMISSAnalyticFieldBurgersEquationOneDim2}
\end{table}

\subsection{Worked example}

Consider the \onedal generalised Burger's equation
\begin{equation}
  a\delby{\fnof{u}{x,t}}{t}+b\deltwosqby{\fnof{u}{x,t}}{x}+c\fnof{u}{x,t}\delby{\fnof{u}{x,t}}{x}=0
  \label{eqn:onedgenburgersexample}
\end{equation}
on the \onedal domain of length $L$ andwith $E$ elements and $N=E+1$ nodes as shown in \figref{fig:Burgers1DDomain}.

\epstexfigure{FluidMechanics/svgs/burgersoneddomain.eps_tex}{One
  dimensional domain for the generalised Burger's equation.}{One
  dimensional domain for the generalised Burger's equation. The domain
  of length $L$ has $E$ elements and $N=E+1$ nodes. The length of each
  element is $L/E$.}{fig:Burgers1DDomain}{1.0}

The weighted residual form of \eqnref{eqn:onedgenburgersexample} is given by
\begin{equation}
  \gint{\Omega}{}{\pbrac{a\delby{\fnof{u}{x,t}}{t}+b\deltwosqby{\fnof{u}{x,t}}{x}+
      c\fnof{u}{x,t}\delby{\fnof{u}{x,t}}{x}}\fnof{w}{x}}{\Omega}=0
\end{equation}

Applying the divergence theorem gives
\begin{equation}
  \begin{split}
    \gint{\Omega}{}{\pbrac{a\delby{\fnof{u}{x,t}}{t}\fnof{w}{x}-b\delby{\fnof{u}{x,t}}{x}\delby{\fnof{w}{x}}{x}+
        c\fnof{u}{x,t}\delby{\fnof{u}{x,t}}{x}\fnof{w}{x}}}{\Omega}\\
    +\gint{\Gamma}{}{\pbrac{\dotprod{b\delby{\fnof{u}{x,t}}{x}}{n}}\fnof{w}{x}}{\Gamma}=0
  \end{split}
\end{equation}
or
\begin{equation}
  \begin{split}
    \gsum{e=1}{E}{\gint{0}{1}{\pbrac{a\delby{\fnof{u}{\xi,t}}{t}\fnof{w}{\xi}
          -b\delby{\fnof{u}{\xi,t}}{\xi}\delby{\xi}{x}\delby{\fnof{w}{\xi}}{\xi}\delby{\xi}{x}
          +c\fnof{u}{\xi,t}\delby{\fnof{u}{\xi,t}}{\xi}\delby{\xi}{x}\fnof{w}{\xi}}\abs{\fnof{J}{\xi}}}{\xi}} \\
    +\gsum{f=1}{F}{\gint{0}{1}{\pbrac{\dotprod{b\delby{\fnof{u}{\xi,t}}{\xi}\delby{\xi}{x}}{n}}\fnof{w}{\xi}
        \abs{\fnof{J}{\xi}}}{\xi}}=0
  \end{split}
\end{equation}

Upon integration the resultant system of equations can be written as
\begin{equation}
  \matr{C}\dotover{\vectr{u}}+\matr{K}\vectr{u}+\fnof{\vectr{g}}{\vectr{u}}=\vectr{f}
\end{equation}

Considering element $i$ we can derive expressions for the elemental
stiffness matrices. Assuming linear basis functions the interpolation
expression for $\fnof{u}{\xi}$ can be written as
\begin{equation}
  \fnof{u}{\xi}=\xi u_{i}+\pbrac{1-\xi}u_{i+1}
\end{equation}
also we have
\begin{align}
  \delby{x}{\xi}&=\dfrac{L}{E} \\
  \delby{\xi}{x}&=\dfrac{E}{L} \\
  \abs{\fnof{J}{\xi}}&=\dfrac{L}{E}
\end{align}

The element damping matrix can be written as
\begin{align}
  C_{i,i}&=\gint{0}{1}{a\delby{\pbrac{\xi \dotover{u}_{i}+\pbrac{1-\xi}\dotover{u}_{i+1}}}{t}\xi\dfrac{L}{E}}{\xi} \\
  &=\dfrac{aL}{E}\gint{0}{1}{\pbrac{\xi\dotover{u}_{i}+\pbrac{1-\xi}\dotover{u}_{i+1}}\xi}{\xi} \\
  &=\dfrac{aL}{E}\gint{0}{1}{\pbrac{\xi^{2}\dotover{u}_{i}+\pbrac{\xi-\xi^{2}}\dotover{u}_{i+1}}}{\xi} \\
  &=\dfrac{aL}{E}\sqbrac{\dfrac{\xi^{3}}{3}\dotover{u}_{i}+\pbrac{\dfrac{\xi^{2}}{2}-\dfrac{\xi^{3}}{3}}\dotover{u}_{i+1}}_{0}^{1}\\
  &=\dfrac{aL}{E}\pbrac{\dfrac{\dotover{u}_{i}}{3}+\dfrac{\dotover{u}_{i+1}}{2}-\dfrac{\dotover{u}_{i+1}}{3}}\\
  &=\dfrac{aL\pbrac{2\dotover{u}_{i}+\dotover{u}_{i+1}}}{6E}  
\end{align}

Similarily
\begin{align}
  C_{i+1,i}&=\gint{0}{1}{a\delby{\pbrac{\xi \dotover{u}_{i}+\pbrac{1-\xi}\dotover{u}_{i+1}}}{t}\pbrac{1-\xi}\dfrac{L}{E}}{\xi} \\
  &=\dfrac{aL}{E}\gint{0}{1}{\pbrac{\xi\dotover{u}_{i}+\pbrac{1-\xi}\dotover{u}_{i+1}}\pbrac{1-\xi}}{\xi} \\
  &=\dfrac{aL}{E}\gint{0}{1}{\pbrac{\pbrac{\xi-\xi^{2}}\dotover{u}_{i}+\pbrac{1-2\xi+\xi^{2}}\dotover{u}_{i+1}}}{\xi} \\
  &=\dfrac{aL}{E}\sqbrac{\pbrac{\dfrac{\xi^{2}}{2}-\dfrac{\xi^{3}}{3}}\dotover{u}_{i}+\pbrac{\xi-\xi^{2}+
      \dfrac{\xi^{3}}{3}}\dotover{u}_{i+1}}_{0}^{1}\\
  &=\dfrac{aL}{E}\pbrac{\dfrac{\dotover{u}_{i}}{2}-\dfrac{\dotover{u}_{i}}{3}+\dotover{u}_{i+1}-\dotover{u}_{i+1}+
    \dfrac{\dotover{u}_{i+1}}{3}}\\
  &=\dfrac{aL\pbrac{\dotover{u}_{i}+2\dotover{u}_{i+1}}}{6E}  
\end{align}

The element damping matrix for element $i$ is thus
\begin{equation}
  \begin{bmatrix}
    C_{i,i} & C_{i,i+1} \\
    C_{i+1,i} & C_{i+1,i+1}
  \end{bmatrix}\begin{bmatrix}
    \dotover{u}_{i} \\
    \dotover{u}_{i+1}
  \end{bmatrix} = \begin{bmatrix}
    \frac{aL}{3E} & \frac{aL}{6E} \\
    \frac{aL}{6E} & \frac{aL}{3E}
   \end{bmatrix}\begin{bmatrix}
    \dotover{u}_{i} \\
    \dotover{u}_{i+1}
  \end{bmatrix}
\end{equation}

The element stiffness matrix can be written as
\begin{align}
  K_{i,i}&=\gint{0}{1}{-b\delby{\pbrac{\xi u_{i}+\pbrac{1-\xi}u_{i+1}}}{\xi}\dfrac{E}{L}\delby{\xi}{\xi}\dfrac{E}{L}\dfrac{L}{E}}{\xi} \\
  &=\dfrac{-bE\pbrac{u_{i}-u_{i+1}}}{L}\gint{0}{1}{}{\xi}\\
  &=\dfrac{-bE\pbrac{u_{i}-u_{i+1}}}{L}\sqbrac{\xi}_{0}^{1}\\
  &=\dfrac{-bE\pbrac{u_{i}-u_{i+1}}}{L}
\end{align}

Similarily
\begin{align}
  K_{i+1,i}&=\gint{0}{1}{-b\delby{\pbrac{\xi u_{i}+\pbrac{1-\xi}u_{i+1}}}{\xi}\dfrac{E}{L}\delby{\pbrac{1-\xi}}{\xi}\dfrac{E}{L}\dfrac{L}{E}}{\xi} \\
  &=\dfrac{bE\pbrac{u_{i}-u_{i+1}}}{L}\gint{0}{1}{}{\xi}\\
  &=\dfrac{bE\pbrac{u_{i}-u_{i+1}}}{L}\sqbrac{\xi}_{0}^{1}\\
  &=\dfrac{bE\pbrac{u_{i}-u_{i+1}}}{L}
\end{align}


The element stiffness matrix for element $i$ is thus
\begin{equation}
  \begin{bmatrix}
    K_{i,i} & K_{i,i+1} \\
    K_{i+1,i} & K_{i+1,i+1}
  \end{bmatrix}\begin{bmatrix}
    u_{i} \\
    u_{i+1}
  \end{bmatrix} = \begin{bmatrix}
    \frac{-bE}{L} & \frac{bE}{L} \\
    \frac{bE}{L} & \frac{-bE}{L}
   \end{bmatrix}\begin{bmatrix}
    u_{i} \\
    u_{i+1}
  \end{bmatrix}
\end{equation}

The nonlinear vector can be written as
\begin{align}
  g_{i}&=\gint{0}{1}{c\pbrac{\xi u_{i}+\pbrac{1-\xi}u_{i+1}}\delby{\pbrac{\xi u_{i}+\pbrac{1-\xi}u_{i+1}}}{\xi}\dfrac{E}{L}\xi\dfrac{L}{E}}{\xi} \\
  &=c\gint{0}{1}{\pbrac{\xi u_{i}+\pbrac{1-\xi}u_{i+1}}\delby{\pbrac{\xi u_{i}+\pbrac{1-\xi}u_{i+1}}}{\xi}\xi}{\xi} \\
  &=c\pbrac{u_{i}-u_{i+1}}\gint{0}{1}{\pbrac{\xi^{2} u_{i}+\pbrac{\xi-\xi^{2}}u_{i+1}}}{\xi} \\
  &=c\pbrac{u_{i}-u_{i+1}}\sqbrac{\dfrac{\xi^{3}}{3}u_{i}+\dfrac{\xi^{2}}{2}u_{i+1} - \dfrac{\xi^{3}}{3}u_{i+1}}_{0}^{1}\\
  &=\dfrac{c\pbrac{u_{i}-u_{i+1}}}{6}\pbrac{2 u_{i}+u_{i+1}}\\
  &=\dfrac{c\pbrac{2 u_{i}^{2}-u_{i}u_{i+1}-u_{i+1}^{2}}}{6}
\end{align}

Similarily
\begin{align}
  g_{i+1}&=\gint{0}{1}{c\pbrac{\xi u_{i}+\pbrac{1-\xi}u_{i+1}}\delby{\pbrac{\xi u_{i}+\pbrac{1-\xi}u_{i+1}}}{\xi}\dfrac{E}{L}\pbrac{1-\xi}\dfrac{L}{E}}{\xi} \\
  &=c\gint{0}{1}{\pbrac{\xi u_{i}+\pbrac{1-\xi}u_{i+1}}\delby{\pbrac{\xi u_{i}+\pbrac{1-\xi}u_{i+1}}}{\xi}\pbrac{1-\xi}}{\xi} \\
  &=c\pbrac{u_{i}-u_{i+1}}\gint{0}{1}{\pbrac{\pbrac{\xi-\xi^{2}}u_{i}+\pbrac{1-2\xi+\xi^{2}}u_{i+1}}}{\xi} \\
  &=c\pbrac{u_{i}-u_{i+1}}\sqbrac{\dfrac{\xi^{2}}{2}u_{i}-\dfrac{\xi^{3}}{3}u_{i}+\xi u_{i+1}-\xi^{2}u_{i+1}+\dfrac{\xi^{3}}{3}u_{i+1}}_{0}^{1}\\
  &=\dfrac{c\pbrac{u_{i}-u_{i+1}}}{6}\pbrac{u_{i}+2u_{i+1}}\\
  &=\dfrac{c\pbrac{u_{i}^{2}+u_{i}u_{i+1}-2u_{i+1}^{2}}}{6}
\end{align}

The element residual vector for element $i$ is thus
\begin{equation}
  \begin{bmatrix}
    g_{i}  \\
    g_{i+1} 
  \end{bmatrix} = \begin{bmatrix}
    \frac{c\pbrac{2 u_{i}^{2}-u_{i}u_{i+1}-u_{i+1}^{2}}}{6} \\
    \frac{c\pbrac{u_{i}^{2}+u_{i}u_{i+1}-2u_{i+1}^{2}}}{6}
   \end{bmatrix}
\end{equation}

The element Jacobian matrix for element $i$ is thus
\begin{equation}
  \begin{bmatrix}
    J_{i,i} & J_{i,i+1} \\
    J_{i+1,i} & J_{i+1,i+1} 
  \end{bmatrix} = \begin{bmatrix}
    \delby{g_{i}}{u_{i}} & \delby{g_{i}}{u_{i+1}} \\
    \delby{g_{i+1}}{u_{i}} & \delby{g_{i+1}}{u_{i+1}}
   \end{bmatrix} = \begin{bmatrix}
    \frac{c\pbrac{4 u_{i}-u_{i+1}}}{6} & \frac{c\pbrac{-u_{i}-2 u_{i+1}}}{6} \\
    \frac{c\pbrac{2 u_{i}+u_{i+1}}}{6} & \frac{c\pbrac{u_{i}-4u_{i+1}}}{6}
   \end{bmatrix}
\end{equation}

The matrix system for the domain shown in \figref{fig:Burgers1DDomain} with no flux conditions is thus
\begin{multline}
  \begin{bmatrix}
    \frac{aL}{3E} & \frac{aL}{6E} & 0 & 0 & 0 & \cdots & 0 & 0 \\
    \frac{aL}{6E} & \frac{2aL}{6E} &  \frac{aL}{6E} & 0 & 0 & \cdots & 0 & 0 \\
    0 & \frac{aL}{6E} & \frac{2aL}{6E} &  \frac{aL}{6E} & 0 & \cdots & 0 & 0 \\
    0 & 0 & \frac{aL}{6E} & \frac{2aL}{6E} &  \frac{aL}{6E} & \cdots & 0 & 0 \\
    0 & 0 & 0 & \frac{aL}{6E} & \frac{2aL}{6E} & \cdots & 0 & 0 \\
    \vdots & \vdots & \vdots & \vdots & \vdots & \ddots & \vdots & \vdots \\
    0 & 0 & 0 & 0 & 0 & \cdots & \frac{2aL}{6E} & \frac{aL}{6E} \\
    0 & 0 & 0 & 0 & 0 & \cdots & \frac{aL}{6E} & \frac{2aL}{6E} \\
  \end{bmatrix} \begin{bmatrix}
    \dotover{u}_{1} \\
    \dotover{u}_{2} \\
    \dotover{u}_{3} \\
    \dotover{u}_{4} \\
    \dotover{u}_{5} \\
    \vdots \\
    \dotover{u}_{N-1} \\
    \dotover{u}_{N} \\
  \end{bmatrix} \\
  + \begin{bmatrix}
    \frac{-bE}{L} & \frac{bE}{L} & 0 & 0 & 0 & \cdots & 0 & 0 \\
    \frac{bE}{L} & \frac{-2bE}{L} &  \frac{bE}{L} & 0 & 0 & \cdots & 0 & 0 \\
    0 & \frac{bE}{L} & \frac{-2bE}{L} &  \frac{bE}{L} & 0 & \cdots & 0 & 0 \\
    0 & 0 & \frac{bE}{L} & \frac{-2bE}{L} &  \frac{bE}{L} & \cdots & 0 & 0 \\
    0 & 0 & 0 & \frac{bE}{L} & \frac{-2bE}{L} & \cdots & 0 & 0 \\
    \vdots & \vdots & \vdots & \vdots & \vdots & \ddots & \vdots & \vdots \\
    0 & 0 & 0 & 0 & 0 & \cdots & \frac{-2bE}{L} & \frac{bE}{L} \\
    0 & 0 & 0 & 0 & 0 & \cdots & \frac{bE}{L} & \frac{-bE}{L} \\
  \end{bmatrix} \begin{bmatrix}
    u_{1} \\
    u_{2} \\
    u_{3} \\
    u_{4} \\
    u_{5} \\
    \vdots \\
    u_{N-1} \\
    u_{N}
  \end{bmatrix} \\
  + \begin{bmatrix}
    \frac{c\pbrac{2 u_{1}^{2}-u_{1}u_{2}-u_{2}^{2}}}{6} \\
    \frac{c\pbrac{u_{1}^{2}+u_{1}u_{2}-u_{2}u_{3}-u_{3}^{2}}}{6} \\
    \frac{c\pbrac{u_{2}^{2}+u_{2}u_{3}-u_{3}u_{4}-u_{4}^{2}}}{6} \\
    \frac{c\pbrac{u_{3}^{2}+u_{3}u_{4}-u_{4}u_{5}-u_{5}^{2}}}{6} \\
    \frac{c\pbrac{u_{4}^{2}+u_{4}u_{5}-u_{5}u_{6}-u_{6}^{2}}}{6} \\
    \vdots \\
    \frac{c\pbrac{u_{N-2}^{2}+u_{N-2}u_{N-1}-u_{N-1}u_{N}-u_{N}^{2}}}{6} \\
    \frac{c\pbrac{u_{N-1}^{2}+u_{N-1}u_{N}-2u_{N}^{2}}}{6}    
  \end{bmatrix} = \begin{bmatrix}
    0 \\
    0 \\
    0 \\
    0 \\
    0 \\
    \vdots \\
    0 \\
    0 \\
  \end{bmatrix}
\end{multline}

And the Jacobian matrix is
\begin{equation}
  \begin{bmatrix}
    \frac{c\pbrac{4u_{1}-u_{2}}}{6} & \frac{c\pbrac{-u_{1}-2u_{2}}}{6} & 0 & 0 & 0 & \cdots & 0 & 0 \\
    \frac{c\pbrac{2u_{1}+u_{2}}}{6} & \frac{c\pbrac{u_{1}-u_{3}}}{6} &  \frac{c\pbrac{-u_{2}-2u_{3}}}{6} & 0 & 0 & \cdots & 0 & 0 \\
    0 & \frac{c\pbrac{2u_{2}+u_{3}}}{6} & \frac{c\pbrac{u_{2}-u_{4}}}{6} &  \frac{c\pbrac{-u_{3}-2u_{4}}}{6} & 0 & \cdots & 0 & 0 \\
    0 & 0 & \frac{c\pbrac{2u_{3}+u_{4}}}{6} & \frac{c\pbrac{u_{3}-u_{5}}}{6} &  \frac{c\pbrac{-u_{4}-2u_{5}}}{6} & \cdots & 0 & 0 \\
    0 & 0 & 0 & \frac{c\pbrac{2u_{4}+u_{5}}}{6} & \frac{c\pbrac{u_{4}-u_{6}}}{6} & \cdots & 0 & 0 \\
    \vdots & \vdots & \vdots & \vdots & \vdots & \ddots & \vdots & \vdots \\
    0 & 0 & 0 & 0 & 0 & \cdots & \frac{c\pbrac{u_{N-2}-u_{N}}}{6} & \frac{c\pbrac{-u_{N-1}-2u_{N}}}{6} \\
    0 & 0 & 0 & 0 & 0 & \cdots & \frac{c\pbrac{2u_{N-1}+u_{N}}}{6} & \frac{c\pbrac{u_{N-1}-4u_{N}}}{6}
  \end{bmatrix}
\end{equation}
