\section{Darcy Equation}

Darcy's equation describes the flow of fluid through a porous material. Ignoring inertial
and body forces, Darcy's equation is:

\begin{equation}
  \vectr{v} = -\frac{\tensor{K}}{\mu} \gradient{}{p}
\end{equation}

$\vectr{v}$ is the relative fluid flow vector, given by
$n \pbrac{\vectr{v}^f - \vectr{v}^s}$ where $n$ is the porosity, and
$\vectr{v}^f$ and $\vectr{v}^s$ are the Eulerian fluid and solid component
velocities.  $\tensor{K}$ is the permeability tensor, $\mu$ is viscosity and
$p$ is the fluid pressure.

Conservation of mass also gives:

\begin{equation}
  \divergence{}{v} = 0
\end{equation}

The weighted residual forms of these equations over a region $\Omega$ with surface $\Gamma$ are:

\begin{equation}
  \gint{\Omega}{}{\pbrac{\vectr{v} \vectr{w} + \frac{\tensor{K}}{\mu}\gradient{}{p}\vectr{w}}}{\Omega} = 0
\end{equation}

\begin{equation}
  \gint{\Omega}{}{q \divergence{}{v}}{\Omega} = 0
\end{equation}

Where $\vectr{w}$ is the weighting function for the flow and $q$ is the weighting function
for pressure.
Applying Green's theorem to the pressure term to give the weak form of the equations, and multiplying through by
$\mu \inverse{\tensor{K}}$:

\begin{equation}
  \gint{\Omega}{}{\pbrac{\mu \inverse{\tensor{K}} \vectr{v} \vectr{w} - p \gradient{}{\vectr{w}}}}{\Omega}  + %
  \gint{\Gamma}{}{p \vectr{n} \vectr{w}}{\Gamma}= 0
\end{equation}

\begin{equation}
  \gint{\Omega}{}{q \divergence{}{\vectr{v}}}{\Omega}= 0
\end{equation}

Where $\vectr{n}$ is the normal vector to the surface $\Gamma$.

The OpenCMISS implementation of Darcy's equation uses the stabilised form of the finite element
equations developed by \citet{masud:2002}. This adds stabilising terms, so that the final form
of the implemented equations is:

\begin{equation}
  \gint{\Omega}{}{\pbrac{\mu \inverse{\tensor{K}} \vectr{v} \vectr{w} - p \gradient{}{\vectr{w}} - %
  \frac{1}{2} \pbrac{\gradient{}{p} \vectr{w} + \mu \inverse{\tensor{K}} \vectr{v} \vectr{w}}}}{\Omega} + %
  \gint{\Gamma}{}{p \vectr{n} \vectr{w}}{\Gamma}= 0
\end{equation}

\begin{equation}
  \gint{\Omega}{}{q \divergence{}{\vectr{v}} + %
  \frac{1}{2} \pbrac{\dotprod{\gradient{}{q}}{\vectr{v}} + %
  \frac{\tensor{K}}{\mu} \dotprod{\gradient{}{p}}{\gradient{}{q}}}}{\Omega}= 0
\end{equation}
