\section{Finite Volumes for Navier Stokes}

Here we give a brief introduction to the process of solving the Navier-Stokes
equations using finite volumes. This section is not intended to cover the
multitude of different solution methods. For further details the reader should
consult the literature.

\subsection{Governing equations:}

Consider the generalised transport (which arises from the Reynolds transport
theorem) equation for a scalar quatity for a control volume. The generalised
transport thereom comes from the conservation principle which states that the
time rate of increase of the scalar inside the control volume plus thr net rate
of decrease of the the scalar due to convection across the control volume
boundary is equal to the net rate of increase of the scalar quantity due to
diffusion across the control volume boundary plus the net rate of the creation
of the scalar inside the control volume. Mathematically for a scalar quantity
$\Phi$ this can be written as
\begin{equation}
  \delby{}{t}\gint{\Omega}{}{\rho\Phi}{\Omega}+\gint{\Gamma}{}{\dotprod{\rho\Phi\vectr{v}}{\vectr{n}}}{\Gamma}=\gint{\Gamma}{}{\dotprod{\vectr{q}_{\Phi}}{\vectr{n}}}{\Gamma}+\gint{\Omega}{}{\dot{q}_{\Phi}}{\Omega}
  \label{eqn:integralgenscalartransport}
\end{equation}
where the control volume is $\Omega$ which has a boundary
$\Gamma=\boundary{\Omega}$ with a normal $\vectr{n}$, $\vectr{v}$ is the
convection velocity, $\vectr{q}_{\Phi}$ is the flux across the control volume
boundary associated with $\Phi$, and $\dot{q}_{\Phi}$ is the volumetric
generation (or source) term. The flux vector can often be related to $\Phi$ by
a gradient law \ie $\vectr{q}_{\Phi}=\gamma\gradient{}{\Phi}$ where $\gamma$
is a proportionality scalar. 

The equivalent differential form of the scalar transport equation by
rearranging \Eqnref{eqn:integralgenscalartransport} and using the Green-Gauss
theorem \ie
\begin{equation}
  \begin{split}
    \delby{}{t}\gint{\Omega}{}{\rho\Phi}{\Omega}+\gint{\Gamma}{}{\dotprod{\rho\Phi\vectr{v}}{\vectr{n}}}{\Gamma}
    &=\gint{\Gamma}{}{\dotprod{\pbrac{\gamma\gradient{}{\Phi}}}{\vectr{n}}}{\Gamma}+\gint{\Omega}{}{\dot{q}_{\Phi}}{\Omega}\\
    \gint{\Omega}{}{\delby{\pbrac{\rho\Phi}}{t}}{\Omega}+\gint{\Omega}{}{\divergence{}{\pbrac{\rho\Phi\vectr{v}}}}{\Omega}
    &=\gint{\Omega}{}{\divergence{}{\pbrac{\gamma\gradient{}{\Phi}}}}{\Omega}+\gint{\Omega}{}{\dot{q}_{\Phi}}{\Omega}
    \\
    \gint{\Omega}{}{\pbrac{\delby{\pbrac{\rho\Phi}}{t}+\divergence{}{\pbrac{\rho\Phi\vectr{v}}}}}{\Omega}
    &=\gint{\Omega}{}{\pbrac{\divergence{}{\pbrac{\gamma\gradient{}{\Phi}}}+\dot{q}_{\Phi}}}{\Omega}
    \\
    \gint{\Omega}{}{\pbrac{\delby{\pbrac{\rho\Phi}}{t}+\divergence{}{\pbrac{\rho\Phi\vectr{v}}}-\divergence{}{\pbrac{\gamma\gradient{}{\Phi}}}-\dot{q}_{\Phi}}}{\Omega}
    &= 0
  \end{split}
\end{equation}
Now for the integral to vanish the integrand must vanish \ie
\begin{equation}
  \delby{\pbrac{\rho\Phi}}{t}+\divergence{}{\pbrac{\rho\Phi\vectr{v}}}
  -\divergence{}{\pbrac{\gamma\gradient{}{\Phi}}}-\dot{q}_{\Phi}=0
\end{equation}
or
\begin{equation}
  \delby{\pbrac{\rho\Phi}}{t}+\divergence{}{\pbrac{\rho\Phi\vectr{v}}}=\divergence{}{\pbrac{\gamma\gradient{}{\Phi}}}+\dot{q}_{\Phi}
  \label{eqn:differentialgenscalartransport}
\end{equation}

The Navier-Stokes equations consist of an equation concerning conservation of
momentum and an equation concerning conservation of mass. These can be thought
of as special cases of the generalised transport theorem. For the conservation
of mass equation we have
\begin{equation}
  \delby{}{t}\gint{\Omega}{}{\rho}{\Omega}+\gint{\Gamma}{}{\dotprod{\rho\vectr{v}}{\vectr{n}}}{\Gamma}=0
  \label{eqn:integralconservationofmass}
\end{equation}

And the differential form for the conservation of mass is given by
\begin{equation}
  \begin{split}
    \delby{}{t}\gint{\Omega}{}{\rho}{\Omega}+\gint{\Gamma}{}{\dotprod{\rho\vectr{v}}{\vectr{n}}}{\Gamma}&=0
    \\
    \gint{\Omega}{}{\delby{\rho}{t}}{\Omega}+\gint{\Omega}{}{\divergence{}{\pbrac{\rho\vectr{v}}}}{\Omega}&=0\\
    \gint{\Omega}{}{\pbrac{\delby{\rho}{t}+\divergence{}{\pbrac{\rho\vectr{v}}}}}{\Omega}&=0
  \end{split}
\end{equation}
which gives us
\begin{equation}
  \delby{\rho}{t}+\divergence{}{\pbrac{\rho\vectr{v}}}=0
  \label{eqn:differentialconservationofmass}
\end{equation}

For the conservation of momentum we have
\begin{equation}
  \delby{}{t}\gint{\Omega}{}{\rho\vectr{v}}{\Omega}+\gint{\Gamma}{}{\dotprod{\tensorprod{\rho\vectr{v}}{\vectr{v}}}{\vectr{n}}}{\Gamma}=\gint{\Gamma}{}{\dotprod{\tensor{T}}{\vectr{n}}}{\Gamma}+\gint{\Omega}{}{\rho\vectr{f}}{\Omega}
\end{equation}
where $\tensor{T}$ is a stress tensor and $\vectr{f}$ is the body force vector. For Newtonian fluids we have
\begin{equation}
  \begin{split}
    \tensor{T}&=\pbrac{-p+\lambda\divergence{}{\vectr{v}}}\tensor{I}+2\mu\tensor{D} \\
    \tensor{T}&=-p\tensor{I}+\mu\pbrac{\pbrac{\gradient{}{\vectr{v}}+\transpose{\gradient{}{\vectr{v}}}}-
      \frac{2}{3}\divergence{}{\vectr{v}}\tensor{I}}\\
    \tensor{T}&=-p\tensor{I}+\tensor{\tau}
  \end{split}
\end{equation}
where $p$ is the pressure, $\tensor{D}$ is the deformation tensor given by
\begin{equation}
  \tensor{D}=\frac{1}{2}\pbrac{\gradient{}{\vectr{v}}+\transpose{\gradient{}{\vectr{v}}}}
\end{equation}
and $\mu$ is
the viscosity and $\tensor{\tau}$ is the shear stress tensor \ie
\begin{equation}
  \tensor{\tau}=\mu\pbrac{\pbrac{\gradient{}{\vectr{v}}+\transpose{\gradient{}{\vectr{v}}}}-
    \frac{2}{3}\divergence{}{\vectr{v}}\tensor{I}}
\end{equation}

Rearranging the integral conservation of momentum equation gives us
\begin{equation}
  \begin{split}
    \delby{}{t}\gint{\Omega}{}{\rho\vectr{v}}{\Omega}+\gint{\Gamma}{}{\dotprod{\tensorprod{\rho\vectr{v}}{\vectr{v}}}{\vectr{n}}}{\Gamma}
    &=\gint{\Gamma}{}{\dotprod{\tensor{T}}{\vectr{n}}}{\Gamma}+\gint{\Omega}{}{\rho\vectr{f}}{\Omega}\\
    \delby{}{t}\gint{\Omega}{}{\rho\vectr{v}}{\Omega}+\gint{\Gamma}{}{\dotprod{\tensorprod{\rho\vectr{v}}{\vectr{v}}}{\vectr{n}}}{\Gamma}
    &=\gint{\Gamma}{}{-p\vectr{n}}{\Gamma}+\gint{\Gamma}{}{\dotprod{\tensor{\tau}}{\vectr{n}}}{\Gamma}+\gint{\Omega}{}{\rho\vectr{f}}{\Omega} \\
    \delby{}{t}\gint{\Omega}{}{\rho\vectr{v}}{\Omega}+\gint{\Gamma}{}{\dotprod{\tensorprod{\rho\vectr{v}}{\vectr{v}}}{\vectr{n}}}{\Gamma}
    &=\gint{\Gamma}{}{-p\vectr{n}}{\Gamma}+\gint{\Gamma}{}{\dotprod{\mu\pbrac{\gradient{}{\vectr{v}}+\transpose{\gradient{}{\vectr{v}}}}}{\vectr{n}}}{\Gamma}+\gint{\Omega}{}{\rho\vectr{f}}{\Omega} \\
  \end{split}
  \label{eqn:integralconservationofmomentum}
\end{equation}

The differential form for the conservation of momentum is given by
\begin{equation}
  \begin{split}
    \delby{}{t}\gint{\Omega}{}{\rho\vectr{v}}{\Omega}+\gint{\Gamma}{}{\dotprod{\tensorprod{\rho\vectr{v}}{\vectr{v}}}{\vectr{n}}}{\Gamma}
    &=\gint{\Gamma}{}{-p\vectr{n}}{\Gamma}+\gint{\Gamma}{}{\dotprod{\tensor{\tau}}{\vectr{n}}}{\Gamma}
    +\gint{\Omega}{}{\rho\vectr{f}}{\Omega} \\
    \gint{\Omega}{}{\delby{\pbrac{\rho\vectr{v}}}{t}}{\Omega}
    +\gint{\Omega}{}{\divergence{}{\pbrac{\tensorprod{\rho\vectr{v}}{\vectr{v}}}}}{\Omega}
    &=\gint{\Omega}{}{-\gradient{}{p}}{\Omega}+\gint{\Omega}{}{\divergence{}{\tensor{\tau}}}{\Omega}
    +\gint{\Omega}{}{\rho\vectr{f}}{\Omega}\\
    \gint{\Omega}{}{\pbrac{\delby{\pbrac{\rho\vectr{v}}}{t}+\divergence{}{\pbrac{\tensorprod{\rho\vectr{v}}{\vectr{v}}}}-\divergence{}{\tensor{\tau}}+\gradient{}{p}-\rho\vectr{f}}}{\Omega}&=0
  \end{split}
\end{equation}
and thus we have
\begin{equation}
  \delby{\pbrac{\rho\vectr{v}}}{t}+\divergence{}{\pbrac{\tensorprod{\rho\vectr{v}}{\vectr{v}}}}-\divergence{}{\tensor{\tau}}+\gradient{}{p}-\rho\vectr{f}=0
\end{equation}
or
\begin{equation}
  \delby{\pbrac{\rho\vectr{v}}}{t}+\divergence{}{\pbrac{\tensorprod{\rho\vectr{v}}{\vectr{v}}}}=\divergence{}{\tensor{\tau}}-\gradient{}{p}+\rho\vectr{f}=0
  \label{eqn:differentialconservationofmomentum}
\end{equation}

For incompressible flows the density doesn't change and so the
integral forms of the conservation of mass and momentum are given by
\begin{equation}
  \gint{\Gamma}{}{\dotprod{\vectr{v}}{\vectr{n}}}{\Gamma}=0
\end{equation}
and
\begin{equation}
  \rho\delby{}{t}\gint{\Omega}{}{\vectr{v}}{\Omega}+\gint{\Gamma}{}{\dotprod{\pbrac{\tensorprod{\rho\vectr{v}}{\vectr{v}}}}{\vectr{n}}}{\Gamma}
  =\gint{\Gamma}{}{-p\vectr{n}}{\Gamma}+\gint{\Gamma}{}{\dotprod{\tensor{\tau}}{\vectr{n}}}{\Gamma}
  +\gint{\Omega}{}{\rho\vectr{f}}{\Omega}
\end{equation}

The differential form for the conservation of mass and momentum for
incompressible flows is given by
\begin{equation}
  \divergence{}{\vectr{v}}=0
\end{equation}
and
\begin{equation}
  \rho\delby{\vectr{v}}{t}+\dotprod{\rho\vectr{v}}{\gradient{}{\vectr{v}}}=\divergence{}{\tensor{\tau}}-\gradient{}{p}+\rho\vectr{f}
\end{equation}

\subsection{Finite volume discretisation:}

A geometry can be discretised into a number of polygonal cells with polygonal
faces forming the boundary of the cells as shown in \Figref{fig:staggeredgrid}.

\epstexfigure{FluidMechanics/svgs/staggeredgrid.eps_tex}{2D
  staggered grid with control volume.}{Diagram of a two-dimensional staggered
  grid with a control volume $\Omega$ with boundary
  $\Gamma$.}{fig:staggeredgrid}{0.75}

For conservation of mass consider the control volume shown in
\Figref{fig:pressurecv}. Note that a two-dimensional incompressible flow problem will be
considered here for clarity but this is easily extendable to three-dimensions.

\epstexfigure{FluidMechanics/svgs/pressurecv.eps_tex}{2D
  control volume centered at a pressure point.}{Diagram of a two-dimensional control volume $\Omega$ with boundary
  $\Gamma$ centered at a pressure point.}{fig:pressurecv}{0.75}

The integral form of the conservation of mass statement is
\begin{equation}
  \gint{\Gamma_{p}}{}{\dotprod{\vectr{v}}{\vectr{n}}}{\Gamma}=0
\end{equation}
thus integrating of the boundary gives
\begin{equation}
  \gint{\Gamma_{p}}{}{\dotprod{\vectr{v}}{\vectr{n}}}{\Gamma}\approx hv^{1,n}_{i+\frac{1}{2},j}-hv^{1,n}_{i-\frac{1}{2},j}+hv^{2,n}_{i+\frac{1}{2},j}-hv^{2,n}_{i-\frac{1}{2},j}=0
\end{equation}
and dividing through by $h$ gives
\begin{equation}
  \gint{\Gamma_{p}}{}{\dotprod{\vectr{v}}{\vectr{n}}}{\Gamma}\approx v^{1,n}_{i+\frac{1}{2},j}-v^{1,n}_{i-\frac{1}{2},j}+v^{2,n}_{i+\frac{1}{2},j}-v^{2,n}_{i-\frac{1}{2},j}=0
\end{equation}

For the conservation of momentum in the $x$ direction consider a control
volume centered on a $v^{1}$ velocity point as shown in \Figref{fig:v1cv}

\epstexfigure{FluidMechanics/svgs/v1cv.eps_tex}{2D
  control volume centered at a $v^{1}$ velocity point.}{Diagram of a two-dimensional control volume $\Omega$ with boundary
  $\Gamma$ centered at a $v^{1}$ velocity point.}{fig:v1cv}{0.75}

The rate of change of momentum in the $x$ direction is thus given by
\begin{equation}
  \rho\delby{}{t}\gint{\Omega_{v^{1}}}{}{v^{1}}{\Omega}\approx\rho h^{2}\dfrac{v^{1,n+1}_{i+\frac{1}{2},j}-v^{1,n}_{i+\frac{1}{2},j}}{\Delta
  t}
\end{equation}

For the conservation of momentum in the $y$ direction consider a control
volume centered on a $v^{2}$ velocity point as shown in
\Figref{fig:v2cv}. Note that there are many different ways of discretising the
momentum equation depending on the degree of accuracy and upwinding etc. 

\epstexfigure{FluidMechanics/svgs/v2cv.eps_tex}{2D
  control volume centered at a $v^{2}$ velocity point.}{Diagram of a two-dimensional control volume $\Omega$ with boundary
  $\Gamma$ centered at a $v^{2}$ velocity point.}{fig:v2cv}{0.75}

The rate of change of momentum in the $y$ direction is thus given by
\begin{equation}
  \rho\delby{}{t}\gint{\Omega_{v^{2}}}{}{v^{2}}{\Omega}\approx\rho h^{2}\dfrac{v^{2,n+1}_{i,j+\frac{1}{2}}-v^{2,n}_{i,j+\frac{1}{2}}}{\Delta
  t}
\end{equation}

Now, the flow of momentum in the $x$ direction is given by
\begin{equation}
  \gint{\Gamma_{v^{1}}}{}{\dotprod{\pbrac{\rho
        v^{1}\vectr{v}}}{\vectr{n}}}{\Gamma}\approx\rho h
  \pbrac{\pbrac{v^{1,n}_{i+1,j}}^{2}-\pbrac{v^{1,n}_{i,j}}^{2}
  +v^{1,n}_{i+\frac{1}{2},j+\frac{1}{2}}v^{2,n}_{i+\frac{1}{2},j+\frac{1}{2}}
  -v^{1,n}_{i+\frac{1}{2},j-\frac{1}{2}}v^{2,n}_{i+\frac{1}{2},j-\frac{1}{2}}}
\end{equation}
where we can find the terms on the right hand side by interpolation \ie
\begin{equation}
  \begin{split}
    \pbrac{v^{1,n}_{i+1,j}}^{2}&=\pbrac{\frac{1}{2}\pbrac{v^{1,n}_{i+\frac{3}{2},j}+v^{1,n}_{i+\frac{1}{2},j}}}^{2}\\
    \pbrac{v^{1,n}_{i,j}}^{2}&=\pbrac{\frac{1}{2}\pbrac{v^{1,n}_{i+\frac{1}{2},j}+v^{1,n}_{i-\frac{1}{2},j}}}^{2}\\
    v^{1,n}_{i+\frac{1}{2},j+\frac{1}{2}}v^{2,n}_{i+\frac{1}{2},j+\frac{1}{2}}&=\pbrac{\frac{1}{2}\pbrac{v^{1,n}_{i+\frac{1}{2},j}+v^{1,n}_{i+\frac{1}{2},j+1}}}
    \pbrac{\frac{1}{2}\pbrac{v^{2,n}_{i,j+\frac{1}{2}}+v^{2,n}_{i+1,j+\frac{1}{2}}}}\\
    v^{1,n}_{i+\frac{1}{2},j-\frac{1}{2}}v^{2,n}_{i+\frac{1}{2},j-\frac{1}{2}}&=\pbrac{\frac{1}{2}\pbrac{v^{1,n}_{i+\frac{1}{2},j}+v^{1,n}_{i+\frac{1}{2},j-1}}}
    \pbrac{\frac{1}{2}\pbrac{v^{2,n}_{i,j-\frac{1}{2}}+v^{2,n}_{i+1,j-\frac{1}{2}}}}
  \end{split}
\end{equation}

Similarily, the flow of momentum in the $y$ direction is given by
\begin{equation}
  \gint{\Gamma_{v^{2}}}{}{\dotprod{\pbrac{\rho
        v^{2}\vectr{v}}}{\vectr{n}}}{\Gamma}\approx\rho h\pbrac{
  v^{1,n}_{i+\frac{1}{2},j+\frac{1}{2}}v^{2,n}_{i+\frac{1}{2},j+\frac{1}{2}}
  -v^{1,n}_{i-\frac{1}{2},j+\frac{1}{2}}v^{2,n}_{i-\frac{1}{2},j+\frac{1}{2}}
  +\pbrac{v^{2,n}_{i,j+1}}^{2}-\pbrac{v^{2,n}_{i,j}}^{2}}
\end{equation}
where we can find the terms on the right hand side by interpolation \ie
\begin{equation}
  \begin{split}
    \pbrac{v^{2,n}_{i,j+1}}^{2}&=\pbrac{\frac{1}{2}\pbrac{v^{2,n}_{i,j+\frac{3}{2}}+v^{2,n}_{i,j+\frac{1}{2}}}}^{2}\\
    \pbrac{v^{2,n}_{i,j}}^{2}&=\pbrac{\frac{1}{2}\pbrac{v^{2,n}_{i,j+\frac{1}{2}}+v^{2,n}_{i,j-\frac{1}{2}}}}^{2}\\
    v^{1,n}_{i+\frac{1}{2},j+\frac{1}{2}}v^{2,n}_{i+\frac{1}{2},j+\frac{1}{2}}&=\pbrac{\frac{1}{2}\pbrac{v^{1,n}_{i+\frac{1}{2},j}+v^{1,n}_{i+\frac{1}{2},j+1}}}
    \pbrac{\frac{1}{2}\pbrac{v^{2,n}_{i,j+\frac{1}{2}}+v^{2,n}_{i+1,j+\frac{1}{2}}}}\\
    v^{1,n}_{i-\frac{1}{2},j+\frac{1}{2}}v^{2,n}_{i-\frac{1}{2},j+\frac{1}{2}}&=\pbrac{\frac{1}{2}\pbrac{v^{1,n}_{i-\frac{1}{2},j}+v^{1,n}_{i-\frac{1}{2},j+1}}}
    \pbrac{\frac{1}{2}\pbrac{v^{2,n}_{i,j+\frac{1}{2}}+v^{2,n}_{i-1,j+\frac{1}{2}}}}
  \end{split}
\end{equation}

The pressure force in the $x$ direction is given by
\begin{equation}
  \gint{\Gamma_{v^{1}}}{}{pn^{1}}{\Gamma}\approx h\pbrac{p^{n}_{i+1,j}-p^{n}_{i,j}}
\end{equation}
and the pressure force in the $y$ direction is given by
\begin{equation}
  \gint{\Gamma_{v^{2}}}{}{pn^{2}}{\Gamma}\approx h\pbrac{p^{n}_{i,j+1}-p^{n}_{i,j}}
\end{equation}

Now, the viscous diffusion of momentum in the $x$ direction is given by
\begin{equation}
  \begin{split}
    \gint{\Gamma_{v^{1}}}{}{\dotprod{\mu\gradient{}{\vectr{v}}}{\vectr{n}}}{\Gamma}
    &=\gint{\Gamma_{v^{1}}}{}{\mu\pbrac{\delby{v^{1}}{x}n^{1}+\delby{v^{1}}{y}n^{2}}}{\Gamma} \\
    &\approx\mu h\pbrac{\pbrac{\delby{v^{1}}{x}}^{n}_{i+1,j}-\pbrac{\delby{v^{1}}{x}}^{n}_{i,j}
      +\pbrac{\delby{v^{1}}{y}}^{n}_{i+\frac{1}{2},j+\frac{1}{2}}-
      \pbrac{\delby{v^{1}}{y}}^{n}_{i+\frac{1}{2},j-\frac{1}{2}}}\\
    &\approx \mu\pbrac{v^{1,n}_{i+\frac{3}{2},j}+v^{1,n}_{i-\frac{1}{2},j}
      +v^{1,n}_{i+\frac{1}{2},j+1}+v^{1,n}_{i+\frac{1}{2},j-1}-4v^{1,n}_{i+\frac{1}{2},j}}
  \end{split}
\end{equation}

Similarily, the viscous diffusion of momentum in the $y$ direction is given by
\begin{equation}
  \gint{\Gamma_{v^{2}}}{}{\dotprod{\mu\gradient{}{\vectr{v}}}{\vectr{n}}}{\Gamma}
  \approx \mu\pbrac{v^{2,n}_{i,j+\frac{3}{2}}+v^{2,n}_{i,j-\frac{1}{2}}
    +v^{2,n}_{i+1,j+\frac{1}{2}}+v^{2,n}_{i-1,j+\frac{1}{2}}-4v^{2,n}_{i,j+\frac{1}{2}}}
\end{equation}

Combining all terms we have that the integral forms of the incompressible Navier-Stokes
equations (ignoring body forces) for a control volume is given by
\begin{equation}
  \begin{split}
    \gint{\Gamma}{}{\dotprod{\vectr{v}}{\vectr{n}}}{\Gamma}&=0 \\
    \rho\delby{}{t}\gint{\Omega}{}{\vectr{v}}{\Omega}+\gint{\Gamma}{}{\dotprod{\pbrac{\tensorprod{\rho\vectr{v}}{\vectr{v}}}}{\vectr{n}}}{\Gamma}
  &=\gint{\Gamma}{}{-p\vectr{n}}{\Gamma}+\gint{\Gamma}{}{\dotprod{\tensor{\tau}}{\vectr{n}}}{\Gamma}
  +\gint{\Omega}{}{\rho\vectr{f}}{\Omega}
  \end{split}
\end{equation}
become
\begin{equation}
  \begin{split}
    v^{1,n+1}_{i+\frac{1}{2},j}-v^{1,n+1}_{i-\frac{1}{2},j}+v^{2,n+1}_{i+\frac{1}{2},j}-v^{2,n+1}_{i-\frac{1}{2},j}&=0
    \\
    \dfrac{v^{1,n+1}_{i+\frac{1}{2},j}-v^{1,n}_{i+\frac{1}{2},j}}{\Delta
      t}&=-\dfrac{1}{h}
    \left(\pbrac{\frac{1}{2}\pbrac{v^{1,n}_{i+\frac{3}{2},j}+v^{1,n}_{i+\frac{1}{2},j}}}^{2}
    -\pbrac{\frac{1}{2}\pbrac{v^{1,n}_{i+\frac{1}{2},j}+v^{1,n}_{i-\frac{1}{2},j}}}^{2}\right. \\
    &\quad+\pbrac{\frac{1}{2}\pbrac{v^{1,n}_{i+\frac{1}{2},j}+v^{1,n}_{i+\frac{1}{2},j+1}}}
    \pbrac{\frac{1}{2}\pbrac{v^{2,n}_{i,j+\frac{1}{2}}+v^{2,n}_{i+1,j+\frac{1}{2}}}} \\
    &\quad\left.-\pbrac{\frac{1}{2}\pbrac{v^{1,n}_{i+\frac{1}{2},j}+v^{1,n}_{i+\frac{1}{2},j-1}}}
    \pbrac{\frac{1}{2}\pbrac{v^{2,n}_{i,j-\frac{1}{2}}+v^{2,n}_{i+1,j-\frac{1}{2}}}}\right)\\
    &\quad+\dfrac{1}{\rho h}\pbrac{p^{n}_{i+1,j}-p^{n}_{i,j}}\\
    &\quad+\dfrac{\mu}{\rho h^{2}}\pbrac{v^{1,n}_{i+\frac{3}{2},j}+v^{1,n}_{i-\frac{1}{2},j}
      +v^{1,n}_{i+\frac{1}{2},j+1}+v^{1,n}_{i+\frac{1}{2},j-1}-4v^{1,n}_{i+\frac{1}{2},j}}
    \\
    \dfrac{v^{2,n+1}_{i,j+\frac{1}{2}}-v^{2,n}_{i,j+\frac{1}{2}}}{\Delta
      t}&=-\dfrac{1}{h}\left(\pbrac{\frac{1}{2}\pbrac{v^{2,n}_{i,j+\frac{3}{2}}+v^{2,n}_{i,j+\frac{1}{2}}}}^{2}
    -\pbrac{\frac{1}{2}\pbrac{v^{2,n}_{i,j+\frac{1}{2}}+v^{2,n}_{i,j-\frac{1}{2}}}}^{2}\right. \\
    &\quad+\pbrac{\frac{1}{2}\pbrac{v^{1,n}_{i+\frac{1}{2},j}+v^{1,n}_{i+\frac{1}{2},j+1}}}
    \frac{1}{2}\pbrac{v^{1,n}_{i-\frac{1}{2},j}+v^{1,n}_{i-\frac{1}{2},j+1}} \\
    &\quad\left.-\pbrac{\frac{1}{2}\pbrac{v^{1,n}_{i-\frac{1}{2},j}+v^{1,n}_{i-\frac{1}{2},j+1}}}
    \pbrac{\frac{1}{2}\pbrac{v^{2,n}_{i,j+\frac{1}{2}}+v^{2,n}_{i-1,j+\frac{1}{2}}}}\right)\\
    &\quad +\dfrac{1}{\rho h}\pbrac{p_{i,j+1}-p_{i,j}}\\
    &\quad +\dfrac{\mu}{\rho h^{2}}\pbrac{v^{2,n}_{i,j+\frac{3}{2}}+v^{2,n}_{i,j-\frac{1}{2}}
    +v^{2,n}_{i+1,j+\frac{1}{2}}+v^{2,n}_{i-1,j+\frac{1}{2}}-4v^{2,n}_{i,j+\frac{1}{2}}}
  \end{split}
\end{equation}

These equations can be written in vector form as
\begin{equation}
  \begin{split}
    \divergence{\Omega^{p}}{\vectr{u}^{n+1}_{i,j}}&=0\\
    \dfrac{\vectr{u}^{n+1}_{i,j}-\vectr{u}^{n}_{i,j}}{\Delta
      t}&=\vectr{h}^{n}_{i,j}-\gradient{\Omega^{\vectr{u}}}{\vectr{p}^{n}_{i,j}}
  \end{split}
\end{equation}
where $\vectr{h}^{n}_{ij}$ contains the advective, diffusive and body momentum
terms.

The divergence operator, $\divergence{\Omega^{p}}{\sqbrac{\cdot}}$, is defined as
\begin{equation}
  \divergence{\Omega^{p}}{\vectr{u}^{n}_{i,j}}=v^{1,n}_{i+\frac{1}{2},j}-v^{1,n}_{i-\frac{1}{2},j}+v^{2,n}_{i+\frac{1}{2},j}-v^{2,n}_{i-\frac{1}{2},j}
\end{equation}
where $V_{\Omega}$ is the volume of the control volume $\Omega^{p}$.

The gradient operator, $\gradient{\Omega^{\vectr{u}}}{\sqbrac{\cdot}}$, is
defined as
\begin{equation}
  \gradient{\Omega^{\vectr{u}}}{p^{n}_{i,j}}=\dfrac{\Delta p^{n}_{i,j}}{\rho h}
\end{equation}
where $\Delta p^{n}_{i,j}$ is the appropriate change in pressure in the $x$ or
$y$ directions as appropriate.

\subsection{Pressure based solutions:}

For incompressible flows density variations are not linked to pressure and
thus the mass convervation is a constraint on the velocity field. The momentum
equation can thus be interpreted as a evolution equation for velocity. Now, if we
discretise in time and consider time point $n$ and $n+1$ we can introduce
another velocity field, $\vectr{u}^{*}_{i,j}$, and split the evolution
equation for velocity
\begin{equation}
  \dfrac{\vectr{u}^{n+1}_{i,j}-\vectr{u}^{n}_{i,j}}{\Delta
    t}=\vectr{h}^{n}_{i,j}-\gradient{\Omega^{\vectr{u}}}{\vectr{p}^{n}_{i,j}}
\end{equation}
into
\begin{equation}
  \dfrac{\vectr{u}^{*}_{i,j}-\vectr{u}^{n}_{i,j}}{\Delta t}=\vectr{h}^{n}_{i,j}
\end{equation}
and
\begin{equation}
  \dfrac{\vectr{u}^{n+1}_{i,j}-\vectr{u}^{*}_{i,j}}{\Delta t}=-\gradient{\Omega^{\vectr{u}}}{\vectr{p}^{n}_{i,j}}
\end{equation}

Or, rewriting, the split is
\begin{equation}
  \vectr{u}^{*}_{i,j}=\vectr{u}^{n}_{i,j}+\Delta t\vectr{h}^{n}_{i,j}
\end{equation}
and
\begin{equation}
  \vectr{u}^{n+1}_{i,j}=\vectr{u}^{*}_{i,j}-\Delta t\gradient{\Omega^{\vectr{u}}}{\vectr{p}^{n}_{i,j}}
\end{equation}

Now, if we take the divergence of the last equation we obtain
\begin{equation}
  \begin{split}
    \divergence{\Omega^{p}}{\vectr{u}^{n+1}_{i,j}}
    &=\divergence{\Omega^{p}}{\vectr{u}^{*}_{i,j}-\Delta
      t\gradient{\Omega^{\vectr{u}}}{\vectr{p}^{n}_{i,j}}}\\
    &=\divergence{\Omega^{p}}{\vectr{u}^{*}_{i,j}}-\divergence{\Omega^{p}}{\Delta
      t\gradient{\Omega^{\vectr{u}}}{\vectr{p}^{n}_{i,j}}}
  \end{split}
\end{equation}

Now, the discretised conservation of mass equation states
\begin{equation}
  \divergence{\Omega^{p}}{\vectr{u}^{n+1}_{i,j}}=0
\end{equation}
and so we have
\begin{equation}
  0=\divergence{\Omega^{p}}{\vectr{u}^{*}_{i,j}}-\divergence{\Omega^{p}}{\Delta
      t\gradient{\Omega^{\vectr{u}}}{\vectr{p}^{n}_{i,j}}}
\end{equation}
or, rearranging, we obtain
\begin{equation}
  \divergence{\Omega^{p}}{\gradient{\Omega^{\vectr{u}}}{\vectr{p}^{n}_{i,j}}}=\dfrac{\divergence{\Omega^{p}}{\vectr{u}^{*}_{i,j}}}{\Delta
    t}
\end{equation}
  
The pressure based solutions for incompressible flow thus proceed by
integrating the velocity evolution equation and computing the pressure field
that is required to ensure mass conservation. Because the equations are
non-linear and coupled for implicit methods directly solving the implicit
methods is computationally expensive. Thus for an implicit (or semi-implicit
methods) velocity and pressure correction methods are often used (especially
for the steady or slower flows that are found in arteries) in which an
approximate velocity and pressure field is used and then correction terms are
applied to find the real velocity and pressure fields at the new time step.

For example consider the SIMPLE (Semi-Implicit Method for Pressure Linked
Equations) algorithm.

The algorithm can be stated as
\begin{enumerate}
\item Guess an initial pressure field $p^{*}$.
\item Use the conservation of momentum and mass to compute a velocity field
  $\vectr{u}^{*}$.
\item Find a correction to the pressure field $p^{'}$ such that
  $p^{**}=p^{*}+p^{'}$.
\item Find a correction to the velocity field $\vectr{u}^{'}$ such that mass
  is conserved and 
  $\vectr{u}^{n+1}=\vectr{u}^{*}+\vectr{u}^{'}$.
\item Iterate for convergence using under-relaxation \ie $p^{n+1}=\alpha
  p^{**}+\pbrac{1-\alpha} p^{n}$ where $\alpha$ is the relaxation parameter.
\end{enumerate}

Note there are other solution algorithms including SIMPLEC (SIMPLE +
Consistent) and PISO (Pressure Implict with Spliting Operators).

