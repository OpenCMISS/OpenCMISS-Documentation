\section{Navier-Stokes Equations}
\label{sec:NavierStokesEquations}

\subsection{Governing equations:}

The Navier-Stokes equations arise from applying Newton's second law to fluid
motion, \ie the temporal and spatial fluid inertia is in equilibrium with
internal (volume/body) and external (surface) forces. The reaction of surface
forces can be described in terms of the fluid stress as the sum of a diffusing
viscous term, plus a pressure term. The equations are named for the 19th
century contributions of Claude-Louis Navier and George Gabriel Stokes. A
solution of the Navier-Stokes equations is called a flow field, \ie velocity
and pressure field, which is a description of the fluid at a given point in
space and time.

Consider the generalised transport (which arises from the Reynolds transport
theorem) equation for a scalar quatity for a control volume. The generalised
transport thereom comes from the conservation principle which states that the
time rate of increase of the scalar inside the control volume plus thr net rate
of decrease of the the scalar due to convection across the control volume
boundary is equal to the net rate of increase of the scalar quantity due to
diffusion across the control volume boundary plus the net rate of the creation
of the scalar inside the control volume. Mathematically for a scalar quantity
$\Phi$ this can be written as
\begin{equation}
  \delby{}{t}\gint{\Omega}{}{\rho\Phi}{\Omega}+\gint{\Gamma}{}{\dotprod{\rho\Phi\vectr{v}}{\vectr{n}}}{\Gamma}=
  \gint{\Gamma}{}{\dotprod{\vectr{q}_{\Phi}}{\vectr{n}}}{\Gamma}+\gint{\Omega}{}{\dot{q}_{\Phi}}{\Omega}
  \label{eqn:integralgenscalartransport}
\end{equation}
where the control volume is $\Omega$ which has a boundary
$\Gamma=\boundary{\Omega}$ with a normal $\vectr{n}$, $\vectr{v}$ is the
convection velocity, $\vectr{q}_{\Phi}$ is the flux across the control volume
boundary associated with $\Phi$, and $\dot{q}_{\Phi}$ is the volumetric
generation (or source) term. The flux vector can often be related to $\Phi$ by
a gradient law \ie $\vectr{q}_{\Phi}=\gamma\gradient{}{\Phi}$ where $\gamma$
is a proportionality scalar.

The equivalent differential form of the scalar transport equation by
rearranging \Eqnref{eqn:integralgenscalartransport} and using the Green-Gauss
theorem \ie
\begin{equation}
  \begin{split}
    \delby{}{t}\gint{\Omega}{}{\rho\Phi}{\Omega}+\gint{\Gamma}{}{\dotprod{\rho\Phi\vectr{v}}{\vectr{n}}}{\Gamma}
    &=\gint{\Gamma}{}{\dotprod{\pbrac{\gamma\gradient{}{\Phi}}}{\vectr{n}}}{\Gamma}+\gint{\Omega}{}{\dot{q}_{\Phi}}{\Omega}\\
    \gint{\Omega}{}{\delby{\pbrac{\rho\Phi}}{t}}{\Omega}+\gint{\Omega}{}{\divergence{}{\pbrac{\rho\Phi\vectr{v}}}}{\Omega}
    &=\gint{\Omega}{}{\divergence{}{\pbrac{\gamma\gradient{}{\Phi}}}}{\Omega}+\gint{\Omega}{}{\dot{q}_{\Phi}}{\Omega}
    \\
    \gint{\Omega}{}{\pbrac{\delby{\pbrac{\rho\Phi}}{t}+\divergence{}{\pbrac{\rho\Phi\vectr{v}}}}}{\Omega}
    &=\gint{\Omega}{}{\pbrac{\divergence{}{\pbrac{\gamma\gradient{}{\Phi}}}+\dot{q}_{\Phi}}}{\Omega}
    \\
    \gint{\Omega}{}{\pbrac{\delby{\pbrac{\rho\Phi}}{t}+\divergence{}{\pbrac{\rho\Phi\vectr{v}}}-
        \divergence{}{\pbrac{\gamma\gradient{}{\Phi}}}-\dot{q}_{\Phi}}}{\Omega}
    &= 0
  \end{split}
\end{equation}
Now for the integral to vanish the integrand must vanish \ie
\begin{equation}
  \delby{\pbrac{\rho\Phi}}{t}+\divergence{}{\pbrac{\rho\Phi\vectr{v}}}
  -\divergence{}{\pbrac{\gamma\gradient{}{\Phi}}}-\dot{q}_{\Phi}=0
\end{equation}
or
\begin{equation}
  \delby{\pbrac{\rho\Phi}}{t}+\divergence{}{\pbrac{\rho\Phi\vectr{v}}}=\divergence{}{\pbrac{\gamma\gradient{}{\Phi}}}+\dot{q}_{\Phi}
  \label{eqn:differentialgenscalartransport}
\end{equation}

The Navier-Stokes equations can formulated in terms of the 'primitive
variables' \ie velocity and pressure where
$\fnof{\vectr{v}}{\vectr{x},t}=\transpose{\sqbrac{v_{1},v_{2},v_{3}}}$ is the
velocity vector which depends on the spatial coordinates
$\vectr{x}=\transpose{\sqbrac{x_{1},x_{2},x_{3}}}$ and the time
$t$. $\fnof{p}{\vectr{x},t}$ is the scalar pressure field. The Navier-Stokes
equations consist of an equation concerning conservation of momentum and an
equation concerning conservation of mass. These can be thought of as special
cases of the generalised transport theorem. For the conservation of momentum
we have
\begin{equation}
  \delby{}{t}\gint{\Omega}{}{\rho\vectr{v}}{\Omega}
  +\gint{\Gamma}{}{\dotprod{\tensorprod{\rho\vectr{v}}{\vectr{v}}}{\vectr{n}}}{\Gamma}
  =\gint{\Gamma}{}{\dotprod{\tensor{\sigma}}{\vectr{n}}}{\Gamma}+\gint{\Omega}{}{\vectr{f}}{\Omega}
\end{equation}
where $\tensor{\sigma}$ is the Cauchy stress tensor and $\vectr{f}$ is the body force
vector.

For Newtonian fluids we have
\begin{equation}
  \begin{split}
    \tensor{\sigma}&=\pbrac{-p+\lambda\divergence{}{\vectr{v}}}\sharptensor{\tensor{g}}+2\mu\tensor{D} \\
    \tensor{\sigma}&=-p\sharptensor{\tensor{g}}+\mu\pbrac{\pbrac{\gradient{}{\vectr{v}}+\transpose{\gradient{}{\vectr{v}}}}-
      \frac{2}{3}\divergence{}{\vectr{v}}\sharptensor{\tensor{g}}}\\
    \tensor{\sigma}&=-p\sharptensor{\tensor{g}}+\tensor{\tau}
  \end{split}
\end{equation}
where $p$ is the pressure, $\tensor{D}$ is the deformation tensor given by
\begin{equation}
  \tensor{D}=D^{ij}\tensorprod{\vectr{g}_{i}}{\vectr{g}_{j}}=\frac{1}{2}\pbrac{\gradient{}{\vectr{v}}+\transpose{\gradient{}{\vectr{v}}}}=\frac{1}{2}\pbrac{g^{ik}\covarderiv{u^{j}}{k}+g^{jk}\covarderiv{u^{i}}{k}}\tensorprod{\vectr{g}_{i}}{\vectr{g}_{j}}
\end{equation}
and $\mu$ is the viscosity and $\tensor{\tau}$ is the shear stress tensor \ie
\begin{equation}
  \tensor{\tau}=\mu\pbrac{\pbrac{\gradient{}{\vectr{v}}+\transpose{\gradient{}{\vectr{v}}}}-
    \frac{2}{3}\divergence{}{\vectr{v}}\sharptensor{\tensor{g}}}
\end{equation}

Rearranging the integral conservation of momentum equation gives us
\begin{equation}
  \begin{split}
    \delby{}{t}\gint{\Omega}{}{\rho\vectr{v}}{\Omega}
    +\gint{\Gamma}{}{\dotprod{\tensorprod{\rho\vectr{v}}{\vectr{v}}}{\vectr{n}}}{\Gamma}
    &=\gint{\Gamma}{}{\dotprod{\tensor{\sigma}}{\vectr{n}}}{\Gamma}+\gint{\Omega}{}{\vectr{f}}{\Omega}\\
    \delby{}{t}\gint{\Omega}{}{\rho\vectr{v}}{\Omega}
    +\gint{\Gamma}{}{\dotprod{\tensorprod{\rho\vectr{v}}{\vectr{v}}}{\vectr{n}}}{\Gamma}
    &=\gint{\Gamma}{}{\dotprod{\pbrac{-p\sharptensor{\tensor{g}}+\tensor{\tau}}}{\vectr{n}}}{\Gamma}
    +\gint{\Omega}{}{\vectr{f}}{\Omega}
    \\ \delby{}{t}\gint{\Omega}{}{\rho\vectr{v}}{\Omega}
    +\gint{\Gamma}{}{\dotprod{\tensorprod{\rho\vectr{v}}{\vectr{v}}}{\vectr{n}}}{\Gamma}
    &=\gint{\Gamma}{}{\dotprod{-p\sharptensor{\tensor{g}}}{\vectr{n}}}{\Gamma}
    +\gint{\Gamma}{}{\dotprod{\tensor{\tau}}{\vectr{n}}}{\Gamma}+\gint{\Omega}{}{\vectr{f}}{\Omega}\\
  \end{split}
  \label{eqn:integralconservationofmomentum}
\end{equation}

The differential form for the conservation of momentum is given by
\begin{equation}
  \begin{split}
    \delby{}{t}\gint{\Omega}{}{\rho\vectr{v}}{\Omega}+
    \gint{\Gamma}{}{\dotprod{\tensorprod{\rho\vectr{v}}{\vectr{v}}}{\vectr{n}}}{\Gamma}
    &=\gint{\Gamma}{}{\dotprod{-p\sharptensor{\tensor{g}}}{\vectr{n}}}{\Gamma}
    +\gint{\Gamma}{}{\dotprod{\tensor{\tau}}{\vectr{n}}}{\Gamma}
    +\gint{\Omega}{}{\vectr{f}}{\Omega} \\
    \gint{\Omega}{}{\delby{\pbrac{\rho\vectr{v}}}{t}}{\Omega}
    +\gint{\Omega}{}{\divergence{}{\pbrac{\tensorprod{\rho\vectr{v}}{\vectr{v}}}}}{\Omega}
    &=\gint{\Omega}{}{\divergence{}{\pbrac{-p\sharptensor{\tensor{g}}}}}{\Omega}
    +\gint{\Omega}{}{\divergence{}{\tensor{\tau}}}{\Omega}
    +\gint{\Omega}{}{\vectr{f}}{\Omega}\\
    \gint{\Omega}{}{\delby{\pbrac{\rho\vectr{v}}}{t}}{\Omega}
    +\gint{\Omega}{}{\divergence{}{\pbrac{\tensorprod{\rho\vectr{v}}{\vectr{v}}}}}{\Omega}
    &=\gint{\Omega}{}{\pbrac{-\dotprod{\gradient{}{p}}{\sharptensor{\tensor{g}}}-p\divergence{}{\sharptensor{\tensor{g}}}}}{\Omega}+\gint{\Omega}{}{\divergence{}{\tensor{\tau}}}{\Omega}
    +\gint{\Omega}{}{\vectr{f}}{\Omega}\\
  \end{split}
\end{equation}
using \eqnref{eqn:TensorIdentity1}.

Now, as $\divergence{}{\sharptensor{\tensor{g}}}=0$ [UNDER WHAT CONDITIONS???]
we have
\begin{equation}
  \gint{\Omega}{}{\pbrac{\delby{\pbrac{\rho\vectr{v}}}{t}
      +\divergence{}{\pbrac{\tensorprod{\rho\vectr{v}}{\vectr{v}}}}
      +\dotprod{\gradient{}{p}}{\sharptensor{\tensor{g}}}-\divergence{}{\tensor{\tau}}-\vectr{f}}}{\Omega}=0
\end{equation}
and thus
\begin{equation}
  \delby{\pbrac{\rho\vectr{v}}}{t}+\divergence{}{\pbrac{\tensorprod{\rho\vectr{v}}{\vectr{v}}}}
  +\dotprod{\gradient{}{p}}{\sharptensor{\tensor{g}}}-\divergence{}{\tensor{\tau}}-\vectr{f}=0
\end{equation}
or
\begin{equation}
  \delby{\pbrac{\rho\vectr{v}}}{t}+\divergence{}{\pbrac{\tensorprod{\rho\vectr{v}}{\vectr{v}}}}
  =-\dotprod{\gradient{}{p}}{\sharptensor{\tensor{g}}}+\divergence{}{\mu\pbrac{\pbrac{\gradient{}{\vectr{v}}
        +\transpose{\gradient{}{\vectr{v}}}}-\frac{2}{3}\divergence{}{\vectr{v}}
      \sharptensor{\tensor{g}}}}+\vectr{f}=0
  \label{eqn:differentialconservationofmomentum}
\end{equation}

For the conservation of mass equation we have
\begin{equation}
  \delby{}{t}\gint{\Omega}{}{\rho}{\Omega}+\gint{\Gamma}{}{\dotprod{\rho\vectr{v}}{\vectr{n}}}{\Gamma}=0
  \label{eqn:integralconservationofmass}
\end{equation}

And the differential form for the conservation of mass is given by
\begin{equation}
  \begin{split}
    \delby{}{t}\gint{\Omega}{}{\rho}{\Omega}+\gint{\Gamma}{}{\dotprod{\rho\vectr{v}}{\vectr{n}}}{\Gamma}&=0
    \\
    \gint{\Omega}{}{\delby{\rho}{t}}{\Omega}+\gint{\Omega}{}{\divergence{}{\pbrac{\rho\vectr{v}}}}{\Omega}&=0\\
    \gint{\Omega}{}{\pbrac{\delby{\rho}{t}+\divergence{}{\pbrac{\rho\vectr{v}}}}}{\Omega}&=0
  \end{split}
\end{equation}
which gives us
\begin{equation}
  \delby{\rho}{t}+\divergence{}{\pbrac{\rho\vectr{v}}}=0
  \label{eqn:differentialconservationofmass}
\end{equation}

For incompressible flows the density doesn't change and so the integral forms
of the conservation of mass and momentum are given by
\begin{equation}
  \gint{\Gamma}{}{\dotprod{\rho\vectr{v}}{\vectr{n}}}{\Gamma}=0
\end{equation}
and
\begin{equation}
  \rho\delby{}{t}\gint{\Omega}{}{\vectr{v}}{\Omega}
  +\gint{\Gamma}{}{\dotprod{\pbrac{\tensorprod{\rho\vectr{v}}{\vectr{v}}}}{\vectr{n}}}{\Gamma}
  =\gint{\Gamma}{}{\dotprod{-p\sharptensor{\tensor{g}}}{\vectr{n}}}{\Gamma}
  +\gint{\Gamma}{}{\dotprod{\tensor{\tau}}{\vectr{n}}}{\Gamma}
  +\gint{\Omega}{}{\vectr{f}}{\Omega}
\end{equation}

The differential form for the conservation of momentum and mass for
incompressible flows is given by
\begin{equation}
  \rho\delby{\vectr{v}}{t}+\dotprod{\rho\vectr{v}}{\gradient{}{\vectr{v}}}
  =-\dotprod{\gradient{}{p}}{\sharptensor{\tensor{g}}}+\divergence{}{\tensor{\tau}}+\vectr{f}
  \label{eqn:DiffIncompNSEMomentum}
\end{equation}
and
\begin{equation}
  \divergence{}{\pbrac{\rho\vectr{v}}}=0
  \label{eqn:DiffIncompNSEMass}
\end{equation}

For a Newtonian fluid \Eqnref{eqn:DiffIncompNSEMomentum} becomes
\begin{equation}
  \rho\delby{\vectr{v}}{t}
  +\dotprod{\rho\vectr{v}}{\gradient{}{\vectr{v}}}
  =\dotprod{-\gradient{}{p}}{\sharptensor{\tensor{g}}}
  +\divergence{}{\mu\pbrac{\gradient{}{\vectr{v}+\transpose{\gradient{}{\vectr{v}}}}}}
  +\vectr{f}
  \label{eqn:DiffIncompNewtonianNSEMomentum}
\end{equation}

The first term on the left hand side of \Eqnref{eqn:DiffIncompNSEMomentum}
represents unsteady accelerative inertial contributions, the second term on
the left hand side represents the nonlinear convective acceleration terms. The
first term on the right hand side of \Eqnref{eqn:DiffIncompNSEMomentum}
represents the viscous stresses in the system, the second term represents the
pressure source and the last term represents the body force.

As with Stokes flow, the incompressibility condition in
\Eqnref{eqn:DiffIncompNSEMass} creates restrictions on the formulation of
the velocity and pressure spaces using finite elements known as the
Ladyzhenkaya, Babuska, and Brezzi (LBB) or inf-sup consistency
condition. Several methods have been devised to define a pressure function
that is consistent with the velocity space using primitive variables. These
include mixed element methods, penalty methods, generalized petrov-galerkin
methods using pressure poisson correction, operator splitting, and
semi-implicit pressure correction \cite{chung:2010}.

\subsection{Non-dimensional form}
\label{subsec:NonDimensionalNavierStokes}

For an incompressible Newtonian fluid the Navier-Stokes equations are
\begin{align}
  \rho\delby{\vectr{v}}{t}
  +\dotprod{\rho\vectr{v}}{\gradient{}{\vectr{v}}}
  &=\dotprod{-\gradient{}{p}}{\sharptensor{\tensor{g}}}
  +\divergence{}{\mu\pbrac{\gradient{}{\vectr{v}+\transpose{\gradient{}{\vectr{v}}}}}}
  +\rho\vectr{a}_{g} \label{eqn:DiffIncompNSEMomentumNonDim} \\
  \divergence{}{\pbrac{\rho\vectr{v}}}&=0 \label{eqn:DiffIncompNSEMassNonDim}
\end{align}
where $\rho$ is the density, $\vectr{v}$ is the fluid velocity, $p$ is
the pressure, $\mu$ is the viscosity and $\vectr{a}_{g}$ is the
acceleration due to gravity.

To obtain a non-dimensional form of the Navier-Stokes equations we
need to determine a scale parameter for each of the base units
involved in the equations. For some base units there may be multiple
choices for the scale parameter and it is important to consider the
type of physics and forces involved in order to make the correct
choice. We will now consider the different scale parameters for the
various base units.

\subsubsection{Length}
The scale parameter for length can be chosen as $L$. The exact value
of the characteristic length $L$ will depend on the geometry of the
flow. Some choices of $L$ include the pipe diameter $D$ for pipe flow,
the length of a plate or aerofoil $L$ for flow over a solid surface,
the depth of a channel for channel flow \etc The scaled length variables are given by
\begin{equation}
  \nondim{x}_{i}=\dfrac{x_{i}}{L}
\end{equation}

Note that the scale parameter for length will also affect differential operators in space \ie
\begin{equation}
  \gradnondim = \grad L
\end{equation}
and
\begin{equation}
  \gradsqnondim = \gradsq L^{2}
\end{equation}

\subsubsection{Velocity}
In terms of primative variables the flow velocity is of importance in
fluid flows. The scale parameter for fluid velocity can be chosen as
$V$. The scaled velocity variables are given by
\begin{equation}
  \nondim{v}_{i}=\dfrac{v_{i}}{V}
\end{equation}

\subsubsection{Time}
For some problems a natural time scale may be available \eg for
dynamic boundary conditions there may be a natural frequency or period
of variation. If such a natural time scale is not available then there
are two time scales that can be formed in terms of the other scale
parameters. The first time scale called the \emph{dynamic time scale},
$T_{d}$, is the time that it takes for a parcel of fluid to cover the
length scale parameter \ie
\begin{equation}
  T_{d}=\dfrac{L}{V}
\end{equation}

The second time scale can be found if the fluid is viscous. This time
scale known as the \emph{viscous diffusion time}, $T_{v}$, is given by
\begin{equation}
  T_{v}=\dfrac{\rho L^{2}}{\mu}=\dfrac{L^{2}}{\nu}
\end{equation}
where $\mu$ is the kinematic viscosity and $\nu$ is the dynamic viscosity.

The scaled time variable is given by
\begin{equation}
  \nondim{t}_{d}=\dfrac{t}{T_{d}}
\end{equation}
or
\begin{equation}
  \nondim{t}_{v}=\dfrac{t}{T_{v}}
\end{equation}

\subsubsection{Acceleration due to gravity}
For the acceleration due to gravity we can pick the gravitational scale factor, $A_{g}$, based on the magnitude of the accerlation vector \ie
\begin{equation}
  A_{g}=\norm{\vectr{a}_{g}}=a_{g}
\end{equation}

The scaled acceleration due to gravity variable is given by
\begin{equation}
  \nondim{a}_{gi}=\dfrac{a_{gi}}{A_{g}}
\end{equation}

\subsubsection{Pressure}
For pressure there is no real natural scale parameter. The parameter
chosen will depend on the physics of the problem. For example, if we
have high velocity flows then dynamic effects will be dominant and
thus we can chose the pressure scale parameter to be based on the
\emph{dynamic pressure}, $P_{d}$, \ie
\begin{equation}
  P_{d}=\rho V^{2}
\end{equation}

However if we have, for example, large viscosity with low flow
velocities then the viscous effects will be dominant. In this case it
is better to scale the pressure using the \emph{viscous pressure}, $P_{v}$, \ie
\begin{equation}
  P_{v}=\dfrac{\mu V}{L}
\end{equation}

The scaled pressure variable is given by
\begin{equation}
  \nondim{p}_{d}=\dfrac{p}{P_{d}}
\end{equation}
or
\begin{equation}
  \nondim{p}_{v}=\dfrac{p}{P_{v}}
\end{equation}

\subsubsection{Non-dimensional numbers}

A number of important non-dimensional numbers can be found in fluid mechanics. These numbers are given in \tabref{tab:NonDimensionalNumbersFluids}

\begin{table}[htb] \centering
  {\renewcommand{\arraystretch}{2.5}
  \begin{tabular}{|c|c|c|c|} \hline
    Name & Symbol & Formula & Interpretation \\ \hline\hline
    Froude Number & $\froudenum$ & $\froudenum = \dfrac{V}{\sqrt{a_{g}L}}$ &
    $\dfrac{\text{Inertial forces}}{\text{Gravitational forces}}$ \\ \hline
    Mach Number & $\machnum$ & $\machnum = \dfrac{V}{c}$ &
    $\dfrac{\text{Flow velocity}}{\text{Speed of sound}}$ \\ \hline
    Reynolds Number & $\reynoldsnum$ & $\reynoldsnum = \dfrac{\rho L V}{\mu} = \dfrac{L V}{\nu}$ &
    $\dfrac{\text{Inertial forces}}{\text{Viscous forces}}$ \\
    & & $\reynoldsnum = \dfrac{T_{d}}{T_{v}}$ &
    $\dfrac{\text{Dynamic time scale}}{\text{Viscous time scale}}$ \\ \hline
    Schmidt Number & $\schmidtnum$ & $\schmidtnum=\dfrac{\mu}{\rho D}=\dfrac{\nu}{D}$ &
    $\dfrac{\text{Viscous diffusion rate}}{\text{Mass diffusion rate}}$ \\ \hline
    Strouhal Number & $\strouhalnum$ & $\strouhalnum=\dfrac{L}{U T}=\dfrac{f L}{U}$ &
    $\dfrac{\text{Eulerian inertia}}{\text{Convective inertia}}$ \\ \hline
  \end{tabular}}
  \caption{Important non-dimensional numbers in fluid mechanics.}
  \label{tab:NonDimensionalNumbersFluids}
\end{table}

\subsubsection{Non-dimensional Derivation}
We can non-dimensionalise the Navier-Stokes equations by substituting the normalised variables. For inviscid flow we substitute
\begin{align}
  x_{i} &= \nondim{x}_{i}L \\
  \grad &= \dfrac{\gradnondim}{L} \\
  v_{i} &= \nondim{v}_{i}V \\
  t &= \nondim{t}T_{d}=\dfrac{\nondim{t}L}{V}\\
  a_{gi} &= \nondim{a}_{gi}A_{g}=\nondim{a}_{gi}a_{g}\\
  p &= \nondim{p}P_{d}=\nondim{p}\rho V^{2}
\end{align}
into \eqnref{eqn:DiffIncompNSEMomentumNonDim} and \eqnref{eqn:DiffIncompNSEMassNonDim} we obtain
\begin{equation}
  \dfrac{\rho V^{2}}{L}\delby{\nondim{\vectr{v}}}{\nondim{t}}
  +\dotprod{\dfrac{\rho V^{2}}{L}\nondim{\vectr{v}}}{\gradientnondim{}{\nondim{\vectr{v}}}}
  =\dotprod{\dfrac{-\rho V^{2}}{L}\gradientnondim{}{\nondim{p}}}{\sharptensor{\tensor{g}}}
  +\dfrac{1}{L}\divergencenondim{}{\dfrac{\mu V}{L}\pbrac{\gradientnondim{}{\nondim{\vectr{v}}+
        \transpose{\gradientnondim{}{\nondim{\vectr{v}}}}}}}
  +\rho a_{g}\nondim{\vectr{a}}_{g}
\end{equation}

If we now multiply the equation above by $\dfrac{L}{\rho V^{2}}$ we obtain
\begin{equation}
  \delby{\nondim{\vectr{v}}}{\nondim{t}}+\dotprod{\nondim{\vectr{v}}}{\gradientnondim{}{\nondim{\vectr{v}}}}
  =\dotprod{-\gradientnondim{}{\nondim{p}}}{\sharptensor{\tensor{g}}}
  +\dfrac{\mu}{\rho L V}\divergencenondim{}{\pbrac{\gradientnondim{}{\nondim{\vectr{v}}+
        \transpose{\gradientnondim{}{\nondim{\vectr{v}}}}}}}
  +\dfrac{a_{g}L}{V^{2}}\nondim{\vectr{a}}_{g}
\end{equation}

After substitution with non-dimensional numbers we obtain
\begin{equation}
  \delby{\nondim{\vectr{v}}}{\nondim{t}}+\dotprod{\nondim{\vectr{v}}}{\gradientnondim{}{\nondim{\vectr{v}}}}
  =\dotprod{-\gradientnondim{}{\nondim{p}}}{\sharptensor{\tensor{g}}}+
  \dfrac{1}{\reynoldsnum}\divergencenondim{}{\pbrac{\gradientnondim{}{\nondim{\vectr{v}}+
        \transpose{\gradientnondim{}{\nondim{\vectr{v}}}}}}}
  +\dfrac{1}{\froudenum^{2}}\nondim{\vectr{a}}_{g}
\end{equation}

For viscous flow we substitute
\begin{align}
  x_{i} &= \nondim{x}_{i}L \\
  \grad &= \gradnondim L \\
  v_{i} &= \nondim{v}_{i}V \\
  t &= \nondim{t}T_{d}=\dfrac{\nondim{t}L}{V}\\
  a_{gi} &= \nondim{a}_{gi}A_{g}=\nondim{a}_{gi}a_{g}\\
  p &= \nondim{p}P_{v}=\dfrac{\nondim{p}\mu V}{L}
\end{align}
[WHY DO WE USE $T_{d}$ here and not $T_{v}$ ???]into
\eqnref{eqn:DiffIncompNSEMomentumNonDim} and
\eqnref{eqn:DiffIncompNSEMassNonDim} we obtain
\begin{equation}
  \dfrac{\rho V^{2}}{L}\delby{\nondim{\vectr{v}}}{\nondim{t}}
  +\dotprod{\dfrac{\rho V^{2}}{L}\nondim{\vectr{v}}}{\gradientnondim{}{\nondim{\vectr{v}}}}
  =\dotprod{\dfrac{-\mu V}{L^{2}}\gradientnondim{}{\nondim{p}}}{\sharptensor{\tensor{g}}}
  +\dfrac{1}{L}\divergencenondim{}{\dfrac{\mu V}{L}\pbrac{\gradientnondim{}{\nondim{\vectr{v}}+
        \transpose{\gradientnondim{}{\nondim{\vectr{v}}}}}}}
  +\rho a_{g}\nondim{\vectr{a}}_{g}
\end{equation}

If we now multiply the equation above by $\dfrac{L^{2}}{\mu V}$ we obtain
\begin{equation}
  \dfrac{\rho L V}{\mu}\delby{\nondim{\vectr{v}}}{\nondim{t}}+
  \dfrac{\rho L V}{\mu}\dotprod{\nondim{\vectr{v}}}{\gradientnondim{}{\nondim{\vectr{v}}}}
  =\dotprod{-\gradientnondim{}{\nondim{p}}}{\sharptensor{\tensor{g}}}
  +\divergencenondim{}{\pbrac{\gradientnondim{}{\nondim{\vectr{v}}+
        \transpose{\gradientnondim{}{\nondim{\vectr{v}}}}}}}
  +\dfrac{\rho a_{g}L^{3}}{\mu V^{3}}\nondim{\vectr{a}}_{g}
\end{equation}

After substitution with non-dimensional numbers we obtain
\begin{equation}
  \reynoldsnum\delby{\nondim{\vectr{v}}}{\nondim{t}}
  +\reynoldsnum\dotprod{\nondim{\vectr{v}}}{\gradientnondim{}{\nondim{\vectr{v}}}}
  =\dotprod{-\gradientnondim{}{\nondim{p}}}{\sharptensor{\tensor{g}}}+
  \divergencenondim{}{\pbrac{\gradientnondim{}{\nondim{\vectr{v}}+
        \transpose{\gradientnondim{}{\nondim{\vectr{v}}}}}}}
  +\dfrac{\rho a_{g}L^{3}}{\mu V^{3}}\nondim{\vectr{a}}_{g}
\end{equation}
or
\begin{equation}
  \reynoldsnum\pbrac{\delby{\nondim{\vectr{v}}}{\nondim{t}}
  +\dotprod{\nondim{\vectr{v}}}{\gradientnondim{}{\nondim{\vectr{v}}}}}
  =\dotprod{-\gradientnondim{}{\nondim{p}}}{\sharptensor{\tensor{g}}}+
  \divergencenondim{}{\pbrac{\gradientnondim{}{\nondim{\vectr{v}}+
        \transpose{\gradientnondim{}{\nondim{\vectr{v}}}}}}}
  +\dfrac{\rho a_{g}L^{3}}{\mu V^{3}}\nondim{\vectr{a}}_{g}
\end{equation}
    [FACTOR ON THE GRAVITATIONAL ACCELERATION TERM LOOKS A BIT FUNNY???]
    
\subsection{Finite element formulation}

Using a classic Galerkin formulation, mixed methods are perhaps conceptually
the most straightforward method of satisfying LBB, in which velocity is
defined over a space one order higher than pressure (\eg quadratic elements
for velocity, linear for pressure), allowing incompressibility to be
satisfied. It should be noted that our use of a mixed formulation to satisfy
LBB will also be reflected in the shape functions that our weak formulation
depends on. For example, using a 2D element with biquadratic velocity and
linear pressure, we will have 9 DOFs and weight functions for each velocity
component and 4 for the pressure.
% meaning of pressure in NSE.

% Galerkin formulation
The weak form of equations \ref{eqn:DiffIncompNewtonianNSEMomentum} and
\ref{eqn:DiffIncompNSEMass} can be found by applying standard Galerkin weight
functions $\vectr{w}$ for the velocity variables \ie
\begin{equation}
  \gint{\Omega}{}{\pbrac{\rho\delby{\vectr{v}}{t}+\dotprod{\rho\vectr{v}}{\gradient{}{\vectr{v}}}
      +\dotprod{\gradient{}{p}}{\sharptensor{\tensor{g}}}
      -\divergence{}{\mu\pbrac{\gradient{}{\vectr{v}+\transpose{\gradient{}{\vectr{v}}}}}}
      -\vectr{f}}{\vectr{w}}}{\Omega} = \vectr{0}
 \label{eqn:BasicGalerkinNSEMomentum}
\end{equation}
and
\begin{equation}
  \gint{\Omega}{}{\divergence{}{\vectr{v}}w^{p}}{\Omega} = 0
 \label{eqn:BasicGalerkinNSEMass}
\end{equation}
where $w^{p}$ is the Galerkin weight function for the pressure variable.

Before applying the Green-Gauss theorem we note that the pressure term in
\eqnref{eqn:BasicGalerkinNSEMomentum} can be written as
\begin{equation}
  \dotprod{\gradient{}{p}}{\sharptensor{\tensor{g}}}=\dotprod{\gradient{}{p}}{\sharptensor{\tensor{g}}}+p\divergence{}{\sharptensor{\tensor{g}}}=\divergence{}{\pbrac{p\sharptensor{\tensor{g}}}}
\end{equation}
using \eqnref{eqn:TensorIdentity1} as
$\divergence{}{\sharptensor{\tensor{g}}}=0$.

Now, the Green-Gauss theorem for a second order tensor, $\vectr{A}$ and a
scalar $\phi$ can be written as
\begin{equation}
  \gint{\Omega}{}{\phi\divergence{}{\tensor{A}}}{\Omega}=\gint{\Gamma}{}{\dotprod{\phi\tensor{A}}{\vectr{n}}}{\Gamma}-\gint{\Omega}{}{\dotprod{\tensor{A}}{\gradient{}{\phi}}}{\Omega}
\end{equation}
and so the pressure integral from the $\nth{i}$ component of
\eqnref{eqn:BasicGalerkinNSEMomentum} can be re-written as
\begin{equation}
  \gint{\Omega}{}{\dotprod{\gradient{}{p}}{\sharptensor{\tensor{g}}}w^{i}}{\Omega}=\gint{\Gamma}{}{\dotprod{p\sharptensor{\tensor{g}}}{\vectr{n}}w^{i}}{\Gamma}-\gint{\Omega}{}{\dotprod{p\sharptensor{\tensor{g}}}{\gradient{}{w^{i}}}}{\Omega}
\end{equation}

Thus applying the Green-Gauss theorem to
\Eqnref{eqn:BasicGalerkinNSEMomentum}, we will get
the weak form of the momentum equation with the associated natural boundary
conditions at the boundary $\Gamma$ \ie
\begin{multline}
  \gint{\Omega}{}{\rho\delby{\vectr{v}}{t}{\vectr{w}}}{\Omega}
  +\gint{\Omega}{}{\dotprod{\rho\vectr{v}}{\gradient{}{\vectr{u}}}\vectr{w}}{\Omega}
  -\gint{\Omega}{}{\dotprod{p\sharptensor{\tensor{g}}}{\gradient{}{\vectr{w}}}}{\Omega}
  +\gint{\Omega}{}{\dotprod{\pbrac{\mu\pbrac{\gradient{}{\vectr{v}}+\gradient{}{\transpose{\vectr{v}}}}}}{\gradient{}{\vectr{w}}}}{\Omega} \\
  -\gint{\Omega}{}{\vectr{f}\vectr{w}}{\Omega}
  =\gint{\Gamma}{}{\dotprod{\pbrac{\mu\pbrac{\gradient{}{\vectr{v}}+\transpose{\gradient{}{\vectr{v}}}}-
        p\sharptensor{\vectr{g}}}}{\vectr{n}}\vectr{w}}{\Gamma}
  \label{eqn:GalerkinNSE}
\end{multline}

% Boundary conditions
For more extensive discussion of this procedure, along with other weak forms
of the PDEs, we refer to \cite{gresho:2000}. From this weak form, we see
natural (Neumann) boundary conditions arising as a direct result of the
integration.

Dirichlet boundary conditions on a boundary
$\Gamma_D$ for velocity will take the form:
\begin{equation}
  \fnof{\vectr{v}}{\vectr{x},t} = \fnof{\vectr{v}_{D}}{\vectr{x},t} \quad \vectr{x}\in\Gamma_{D}\\
  \label{eqn:NSEVelocityDirichletBC} 
\end{equation}
and for pressure they will take the form
\begin{equation}
  \fnof{p}{\vectr{x},t} = \fnof{p_{D}}{\vectr{x},t} \quad \vectr{x}\in\Gamma_{D}\\
  \label{eqn:NSEPressureDirichletBC} 
\end{equation}

Neumann boundary conditions will consist of a pressure term and
viscous stress acting normal to a given boundary.
\begin{equation}
  \fnof{\vectr{q}}{\vectr{x},t} =
  \dotprod{\pbrac{\mu\pbrac{\gradient{}{\fnof{\vectr{v}}{\vectr{x},t}}+\transpose{\gradient{}{\fnof{\vectr{v}}{\vectr{x},t}}}}-
      \fnof{p}{\vectr{x},t}\sharptensor{\vectr{g}}}}{\fnof{\vectr{n}}{\vectr{x},t}}  \quad\quad \vectr{x}\in\Gamma_{N}
  \label{eqn:NSENeumannBC}  
\end{equation}
or, in component notation,
\begin{equation}
  \fnof{\vectr{q}}{\vectr{x},t}=\fnof{q^{i}}{\vectr{x},t}\vectr{g}_{i}=
  \pbrac{\mu\pbrac{
    g^{jk}\pbrac{\partialderiv{v^{i}}{k}+\christoffel{i}{k}{h}v^{h}}+g^{ik}\pbrac{\partialderiv{v^{j}}{k}+\christoffel{j}{k}{h}v^{h}}}-pg^{ij}}n_{j}\vectr{g}_{i}
\end{equation}
or
\begin{equation}
  \begin{split}
    \fnof{\vectr{q}}{\vectr{x},t}=\fnof{q^{i}}{\vectr{x},t}\vectr{g}_{i}&=\pbrac{\mu\pbrac{
      g^{jk}\partialderiv{v^{i}}{k}+g^{ik}\partialderiv{v^{j}}{k}}-pg^{ij}}n_{j}\vectr{g}_{i}+\mu
    \pbrac{g^{jk}\christoffel{i}{k}{h}v^{h}+g^{ik}\christoffel{j}{k}{h}v^{h}}n_{j}\vectr{g}_{i}\\
    &=\bar{q}^{i}\vectr{g}_{i}+\tilde{q}^{i}\vectr{g}_{i}
  \end{split}
\end{equation}

Specification of Neumann boundaries will simply require the specification of
the terms across element DOFs.

\subsection{Tensor notation}

\Eqnref{eqn:GalerkinNSE} in tensor notation becomes
\begin{multline}
  \gint{\Omega}{}{\rho\dot{v}^{i}w^{i}}{\Omega}
 +\gint{\Omega}{}{\rho v^{j}\covarderiv{v^{i}}{j}w^{i}}{\Omega}
 -\gint{\Omega}{}{pg^{ik}\covarderiv{w^{i}}{k}}{\Omega}\\
 +\gint{\Omega}{}{{\mu}g^{jk}\covarderiv{v^{i}}{k}\covarderiv{w^{i}}{j}}{\Omega}
 +\gint{\Omega}{}{{\mu}g^{ik}\covarderiv{v^{j}}{k}\covarderiv{w^{i}}{j}}{\Omega}\\
 -\gint{\Omega}{}{f^{i}w^{i}}{\Omega}
 -\gint{\Gamma_N}{}{\pbrac{\mu\pbrac{g^{jk}\covarderiv{v^{i}}{k}+g^{ik}\covarderiv{v^{j}}{k}}-pg^{ij}}n_{j}w^{i}}{\Gamma}=0
 \label{eqn:TensorNSE}
\end{multline}
or
\begin{multline}
  \gint{\Omega}{}{\rho\dot{v}^{i}w^{i}}{\Omega}
 +\gint{\Omega}{}{\rho v^{j}\pbrac{\partialderiv{v^{i}}{j}+\christoffel{i}{j}{h}v^{h}}w^{i}}{\Omega}
 -\gint{\Omega}{}{pg^{ik}\partialderiv{w^{i}}{k}}{\Omega}\\
 +\gint{\Omega}{}{{\mu}g^{jk}\pbrac{\partialderiv{v^{i}}{k}+\christoffel{i}{k}{h}v^{h}}\partialderiv{w^{i}}{j}}{\Omega}
 +\gint{\Omega}{}{{\mu}g^{ik}\pbrac{\partialderiv{v^{j}}{k}+\christoffel{j}{k}{h}v^{h}}\partialderiv{w^{i}}{j}}{\Omega}\\
 -\gint{\Omega}{}{f^{i}w^{i}}{\Omega}
 -\gint{\Gamma_N}{}{\pbrac{\mu\pbrac{g^{jk}\pbrac{\partialderiv{v^{i}}{k}+\christoffel{i}{k}{h}v^{h}}+g^{ik}\pbrac{\partialderiv{v^{j}}{k}+\christoffel{j}{k}{h}v^{h}}}-pg^{ij}}n_{j}w^{i}}{\Gamma}=0
 \label{eqn:Tensor2NSE}
\end{multline}
or
\begin{multline}
  \gint{\Omega}{}{\rho\dot{v}^{i}w^{i}}{\Omega}
 +\gint{\Omega}{}{\rho v^{j}\partialderiv{v^{i}}{j}w^{i}}{\Omega}
 -\gint{\Omega}{}{pg^{ik}\partialderiv{w^{i}}{k}}{\Omega}\\
 +\gint{\Omega}{}{\mu\pbrac{g^{jk}\partialderiv{v^{i}}{k}+g^{ik}\partialderiv{v^{j}}{k}}\partialderiv{w^{i}}{j}}{\Omega}
 -\gint{\Omega}{}{f^{i}w^{i}}{\Omega}
 -\gint{\Gamma_N}{}{\pbrac{\mu\pbrac{g^{jk}\partialderiv{v^{i}}{k}+g^{ik}\partialderiv{v^{j}}{k}}-pg^{ij}}n_{j}w^{i}}{\Gamma}\\
 +\gint{\Omega}{}{\rho v^{j}\christoffel{i}{j}{h}v^{h}w^{i}+\mu\pbrac{g^{jk}\christoffel{i}{k}{h}v^{h}+
     g^{ik}\christoffel{j}{k}{h}v^{h}}\partialderiv{w^{i}}{j}}{\Omega}
 -\gint{\Gamma_N}{}{\mu\pbrac{g^{jk}\christoffel{i}{k}{h}v^{h}+g^{ik}\christoffel{j}{k}{h}v^{h}}n_{j}w^{i}}{\Gamma}=0
 \label{eqn:Tensor3NSE}
\end{multline}
where $g^{jk}$ is the contravariant metric tensor and $\christoffel{i}{j}{k}$
is the Christoffel symbol of the second kind for the spatial coordinates.

The conservation of mass equation can be written as
\begin{equation}
  \gint{\Omega}{}{\covarderiv{v^{i}}{i}w^{p}}{\Omega}=
  \gint{}{}{\pbrac{\partialderiv{v^{i}}{i}+\christoffel{i}{i}{j}v^{j}}w^{p}}{\Omega}=
  \gint{\Omega}{}{\partialderiv{v^{i}}{i}w^{p}}{\Omega}+\gint{\Omega}{}{\christoffel{i}{i}{j}v^{j}w^{p}}{\Omega}=0
  \label{eqn:TensorConvOfMass}
\end{equation}

\subsection{Finite Element Formulation}

We can now discretise the domain into finite elements \ie
$\Omega=\displaystyle{\bigcup_{e=1}^{E}}\Omega_{e}$ with
$\Gamma=\displaystyle{\bigcup_{f=1}^{F}}\Gamma_{f}$. \Eqnref{eqn:Tensor3NSE} now
becomes:
\begin{multline}
  \gsum{e=1}{E}{\gint{\Omega_{e}}{}{\rho\dot{v}^{i}w^{i}}{\Omega}}
 +\gsum{e=1}{E}{\gint{\Omega_{e}}{}{\rho v^{j}\partialderiv{v^{i}}{j}{w^{i}}}{\Omega}}
 -\gsum{e=1}{E}{\gint{\Omega_{e}}{}{pg^{ik}\partialderiv{w^{i}}{k}}{\Omega}}\\
 +\gsum{e=1}{E}{\gint{\Omega_{e}}{}{\mu\pbrac{g^{jk}\partialderiv{v^{i}}{k}+
       g^{ik}\partialderiv{v^{j}}{k}}\partialderiv{w^{i}}{j}}{\Omega}}
 -\gsum{e=1}{E}{\gint{\Omega_{e}}{}{f^{i}w^{i}}{\Omega}}
 -\gsum{f=1}{F}{\gint{\Gamma_{N_{f}}}{}{\pbrac{\mu\pbrac{g^{jk}\partialderiv{v^{i}}{k}+g^{ik}\partialderiv{v^{j}}{k}}-pg^{ij}}n_{j}w^{i}}{\Gamma}}\\
 +\gsum{e=1}{E}{\gint{\Omega_{e}}{}{\rho
   v^{j}\christoffel{i}{j}{h}v^{h}w^{i}+\mu\pbrac{g^{jk}\christoffel{i}{k}{h}v^{h}+
     g^{ik}\christoffel{j}{k}{h}v^{h}}\partialderiv{w^{i}}{j}}{\Omega}}
 -\gsum{f=1}{F}{\gint{\Gamma_{N_{f}}}{}{\mu\pbrac{g^{jk}\christoffel{i}{k}{h}v^{h}+g^{ik}\christoffel{j}{k}{h}v^{h}}n_{j}w^{i}}{\Gamma}}=0
 \label{eqn:FEMNSE}
\end{multline}

\Eqnref{eqn:TensorConvOfMass} now becomes
\begin{equation}
  \gsum{e=1}{E}{\gint{\Omega_{e}}{}{\partialderiv{v^{i}}{i}w^{p}}{\Omega}}
  +\gsum{e=1}{E}{\gint{\Omega_{e}}{}{\christoffel{i}{i}{j}v^{j}w^{p}}{\Omega}}=0
\end{equation}

We will assume that the dependent variables $\vectr{v}$ and $p$ can be
interpolated separately in space and time. Here we must also be careful to
note again the discrepancy between the functional spaces for velocity and
pressure using a mixed formulation to satisfy the LBB consistency
requirement. We will therefore define two separate weighting functions: for
the velocity space on $\Omega$ and for the pressure space giving:

\begin{gather}
  \fnof{\vectr{v}}{\vectr{x},t}=\gbfn{n}{}{\vectr{x}}\fnof{\nodept{\vectr{v}}{n}}{t}\\
  \fnof{p}{\vectr{x},t}=\altgbfn{n}{}{\vectr{x}}\fnof{\nodept{p}{n}}{t}
\end{gather}

In standard interpolation notation within an element, we transform from
$\vectr{x}$ to $\vectr{\xi}$:

\begin{gather}
  \fnof{v^{i}}{\vectr{\xi},t}=\idxgbfn{i}{n}{\beta}{\vectr{\xi}}
  \fnof{\idxnodedof{v}{i}{n}{\beta}}{t}\idxgsf{i}{n}{\beta}\\
  \fnof{p}{\vectr{\xi},t}=\altgbfn{n}{\beta}{\vectr{\xi}}
  \fnof{\nodedof{p}{n}{\beta}}{t}\gsf{n}{\beta}
\end{gather}

where $\fnof{\idxnodedof{v}{i}{n}{\beta}}{t}$ are the time varying nodal
degrees-of-freedom for velocity component $i$, node $n$, global derivative
$\beta$, $\idxgbfn{i}{n}{\beta}{\vectr{\xi}}$ are the corresponding velocity
basis functions and $\idxgsf{i}{n}{\beta}$ are the scale factors. The scalar
pressure DOFs, $\fnof{\nodedof{p}{n}{\beta}}{t}$ are interpolated
similarly.

For the natural boundary, we can separate $q^{i}$ into its component velocity
and pressure terms as noted in \ref{eqn:NSENeumannBC} and shown in
\ref{eqn:FEM3NSE}. For our current treatment, we will also assume the values
of viscosity $\mu$ and density $\rho$ are constant. These can be interpolated:

\begin{equation}
  \begin{split}
    \fnof{q^{i}}{\vectr{\xi},t} &= \idxgbfn{i}{o}{\gamma}{\vectr{\xi}}
      \fnof{\idxnodedof{q}{i}{o}{\gamma}}{t}\idxgsf{i}{o}{\gamma} \\
    \fnof{\mu}{\vectr{\xi}} &=\gbfn{r}{\delta}{\vectr{\xi}}
    \nodedof{\mu}{r}{\delta}\gsf{r}{\delta} \\
    \fnof{\rho}{\vectr{\xi}} &=\gbfn{r}{\delta}{\vectr{\xi}}\nodedof{\rho}{r}{\delta}
    \gsf{r}{\delta} \\
  \end{split}
\end{equation}

Using the spatial weighting functions for a Galerkin finite element
formulation:
\begin{equation}
  \fnof{w^{i}}{\vectr{\xi}}=\idxgbfn{i}{m}{\alpha}{\vectr{\xi}}\idxgsf{i}{m}{\alpha}
\end{equation}
and
\begin{equation}
  \fnof{w^{p}}{\vectr{\xi}}=\altgbfn{m}{\alpha}{\vectr{\xi}}\gsf{m}{\alpha}
\end{equation}


\subsection{Spatial Integration}

Using standard integration notation, we can rewrite our Galerkin FEM
formulation from \ref{eqn:NSEFEM2}:

\begin{multline}
  %time dependence
  \gsum{e=1}{E}{\gint{\vectr{0}}{\vectr{1}}{\fnof{\rho}{\vectr{\xi}}\delby{\idxgbfn{i}{n}{\beta}{\vectr{\xi}}
  \fnof{\idxnodedof{v}{i}{n}{\beta}}{t}\idxgsf{i}{n}{\beta}}{t}\idxgbfn{i}{m}{\alpha}{\vectr{\xi}}\idxgsf{i}{m}{\alpha}\abs{\fnof{\matr{J}}{\vectr{\xi}}}}{\vectr{\xi}}}\\
  %convective term
  +\gsum{e=1}{E}{\gint{\vectr{0}}{\vectr{1}}{\fnof{\rho}{\vectr{\xi}}\idxgbfn{j}{n}{\beta}{\vectr{\xi}}
  \fnof{\idxnodedof{v}{j}{n}{\beta}}{t}\idxgsf{j}{n}{\beta}\delby{\idxgbfn{i}{n}{\beta}{\vectr{\xi}}
  \fnof{\idxnodedof{v}{i}{n}{\beta}}{t}\idxgsf{i}{n}{\beta}}{x^{j}}\idxgbfn{i}{m}{\alpha}{\vectr{\xi}}\idxgsf{i}{m}{\alpha}\abs{\fnof{\matr{J}}{\vectr{\xi}}}}{\vectr{\xi}}}\\
  %pressure
  +\gsum{e=1}{E}{\gint{\vectr{0}}{\vectr{1}}{\fnof{g^{ik}}{\vectr{\xi}}\altgbfn{n}{\beta}{\vectr{\xi}}
  \fnof{\nodedof{p}{n}{\beta}}{t}\gsf{n}{\beta}\delby{\idxgbfn{i}{m}{\alpha}{\vectr{\xi}}\idxgsf{i}{m}{\alpha}}{x^{k}}\abs{\fnof{\matr{J}}{\vectr{\xi}}}}{\vectr{\xi}}}\\
  %viscous stress
  -\dsuml{e=1}{E}\dintl{\vectr{0}}{\vectr{1}}\fnof{\mu}{\vectr{\xi}}\left(\fnof{g^{jk}}{\vectr{\xi}}\delby{\idxgbfn{i}{n}{\beta}{\vectr{\xi}}
      \fnof{\idxnodedof{v}{i}{n}{\beta}}{t}\idxgsf{i}{n}{\beta}}{x^{k}}\right.\\
    \left.+\fnof{g^{ik}}{\vectr{\xi}}\delby{\idxgbfn{j}{n}{\beta}{\vectr{\xi}}
  \fnof{\idxnodedof{v}{j}{n}{\beta}}{t}\idxgsf{j}{n}{\beta}}{x^{k}}\right)\delby{\idxgbfn{i}{m}{\alpha}{\vectr{\xi}}\idxgsf{i}{m}{\alpha}}{x^{j}}\abs{\fnof{\matr{J}}{\vectr{\xi}}}\exteriorderiv{\vectr{\xi}} \\
  %body force
  -\gsum{e=1}{E}{\gint{\vectr{0}}{\vectr{1}}{\fnof{f^{i}}{\vectr{\xi},t}\idxgbfn{i}{m}{\alpha}{\vectr{\xi}}\idxgsf{i}{m}{\alpha}\abs{\fnof{\matr{J}}{\vectr{\xi}}}}{\vectr{\xi}}}\\
  %boundary terms
  -\gsum{f=1}{F}{\gint{\vectr{0}}{\vectr{1}}{\idxgbfn{i}{o}{\gamma}{\vectr{\xi}}
      \fnof{\idxnodedof{\bar{q}}{i}{o}{\gamma}}{t}\idxgsf{i}{o}{\gamma}\idxgbfn{i}{m}{\alpha}{\vectr{\xi}}\idxgsf{i}{m}{\alpha}\abs{\fnof{\matr{J}}{\vectr{\xi}}}}{\vectr{\xi}}}\\
  %Christofell terms
  -\dsuml{e=1}{E} \dintl{\vectr{0}}{\vectr{1}} \fnof{\rho}{\vectr{\xi}}
  \idxgbfn{j}{n}{\beta}{\vectr{\xi}}
  \fnof{\idxnodedof{v}{j}{n}{\beta}}{t}\idxgsf{j}{n}{\beta}\fnof{\christoffel{i}{j}{h}}{\vectr{\xi}}
  \idxgbfn{h}{n}{\beta}{\vectr{\xi}}\fnof{\idxnodedof{v}{h}{n}{\beta}}{t}\idxgsf{h}{n}{\beta}
  \idxgbfn{i}{m}{\alpha}{\vectr{\xi}}\idxgsf{i}{m}{\alpha}\\
  +\fnof{\mu}{\vectr{\xi}}\left(\fnof{g^{jk}}{\vectr{\xi}}\fnof{\christoffel{i}{k}{h}}{\vectr{\xi}}
    \idxgbfn{h}{n}{\beta}{\vectr{\xi}}
    \fnof{\idxnodedof{v}{h}{n}{\beta}}{t}\idxgsf{h}{n}{\beta}\right.\\
    \left.+\fnof{g^{ik}}{\vectr{\xi}}\fnof{\christoffel{j}{k}{h}}{\vectr{\xi}}\idxgbfn{h}{n}{\beta}{\vectr{\xi}}
    \fnof{\idxnodedof{v}{h}{n}{\beta}}{t}\idxgsf{h}{n}{\beta}\right)
    \delby{\idxgbfn{i}{m}{\alpha}{\vectr{\xi}}\idxgsf{i}{m}{\alpha}}{x^{j}}
    \abs{\fnof{\matr{J}}{\vectr{\xi}}}\exteriorderiv{\vectr{\xi}}\\
  -\gsum{f=1}{F}{\gint{\vectr{0}}{\vectr{1}}{\fnof{\mu}{\vectr{\xi}}
      \fnof{g^{jk}}{\vectr{\xi}}\fnof{\christoffel{i}{k}{h}}{\vectr{\xi}}\idxgbfn{h}{n}{\beta}{\vectr{\xi}}
      \fnof{\idxnodedof{v}{h}{n}{\beta}}{t}\idxgsf{h}{n}{\beta}\fnof{n_{j}}{\vectr{\xi}}
      \idxgbfn{i}{m}{\alpha}{\vectr{\xi}}\idxgsf{i}{m}{\alpha}\abs{\fnof{\matr{J}}{\vectr{\xi}}}}{\vectr{\xi}}}
  =0
  \label{eqn:FEM4NSE}
\end{multline}
where the appropriate face bases are used for the surface integrals. Note that
$\fnof{\matr{J}}{\vectr{\xi}}$ represents the jacobian matrix from $\vectr{x}$
to $\vectr{\xi}$ coordinates.

Taking constant terms outside the integral signs and converting derivatives
with respect to $\vectr{x}$ to derivatives with respect to $\vectr{\xi}$ gives
\begin{multline}
  %time dependence
  \gsum{e=1}{E}{\idxgsf{i}{m}{\alpha}\idxgsf{i}{n}{\beta}\fnof{\idxnodedof{\dot{v}}{i}{n}{\beta}}{t}
    \gint{\vectr{0}}{\vectr{1}}{\fnof{\rho}{\vectr{\xi}}
      \idxgbfn{i}{m}{\alpha}{\vectr{\xi}}\idxgbfn{i}{n}{\beta}{\vectr{\xi}}
      \abs{\fnof{\matr{J}}{\vectr{\xi}}}}{\vectr{\xi}}}\\
  %convective term
  +\gsum{e=1}{E}{\idxgsf{i}{m}{\alpha}\gint{\vectr{0}}{\vectr{1}}{\fnof{\rho}{\vectr{\xi}}
      \idxgbfn{i}{m}{\alpha}{\vectr{\xi}}\idxgbfn{j}{n}{\beta}{\vectr{\xi}}
      \fnof{\idxnodedof{v}{j}{n}{\beta}}{t}\idxgsf{j}{n}{\beta}\delby{\idxgbfn{i}{n}{\beta}{\vectr{\xi}}}{\xi^{r}}
      \delby{\xi^{r}}{x^{j}}\fnof{\idxnodedof{v}{i}{n}{\beta}}{t}\idxgsf{i}{n}{\beta}
      \abs{\fnof{\matr{J}}{\vectr{\xi}}}}{\vectr{\xi}}}\\
  %pressure
  +\gsum{e=1}{E}{\idxgsf{i}{m}{\alpha}\gsf{n}{\beta}\fnof{\nodedof{p}{n}{\beta}}{t}\gint{\vectr{0}}{\vectr{1}}{
      \fnof{g^{ik}}{\vectr{\xi}}\delby{\idxgbfn{i}{m}{\alpha}{\vectr{\xi}}}{\xi^{r}}\delby{\xi^{r}}{x^{k}}
      \altgbfn{n}{\beta}{\vectr{\xi}}\abs{\fnof{\matr{J}}{\vectr{\xi}}}}{\vectr{\xi}}}\\
  %viscous stress
  -\gsum{e=1}{E}{\idxgsf{i}{m}{\alpha}\idxgsf{i}{n}{\beta}\fnof{\idxnodedof{v}{i}{n}{\beta}}{t}\gint{\vectr{0}}{\vectr{1}}{
      \fnof{\mu}{\vectr{\xi}}\fnof{g^{jk}}{\vectr{\xi}}\delby{\idxgbfn{i}{m}{\alpha}{\vectr{\xi}}}{\xi^{s}}
      \delby{\xi^{s}}{x^{j}}\delby{\idxgbfn{i}{n}{\beta}{\vectr{\xi}}}{\xi^{r}}\delby{\xi^{r}}{x^{k}}
      \abs{\fnof{\matr{J}}{\vectr{\xi}}}}{\vectr{\xi}}}\\
  -\gsum{e=1}{E}{\idxgsf{i}{m}{\alpha}\idxgsf{j}{n}{\beta}\fnof{\idxnodedof{v}{j}{n}{\beta}}{t}
    \gint{\vectr{0}}{\vectr{1}}{\fnof{\mu}{\vectr{\xi}}\fnof{g^{ik}}{\vectr{\xi}}\delby{\idxgbfn{i}{m}{\alpha}{\vectr{\xi}}}{\xi^{s}}
      \delby{\xi^{s}}{x^{j}}\delby{\idxgbfn{j}{n}{\beta}{\vectr{\xi}}}{\xi^{r}}\delby{\xi^{r}}{x^{k}}
      \abs{\fnof{\matr{J}}{\vectr{\xi}}}}{\vectr{\xi}}} \\
  %body force
  -\gsum{e=1}{E}{\idxgsf{i}{m}{\alpha}\gint{\vectr{0}}{\vectr{1}}{\fnof{f^{i}}{\vectr{\xi},t}\idxgbfn{i}{m}{\alpha}{\vectr{\xi}}\abs{\fnof{\matr{J}}{\vectr{\xi}}}}{\vectr{\xi}}}\\
  %boundary terms
  -\gsum{f=1}{F}{\idxgsf{i}{m}{\alpha}\idxgsf{i}{o}{\gamma}\fnof{\idxnodedof{\bar{q}}{i}{o}{\gamma}}{t}
    \gint{\vectr{0}}{\vectr{1}}{\idxgbfn{i}{m}{\alpha}{\vectr{\xi}}\idxgbfn{i}{o}{\gamma}{\vectr{\xi}}
      \abs{\fnof{\matr{J}}{\vectr{\xi}}}}{\vectr{\xi}}}\\
  %Christofell terms
  -\gsum{e=1}{E}{\idxgsf{i}{m}{\alpha}\gint{\vectr{0}}{\vectr{1}}{\fnof{\rho}{\vectr{\xi}}
      \idxgbfn{i}{m}{\alpha}{\vectr{\xi}}\idxgbfn{j}{n}{\beta}{\vectr{\xi}}
      \fnof{\idxnodedof{v}{j}{n}{\beta}}{t}\idxgsf{j}{n}{\beta}
      \fnof{\christoffel{i}{j}{h}}{\vectr{\xi}}
      \idxgbfn{h}{n}{\beta}{\vectr{\xi}}\fnof{\idxnodedof{v}{h}{n}{\beta}}{t}\idxgsf{h}{n}{\beta}
      \abs{\fnof{\matr{J}}{\vectr{\xi}}}}{\vectr{\xi}}}\\
  -\gsum{e=1}{E}{\idxgsf{i}{m}{\alpha}\idxgsf{h}{n}{\beta}\fnof{\idxnodedof{v}{h}{n}{\beta}}{t}
    \gint{\vectr{0}}{\vectr{1}}{\fnof{\mu}{\vectr{\xi}}\fnof{g^{jk}}{\vectr{\xi}}
      \delby{\idxgbfn{i}{m}{\alpha}{\vectr{\xi}}}{\xi^{s}}\delby{\xi^{s}}{x^{j}}
      \fnof{\christoffel{i}{k}{h}}{\vectr{\xi}}\idxgbfn{h}{n}{\beta}{\vectr{\xi}}    
    \abs{\fnof{\matr{J}}{\vectr{\xi}}}}{\vectr{\xi}}}\\
  -\gsum{e=1}{E}{\idxgsf{i}{m}{\alpha}\idxgsf{h}{n}{\beta}\fnof{\idxnodedof{v}{h}{n}{\beta}}{t}
    \gint{}{}{\fnof{\mu}{\vectr{\xi}}\fnof{g^{ik}}{\vectr{\xi}}
      \delby{\idxgbfn{i}{m}{\alpha}{\vectr{\xi}}}{\xi^{s}}\delby{\xi^{s}}{x^{j}}
      \fnof{\christoffel{j}{k}{h}}{\vectr{\xi}}\idxgbfn{h}{n}{\beta}{\vectr{\xi}}        
    \abs{\fnof{\matr{J}}{\vectr{\xi}}}}{\vectr{\xi}}}\\
  -\gsum{f=1}{F}{\idxgsf{i}{m}{\alpha}\idxgsf{h}{n}{\beta}\fnof{\idxnodedof{v}{h}{n}{\beta}}{t}
    \gint{\vectr{0}}{\vectr{1}}{\fnof{\mu}{\vectr{\xi}}\fnof{g^{jk}}{\vectr{\xi}}\idxgbfn{i}{m}{\alpha}{\vectr{\xi}}
      \fnof{\christoffel{i}{k}{h}}{\vectr{\xi}}\idxgbfn{h}{n}{\beta}{\vectr{\xi}}\fnof{n_{j}}{\vectr{\xi}}
      \abs{\fnof{\matr{J}}{\vectr{\xi}}}}{\vectr{\xi}}}=0
  \label{eqn:NSEFEM4}
\end{multline}

If we assume a rectangular cartesian coordinate system, this simplifies
significantly, as the metric tensor $g^{jk}$ will become $\delta^{jk}$ and the
Christoffel symbols $\christoffel{i}{j}{k}$ will all be zero. This gives:
\begin{multline}
  %time dependence
  \gsum{e=1}{E}{\idxgsf{i}{m}{\alpha}\idxgsf{i}{n}{\beta}\fnof{\idxnodedof{\dot{v}}{i}{n}{\beta}}{t}
    \gint{\vectr{0}}{\vectr{1}}{\fnof{\rho}{\vectr{\xi}}
      \idxgbfn{i}{m}{\alpha}{\vectr{\xi}}\idxgbfn{i}{n}{\beta}{\vectr{\xi}}
      \abs{\fnof{\matr{J}}{\vectr{\xi}}}}{\vectr{\xi}}}\\
  %convective term
  +\gsum{e=1}{E}{\idxgsf{i}{m}{\alpha}\gint{\vectr{0}}{\vectr{1}}{\fnof{\rho}{\vectr{\xi}}
      \idxgbfn{i}{m}{\alpha}{\vectr{\xi}}\idxgbfn{j}{n}{\beta}{\vectr{\xi}}
      \fnof{\idxnodedof{v}{j}{n}{\beta}}{t}\idxgsf{j}{n}{\beta}\delby{\idxgbfn{i}{n}{\beta}{\vectr{\xi}}}{\xi^{r}}
      \delby{\xi^{r}}{x^{j}}\fnof{\idxnodedof{v}{i}{n}{\beta}}{t}\idxgsf{i}{n}{\beta}
      \abs{\fnof{\matr{J}}{\vectr{\xi}}}}{\vectr{\xi}}}\\
  %pressure
  +\gsum{e=1}{E}{\idxgsf{i}{m}{\alpha}\gsf{n}{\beta}\fnof{\nodedof{p}{n}{\beta}}{t}\gint{\vectr{0}}{\vectr{1}}{
      \fnof{\delta^{ik}}{\vectr{\xi}}\delby{\idxgbfn{i}{m}{\alpha}{\vectr{\xi}}}{\xi^{r}}\delby{\xi^{r}}{x^{k}}
      \altgbfn{n}{\beta}{\vectr{\xi}}\abs{\fnof{\matr{J}}{\vectr{\xi}}}}{\vectr{\xi}}}\\
  %viscous stress
  -\gsum{e=1}{E}{\idxgsf{i}{m}{\alpha}\idxgsf{i}{n}{\beta}\fnof{\idxnodedof{v}{i}{n}{\beta}}{t}\gint{\vectr{0}}{\vectr{1}}{
      \fnof{\mu}{\vectr{\xi}}\fnof{\delta^{jk}}{\vectr{\xi}}\delby{\idxgbfn{i}{m}{\alpha}{\vectr{\xi}}}{\xi^{s}}
      \delby{\xi^{s}}{x^{j}}\delby{\idxgbfn{i}{n}{\beta}{\vectr{\xi}}}{\xi^{r}}\delby{\xi^{r}}{x^{k}}
      \abs{\fnof{\matr{J}}{\vectr{\xi}}}}{\vectr{\xi}}}\\
  -\gsum{e=1}{E}{\idxgsf{i}{m}{\alpha}\idxgsf{j}{n}{\beta}\fnof{\idxnodedof{v}{j}{n}{\beta}}{t}
    \gint{\vectr{0}}{\vectr{1}}{\fnof{\mu}{\vectr{\xi}}\fnof{\delta^{ik}}{\vectr{\xi}}\delby{\idxgbfn{i}{m}{\alpha}{\vectr{\xi}}}{\xi^{s}}
      \delby{\xi^{s}}{x^{j}}\delby{\idxgbfn{j}{n}{\beta}{\vectr{\xi}}}{\xi^{r}}\delby{\xi^{r}}{x^{k}}
      \abs{\fnof{\matr{J}}{\vectr{\xi}}}}{\vectr{\xi}}} \\
  %body force
  -\gsum{e=1}{E}{\idxgsf{i}{m}{\alpha}\gint{\vectr{0}}{\vectr{1}}{\fnof{f^{i}}{\vectr{\xi}}\idxgbfn{i}{m}{\alpha}{\vectr{\xi}}\abs{\fnof{\matr{J}}{\vectr{\xi}}}}{\vectr{\xi}}}\\
  %boundary terms
  -\gsum{f=1}{F}{\idxgsf{i}{m}{\alpha}\idxgsf{i}{o}{\gamma}\fnof{\idxnodedof{\bar{q}}{i}{o}{\gamma}}{t}
    \gint{\vectr{0}}{\vectr{1}}{\idxgbfn{i}{m}{\alpha}{\vectr{\xi}}\idxgbfn{i}{o}{\gamma}{\vectr{\xi}}
      \abs{\fnof{\matr{J}}{\vectr{\xi}}}}{\vectr{\xi}}}=0
  \label{eqn:NSEFEM5RC}
\end{multline}

\subsection{General form}

We now seek to assemble this into the corresponding general form for dynamic equations, as outlined in \secref{sec:dynamicequations}:

\begin{equation}
  \matr{M}\fnof{\ddot{\vectr{d}}}{t}+\matr{C}\fnof{\dot{\vectr{d}}}{t}+\matr{K}\fnof{\vectr{d}}{t}+
  \fnof{\vectr{g}}{\fnof{\vectr{d}}{t}}+\fnof{\vectr{f}}{t}=\vectr{0}
  \label{eqn:NSEGeneralDynamic}
\end{equation}
where $\dot{\vectr{d}}(t)$ and $\ddot{\vectr{d}}(t)$ represent the first and
second derivatives (respectively) of the degrees of freedom vector
$\vectr{d}(t)$, which consists of the dependent variables
$\vectr{v}(\vectr{x},t)$ and $p(\vectr{x},t)$. $\matr{M}$ is the mass matrix, which
provides the shape function based weights and $\matr{C}$ is the transient
damping matrix. $\matr{K}$ represents the stiffness matrix, which will contain
the linear parts of the operators. $\fnof{\vectr{g}}{\fnof{\vectr{d}}{t}}$ is
the nonlinear vector function that will be used to represent the convective
term and $\fnof{\vectr{f}}{t}$ the forcing vector.

We will assume cartesian coordinates $\vectr{x}=\bbrac{x^{1},x^{2},x^{3}}$ and denote the
corresponding velocity components $\vectr{v}=\bbrac{v^{1},v^{2},v^{3}}$, with $N$
representing the number of velocity DOFs and $M$ the number of pressure
DOFs. The vector $\fnof{\vectr{d}}{t}$ then becomes:
\begin{equation}
  \fnof{\vectr{d}}{t}=\begin{bmatrix}
  \fnof{\vectr{v}^{1}}{t} \\
  \fnof{\vectr{v}^{2}}{t} \\
  \fnof{\vectr{v}^{3}}{t} \\
  \fnof{\vectr{p}}{t}
  \end{bmatrix}
\end{equation}
and the vector $\fnof{\dot{\vectr{d}}}{t}$ then becomes:
\begin{equation}
  \fnof{\dot{\vectr{d}}}{t}=\begin{bmatrix}
  \fnof{\dot{\vectr{v}}^{1}}{t} \\
  \fnof{\dot{\vectr{v}}^{2}}{t} \\
  \fnof{\dot{\vectr{v}}^{3}}{t} \\
  \fnof{\dot{\vectr{p}}}{t}
  \end{bmatrix}
\end{equation}

All the components of \ref{eqn:NSEGeneralDynamic} will depend on the
number of dimensions, $n_{dim}$.

Returning to the general case, the Navier-Stokes equations do not have a mass
matrix $\matr{M}$ and so
\begin{equation}
  \matr{M}=\matr{0}
\end{equation}

The elemental damping matrix $\matr{C}$ is given by
\begin{equation}
  C^{i\alpha\beta}_{mn}=\matr{C}^{i}=\idxgsf{i}{m}{\alpha}\idxgsf{i}{n}{\beta}
  \gint{\vectr{0}}{\vectr{1}}{\fnof{\rho}{\vectr{\xi}}
    \idxgbfn{i}{m}{\alpha}{\vectr{\xi}}\idxgbfn{i}{n}{\beta}{\vectr{\xi}}
    \abs{\fnof{\matr{J}}{\vectr{\xi}}}}{\vectr{\xi}}
\end{equation}

The element stiffness matrix $\matr{K}$ will contain the linear components of
the system. For the classic Galerkin formulation this includes the viscous
terms and the pressure gradient term. As the velocity and pressure DOFs are
both included in $\vectr{d}$, $\matr{K}$ will be assembled so that the
corresponding operators are applied to the DOFs according. If we combine the
conservation of mass equations we obtain a K of the form.

\begin{equation}
  \matr{K}=
   \begin{bmatrix}
     \matr{A}^{1}+\matr{D}^{11} & \matr{D}^{12} & \matr{D}^{13} & \matr{B}^{1}\\
     \matr{D}^{21} & \matr{A}^{2}+\matr{D}^{22} & \matr{D}^{23} & \matr{B}^{2}\\
     \matr{D}^{31} & \matr{D}^{32} & \matr{A}^{3}+\matr{D}^{13} & \matr{B}^{3}\\
     -\transpose{\matr{B}^{1}} & -\transpose{\matr{B}^{2}} & -\transpose{\matr{B}^{3}} & \matr{0}
   \end{bmatrix}
\end{equation}

where
\begin{equation}
  A^{i\alpha\beta}_{mn}=\matr{A}^{i}=\idxgsf{i}{m}{\alpha}\idxgsf{i}{n}{\beta}\gint{\vectr{0}}{\vectr{1}}{
      \fnof{\mu}{\vectr{\xi}}\fnof{g^{jk}}{\vectr{\xi}}\delby{\idxgbfn{i}{m}{\alpha}{\vectr{\xi}}}{\xi^{s}}
      \delby{\xi^{s}}{x^{j}}\delby{\idxgbfn{i}{n}{\beta}{\vectr{\xi}}}{\xi^{r}}\delby{\xi^{r}}{x^{k}}
      \abs{\fnof{\matr{J}}{\vectr{\xi}}}}{\vectr{\xi}}
\end{equation}
and
\begin{equation}
  B^{i\alpha\beta}_{mn}=\matr{B}^{i}=\idxgsf{i}{m}{\alpha}\gsf{n}{\beta}\gint{\vectr{0}}{\vectr{1}}{
      \fnof{g^{ik}}{\vectr{\xi}}\delby{\idxgbfn{i}{m}{\alpha}{\vectr{\xi}}}{\xi^{r}}\delby{\xi^{r}}{x^{k}}
      \altgbfn{n}{\beta}{\vectr{\xi}}\abs{\fnof{\matr{J}}{\vectr{\xi}}}}{\vectr{\xi}}
\end{equation}
and
\begin{equation}
  D^{ij\alpha\beta}_{mn}=\bar{D}^{ij\alpha\beta}_{mn}+\tilde{D}^{ij\alpha\beta}_{mn}
\end{equation}
where
\begin{equation}
  \bar{D}^{ij\alpha\beta}_{mn}=\bar{\matr{D}}^{ij}=\idxgsf{i}{m}{\alpha}\idxgsf{j}{n}{\beta}
  \gint{\vectr{0}}{\vectr{1}}{\fnof{\mu}{\vectr{\xi}}\fnof{g^{ik}}{\vectr{\xi}}\delby{\idxgbfn{i}{m}{\alpha}{\vectr{\xi}}}{\xi^{s}}
    \delby{\xi^{s}}{x^{j}}\delby{\idxgbfn{j}{n}{\beta}{\vectr{\xi}}}{\xi^{r}}\delby{\xi^{r}}{x^{k}}
    \abs{\fnof{\matr{J}}{\vectr{\xi}}}}{\vectr{\xi}}
\end{equation}
and
\begin{multline}
  \tilde{D}^{ij\alpha\beta}_{mn}=\tilde{\matr{D}}^{ij}=\idxgsf{i}{m}{\alpha}\idxgsf{j}{n}{\beta}
    \dintl{\vectr{0}}{\vectr{1}}\fnof{\mu}{\vectr{\xi}}\left(\fnof{g^{hk}}{\vectr{\xi}}
      \delby{\idxgbfn{i}{m}{\alpha}{\vectr{\xi}}}{\xi^{s}}\delby{\xi^{s}}{x^{h}}
      \fnof{\christoffel{i}{k}{j}}{\vectr{\xi}}\idxgbfn{j}{n}{\beta}{\vectr{\xi}}\right.\\
      \left.+\fnof{g^{ik}}{\vectr{\xi}}
      \delby{\idxgbfn{i}{m}{\alpha}{\vectr{\xi}}}{\xi^{s}}\delby{\xi^{s}}{x^{h}}
      \fnof{\christoffel{h}{k}{j}}{\vectr{\xi}}\idxgbfn{j}{n}{\beta}{\vectr{\xi}}\right)   
    \abs{\fnof{\matr{J}}{\vectr{\xi}}}\exteriorderiv{\vectr{\xi}}
\end{multline}

The nonlinear vector, $\fnof{\vectr{g}}{\fnof{\vectr{d}}{t}}$ will provide the
convective operators and for the standard Galerkin formulation is given by
\begin{equation}
  g^{i\alpha}_{m}=\bar{g}^{i\alpha}_{m}+\tilde{g}^{i\alpha}_{m}
\end{equation}
where
\begin{multline}
  \bar{g}^{i\alpha}_{m}=\bar{\vectr{g}}^{i}=\idxgsf{i}{m}{\alpha}\dintl{\vectr{0}}{\vectr{1}}\fnof{\rho}{\vectr{\xi}}
    \idxgbfn{i}{m}{\alpha}{\vectr{\xi}}\idxgbfn{j}{n}{\beta}{\vectr{\xi}}
    \fnof{\idxnodedof{v}{j}{n}{\beta}}{t}\idxgsf{j}{n}{\beta}\\
    \delby{\idxgbfn{i}{n}{\beta}{\vectr{\xi}}}{\xi^{r}}
    \delby{\xi^{r}}{x^{j}}\fnof{\idxnodedof{v}{i}{n}{\beta}}{t}\idxgsf{i}{n}{\beta}
    \abs{\fnof{\matr{J}}{\vectr{\xi}}}\exteriorderiv{\vectr{\xi}}
\end{multline}
and
\begin{multline}
  \tilde{g}^{i\alpha}_{m}=\tilde{\vectr{g}}^{i}=\idxgsf{i}{m}{\alpha}\dintl{\vectr{0}}{\vectr{1}}\fnof{\rho}{\vectr{\xi}}
    \idxgbfn{i}{m}{\alpha}{\vectr{\xi}}\idxgbfn{j}{n}{\beta}{\vectr{\xi}}
    \fnof{\idxnodedof{v}{j}{n}{\beta}}{t}\idxgsf{j}{n}{\beta}\\
    \fnof{\christoffel{i}{j}{h}}{\vectr{\xi}}
    \idxgbfn{h}{n}{\beta}{\vectr{\xi}}\fnof{\idxnodedof{v}{h}{n}{\beta}}{t}\idxgsf{h}{n}{\beta}
    \abs{\fnof{\matr{J}}{\vectr{\xi}}}\exteriorderiv{\vectr{\xi}}
\end{multline}

The forcing vector $\fnof{\vectr{f}}{t}$ is given by
\begin{equation}
  f^{i\alpha}_{m}=\vectr{f}^{i}=\idxgsf{i}{m}{\alpha}\gint{\vectr{0}}{\vectr{1}}{\fnof{f^{i}}{\vectr{\xi},t}\idxgbfn{i}{m}{\alpha}{\vectr{\xi}}\abs{\fnof{\matr{J}}{\vectr{\xi}}}}{\vectr{\xi}}
\end{equation}

The final system is given by
\begin{multline}
  \begin{bmatrix}
    \matr{C}^{1} & \matr{0} & \matr{0} & \matr{0} \\
    \matr{0} & \matr{C}^{2} & \matr{0} & \matr{0} \\
    \matr{0} & \matr{0} & \matr{C}^{3} & \matr{0} \\
    \matr{0} & \matr{0} & \matr{0} & \matr{0}
  \end{bmatrix}\begin{bmatrix}
      \fnof{\dot{\vectr{v}}^{1}}{t} \\
      \fnof{\dot{\vectr{v}}^{2}}{t} \\
      \fnof{\dot{\vectr{v}}^{3}}{t} \\
      \vectr{0}
  \end{bmatrix}+\begin{bmatrix}
     \matr{A}^{1}+\matr{D}^{11} & \matr{D}^{12} & \matr{D}^{13} & \matr{B}^{1}\\
     \matr{D}^{21} & \matr{A}^{2}+\matr{D}^{22} & \matr{D}^{23} & \matr{B}^{2}\\
     \matr{D}^{31} & \matr{D}^{32} & \matr{A}^{3}+\matr{D}^{13} & \matr{B}^{3}\\
     -\transpose{\matr{B}^{1}} & -\transpose{\matr{B}^{2}} & -\transpose{\matr{B}^{3}} & \matr{0}
  \end{bmatrix}\begin{bmatrix}
    \fnof{\vectr{v}^{1}}{t} \\
    \fnof{\vectr{v}^{2}}{t} \\
    \fnof{\vectr{v}^{3}}{t} \\
    \fnof{\vectr{p}}{t}
  \end{bmatrix}\\
  +\begin{bmatrix}
    \fnof{\vectr{g}^{1}}{\fnof{\vectr{d}}{t}} \\
    \fnof{\vectr{g}^{2}}{\fnof{\vectr{d}}{t}} \\
    \fnof{\vectr{g}^{3}}{\fnof{\vectr{d}}{t}} \\
    \vectr{0}
  \end{bmatrix}+\begin{bmatrix}
    \fnof{\vectr{f}^{1}}{t} \\
    \fnof{\vectr{f}^{2}}{t} \\
    \fnof{\vectr{f}^{3}}{t} \\
    \vectr{0}
  \end{bmatrix}=\begin{bmatrix}
  \vectr{0} \\
  \vectr{0} \\
  \vectr{0} \\
  \vectr{0}
  \end{bmatrix}
\end{multline}

\subsection{Streamline Upwind/Petrov-Galerkin (SUPG)}

For convection dominated flows, it is often useful to modify the Galerkin
formulation account for numerical problems associated with large asymmetric
velocity gradients. We describe augmenting the Galerkin Navier-Stokes
formulation to form a Petrov-Galerkin form, which uses a different test
function for the convective term to stabilize the algorithm with a balancing
diffusive term. Please refer to \secref{sec:Convective Transport} for
more details.

For the Navier-Stokes equations (and most nonlinear problems), the use of a
Petrov-Galerkin formulation becomes more difficult than for linear cases like
advection-diffusion, as consistent weighting can cause instabilities when used
with additional terms like body forces \cite{heinrich:1999}. These issues can
be overcome with more complex Petrov-Galerkin formulations, as those derived
by Hughes and colleagues in the 1980s. A useful alternative route is to apply
SUPG weights to the convective operator in elements as a function of each
element's Reynolds number (referred to subsequently as the cell Reynolds
number). This can provide the desired stabilizing effect in the direction of
large velocity gradients but can be easily implemented and is practically
useful.

Assuming cartesian coordinates and applying Petrov-Galerkin weights to the
convective term, the finite element formulation described in equation
\ref{eqn:FEMNSE} can be written:

\begin{multline}
  \gsum{e=1}{E}{\gint{\Omega_{e}}{}{\rho\dot{v}^{i}w^{i}}{\Omega}}
 +\gsum{e=1}{E}{\gint{\Omega_{e}}{}{{\rho}g^{k}_{l}{v^{l}}\partialderiv{v^{i}}{k}\pbrac{w^{i}+\Psi^{i}}}{\Omega}}
 +\gsum{e=1}{E}{\gint{\Omega_{e}}{}{g^{ik}{p}\partialderiv{w^{i}}{k}}{\Omega}}\\
 -\gsum{e=1}{E}{\gint{\Omega_{e}}{}{\mu\pbrac{g^{jk}\partialderiv{v^{i}}{k}+g^{ik}\partialderiv{v^{j}}{k}}\partialderiv{w^{i}}{j}}{\Omega}}
 -\gsum{e=1}{E}{\gint{\Omega_{e}}{}{{\fnof{f^{i}}{t}}{w^{i}}}{\Omega}}
 -\gsum{f=1}{F}{\gint{\Gamma_{N_{f}}}{}{\pbrac{\mu g^{jk}\partialderiv{v^{i}}{k}-pg^{ij}}n_{j}w^{i}}{\Gamma}}\\
 -\gsum{e=1}{E}{\gint{\Omega_{e}}{}{\rho
   g^{k}_{l}v^{l}\christoffel{i}{k}{h}v^{h}w^{i}+\mu\pbrac{g^{jk}\christoffel{i}{k}{h}v^{h}+
     g^{ik}\christoffel{j}{k}{h}v^{h}}\partialderiv{w^{i}}{j}}{\Omega}}
 -\gsum{f=1}{F}{\gint{\Gamma_{N_{f}}}{}{\mu g^{jk}\christoffel{i}{k}{h}v^{h}n_{j}w^{i}}{\Gamma}}=0
  \label{eqn:NSEPGFEM1}
\end{multline}

where $w^{i}$ will be the classic Galerkin weight and $\Psi^{i}$ will be the
Petrov-Galerkin for the Navier-Stokes equations. We must also define the
Peclet number, $Pe$, for the Navier-Stokes equations to describe the ratio
of the convective:viscous operators. Here, the effective Peclet number will be
based on the cell Reynolds number:

\begin{equation}
  Re = \frac{\rho\bar{V}L}{\mu}  =\frac{\bar{V}L}{\nu}
\end{equation}
where $\bar{V}$ is the characteristic velocity and $\nu=\frac{\mu}{\rho}$ is
the kinematic viscosity. $L$ will be the characteristic length scale for a
given element. The effective Peclet number $Pe$ will be:

\begin{equation}
  Pe = \frac{Re}{2} = \frac{\bar{V}L}{2\nu}
\end{equation}

We will retain the same value for
$\alpha={\coth{\frac{Pe}{2}} - \frac{2}{Pe}}$ as found for the
linearized advection-diffusion equation.

\begin{equation}
 \label{PGTest}
  \Psi_i = \pbrac{\frac{\alpha{L}}{2}{\frac{u_j}{\abs{u_j}}}}{\partialderiv{w_{i}}{k}}
\end{equation},

\Eqnref{eqn:NSEPGFEM1} then becomes after integration and simplification:

\begin{multline}
%time dependence
  \dsum_{e=1}^{E}{\idxnodedof{\dot{u}}{i}{n}{\beta}}(t){\idxgsf{i}{n}{\beta}}{\idxgsf{i}{m}{\alpha}}
  \gint{\vectr{0}}{\vectr{1}}{\fnof{\rho}{\vectr{\xi}}{\idxgbfn{i}{n}{\beta}{\vectr{\xi}}\idxgbfn{i}{m}{\alpha}{\vectr{\xi}}
  \abs{\fnof{\matr{J}}{\vectr{\xi}}}}}{\vectr{\xi}}\\
 %convective term
  -\dsum_{e=1}^{E}{\idxnodedof{u}{j}{n}{\beta}}(t){\idxgsf{i}{n}{\beta}}^2{\idxgsf{i}{m}{\alpha}}
   \dintl{\vectr{0}}{\vectr{1}}\delta^{jk}{\fnof{\rho}{\vectr{\xi}}\idxgbfn{j}{n}{\beta}{\vectr{\xi}}
    \idxgbfn{i}{m}{\alpha}{\vectr{\xi}}
     \pbrac{\delby{\idxgbfn{i}{n}{\beta}{\vectr{\xi}}}{x^{k}}}
  \abs{\fnof{\matr{J}}{\vectr{\xi}}}}d{\vectr{\xi}}\\
 %P-G term
  -\dsum_{e=1}^{E}{\idxnodedof{u}{j}{n}{\beta}}(t){\idxgsf{i}{n}{\beta}}^2{\idxgsf{i}{m}{\alpha}}
   \dintl{\vectr{0}}{\vectr{1}}\delta^{jk}{\fnof{\rho}{\vectr{\xi}}\idxgbfn{j}{n}{\beta}{\vectr{\xi}}
    \pbrac{\frac{(\alpha){L(\vectr{\xi})}}{2}{\frac{u_j}{\abs{u_j}}}}\idxgbfn{i}{m}{\alpha}{\vectr{\xi}}
     \pbrac{\delby{\idxgbfn{i}{n}{\beta}{\vectr{\xi}}}{x^{k}}}
  \abs{\fnof{\matr{J}}{\vectr{\xi}}}}d{\vectr{\xi}}\\
 %pressure
    +\dsum_{e=1}^{E}{\fnof{\idxnodedof{p}{}{n}{\beta}}{t}\idxgsf{i}{n}{\beta}\idxgsf{i}{m}{\alpha}}
    \dintl{\vectr{0}}{\vectr{1}}\delta^{jk}\phi_{in}^{\beta}({\vectr{\xi}})
    \pbrac{\delby{\psi_{im}^{\alpha}({\vectr{\xi}})}{x^{k}}}\abs{\fnof{\matr{J}}{\vectr{\xi}}}d\vectr{\xi}\\
 % %viscous stress
    -\dsum_{e=1}^{E}\fnof{\idxnodedof{u}{i}{n}{\beta}}{t}\idxgsf{i}{n}{\beta}\idxgsf{i}{m}{\alpha}
    \dintl{\vectr{0}}{\vectr{1}}{\delta^{jk}\fnof{\mu}{\vectr{\xi}}}\pbrac{\delby{\idxgbfn{i}{n}{\beta}{\vectr{\xi}}}{x^{j}}}
      \pbrac{\delby{\idxgbfn{i}{m}{\alpha}{\vectr{\xi}}}{x^{k}}}
      \abs{\fnof{\matr{J}}{\vectr{\xi}}}d{\vectr{\xi}}\\
% %boundary terms
  =\dsum_{f=1}^{F}\fnof{\idxnodedof{q}{i}{o}{\gamma}}{t}\idxgsf{i}{m}{\alpha}\idxgsf{i}{o}{\gamma}
   \gint{\vectr{0}}{\vectr{1}}{\gbfn{m}{\alpha}{\vectr{\xi}}\gbfn{o}{\gamma}{\vectr{\xi}}
    \abs{\fnof{\matr{J}}{\vectr{\xi}}}}{\vectr{\xi}}\\
  \label{eqn:NSEFEM4}
\end{multline}

Notice the integration by parts is done only to the standard Galerkin weights-
this is because the artificial diffusion added by the Petrov-Galerkin terms
should only contribute within the elements

The use of the Petrov-Galerkin formulation in this way allows for the
stabilization of moderately convective problems. However, it also results in
nonsymmetric mass matrices with the additional term, making classical
mass-lumping more difficult. As a result, the use of explicit methods also
becomes more difficult for time-dependent problems.


\subsection{Arbitrary Lagrangian-Eulerian (ALE) Formulation}
%ALE form
Whereas \eqnref{eqn:NavierStokesequation1} has been formulated in Eulerian
form, moving domain approaches often require the ALE modification taking an
additional term into account, depending on the fluid density $\rho$ and a
correction velocity $\vectr{v}^{*}$ which yields to:
\begin{equation}
  \rho\pbrac{\dotprod{\pbrac{\vectr{v}-\vectr{v}^*}}{\grad}}\vectr{u}=\vectr{f}-\gradient{}{p}+\mu\laplacian{}{\vectr{u}}
  \label{eqn:NavierStokesequationALE}
\end{equation}
So far, the nonlinear term in \eqnref{eqn:NavierStokesequation1} represents
the fluid spatial acceleration only. \eqnref{eqn:NavierStokesequation2} now
also takes the dynamic inertia terms into account
\begin{equation}
  \rho\delby{\vectr{u}}{t}+ \dotprod{\rho\pbrac{\vectr{u}-\vectr{u}^*}}{\grad}\vectr{u}=\vectr{f}-\gradient{}{p}+\mu\laplacian{}{\vectr{u}}
  \label{eqn:NavierStokesequation2}
\end{equation}
which gives us the complete Navier-Stokes equations in ALE formulation.  The
following section, however, describes the reordered quasi-static formulation
of \eqnref{eqn:NavierStokesequationALE}:
\begin{equation}
  \rho\pbrac{\dotprod{\pbrac{\vectr{u}-\vectr{u}^*}}{\grad}}\vectr{u}-\mu\laplacian{}{\vectr{u}}+\gradient{}{p}=\vectr{f}
  % -\gradient{}{p}+\mu\laplacian{}{\vectr{u}}-\rho\pbrac{\dotprod{\vectr{u}^*}{\grad}\vectr{u}=\vectr{f}
  \label{eqn:NavierStokesequationALE2}
\end{equation}

\paragraph{Weak formulation:}

The corresponding weak form of the equation system consisting of
\eqnref{eqn:NavierStokesequation1} and \eqnref{eqn:NavierStokesmasequation}
can be written in the general dynamic form (see \secref{sec:dynamicequation})
\begin{equation}
  \matr{M}\fnof{\ddot{\vectr{u}}}{t}+\matr{C}\fnof{\dot{\vectr{u}}}{t}+\matr{K}\fnof{\vectr{u}}{t}+
  \fnof{\vectr{g}}{\fnof{\vectr{u}}{t}}+\fnof{\vectr{f}}{t}=\vectr{0}
  \label{eqn:generaldynamicnonlinear}
\end{equation}
where $\fnof{u}{t}$ is the dependent variables vector $\vectr{u}(\vectr{x},t)$ and $p$
for the degrees of freedom. $\matr{M}$ is the mass matrix, which provides the
shape function based weights, $\matr{C}$ is the transient damping matrix
(which we will discuss further below). $\matr{K}$ represents the stiffness
matrix, which will contain the linear parts of the operator, including the
viscous terms, the conservation of mass terms, and pressure
terms. $\fnof{\vectr{g}}{\fnof{\vectr{u}}{t}}$ is the nonlinear vector
function for the convective terms and $\fnof{\vectr{f}}{t}$ the forcing
vector.

The corresponding weak form of the equation system consisting of
\eqnref{eqn:NavierStokesequation1} and \eqnref{eqn:NavierStokesmasequation}
can be written as:
\begin{equation}
  \gint{\Omega}{}{\rho\pbrac{\dotprod{\vectr{u}}{\grad}}\vectr{u}\vectr{v} }{\Omega}
  -\gint{\Omega}{}{\rho\pbrac{\dotprod{\vectr{u}^*}{\grad}}\vectr{u}\vectr{v} }{\Omega}
  +\gint{\Omega}{}{\mu\laplacian{}{\vectr{u}}\vectr{v}}{\Omega}
  -\gint{\Omega}{}{\gradient{}{p}\vectr{v}}{\Omega}
  +\gint{\Omega}{}{\divergence{}{\vectr{u}}q}{\Omega}=
  \gint{\Omega}{}{\vectr{f}\vectr{v}}{\Omega}  
  \label{eqn:NavierStokesweakform}
\end{equation}
The general form for this kind of equation system is
\begin{equation}
  \matr{K}{\vect{\hat{u}}}+
  \fnof{\vect{\hat{g}}}{{\vect{{u}}}}={\vect{\hat{f}}}
  \label{eqn:NavierStokesequationALE2general}
\end{equation}
where ${\vect{\hat{u}}}$ is the vector of unknown ``DOFs'', $\matr{K}$ is the
stiffness matrix, $\fnof{\vect{\hat{g}}}{{\vect{{u}}}}$ a non-linear vector
function and ${\vect{\hat{f}}}$ the forcing vector. In
\eqnref{eqn:NavierStokesweakform} the only real non-linear term is represented
by
$\fnof{\vect{\hat{g}}}{\vect{{u}}}=\gint{\Omega}{}{\rho\pbrac{\dotprod{\vectr{u}}{\grad}}\vectr{u}\vectr{v}
}{\Omega}$.  If $\fnof{\vect{\hat{g}}}{\vect{u}}$ is not $\equiv\vect{0}$ then
we use Newton's method \ie
\begin{equation}
  \begin{split}
    \text{1.  } & \fnof{\matr{J}}{\vect{u}_{i}}.\delta
    \vect{u}_{i} = 
    -\fnof{\vect{\psi}}{\vectr{u}_{i}} \\
    \text{2.  } & \vect{u}_{i+1}=\vect{u}_{i}+\delta
    \vect{u}_{i}
  \end{split}
\end{equation}
where $\fnof{\matr{J}}{\vect{u}}$ is the Jacobian and is given by
\begin{equation}
  \fnof{\matr{J}}{\vect{u}}=\matr{K}+
    \delby{\fnof{\vect{\hat{g}}}{\vect{u}}}{\vect{u}}
\end{equation}
with the stiffness matrix $\matr{K}$ derived from
\eqnref{eqn:NavierStokesequationALE2general} by applying Green's theorem as
follows:
\begin{equation}
  \begin{split}
  \matr{K}\vectr{\hat{u}}=
  \gint{\Omega}{}{\divergence{}{\vectr{v}}p}{\Omega}
  -\gint{\Omega}{}{\doubledotprod{\mu\gradient{}{\vectr{v}}}{\gradient{}{\vectr{u}}}}{\Omega}
  -\gint{\Omega}{}{\rho\pbrac{\dotprod{\vectr{u}^*}{\grad}}\vectr{u}\vectr{v}}{\Omega}
  +\gint{\Omega}{}{\divergence{}{\vectr{u}}q}{\Omega}
  \end{split}
  \label{eqn:NavierStokesweakform2}
\end{equation}
and $\fnof{\vect{\psi}}{\vect{\hat{u}}}=\matr{K}\vect{\hat{u}}+\fnof{\vect{\hat{g}}}{\vect{u}}+\vect{\hat{f}}$.





%%% Local Variables: 
%%% mode: latex
%%% TeX-master: "../../OpenCMISSNotes"
%%% End: 
