\chapter{Introduction}
\label{cha:Introduction}

\section{Background - \CMISS}
\label{sec:IntroBackgroundCMISS}

\OpenCMISS is a computational code designed to solve bioengineering
problems. \OpenCMISS is a re-engineering of the \CMISS computational
environment. \CMISS was originally started in the late 1970's/early
1980's by Peter Hunter in order to solve problems in bioengineering
that required features that were not available in commercial
computational codes. Since then it has been extended by a large number
of people from the Department of Engineering Science and then later
the Auckland Bioengineering Institute at the University of
Auckland. \CMISS is an acronym and starts for Continuum Mechanics,
Imaging, Signal processing and System identification. \CMISS currently
consists of two main parts - a graphical front end called \CMGUI and a
computational engine called \CM. \CMGUI is written in C/C++ and is
cross-platform and open source.

The computational engine \CM is written in legacy Fortran 77 and is
approximately $500,000$ lines in length. It is an application in that
it has its own run-time environment and defined input an output
files. \CM uses \OpenMP for shared memory parallelism. Some of the
unique features of \CMISS include the use of high order elements like
cubic Hermite that preserve \cont{1} or \gcont{1} continuity across
element boundaries, the use of variable order basis functions, the use
of embedded material fields to preserve the close relationship of
tissue structure and anatomy and the use of Physiome standards to
represent models.

Despite the success of \CMISS in solving bioengineering problems over
the last $40$ years it has recently become clear that \CMISS has a
number of limitations that will be difficult to overcome. For example,
it is difficult to run very large models in \CM as it has a large
memory footprint and parallel speedup is limited due to the shared
memory model used. In addition it was desired to take advantage of new
modelling standards and repositories, to move towards an open source
framework, and to take advantage of modern high performance computer
architectures.

\section{\OpenCMISS Design Goals}
\label{sec:IntroOpenCMISSDesignGoals}

In order to overcome the limitations of \CMISS it was decided to start
a fresh with the \OpenCMISS project. From the start it was that
\OpenCMISS would be a general collaborative code. The reasons for this
were as follows:
\begin{itemize}
  \item Science and models are becoming more complex. The increase in
    problem complexity inevitably leads to a corresponding increase in
    code complexity. This means that modern scientific codes take a
    greate deal of time and effort to develop, test and maintain them.
  \item Science is collaborative. In order to facilitate collaborative
    science codes are often developed collaboratively and used by many
    groups.
  \item Science builds on previous science. Most work builds on
    previous projects. This becomes difficult if a student writes
    independent code and then leaves the group on completion of their
    studies. The next student may find the previous code difficult to
    understand or adapt. Reinventing the wheel is then costly in both
    time and effort.
  \item Tomorrows science is upredictable. Assumptions made today may
    be costly to fix in the future. A general code allows for a
    greater potential for future use.
\end{itemize}

\OpenCMISS also had a number of design goals in mind. These were
\begin{itemize}
  \item That \OpenCMISS would be a library. A library is more flexible
    rather than a large monolithic application such as \CMISS. A
    library-based code means that it is considerably easier to
    incorporate physiome and bioengineering models into clinical or
    commercial applications as a library that can be wrapped by a
    customised interface. The library should be modular, extensible,
    and programmable. This allows for the library itself to be
    customised and/or extended in whatever way is appropriate for the
    end application.
\item That \OpenCMISS would be a general code. Previous experience
  with the \CMISS modelling environment indicated the importance of
  developing code in as general a way as possible. Generalised data
  structures, in which the data for diverse modelling problems are
  expressed in a common format, allow for easier coupling between
  different problems. This is especially true for unforeseen, coupled
  problems that may arise from future applications. The goal of
  generality does, however, often mean that there is some trade-off
  with the computational performance of code. As the computational
  size of bioengineering models can be very large it is extremely
  important that computational performance is carefully
  considered. But it is our view that it is better to optimise a more
  general code armed with the knowledge of exactly what the problem is
  than to prematurely optimise a specific code which could then limit
  the applicability of that code.
\item That \OpenCMISS should be an inherently parallel code and that
  the parallel environment should be as general as possible. Parallel
  processing is required as the computational demands of solving
  models increases due to increased resolution or complexity of the
  models. However, optimal parallel processing strategies depend on
  the particular problem being solved. Also the lifetime of modelling
  codes is often an order of magnitude greater than the lifetime of
  the computer hardware, and it is notable that the architecture of
  parallel machines has changed over the last few decades from vector
  processors, to symmetric multiple processors (SMPs), to clusters of
  processors, to clusters of multiple core processors, through to
  using General Purpose Graphical Processing Units (GPGPUs). Code that
  assumes a particular parallel algorithm or a particular parallel
  architecture may not be appropriate for a future problem or future
  parallel hardware. 
\item That \OpenCMISS should be able to be used, understood, and
  developed by novices and experts alike. Modern bioengineering and
  physiome science requires a team of scientists, graduate students,
  and post-doctoral researchers from varied backgrounds, each with a
  different skill set. It is unrealistic to expect that each member of
  the team will become an expert in every area of modelling and
  computation. The design of OpenCMISS thus abstracts and encapsulates
  model details in a number of objects of hierarchical complexity. The
  hierarchy of these objects allows complex details to be hidden from
  the users, if required, and the object interface allows an expert to
  manipulate object parameters whilst the novice user makes use of
  sensible default parameter values for the common cases.
\item That \OpenCMISS incorporate Application Programming Interfaces
  (APIs) for the physiome markup languages CellML and FieldML.
\end{itemize}

\section{Notation} 
\label{sec:IntroNotation}
 
\subsection{Vector and Tensors}
\label{subsec:IntroVectorTensorNotation}

The following formatting will be used throughout theses notes for scalars, vectors, and tensors.
\begin{equation}
  \begin{split}
    a, b, c, \ldots & \quad\text{Scalars or $0^{th}$ order tensors} \\
    \vectr{a},\vectr{b},\vectr{c}, \ldots & \quad\text{Vectors or $1^{st}$ order
      tensors} \\
    \tensortwo{A},\tensortwo{B},\tensortwo{C}, \ldots & \quad\text{Dyadics or $2^{nd}$ order
      tensors} \\
    \tensorthree{A},\tensorthree{B},\tensorthree{C}, \ldots & \quad\text{Triadics or $3^{rd}$
      order tensors} \\
    \tensorfour{A},\tensorfour{B},\tensorfour{C}, \ldots & \quad\text{Tetradics or $4^{th}$
      order tensors}
  \end{split}
\end{equation}

\subsection{Summation Notation}
\label{subsec:IntroSummationNotation}

The following (Einstein\footnote{named after
\link{https://en.wikipedia.org/wiki/Albert_Einstein}{Albert Einstein}
(1879-1955), a German-born physicist.}) summation
notation\index{Summation notation} will be used throughout these
notes. In order to eliminate summation symbols repeated ``dummy''
indices will be used \ie
\begin{equation}
  \gsum{i=1}{n}{a^{i}b_{i}}=a^{i}b_{i}
\end{equation}

To indicate an index that is not summed, parentheses will be used
\ie $a^{(i)}b_{(i)}$ is talking about the singular expression for $i$ \eg
$a^{1}b_{1}$, $a^{2}b_{2}$ \etc

In order to indicate a summation the sum must occur over indices that are
different sub/super-script \ie the sum must be over an ``upper'' and a
``lower'' index or a ``lower'' and an ``upper'' index. Note that it may be
useful to remember that if an index appears in the denominator of a fractional
expression then the index upper- or lower- ness is ``reversed''. 

For some quantities with both upper and lower indices a dot will be used to
indicate the ``second'' index \eg in the expression $A^{i}_{.j}$ then $i$ can
be considered the first index and $j$ the second index.

\section{OpenCMISS notes} 
\label{sec:IntroOpenCMISSNotes}

\subsection{Building these notes}
\label{subsec:IntroBuildingOpenCMISSNotes}

These notes are written in \LaTeX\xspace
(\url{https://www.latex-project.org/}) and your will need to install
an appropriate \LaTeX\xspace environment on your system via your package
manager.

In order to create and/or edit the Scalable Vector Graphics (SVG)
diagrams in these notes it is recommeded that you install \Inkscape from either your
package manager or from \url{https://inkscape.org/}. In order to allow
for \Inkscape to use \LaTeX\xspace it is recommended that you also install
the TexText plugin for \Inkscape - see
\url{https://textext.github.io/textext/}.

In order to create and/or edit the graphs and plots in these notes it
is recommeded that you install \Gnuplot from either your package
manager of from \url{http://www.gnuplot.info/}.

The main \LaTeX\xspace source for these notes can be found in the \OpenCMISS
\GitHub (\url{https://github.com/}) documentation repository at
\url{https://github.com/OpenCMISS/documentation}.

In order to make changes to the notes and compile the \LaTeX\xspace source
you should set yourself up as an \OpenCMISS developer. The main
instructions to do this are at
\url{http://opencmiss.org/documentation/contribute/dev_setup.html}
but, in brief, the steps to do this for a Linux/UNIX operating system
are as follows:
\begin{enumerate}
\item Create an \OpenCMISS main directory \ie
  \begin{code}
    mkdir OpenCMISS
  \end{code}
\item Change directory into this directory \ie
  \begin{code}
    cd OpenCMISS
  \end{code}
\item Fork the \OpenCMISS \GitHub utilites repository at \url{https://github.com/OpenCMISS/utilities}.
\item Clone the \OpenCMISS utilities repository \ie
  \begin{code}
    git clone git@github.com:GITHUBUSERNAME/documentation.git
  \end{code} where \oscommand{GITHUBUSERNAME} is your \GitHub username.
\item Set up to use the developer scripts. The main instructions are
  at
  \url{http://opencmiss.org/documentation/contribute/dev_login_scripts.html}
  but, in brief, edit your \file{\midtilde/.bashrc} (if you have a bash shell or
  similar) or \file{\midtilde/.cshrc} file (if you have a csh shell or
  similar). If you are unsure which shell you have use the command
  \oscommand{echo \$SHELL}. Add the lines
  \begin{code}
    export OPENCMISS_ROOT=<path-to-where-you-put-your-OpenCMISS-files>
    if [ -r "$OPENCMISS_ROOT/utilities/scripts/opencmiss_developer.sh" ]; then
      . $OPENCMISS_ROOT/utilities/scripts/opencmiss_developer.sh
    fi
  \end{code} if you have a bash shell or
  \begin{code}
    setenv OPENCMISS_ROOT <path-to-where-you-put-your-OpenCMISS-files>
    if ( -r "$\{OPENCMISS_ROOT\}/utilities/scripts/opencmiss_developer.csh" ) then
      source $\{OPENCMISS_ROOT\}/utilities/scripts/opencmiss_developer.csh
    endif
  \end{code} if you have a csh shell. Note that
  \file{<path-to-where-you-put-your-OpenCMISS-files>} is the path to
  the main directory created in step 1.
\item Fork the \OpenCMISS \GitHub utilites repository at \url{https://github.com/OpenCMISS/documentation}.
\item Clone the OpenCMISS documentation repository \ie
  \begin{code}
    git clone git@github.com:GITHUBUSERNAME/documentation.git
  \end{code} where \oscommand{GITHUBUSERNAME} is your \GitHub username.
\item Log out and back in again to start using the developer scripts.
\end{enumerate}

The developer setup scripts should ensure that all necessary paths and
links are set to ensure that you will be able to compile the
\OpenCMISS notes document.

Once you have cloned a copy of the notes to disk the notes are in the
\directory{notes} under the \directory{OpenCMISS/documentation}
directory.

To compile the notes use the \oscommand{latexmake} command \ie
\begin{code}
  cd $OPENCMISS_ROOT/documentation/notes/OpenCMISSNotes
  latexmake pdf
\end{code}
to compile the \LaTeX\xspace and generate a PDF file.

You can use the \oscommand{latexmake help} for other options.

\subsection{Acknowledgements}
\label{subsec:IntroAcknowledgements}

A number of formulae and figures have been adapted from the ABI Notes
and various student theses that have been completed at the
\link{Auckland Bioengineering
  Institute}{http://www.abi.auckland.ac.nz}.

A number of elements in the figures used in these notes have been
taken from the Free {SVG} library located at
\urllink{https://freesvg.org/}. These {SVG} images are licensed with
the \link{Creative Commons License}{https://creativecommons.org/}.
