\section{Material Coordinate Transformations}
\label{sec:MaterialCoordinateTransformations}

\subsection{Two dimensions}
\label{subsec:MaterialCoordinateTransformationsTwoD}

Consider the two dimensional system shown in
\figref{fig:MaterialCoordinateTransformationTwoD}. There are three
coordinate systems of interest. The first is the geometric coordinate
system, $\vectr{x}$, and the second is the $\vectr{\xi}$ coordinate
system of an element. The final coordinate system is the $\vectr{\nu}$
coordinate system of the material fibre directions.

\epstexfigure{Theory/svgs/TwoDMaterialTransformation.eps_tex}{Two
  dimensional material coordinate transformation.}{Two dimensional
  material coordinate
  transformation.}{fig:MaterialCoordinateTransformationTwoD}{0.5}

To work out the transformations between these coordinate systems we
first define the geometric position vector (with respect to the base vectors $\generalbasevector_{i}$) of a point inside the element
using interpolation \ie
\begin{equation}
  \fnof{\vectr{x}}{\vectr{\xi}}=\fnof{x^{i}}{\vectr{\xi}}\generalbasevector_{i}=
  \gsf{n}{\beta}\gbfn{n}{\beta}{\vectr{\xi}}\nodedof{\vectr{x}}{n}{\beta}
\end{equation}

The transformation matrix between geometric coordinates, $\vectr{x}$, and element coordinates, $\vectr{\xi}$, is thus given by
\begin{equation}
  \delby{x^{i}}{\xi^{r}}=\begin{bmatrix}
  \delby{x^{1}}{\xione} & \delby{x^{2}}{\xione} \\
  \delby{x^{1}}{\xitwo} & \delby{x^{2}}{\xitwo}
  \end{bmatrix}
  \label{eqn:GeometricToElementTransformationTwoD}
\end{equation}
and the inverse transformation between element coordinates, $\vectr{\xi}$, and geometric coordinates, $\vectr{x}$, is thus given by
\begin{equation}
  \delby{\xi^{r}}{x^{i}}=\inverse{\sqbrac{\delby{x^{i}}{\xi^{r}}}}=
  \dfrac{1}{\delby{x^{1}}{\xione}\delby{x^{2}}{\xitwo}-\delby{x^{2}}{\xione}\delby{x^{1}}{\xitwo}}\begin{bmatrix}
  \delby{x^{2}}{\xitwo} & -\delby{x^{2}}{\xione} \\
  -\delby{x^{1}}{\xitwo} & \delby{x^{1}}{\xione}    
  \end{bmatrix}=\begin{bmatrix}
  \delby{\xi^{1}}{x^{1}} & \delby{\xi^{2}}{x^{1}} \\
  \delby{\xi^{1}}{x^{2}} & \delby{\xi^{2}}{x^{2}}
  \end{bmatrix}
  \label{eqn:ElementToGeometricTransformationTwoD}
\end{equation}

The derivative of the geometric interpolation also gives the tangent
direction with respect to $\xione$. We define the normalisation of
this direction as the vector, $\vectr{a}$, \ie
\begin{equation}
  \fnof{\vectr{a}}{\vectr{\xi}}=\fnof{a^{i}}{\vectr{\xi}}\generalbasevector_{i}=\norm{\delby{\fnof{\vectr{x}}{\vectr{\xi}}}{\xione}}
\end{equation}

We now define the vector, $\vectr{b}$, to be orthogonal to the vector, $\vectr{a}$, by rotating $\vectr{a}$ counter clockwise
$90\degree$ \ie
\begin{equation}
  \fnof{\vectr{b}}{\vectr{\xi}}=\fnof{b^{i}}{\vectr{\xi}}\generalbasevector_{i}=
  \fnof{a^{2}}{\vectr{\xi}}\generalbasevector_{1}-\fnof{a^{1}}{\vectr{\xi}}\generalbasevector_{2}
\end{equation}

Now the material fibre direction is defined by a rotation angle,
$\theta$, from the $\xione$ direction. This angle is given by
interpolation within the element \ie
\begin{equation}
  \fnof{\theta}{\vectr{\xi}}=\gsf{o}{\gamma}\gbfn{o}{\gamma}{\vectr{\xi}}\nodedof{\vectr{\theta}}{o}{\gamma}
\end{equation}

We can now find the fibre vector, $\vectr{f}$, and sheet vector,
$\vectr{s}$, by rotating the vectors $\vectr{a}$ and $\vectr{b}$ by
$\theta$. To do this we make use of the rotation matrix
\begin{equation}
  \fnof{\matr{R}}{\fnof{\theta}{\vectr{\xi}}}=\begin{bmatrix}
  \cosine{\fnof{\theta}{\vectr{\xi}}} & \sine{\fnof{\theta}{\vectr{\xi}}} \\
  -\sine{\fnof{\theta}{\vectr{\xi}}} & \cosine{\fnof{\theta}{\vectr{\xi}}}
  \end{bmatrix}
\end{equation}
as
\begin{equation}
  \fnof{\vectr{f}}{\vectr{\xi}}=\fnof{f^{i}}{\vectr{\xi}}\generalbasevector_{i}=\fnof{\matr{R}}{\fnof{\theta}{\vectr{\xi}}}\fnof{\vectr{a}}{\vectr{\xi}}
\end{equation}
and
\begin{equation}
  \fnof{\vectr{s}}{\vectr{\xi}}=\fnof{s^{i}}{\vectr{\xi}}\generalbasevector_{i}=\fnof{\matr{R}}{\fnof{\theta}{\vectr{\xi}}}\fnof{\vectr{b}}{\vectr{\xi}}
\end{equation}
  
The rotation matrix thus gives the transformation between element coordinates, $\vectr{\xi}$, and material/fibre coordinates, $\vectr{\nu}$, \ie
\begin{equation}
  \delby{\nu^{a}}{\xi^{r}}=\fnof{\matr{R}}{\fnof{\theta}{\vectr{\xi}}}\begin{bmatrix}
  \cosine{\fnof{\theta}{\vectr{\xi}}} & \sine{\fnof{\theta}{\vectr{\xi}}} \\
  -\sine{\fnof{\theta}{\vectr{\xi}}} & \cosine{\fnof{\theta}{\vectr{\xi}}}
  \end{bmatrix}  
  \label{eqn:ElementToMaterialTransformationTwoD}
\end{equation}

The transformation between material/fibre coordinates, $\vectr{\nu}$,
and element coordinates, $\vectr{\xi}$, can thus be found from
\begin{equation}
  \delby{\xi^{r}}{\nu^{a}}=\inverse{\sqbrac{\delby{\nu^{a}}{\xi^{r}}}}=\inverse{\fnof{\matr{R}}{\fnof{\theta}{\vectr{\xi}}}}=
  \transpose{\fnof{\matr{R}}{\fnof{\theta}{\vectr{\xi}}}}=\begin{bmatrix}
  \cosine{\fnof{\theta}{\vectr{\xi}}} & -\sine{\fnof{\theta}{\vectr{\xi}}} \\
  \sine{\fnof{\theta}{\vectr{\xi}}} & \cosine{\fnof{\theta}{\vectr{\xi}}}
  \end{bmatrix}  
  \label{eqn:MaterialToElementTransformationTwoD}
\end{equation}
as the rotation matrix is orthogonal.

\subsection{Three dimensions}
\label{subsec:MaterialCoordinateTransformationsThreeD}

Unlike in two-dimensions, rotation of coordinates in three-dimensions
is not straightforward and there a number of different ways to define
the coordinate transformations.

\subsubsection{Euler and Tait-Bryan angles}
\label{subsubsec:MaterialCoordinateTransformationsThreeDEulerTaitBryan}

\subsubsection{Quaternions}
\label{subsubsec:MaterialCoordinateTransformationsThreeDQuaternions}


\subsection{Material tensor transformations}
\label{subsec:MaterialCoordinateTransformationsTensor}

Having defined the relationships between the various material, element
and geometric coordinate systems we can transform material tensors to
geometric tensors via the element coordinate system.

For example consider transforming a material vector, $\vectr{a}$, to a
geometric vector, $\vectr{b}$. The transformations are
\begin{equation}
  b^{i}=\delby{x^{i}}{\xi^{r}}\delby{\xi^{r}}{\nu^{a}}a^{a}
\end{equation}
for rank(1,0) vectors, and,
\begin{equation}
  b_{i}=\delby{\xi^{r}}{x^{i}}\delby{\nu^{a}}{\xi^{r}}a_{a}
\end{equation}
for rank(0,1) vectors.

For a second order material tensors, $\tensortwo{A}$, to geometric
tensors, $\tensortwo{B}$, the transformations are
\begin{equation}
  B^{ij}=\delby{x^{i}}{\xi^{r}}\delby{x^{j}}{\xi^{s}}\delby{\xi^{r}}{\nu^{a}}\delby{\xi^{s}}{\nu^{b}}A^{ab}
\end{equation}
for rank(2,0) tensors, and,
\begin{equation}
  B_{ij}=\delby{\xi^{r}}{x^{i}}\delby{\xi^{s}}{x^{j}}\delby{\nu^{a}}{\xi^{r}}\delby{\nu^{b}}{\xi^{s}}A_{ab}
\end{equation}
for rank(0,2) tensors, and,
\begin{equation}
  B^{i.}_{.j}=\delby{x^{i}}{\xi^{r}}\delby{\xi^{s}}{x^{j}}\delby{\xi^{r}}{\nu^{a}}\delby{\nu^{b}}{\xi^{s}}A^{a.}_{.b}
\end{equation}
and
\begin{equation}
  B^{.j}_{i.}=\delby{\xi^{r}}{x^{i}}\delby{x^{j}}{\xi^{s}}\delby{\nu^{a}}{\xi^{r}}\delby{\xi^{s}}{\nu^{b}}A^{.b}_{a.}
\end{equation}
for rank(1,1) tensors.

For fourth order material tensors, $\tensorfour{A}$, to geometric
tensors, $\tensorfour{B}$, the transformations are
\begin{equation}
  B^{ijkl}=\delby{x^{i}}{\xi^{r}}\delby{x^{j}}{\xi^{s}}\delby{x^{k}}{\xi^{t}}\delby{x^{l}}{\xi^{u}}
  \delby{\xi^{r}}{\nu^{a}}\delby{\xi^{s}}{\nu^{b}}\delby{\xi^{t}}{\nu^{c}}\delby{\xi^{u}}{\nu^{d}}A^{abcd}
\end{equation}
for rank(4,0) tensors, and,
\begin{equation}
  B_{ijkl}=\delby{\xi^{r}}{x^{i}}\delby{\xi^{s}}{x^{j}}\delby{\xi^{t}}{x^{k}}\delby{\xi^{u}}{x^{l}}
  \delby{\nu^{a}}{\xi^{r}}\delby{\nu^{b}}{\xi^{s}}\delby{\nu^{c}}{\xi^{t}}\delby{\nu^{d}}{\xi^{u}}A_{abcd}
\end{equation}
for rank(0,4) tensors, and,
\begin{equation}
  B^{ij..}_{..kl}=\delby{x^{i}}{\xi^{r}}\delby{x^{j}}{\xi^{s}}\delby{\xi^{t}}{x^{k}}\delby{\xi^{u}}{x^{l}}
  \delby{\xi^{r}}{\nu^{a}}\delby{\xi^{s}}{\nu^{b}}\delby{\nu^{c}}{\xi^{t}}\delby{\nu^{d}}{\xi^{u}}A^{ab..}_{..cd}
\end{equation}
and
\begin{equation}
  B^{..kl}_{ij..}=\delby{\xi^{r}}{x^{i}}\delby{\xi^{s}}{x^{j}}\delby{x^{k}}{\xi^{t}}\delby{x^{l}}{\xi^{u}}
  \delby{\nu^{a}}{\xi^{r}}\delby{\nu^{b}}{\xi^{s}}\delby{\xi^{t}}{\nu^{c}}\delby{\xi^{u}}{\nu^{d}}A^{..cd}_{ab..}
\end{equation}
\etc, for rank(2,2) tensors.

Here,
\eqnrefsfour{eqn:GeometricToElementTransformationTwoD}{eqn:ElementToGeometricTransformationTwoD}{eqn:ElementToMaterialTransformationTwoD}{eqn:MaterialToElementTransformationTwoD}
define the transformations in two-dimensions, and
\eqnrefsfour{eqn:GeometricToElementTransformationThreeD}{eqn:ElementToGeometricTransformationThreeD}{eqn:ElementToMaterialTransformationThreeD}{eqn:MaterialToElementTransformationThreeD}
define the transformations in three-dimensions. The transformations in one-dimension are trivial.
