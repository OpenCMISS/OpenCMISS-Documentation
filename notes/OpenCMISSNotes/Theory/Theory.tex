\clearemptydoublepage
\chapter{Theory}
\label{cha:theory}

\section{Tensor Analysis}
\subsection{Base vectors}

Now, if we have a vector, $\vectr{v}$ we can write
\begin{equation}
  \vectr{v}=v^{i}\vectr{g}_{i}
\end{equation}
where $v^{i}$ are the components of the contravariant vector, and
$\vectr{g}_{i}$ are the covariant base vectors.

Similarly, the vector $\vectr{v}$ can also be written as 
\begin{equation}
  \vectr{v}=v_{i}\vectr{g}^{i}
\end{equation}
where $v_{i}$ are the components of the covariant vector, and
$\vectr{g}^{i}$ are the contravariant base vectors. 

We now note that
\begin{equation}
  \vectr{v}=v^{i}\vectr{g}_{i}=v^{i}\sqrt{g_{ii}}\hat{\vectr{g}_{i}}
\end{equation}
where $v^{i}\sqrt{g_{ii}}$ are the physical components of the vector and
$\hat{\vectr{g}_{i}}$ are the unit vectors given by
\begin{equation}
  \hat{\vectr{g}_{i}}=\dfrac{\vectr{g}_{i}}{\sqrt{g_{ii}}}
\end{equation}

\subsection{Metric Tensors}
\label{sec:metric tensors}

Metric tensors are the inner product of base vectors. If $\vectr{g}_{i}$ are the
covariant base vectors then the covariant metric tensor is given by
\begin{equation}
  g_{ij}=\dotprod{\vectr{g}_{i}}{\vectr{g}_{j}}
\end{equation}

Similarily if $\vectr{g}^{i}$ are the contravariant base vectors then the
contravariant metric tensor is given by 
\begin{equation}
  g^{ij}=\dotprod{\vectr{g}^{i}}{\vectr{g}^{j}}
\end{equation}

We can also form a mixed metric tensor from the dot product of a contravariant
and a covariant base vector \ie
\begin{equation}
  g^{i}_{.j}=\dotprod{\vectr{g}^{i}}{\vectr{g}_{j}}
\end{equation}
and 
\begin{equation}
  g_{i}^{.j}=\dotprod{\vectr{g}_{i}}{\vectr{g}^{j}}
\end{equation}

Note that for mixed tensors the ``.'' indicates the order of the index \ie
$g^{i}_{.j}$ indicates that the first index is contravariant and the second
index is covariant whereas $g_{i}^{.j}$ indicates that the first index is
covariant and the second index is contravariant.

If the base vectors are all mutually orthogonal and constant then
$\vectr{g}_{i}=\vectr{g}^{i}$ and $g_{ij}=g^{ij}$.

The metric tensors generalise (Euclidean) distance \ie
\begin{equation}
  ds^{2}=g_{ij}dx^{i}dx^{j}
\end{equation}

Note that multiplying by the covariant metric tensor lowers indices \ie
\begin{equation}
  \begin{split}
    \vectr{A}_{i} &= g_{ij}\vectr{A}^{j} \\
    A_{ij} &= g_{ik}g_{jl}A^{kl} = g_{jk}A_{i}^{.k} = g_{ik}A^{k}_{.j} 
  \end{split}
\end{equation}
and that multiplying by the contravariant metric tensor raises indices \ie
\begin{equation}
  \begin{split}
  \vectr{A}^{i} &=  g^{ij}\vectr{A}_{j} \\
   A^{ij} &= g^{ik}g^{jl}A_{kl} = g^{ik}A_{k}^{.j} = g^{jk}A^{i}_{.k}
  \end{split}
\end{equation}
and for the mixed tensors
\begin{equation}
  \begin{split}
  A_{i}^{.j} &= g^{jk}A_{ik} = g_{ik}A^{kj} \\
  A^{i}_{.j} &= g^{ik}A_{kj} = g_{jk}A^{ik} \\
  \end{split}
\end{equation}

\subsection{Transformations}

The transformation rules for tensors in going from a $\vectr{\nu}$ coordinate
system to a $\vectr{\xi}$ coordinate system are as follows: 


For a covariant vector (a rank (0,1) tensor)
\begin{equation}
  {\tilde{a}}_{i}=\delby{\nu^{a}}{\xi^{i}}a_{a}
\end{equation}

For a contravariant vector (a rank (1,0) tensor)
\begin{equation}
  {\tilde{a}}^{i}=\delby{\xi^{i}}{\nu^{a}}a^{a}
\end{equation}

For a covariant tensor (a rank (0,2) tensor)
\begin{equation}
  {\tilde{A}}_{ij}=\delby{\nu^{a}}{\xi^{i}}\delby{\nu^{b}}{\xi^{j}}A_{ab} 
\end{equation}

For a contravariant tensor (a rank (2,0) tensor)
\begin{equation}
  {\tilde{A}}^{ij}=\delby{\xi^{i}}{\nu^{a}}\delby{\xi^{j}}{\nu^{b}}A^{ab}
\end{equation}

and for Mixed tensors (rank (1,1) tensors)
\begin{equation}
  {\tilde{A}}^{i}_{.j}=\delby{\xi^{i}}{\nu^{a}}\delby{\nu^{b}}{\xi^{j}}A^{a}_{.b}
\end{equation}
and
\begin{equation}
  {\tilde{A}}_{i}^{.j}=\delby{\nu^{a}}{\xi^{i}}\delby{\xi^{j}}{\nu^{b}}A_{a}^{.b}
\end{equation}

\subsection{Derivatives}
\label{subsec:function derivatives}

\subsubsection{Scalars}

We note that a scalar quantity $\fnof{u}{\vectr{\xi}}$ has derivatives
\begin{equation}
  \delby{u}{\xi^{i}}=\partialderiv{u}{i}
\end{equation}

Or more formally, the covariant derivative ($\covarderiv{\cdot}{\cdot}$) of a
rank 0 tensor $u$ is
\begin{equation}
  \covarderiv{u}{i}=\delby{u}{\xi^{i}}=\partialderiv{u}{i}
\end{equation}

\subsubsection{Vectors}

The derivatives of a vector $\vectr{v}$ are given by
\begin{equation}
  \begin{split}
    \delby{\vectr{v}}{\xi^{i}} &=
    \delby{}{\xi^{i}}\pbrac{v^{k}\vectr{g}_{k}} \\
    &= \delby{v^{k}}{\xi^{i}}\vectr{g}_{k}+v^{k}\delby{\vectr{g}_{k}}{\xi^{i}} \\
    &= \partialderiv{v^{k}}{i}\vectr{g}_{k}+v^{k}\partialderiv{\vectr{g}_{k}}{i}
  \end{split}
\end{equation}

Now introducing the notation
\begin{equation}
  \christoffelsecond{i}{j}{k} = \dotprod{\vectr{g}^{i}}{\delby{\vectr{g}_{j}}{x^{k}}}
\end{equation}
where $\christoffelsecond{i}{j}{k}$ are the Christoffel symbols of the second
kind. 

Note that the Christoffel symbols of the first kind are given by
\begin{equation}
  \christoffelfirst{i}{j}{k} = \dotprod{\vectr{g}_{i}}{\delby{\vectr{g}_{j}}{x^{k}}}
\end{equation}

Note that
\begin{equation}
  \begin{split}
    \christoffel{i}{j}{k} &= \dotprod{\vectr{g}^{i}}{\partialderiv{\vectr{g}_{j}}{k}} \\
    &=\dotprod{\vectr{g}^{i}}{\christoffelsecond{l}{j}{k}\vectr{g}_{l}} \\
    &= \christoffel{i}{j}{l}g^{i}_{.l} 
  \end{split}
\end{equation}

The Christoffel symbols of the first kind are also given by
\begin{equation}
  \christoffelfirst{i}{j}{k}=\frac{1}{2}\pbrac{\delby{g_{ij}}{\xi^{k}}+\delby{g_{ik}}{\xi^{j}}-\delby{g_{jk}}{\xi^{i}}}
\end{equation}
and that Christoffel symbols of the second kind are given by
\begin{equation}
  \begin{split}
    \christoffelsecond{i}{j}{k} &= g^{il}\christoffelfirst{l}{j}{k} \\
    &= \frac{1}{2}g^{il}\pbrac{\delby{g_{lj}}{\xi^{k}}+\delby{g_{lk}}{\xi^{j}}-\delby{g_{jk}}{\xi^{l}}} 
  \end{split}
\end{equation}

Note that Christoffel symbols are not tensors and the have the following
transformation laws from $\vectr{\nu}$ to $\vectr{\xi}$ coordinates
\begin{align}
  \christoffelfirst{i}{j}{k} &=
  \christoffelfirst{a}{b}{c}\delby{\nu^{b}}{\xi^{j}}\delby{\nu^{c}}{\xi^{k}}\delby{\nu^{a}}{\xi^{i}}+
  g_{ab}\delby{\nu^{c}}{\xi^{i}}\deltwoby{\nu^{c}}{\xi^{j}}{\xi^{k}} \\
  \christoffelsecond{i}{j}{k} &= \christoffelsecond{a}{b}{c}\delby{\xi^{i}}{\nu^{a}}\delby{\nu^{b}}{\xi^{k}}\delby{\nu^{c}}{\xi^{j}}+
  \delby{\xi^{i}}{\nu^{a}}\deltwoby{\nu^{a}}{\xi^{j}}{\xi^{k}} \\
\end{align}

We can now write (BELOW SEEMS WRONG - CHECK)
\begin{equation}
  \begin{split}
    \partialderiv{\vectr{v}}{i}&=\partialderiv{v^{k}}{i}\vectr{g}_{k}+\christoffel{k}{i}{j}v^{j}\vectr{g}_{j}\\
    &=\partialderiv{v^{k}}{i}\vectr{g}_{k}+\christoffel{j}{i}{k}v^{k}\vectr{g}_{k}\\
    &=\pbrac{\partialderiv{v^{k}}{i}+\christoffel{j}{i}{k}v^{k}}\vectr{g}_{k}\\
    &=\covarderiv{v^{k}}{i}\vectr{g}_{k}
  \end{split}
\end{equation}
where $\covarderiv{v^{k}}{i}$ is the covariant derivative of $v^{k}$ . 

The covariant derivative of a contravariant (rank (0,1)) tensor $v^{k}$ is
\begin{equation}
  \covarderiv{v^{k}}{i} =\partialderiv{v^{k}}{i}+\christoffel{k}{i}{j}v^{j}
\end{equation}
and the covariant derivative of a covariant tensor  (rank (1,0)) $v_{k}$ is
\begin{equation}
  \covarderiv{v_{k}}{i} =\partialderiv{v_{k}}{i}-\christoffel{j}{k}{i}v_{j}
\end{equation}

\subsubsection{Tensors}

The covariant derivative of a contravariant (rank (0,2)) tensor $W^{mn}$ is
\begin{equation}
  \covarderiv{W^{mn}}{i}=\partialderiv{W^{mn}}{i}+\christoffel{m}{j}{i}W^{jn}+\christoffel{n}{j}{i}W^{mj}
\end{equation}
and the covariant derivative of a covariant (rank (2,0)) tensor $W_{mn}$ is
\begin{equation}
  \covarderiv{W_{mn}}{i}=\partialderiv{W_{mn}}{i}-\christoffel{j}{m}{i}W_{jn}-\christoffel{j}{n}{i}W_{mj}
\end{equation}
and the covariant derivative of a mixed (rank (1,1)) tensor $W^{m}_{.n}$ is
\begin{equation}
  \covarderiv{W^{m}_{.n}}{i}=\partialderiv{W^{m}_{.n}}{i}+\christoffel{m}{j}{i}W^{j}_{.n}-\christoffel{j}{n}{i}W^{m}_{.j}
\end{equation}

\subsection{Common Operators}
For tensor equations to hold in any coordinate system the equations must
involve tensor quantities \ie covariant derivatives rather than partial derivatives.

\subsubsection{Gradient}

As the covariant derivative of a scalar is just the partial derivative the
gradient of a scalar function $\phi$ using covariant derivatives is
\begin{equation}
  \text{grad } \phi = \gradient{}{\phi}=\covarderiv{\phi}{i}\vectr{g}^{i}=\partialderiv{\phi}{i}\vectr{g}^{i}
\end{equation}
and
\begin{equation}
  \gradient{}{\phi}=\partialderiv{\phi}{i}\vectr{g}^{i}=\partialderiv{\phi}{i}g^{ij}\vectr{g}_{j}
\end{equation}

\subsubsection{Divergence}

The divergence of a vector using covariant derivatives is
\begin{equation}
  \text{div } \vectr{\phi} = \divergence{}{\vectr{\phi}}=\covarderiv{\phi^{i}}{i}=\frac{1}{\sqrt{\abs{g}}}\partialderiv{\pbrac{\sqrt{\abs{g}}\phi^{i}}}{i}
\end{equation}
where $g$ is the determinant of the covariant metric tensor $g_{ij}$.

\subsubsection{Curl}

The curl of a vector using covariant derivatives is
\begin{equation}
  \text{curl } \vectr{\phi} = \curl{}{\vectr{\phi}}=\frac{1}{\sqrt{g}}\pbrac{\covarderiv{\phi_{j}}{i}-\covarderiv{\phi_{i}}{j}}\vectr{g}_{k}
\end{equation}
where $g$ is the determinant of the covariant metric tensor $g_{ij}$.

\subsubsection{Laplacian}

The Laplacian of a scalar using covariant derivatives is
\begin{equation}
  \laplacian{\phi}=\text{div}\pbrac{\text{grad }\phi}=\divergence{}{\gradient{}{\phi}}=\mixedderiv{\phi}{i}{i}=\frac{1}{\sqrt{g}}\partialderiv{\pbrac{\sqrt{g}g^{ij}\partialderiv{\phi}{j}}}{i}
\end{equation}
where $g$ is the determinant of the covariant metric tensor $g_{ij}$.

The Laplacian of a vector using covariant derivatives is
\begin{equation}
  \laplacian{\vectr{\phi}}=\text{grad }\pbrac{\text{div }\vectr{\phi}}-\text{curl } \pbrac{\text{curl }\vectr{\phi}}==\mixedderiv{\vectr{\phi}}{i}{i}
\end{equation}

The Laplacian of a contravariant (rank (0,1)) tensor $\phi^{k}$ is
\begin{equation}
  \laplacian{\vectr{\phi}}=\pbrac{\laplacian{\phi_{k}}-2g^{ij}\christoffel{K}{j}{H}\delby{\phi^{h}}{x^{i}}+\phi^{h}\delby{g^{ij}\christoffel{K}{i}{j}}{x^{h}}}\vectr{e}^{k}
\end{equation}
and the covariant derivative of a covariant tensor  (rank (1,0)) $\phi_{k}$ is
\begin{equation}
  \laplacian{\vectr{\phi}}=\pbrac{\laplacian{\phi_{k}}-2g^{ij}\christoffel{h}{j}{k}\delby{\phi_{h}}{x^{i}}+\phi_{h}g^{ij}\delby{\christoffel{h}{i}{j}}{x^{i}}}\vectr{e}_{k}
\end{equation}

\section{Equation set types}

\subsection{Static Equations}

The general form for static equations is

\subsection{Dynamic Equations}
\label{sec:dynamicequations}

The general form for dynamic equations is
\begin{equation}
  \matr{M}\fnof{\ddot{\vectr{u}}}{t}+\matr{C}\fnof{\dot{\vectr{u}}}{t}+\matr{K}\fnof{\vectr{u}}{t}+
  \fnof{\vectr{g}}{\fnof{\vectr{u}}{t}}+\fnof{\vectr{f}}{t}=\vectr{0}
  \label{eqn:generaldynamicnonlinear}
\end{equation}
where $\fnof{\vectr{u}}{t}$ is the unknown ``displacement vector'', $\matr{M}$
is the mass matrix, $\matr{C}$ is the damping matrix, $\matr{K}$ is the
stiffness matrix, $\fnof{\vectr{g}}{\fnof{\vectr{u}}{t}}$ a non-linear vector
function and $\fnof{\vectr{f}}{t}$ the forcing vector.

From \cite{zienkiewicz:2006_1} we now expand the unknown vector $\fnof{\vectr{u}}{t}$ in terms of a polynomial of degree
$p$. With the known values of $\vectr{u}_{n}$, $\dot{\vectr{u}}_{n}$,
$\ddot{\vectr{u}}_{n}$ up to $\symover{p-1}{\vectr{u}}_{n}$ at the beginning of
the time step $\Delta t$ we can write the polynomial expansion as
\begin{equation}
  \fnof{\vectr{u}}{t_{n}+\tau}\approx\fnof{\tilde{\vectr{u}}}{t_{n}+\tau}=\vectr{u}_{n}+\tau\dot{\vectr{u}}_{n}+
  \frac{1}{2!}\tau^{2}\ddot{\vectr{u}}_{n}+\cdots+\dfrac{1}{\factorial{p-1}}\tau^{p-1}\symover{p-1}{\vectr{u}}_{n}+
  \dfrac{1}{p!}\tau^{p}\vectr{\alpha}^{p}_{n}
  \label{eqn:timepolyexpansion}
\end{equation}
where the only unknown is the the vector $\vectr{\alpha}^{p}_{n}$,
\begin{equation}
  \vectr{\alpha}^{p}_{n}\approx\symover{p}{\vectr{u}}\equiv\dnby{p}{\vectr{u}}{t}
\end{equation}

A recurrance relationship can be established by substituting
\eqnref{eqn:timepolyexpansion} into \eqnref{eqn:generaldynamicnonlinear} and
taking a weighted residual approach \ie
\begin{multline}
  \dintl{0}{\Delta
    t}\fnof{W}{\tau}\left[\matr{M}\pbrac{\ddot{\vectr{u}}_{n}+\tau\dddot{\vectr{u}}_{n}+\cdots+
    \dfrac{1}{\factorial{p-2}}\tau^{p-2}\vectr{\alpha}^{p}_{n}} \right.\\
  +\matr{C}\pbrac{\dot{\vectr{u}}_{n}+\tau\ddot{\vectr{u}}_{n}+\cdots+
    \dfrac{1}{\factorial{p-1}}\tau^{p-1}\vectr{\alpha}^{p}_{n}} \\
  +\matr{K}\pbrac{\vectr{u}_{n}+\tau\dot{\vectr{u}}_{n}+\cdots+
    \dfrac{1}{p!}\tau^{p}\vectr{\alpha}^{p}_{n}} \\
  +\left.\fnof{\vectr{g}}{\vectr{u}_{n}+\tau\dot{\vectr{u}}_{n}+\cdots+
    \dfrac{1}{p!}\tau^{p}\vectr{\alpha}^{p}_{n}}+\fnof{\vectr{f}}{t_{n}+\tau}\right] d\tau = \vectr{0}
\end{multline}
where $\fnof{W}{\tau}$ is some weight function, $\tau=t-t_{n}$ and $\Delta
t=t_{n+1}-t_{n}$. Dividing by $\gint{0}{\Delta t}{\fnof{W}{\tau}}{\tau}$ we obtain
\begin{multline}
  \dfrac{\gint{0}{\Delta t}{\fnof{W}{\tau}\matr{M}\pbrac{\ddot{\vectr{u}}_{n}+\tau\dddot{\vectr{u}}_{n}+\cdots+
        \dfrac{1}{\factorial{p-2}}\tau^{p-2}\vectr{\alpha}^{p}_{n}}}{\tau}}{\gint{0}{\Delta
      t}{\fnof{W}{\tau}}{\tau}} \\
  + \dfrac{\gint{0}{\Delta t}{\fnof{W}{\tau}\matr{C}\pbrac{\dot{\vectr{u}}_{n}+\tau\ddot{\vectr{u}}_{n}+\cdots+
        \dfrac{1}{\factorial{p-1}}\tau^{p-1}\vectr{\alpha}^{p}_{n}}}{\tau}}{\gint{0}{\Delta
      t}{\fnof{W}{\tau}}{\tau}} \\
  + \dfrac{\gint{0}{\Delta t}{\fnof{W}{\tau}\matr{K}\pbrac{\vectr{u}_{n}+\tau\dot{\vectr{u}}_{n}+\cdots+
        \dfrac{1}{p!}\tau^{p}\vectr{\alpha}^{p}_{n}}}{\tau}}{\gint{0}{\Delta
      t}{\fnof{W}{\tau}}{\tau}} \\
  + \dfrac{\gint{0}{\Delta t}{\fnof{W}{\tau}\fnof{\vectr{g}}{\vectr{u}_{n}+\tau\dot{\vectr{u}}_{n}+\cdots+
        \dfrac{1}{p!}\tau^{p}\vectr{\alpha}^{p}_{n}}}{\tau}}{\gint{0}{\Delta
      t}{\fnof{W}{\tau}}{\tau}}  
  + \dfrac{\gint{0}{\Delta t}{\fnof{W}{\tau}\fnof{\vectr{f}}{t_{n}+
        \tau}}{\tau}}{\gint{0}{\Delta t}{\fnof{W}{\tau}}{\tau}}=\vectr{0}
\end{multline}

Now we define 
\begin{equation}
  \bar{\vectr{g}}=\dfrac{\gint{0}{\Delta
      t}{\fnof{W}{\tau}\fnof{\vectr{g}}{\fnof{\vectr{u}}{t_{n}+\tau}}}{\tau}}{
    \gint{0}{\Delta t}{\fnof{W}{\tau}}{\tau}}
  \label{eqn:meanweightednonlinearvector}
\end{equation}
and
\begin{equation}
  \bar{\vectr{f}}=\dfrac{\gint{0}{\Delta
      t}{\fnof{W}{\tau}\fnof{\vectr{f}}{t_{n}+\tau}}{\tau}}{
    \gint{0}{\Delta t}{\fnof{W}{\tau}}{\tau}}
  \label{eqn:meanweightedloadvector}
\end{equation}
and
\begin{equation}
  \theta_{k}=\dfrac{\gint{0}{\Delta t}{\fnof{W}{\tau}\tau^{k}}{\tau}}{{\Delta
      t}^{k}\gint{0}{\Delta t}{\fnof{W}{\tau}}{\tau}} \text{  for  } k=0,1,\ldots,p
  \label{eqn:thetakdefinition}
\end{equation}

We can now write
\begin{multline}
  \matr{M}\pbrac{\ddot{\bar{\vectr{u}}}_{n+1}+\dfrac{\theta_{p-2}{\Delta
        t}^{p-2}}{\factorial{p-2}}\vectr{\alpha}^{p}_{n}}+
  \matr{C}\pbrac{\dot{\bar{\vectr{u}}}_{n+1}+\dfrac{\theta_{p-1}{\Delta
        t}^{p-1}}{\factorial{p-1}}\vectr{\alpha}^{p}_{n}}+
  \matr{K}\pbrac{\bar{\vectr{u}}_{n+1}+\dfrac{\theta_{p}{\Delta
        t}^{p}}{p!}\vectr{\alpha}^{p}_{n}}+ \\
  + \dfrac{\gint{0}{\Delta t}{\fnof{W}{\tau}\fnof{\vectr{g}}{\vectr{u}_{n}+\tau\dot{\vectr{u}}_{n}+\cdots+
        \dfrac{1}{p!}\tau^{p}\vectr{\alpha}^{p}_{n}}}{\tau}}{\gint{0}{\Delta
      t}{\fnof{W}{\tau}}{\tau}}+\bar{\vectr{f}}=\vectr{0}
  \label{eqn:dynamic1}
\end{multline}
where
\begin{equation}
  \begin{split}
    \bar{\vectr{u}}_{n+1} &= \gsum{q=0}{p-1}{\dfrac{\theta_{q}{\Delta
            t}^{q}}{q!}\symover{q}{\vectr{u}}_{n}} \\
    \dot{\bar{\vectr{u}}}_{n+1} &= \gsum{q=1}{p-1}{\dfrac{\theta_{q-1}{\Delta
            t}^{q-1}}{\factorial{q-1}}\symover{q}{\vectr{u}}_{n}} \\
    \ddot{\bar{\vectr{u}}}_{n+1} &= \gsum{q=2}{p-1}{\dfrac{\theta_{q-2}{\Delta
            t}^{q-2}}{\factorial{q-2}}\symover{q}{\vectr{u}}_{n}} 
  \end{split}
\end{equation}

We note that as $\fnof{\vectr{g}}{\fnof{\vectr{u}}{t}}$ is nonlinear we need to
evaluate an integral of the form
\begin{equation}
  \gint{0}{\Delta
    t}{\fnof{W}{\tau}\fnof{\vectr{g}}{\fnof{\vectr{u}}{t_{n}+\tau}}}{\tau}
  \label{eqn:nonlineartimeweightintegral}
\end{equation}

In order to ensure that the accuracy of the integration scheme is not
compromised we need to evaluate \eqnref{eqn:nonlineartimeweightintegral} with
$\nth{p}$ order accuracy in time. For a first order in time approximation we form Taylor's series expansions for
$\fnof{\vectr{g}}{\fnof{\vectr{u}}{t}}$ about the point
$\fnof{\vectr{u}}{t_{n}+\tau}$ \ie
\begin{equation}
  \fnof{\vectr{g}}{\fnof{\vectr{u}}{t_{n}}}=\fnof{\vectr{g}}{\fnof{\vectr{u}}{t_{n}+\tau}}-
  \tau\fnof{\dot{\vectr{g}}}{\fnof{\vectr{u}}{t_{n}+\tau}}+\orderof{\tau^{2}}
  \label{eqn:firstordercurrentTaylorexpansion}
\end{equation}
and
\begin{equation}
  \fnof{\vectr{g}}{\fnof{\vectr{u}}{t_{n+1}}}=\fnof{\vectr{g}}{\fnof{\vectr{u}}{t_{n}+\tau}}+
  \pbrac{\Delta
    t-\tau}\fnof{\dot{\vectr{g}}}{\fnof{\vectr{u}}{t_{n}+\tau}}+\orderof{\pbrac{\Delta
      t-\tau}^{2}}
  \label{eqn:firstordernextTaylorexpansion}
\end{equation}

Now if we sum \eqnref{eqn:firstordercurrentTaylorexpansion} times
$\dfrac{\Delta t -\tau}{\Delta t}$ and
\eqnref{eqn:firstordernextTaylorexpansion} times 
$\dfrac{\tau}{\Delta t}$ we obtain
\begin{equation}
  \dfrac{\Delta t-\tau}{\Delta t}\fnof{\vectr{g}}{\fnof{\vectr{u}}{t_{n}}}+
  \dfrac{\tau}{\Delta t}\fnof{\vectr{g}}{\fnof{\vectr{u}}{t_{n+1}}}=
  \fnof{\vectr{g}}{\fnof{\vectr{u}}{t_{n}+\tau}}+\orderof{\tau^{2}}
\end{equation}

\Eqnref{eqn:nonlineartimeweightintegral} now becomes
\begin{equation}
  \gint{0}{\Delta t}{\fnof{W}{\tau}\fnof{\vectr{g}}{\fnof{\vectr{u}}{t_{n}+\tau}}}{\tau}
  =\gint{0}{\Delta t}{\fnof{W}{\tau}
      \pbrac{\dfrac{\Delta t-\tau}{\Delta t}\fnof{\vectr{g}}{\fnof{\vectr{u}}{t_{n}}}+
        \dfrac{\tau}{\Delta t}\fnof{\vectr{g}}{\fnof{\vectr{u}}{t_{n+1}}}-\orderof{\tau^{2}}}}{\tau}
\end{equation}
or
\begin{equation}
  \gint{0}{\Delta t}{\fnof{W}{\tau}\fnof{\vectr{g}}{\fnof{\vectr{u}}{t_{n}+\tau}}}{\tau}
  =\gint{0}{\Delta t}{\fnof{W}{\tau}
      \pbrac{\pbrac{1-\dfrac{\tau}{{\Delta t}}}\fnof{\vectr{g}}{\fnof{\vectr{u}}{t_{n}}}+
        \dfrac{\tau}{\Delta t}\fnof{\vectr{g}}{\fnof{\vectr{u}}{t_{n+1}}}-\orderof{\tau^{2}}}}{\tau}
\end{equation}
and so
\begin{multline}
\dfrac{\gint{0}{\Delta t}{\fnof{W}{\tau}
      \fnof{\vectr{g}}{\fnof{\vectr{u}}{t_{n}+\tau}}}{\tau}}{\gint{0}{\Delta
      t}{\fnof{W}{\tau}}{\tau}}=\dfrac{\gint{0}{\Delta
      t}{\fnof{W}{\tau}\fnof{\vectr{g}}{\fnof{\vectr{u}}{t_{n}}}}{\tau}}{\gint{0}{\Delta
    t}{\fnof{W}{\tau}}{\tau}}
-\dfrac{\gint{0}{\Delta
    t}{\fnof{W}{\tau}\tau\fnof{\vectr{g}}{\fnof{\vectr{u}}{t_{n}}}}{\tau}}{\Delta
  t\gint{0}{\Delta
    t}{\fnof{W}{\tau}}{\tau}} \\
+\dfrac{\gint{0}{\Delta
    t}{\fnof{W}{\tau}\tau\fnof{\vectr{g}}{\fnof{\vectr{u}}{t_{n+1}}}}{\tau}}{\Delta
  t\gint{0}{\Delta
    t}{\fnof{W}{\tau}}{\tau}}- 
  \dfrac{\gint{0}{\Delta
      t}{\fnof{W}{\tau}\orderof{\tau^{2}}}{\tau}}{\gint{0}{\Delta t}{\fnof{W}{\tau}}{\tau}}
\end{multline}

Now if we recall that $\theta_{0}=\dfrac{\gint{0}{\Delta
    t}{\fnof{W}{\tau}}{\tau}}{\gint{0}{\Delta t}{\fnof{W}{\tau}}{\tau}}=1$ and
$\theta_{1}=\dfrac{\gint{0}{\Delta t}{\fnof{W}{\tau}\tau}{\tau}}{\Delta
  t\gint{0}{\Delta t}{\fnof{W}{\tau}}{\tau}}$
we can write
\begin{equation}
  \bar{\vect{g}}=\pbrac{1-\theta_{1}}\fnof{\vectr{g}}{\fnof{\vectr{u}}{t_{n}}}+
  \theta_{1}\fnof{\vectr{g}}{\fnof{\vectr{u}}{t_{n+1}}}-\text{Error}
\end{equation}
where
\begin{equation}
  \text{Error}=\dfrac{\gint{0}{\Delta t}{\fnof{W}{\tau}\orderof{\tau^{2}}}{\tau}}{
    \gint{0}{\Delta t}{\fnof{W}{\tau}}{\tau}}
\end{equation}

For a second order in time approximation we need to consider three points. We form Taylor's series expansions for
$\fnof{\vectr{g}}{\fnof{\vectr{u}}{t}}$ about the point $\fnof{\vectr{u}}{t_{n}+\tau}$ \ie
\begin{multline}
  \fnof{\vectr{g}}{\fnof{\vectr{u}}{t_{n-1}}}=\fnof{\vectr{g}}{\fnof{\vectr{u}}{t_{n}+\tau}}
  - \pbrac{\Delta t+\tau}\fnof{\dot{\vectr{g}}}{\fnof{\vectr{u}}{t_{n}+\tau}} \\
  + \dfrac{\pbrac{\Delta t+\tau}^{2}}{2}\fnof{\ddot{\vectr{g}}}{\fnof{\vectr{u}}{t_{n}+\tau}}
  + \orderof{\pbrac{\Delta t+\tau}^{3}}
  \label{eqn:secondorderpreviousTaylorexpansion}
\end{multline}
and
\begin{equation}
  \fnof{\vectr{g}}{\fnof{\vectr{u}}{t_{n}}}=\fnof{\vectr{g}}{\fnof{\vectr{u}}{t_{n}+\tau}}
  - \tau\fnof{\dot{\vectr{g}}}{\fnof{\vectr{u}}{t_{n}+\tau}}
  + \dfrac{\tau^{2}}{2}\fnof{\ddot{\vectr{g}}}{\fnof{\vectr{u}}{t_{n}+\tau}}
  + \orderof{\tau^{3}}
  \label{eqn:secondordercurrentTaylorexpansion}
\end{equation}
and
\begin{multline}
  \fnof{\vectr{g}}{\fnof{\vectr{u}}{t_{n+1}}}=\fnof{\vectr{g}}{\fnof{\vectr{u}}{t_{n}+\tau}}
  + \pbrac{\Delta t-\tau}\fnof{\dot{\vectr{g}}}{\fnof{\vectr{u}}{t_{n}+\tau}} \\
  + \dfrac{\pbrac{\Delta t-\tau}^{2}}{2}\fnof{\ddot{\vectr{g}}}{\fnof{\vectr{u}}{t_{n}+\tau}}
  + \orderof{\pbrac{\Delta t-\tau}^{3}}
  \label{eqn:secondordernextTaylorexpansion}
\end{multline}

If we sum the equations \eqnref{eqn:secondorderpreviousTaylorexpansion} times
$\dfrac{-\tau\pbrac{\Delta t - \tau}}{2{\Delta t}^{2}}$,
\eqnref{eqn:secondordercurrentTaylorexpansion} times
$\dfrac{2\pbrac{\Delta t - \tau}\pbrac{\Delta t + \tau}}{2{\Delta t}^{2}}$, and
\eqnref{eqn:secondordernextTaylorexpansion} times
$\dfrac{\tau\pbrac{\Delta t + \tau}}{2{\Delta t}^{2}}$ we obtain
\begin{multline}
  \dfrac{-\tau\pbrac{\Delta t - \tau}}{2\Delta
    t^{2}}\fnof{\vectr{g}}{\fnof{\vectr{u}}{t_{n-1}}}+
  \dfrac{2\pbrac{\Delta t - \tau}\pbrac{\Delta t + \tau}}{2\Delta
    t^{2}}\fnof{\vectr{g}}{\fnof{\vectr{u}}{t_{n}}}+
  \dfrac{\tau\pbrac{\Delta t + \tau}}{2\Delta
    t^{2}}\fnof{\vectr{g}}{\fnof{\vectr{u}}{t_{n+1}}}\\
  =\fnof{\vectr{g}}{\fnof{\vectr{u}}{t_{n}+\tau}}+
  \orderof{\tau^{3}}
\end{multline}

\Eqnref{eqn:nonlineartimeweightintegral} now becomes
\begin{multline}
  \gint{0}{\Delta
    t}{\fnof{W}{\tau}\fnof{\vectr{g}}{\fnof{\vectr{u}}{t_{n}+\tau}}}{\tau} = 
  \dintl{0}{\Delta t}\,\fnof{W}{\tau}\left(\dfrac{-\tau\pbrac{\Delta t - \tau}}{2\Delta
    t^{2}}\fnof{\vectr{g}}{\fnof{\vectr{u}}{t_{n-1}}}+\right. \\
  \left.\dfrac{2\pbrac{\Delta t - \tau}\pbrac{\Delta t + \tau}}{2\Delta
    t^{2}}\fnof{\vectr{g}}{\fnof{\vectr{u}}{t_{n}}}+
  \dfrac{\tau\pbrac{\Delta t + \tau}}{2\Delta
    t^{2}}\fnof{\vectr{g}}{\fnof{\vectr{u}}{t_{n+1}}}-\orderof{\tau^{3}}\right)\exteriorderivop\tau
\end{multline}
or
\begin{multline}
  \gint{0}{\Delta t}{\fnof{W}{\tau}\fnof{\vectr{g}}{\fnof{\vectr{u}}{t_{n}+\tau}}}{\tau} = 
  \dintl{0}{\Delta t}\,\fnof{W}{\tau}\left(\pbrac{\dfrac{-\tau}{2\Delta t}+\dfrac{\tau^{2}}{2\Delta
      t^{2}}}\fnof{\vectr{g}}{\fnof{\vectr{u}}{t_{n-1}}}\right. \\
  \left.+\pbrac{1-\dfrac{\tau^{2}}{{\Delta t}^{2}}}\fnof{\vectr{g}}{\fnof{\vectr{u}}{t_{n}}}
  +\pbrac{\dfrac{\tau}{2\Delta t}+\dfrac{\tau^{2}}{2{\Delta t}^{2}}}\fnof{\vectr{g}}{\fnof{\vectr{u}}{t_{n+1}}}
    -\orderof{\tau^{3}}\right)\exteriorderivop\tau
\end{multline}
and so 
\begin{multline}
  \dfrac{\gint{0}{\Delta t}{\fnof{W}{\tau}\fnof{\vectr{g}}{\fnof{\vectr{u}}{t_{n}+\tau}}}{\tau}}
        {\gint{0}{\Delta t}{\fnof{W}{\tau}}{\tau}}
        =\dfrac{\gint{0}{\Delta
            t}{\fnof{W}{\tau}\tau\fnof{\vectr{g}}{\fnof{\vectr{u}}{t_{n-1}}}}{\tau}}{2\Delta
          t\gint{0}{\Delta t}{\fnof{W}{\tau}}{\tau}}+ \dfrac{\gint{0}{\Delta
            t}{\fnof{W}{\tau}\tau^{2}\fnof{\vectr{g}}{\fnof{\vectr{u}}{t_{n-1}}}}{\tau}}{2\Delta
          t^{2}\gint{0}{\Delta t}{\fnof{W}{\tau}}{\tau}}+
        \\ \dfrac{\gint{0}{\Delta
            t}{\fnof{W}{\tau}\fnof{\vectr{g}}{\fnof{\vectr{u}}{t_{n}}}}{\tau}}{\gint{0}{\Delta
            t}{\fnof{W}{\tau}}{\tau}}- \dfrac{\gint{0}{\Delta
            t}{\fnof{W}{\tau}\tau^{2}\fnof{\vectr{g}}{\fnof{\vectr{u}}{t_{n}}}}{\tau}}{\Delta
          t^{2}\gint{0}{\Delta t}{\fnof{W}{\tau}}{\tau}}+
        \dfrac{\gint{0}{\Delta
            t}{\fnof{W}{\tau}\tau\fnof{\vectr{g}}{\fnof{\vectr{u}}{t_{n+1}}}}{\tau}}{2\Delta
          t\gint{0}{\Delta t}{\fnof{W}{\tau}}{\tau}}+
        \\ \dfrac{\gint{0}{\Delta
            t}{\fnof{W}{\tau}\tau^{2}\fnof{\vectr{g}}{\fnof{\vectr{u}}{t_{n+1}}}}{\tau}}{2\Delta
          t^{2}\gint{0}{\Delta t}{\fnof{W}{\tau}}{\tau}}-
        \dfrac{\gint{0}{\Delta
            t}{\fnof{W}{\tau}\orderof{\tau^{3}}}{\tau}}{\gint{0}{\Delta
            t}{\fnof{W}{\tau}}{\tau}}
\end{multline}

Now if we recall $\theta_{0}=\dfrac{\gint{0}{\Delta
    t}{\fnof{W}{\tau}}{\tau}}{\gint{0}{\Delta t}{\fnof{W}{\tau}}{\tau}}=1$,
$\theta_{1}=\dfrac{\gint{0}{\Delta t}{\fnof{W}{\tau}\tau}{\tau}}{\Delta
  t\gint{0}{\Delta t}{\fnof{W}{\tau}}{\tau}}$, and
$\theta_{2}=\dfrac{\gint{0}{\Delta t}{\fnof{W}{\tau}\tau^{2}}{\tau}}{\Delta
  t^{2}\gint{0}{\Delta t}{\fnof{W}{\tau}}{\tau}}$ we can write
\begin{equation}
  \bar{\vectr{g}}=
  \dfrac{\pbrac{\theta_{2}-\theta_{1}}}{2}\fnof{\vectr{g}}{\fnof{\vectr{u}}{t_{n-1}}}+
  \pbrac{1-\theta_{2}}\fnof{\vectr{g}}{\fnof{\vectr{u}}{t_{n}}}+
  \dfrac{\pbrac{\theta_{2}+\theta_{1}}}{2}\fnof{\vectr{g}}{\fnof{\vectr{u}}{t_{n+1}}}-\text{Error}
\end{equation}
where
\begin{equation}
  \text{Error}=\dfrac{\gint{0}{\Delta t}{\fnof{W}{\tau}\orderof{\tau^{3}}}{\tau}}{
    \gint{0}{\Delta t}{\fnof{W}{\tau}}{\tau}}
\end{equation}

For a third order in time approximation we need to consider four points. We
form Taylor's series expansions for $\fnof{\vectr{g}}{\fnof{\vectr{u}}{t}}$
about the point $\fnof{\vectr{u}}{t_{n}+\tau}$ \ie
\begin{multline}
  \fnof{\vectr{g}}{\fnof{\vectr{u}}{t_{n-2}}}=\fnof{\vectr{g}}{\fnof{\vectr{u}}{t_{n}+\tau}}
  - \pbrac{2\Delta t+\tau}\fnof{\dot{\vectr{g}}}{\fnof{\vectr{u}}{t_{n}+\tau}} 
  + \dfrac{\pbrac{2\Delta t+\tau}^{2}}{2}\fnof{\ddot{\vectr{g}}}{\fnof{\vectr{u}}{t_{n}+\tau}} \\
  - \dfrac{\pbrac{2\Delta t+\tau}^{3}}{6}\fnof{\dddot{\vectr{g}}}{\fnof{\vectr{u}}{t_{n}+\tau}}
  + \orderof{\pbrac{2\Delta t+\tau}^{4}}
  \label{eqn:thirdorderpreviouspreviousTaylorexpansion}
\end{multline}
and
\begin{multline}
  \fnof{\vectr{g}}{\fnof{\vectr{u}}{t_{n-1}}}=\fnof{\vectr{g}}{\fnof{\vectr{u}}{t_{n}+\tau}}
  - \pbrac{\Delta t+\tau}\fnof{\dot{\vectr{g}}}{\fnof{\vectr{u}}{t_{n}+\tau}} 
  + \dfrac{\pbrac{\Delta t+\tau}^{2}}{2}\fnof{\ddot{\vectr{g}}}{\fnof{\vectr{u}}{t_{n}+\tau}} \\
  - \dfrac{\pbrac{\Delta t+\tau}^{3}}{6}\fnof{\dddot{\vectr{g}}}{\fnof{\vectr{u}}{t_{n}+\tau}}
  + \orderof{\pbrac{\Delta t+\tau}^{4}}
  \label{eqn:thirdorderpreviousTaylorexpansion}
\end{multline}
and
\begin{multline}
  \fnof{\vectr{g}}{\fnof{\vectr{u}}{t_{n}}}=\fnof{\vectr{g}}{\fnof{\vectr{u}}{t_{n}+\tau}}
  - \tau\fnof{\dot{\vectr{g}}}{\fnof{\vectr{u}}{t_{n}+\tau}}
  + \dfrac{\tau^{2}}{2}\fnof{\ddot{\vectr{g}}}{\fnof{\vectr{u}}{t_{n}+\tau}} \\
  - \dfrac{\tau^{3}}{6}\fnof{\dddot{\vectr{g}}}{\fnof{\vectr{u}}{t_{n}+\tau}}
  + \orderof{\tau^{4}}
  \label{eqn:thirdordercurrentTaylorexpansion}
\end{multline}
and
\begin{multline}
  \fnof{\vectr{g}}{\fnof{\vectr{u}}{t_{n+1}}}=\fnof{\vectr{g}}{\fnof{\vectr{u}}{t_{n}+\tau}}
  + \pbrac{\Delta t-\tau}\fnof{\dot{\vectr{g}}}{\fnof{\vectr{u}}{t_{n}+\tau}} 
  + \dfrac{\pbrac{\Delta t-\tau}^{2}}{2}\fnof{\ddot{\vectr{g}}}{\fnof{\vectr{u}}{t_{n}+\tau}} \\
  + \dfrac{\pbrac{\Delta t-\tau}^{3}}{6}\fnof{\dddot{\vectr{g}}}{\fnof{\vectr{u}}{t_{n}+\tau}}
  + \orderof{\pbrac{\Delta t-\tau}^{4}}
  \label{eqn:thirdordernextTaylorexpansion}
\end{multline}

If we sum \eqnref{eqn:thirdorderpreviouspreviousTaylorexpansion} times
$\dfrac{\tau\pbrac{\Delta t - \tau}\pbrac{\Delta t + \tau}}{6\Delta t^{3}}$,
\eqnref{eqn:thirdorderpreviousTaylorexpansion} times
$\dfrac{-3\tau\pbrac{\Delta t - \tau}\pbrac{2\Delta t + \tau}}{6\Delta
  t^{3}}$, \eqnref{eqn:thirdordercurrentTaylorexpansion} times
$\dfrac{3\pbrac{\Delta t - \tau}\pbrac{\Delta t + \tau}\pbrac{2\Delta t +
    \tau}}{6\Delta t^{3}}$, and \eqnref{eqn:thirdordernextTaylorexpansion}
times $\dfrac{\tau\pbrac{\Delta t + \tau}\pbrac{2\Delta t + \tau}}{6\Delta
  t^{3}}$ we obtain
\begin{multline}
  \dfrac{\tau\pbrac{\Delta t - \tau}\pbrac{\Delta t + \tau}}{6\Delta
    t^{3}}\fnof{\vectr{g}}{\fnof{\vectr{u}}{t_{n-2}}}
  -\dfrac{3\tau\pbrac{\Delta t - \tau}\pbrac{2\Delta t + \tau}}{6\Delta
    t^{3}}\fnof{\vectr{g}}{\fnof{\vectr{u}}{t_{n-1}}} \\
  +\dfrac{3\pbrac{\Delta t - \tau}\pbrac{\Delta t + \tau}\pbrac{2\Delta t + \tau}}{6\Delta
    t^{3}}\fnof{\vectr{g}}{\fnof{\vectr{u}}{t_{n}}}
  +\dfrac{\tau\pbrac{\Delta t + \tau}\pbrac{2\Delta t + \tau}}{6\Delta
    t^{3}}\fnof{\vectr{g}}{\fnof{\vectr{u}}{t_{n+1}}} \\
  =\fnof{\vectr{g}}{\fnof{\vectr{u}}{t_{n}+\tau}}+
  \orderof{\tau^{4}}
\end{multline}

\Eqnref{eqn:nonlineartimeweightintegral} now becomes
\begin{multline}
  \gint{0}{\Delta
    t}{\fnof{W}{\tau}\fnof{\vectr{g}}{\fnof{\vectr{u}}{t_{n}+\tau}}}{\tau} =
  \dintl{0}{\Delta t}\,\fnof{W}{\tau}\left(\dfrac{\tau\pbrac{\Delta
      t-\tau}\pbrac{\Delta t+\tau}}{6\Delta
    t^{3}}\fnof{\vectr{g}}{\fnof{\vectr{u}}{t_{n-2}}}\right. \\ \left.-\dfrac{3\tau\pbrac{\Delta
      t-\tau}\pbrac{2\Delta t+\tau}}{6\Delta
    t^{3}}\fnof{\vectr{g}}{\fnof{\vectr{u}}{t_{n-1}}} +\dfrac{3\pbrac{\Delta
      t-\tau}\pbrac{\Delta t+\tau}\pbrac{2\Delta t+\tau}}{6\Delta
    t^{3}}\fnof{\vectr{g}}{\fnof{\vectr{u}}{t_{n}}}\right.\\ \left.+\dfrac{\tau\pbrac{\Delta
      t+\tau}\pbrac{2\Delta t+\tau}}{6\Delta
    t^{3}}\fnof{\vectr{g}}{\fnof{\vectr{u}}{t_{n+1}}}-\orderof{\tau^{4}}\right)\exteriorderivop\tau
\end{multline}
or
\begin{multline}
  \gint{0}{\Delta
    t}{\fnof{W}{\tau}\fnof{\vectr{g}}{\fnof{\vectr{u}}{t_{n}+\tau}}}{\tau} =
  \dintl{0}{\Delta t}\,\fnof{W}{\tau}
  \left(\pbrac{\dfrac{\tau}{6{\Delta t}}-\dfrac{\tau^{3}}{6{\Delta t}^{3}}}
  \fnof{\vectr{g}}{\fnof{\vectr{u}}{t_{n-2}}}\right. \\
  \left.+\pbrac{\dfrac{-\tau}{{\Delta t}}+\dfrac{\tau^{2}}{2{\Delta t}^{2}}+\dfrac{\tau^{3}}{2{\Delta t}^{3}}}
  \fnof{\vectr{g}}{\fnof{\vectr{u}}{t_{n-1}}}
  +\pbrac{1+\dfrac{\tau}{2{\Delta t}}-\dfrac{\tau^{2}}{{\Delta t}^{2}}-\dfrac{\tau^{3}}{2{\Delta t}^{3}}}
  \fnof{\vectr{g}}{\fnof{\vectr{u}}{t_{n}}}\right.\\
  \left.+\pbrac{\dfrac{\tau}{3{\Delta t}}+\dfrac{\tau^{2}}{2{\Delta t}^{2}}+\dfrac{\tau^{3}}{6{\Delta t}^{3}}}
  \fnof{\vectr{g}}{\fnof{\vectr{u}}{t_{n+1}}}
  -\orderof{\tau^{4}}\right)\exteriorderivop\tau
\end{multline}
and so
\begin{multline}
  \dfrac{\gint{0}{\Delta t}{\fnof{W}{\tau}
      \fnof{\vectr{g}}{\fnof{\vectr{u}}{t_{n}+\tau}}}{\tau}}{\gint{0}{\Delta
      t}{\fnof{W}{\tau}}{\tau}}=\dfrac{\gint{0}{\Delta
      t}{\fnof{W}{\tau}\tau\fnof{\vectr{g}}{\fnof{\vectr{u}}{t_{n-2}}}}{\tau}}{6\Delta
    t\gint{0}{\Delta t}{\fnof{W}{\tau}}{\tau}}-
  \dfrac{\gint{0}{\Delta
      t}{\fnof{W}{\tau}\tau^{3}\fnof{\vectr{g}}{\fnof{\vectr{u}}{t_{n-2}}}}{\tau}}{6\Delta
    t^{3}\gint{0}{\Delta t}{\fnof{W}{\tau}}{\tau}} \\
  -\dfrac{\gint{0}{\Delta
      t}{\fnof{W}{\tau}\tau\fnof{\vectr{g}}{\fnof{\vectr{u}}{t_{n-1}}}}{\tau}}{{\Delta
      t}\gint{0}{\Delta t}{\fnof{W}{\tau}}{\tau}}
  +\dfrac{\gint{0}{\Delta
      t}{\fnof{W}{\tau}\tau^{2}\fnof{\vectr{g}}{\fnof{\vectr{u}}{t_{n-1}}}}{\tau}}{2{\Delta
    t}^{2}\gint{0}{\Delta t}{\fnof{W}{\tau}}{\tau}}
  +\dfrac{\gint{0}{\Delta
      t}{\fnof{W}{\tau}\tau^{3}\fnof{\vectr{g}}{\fnof{\vectr{u}}{t_{n-1}}}}{\tau}}{2{\Delta
    t}^{3}\gint{0}{\Delta t}{\fnof{W}{\tau}}{\tau}} \\
  +\dfrac{\gint{0}{\Delta
      t}{\fnof{W}{\tau}\fnof{\vectr{g}}{\fnof{\vectr{u}}{t_{n}}}}{\tau}}{\gint{0}{\Delta t}{\fnof{W}{\tau}}{\tau}}
  +\dfrac{\gint{0}{\Delta
      t}{\fnof{W}{\tau}\tau\fnof{\vectr{g}}{\fnof{\vectr{u}}{t_{n}}}}{\tau}}{2{\Delta
    t}\gint{0}{\Delta t}{\fnof{W}{\tau}}{\tau}}
  -\dfrac{\gint{0}{\Delta
      t}{\fnof{W}{\tau}\tau^{2}\fnof{\vectr{g}}{\fnof{\vectr{u}}{t_{n}}}}{\tau}}{{\Delta
    t}^{2}\gint{0}{\Delta t}{\fnof{W}{\tau}}{\tau}} \\
  -\dfrac{\gint{0}{\Delta
      t}{\fnof{W}{\tau}\tau^{3}\fnof{\vectr{g}}{\fnof{\vectr{u}}{t_{n}}}}{\tau}}{2{\Delta
      t}^{3}\gint{0}{\Delta t}{\fnof{W}{\tau}}{\tau}}
  +\dfrac{\gint{0}{\Delta
      t}{\fnof{W}{\tau}\tau\fnof{\vectr{g}}{\fnof{\vectr{u}}{t_{n+1}}}}{\tau}}{3{\Delta
    t}\gint{0}{\Delta t}{\fnof{W}{\tau}}{\tau}}
  +\dfrac{\gint{0}{\Delta
      t}{\fnof{W}{\tau}\tau^{2}\fnof{\vectr{g}}{\fnof{\vectr{u}}{t_{n+1}}}}{\tau}}{2{\Delta
    t}^{2}\gint{0}{\Delta t}{\fnof{W}{\tau}}{\tau}} \\
  +\dfrac{\gint{0}{\Delta
      t}{\fnof{W}{\tau}\tau^{3}\fnof{\vectr{g}}{\fnof{\vectr{u}}{t_{n+1}}}}{\tau}}{6\Delta
    t^{3}\gint{0}{\Delta t}{\fnof{W}{\tau}}{\tau}}
  -\dfrac{\gint{0}{\Delta
      t}{\fnof{W}{\tau}\orderof{\tau^{4}}}{\tau}}{\gint{0}{\Delta t}{\fnof{W}{\tau}}{\tau}}
\end{multline}

If we recall that $\theta_{0}=\dfrac{\gint{0}{\Delta
    t}{\fnof{W}{\tau}}{\tau}}{\gint{0}{\Delta t}{\fnof{W}{\tau}}{\tau}}=1$,
$\theta_{1}=\dfrac{\gint{0}{\Delta t}{\fnof{W}{\tau}\tau}{\tau}}{\Delta
  t\gint{0}{\Delta t}{\fnof{W}{\tau}}{\tau}}$,
$\theta_{2}=\dfrac{\gint{0}{\Delta t}{\fnof{W}{\tau}\tau^{2}}{\tau}}{\Delta
  t^{2}\gint{0}{\Delta t}{\fnof{W}{\tau}}{\tau}}$, and
$\theta_{3}=\dfrac{\gint{0}{\Delta t}{\fnof{W}{\tau}\tau^{3}}{\tau}}{{\Delta
    t}^{3}\gint{0}{\Delta t}{\fnof{W}{\tau}}{\tau}}$ we can write
\begin{multline}
  \bar{\vect{g}}=\dfrac{\pbrac{\theta_{1}-\theta_{3}}}{6}\fnof{\vectr{g}}{\fnof{\vectr{u}}{t_{n-2}}}
  +\dfrac{\pbrac{\theta_{3}+\theta_{2}-2\theta_{1}}}{2}\fnof{\vectr{g}}{\fnof{\vectr{u}}{t_{n-1}}}
  +\pbrac{1-\dfrac{\pbrac{\theta_{3}+2\theta_{2}-\theta_{1}}}{2}}\fnof{\vectr{g}}{\fnof{\vectr{u}}{t_{n}}} \\
  +\dfrac{\pbrac{\theta_{3}+3\theta_{2}+2\theta_{1}}}{6}\fnof{\vectr{g}}{\fnof{\vectr{u}}{t_{n+1}}}
  -\text{Error}
\end{multline}
where
\begin{equation}
  \text{Error}=\dfrac{\gint{0}{\Delta t}{\fnof{W}{\tau}\orderof{\tau^{4}}}{\tau}}{
    \gint{0}{\Delta t}{\fnof{W}{\tau}}{\tau}}
\end{equation}

\Eqnref{eqn:dynamic1} now becomes
\begin{multline}
  \matr{M}\pbrac{\ddot{\bar{\vectr{u}}}_{n+1}+\dfrac{\theta_{p-2}{\Delta
        t}^{p-2}}{\factorial{p-2}}\vectr{\alpha}^{p}_{n}}+
  \matr{C}\pbrac{\dot{\bar{\vectr{u}}}_{n+1}+\dfrac{\theta_{p-1}{\Delta
        t}^{p-1}}{\factorial{p-1}}\vectr{\alpha}^{p}_{n}}
  +\matr{K}\pbrac{\bar{\vectr{u}}_{n+1}+\dfrac{\theta_{p}{\Delta
        t}^{p}}{p!}\vectr{\alpha}^{p}_{n}}\\
  +\pbrac{1-\theta_{1}}\fnof{\vectr{g}}{\vectr{u}_{n}}+\theta_{1}\fnof{\vectr{g}}{\vectr{u}_{n+1}}+\bar{\vectr{f}}+
  \text{Error}=\vectr{0}
  \label{eqn:firstorderdynamic2}
\end{multline}
with a first order approximation in time or 
\begin{multline}
  \matr{M}\pbrac{\ddot{\bar{\vectr{u}}}_{n+1}+\dfrac{\theta_{p-2}{\Delta
        t}^{p-2}}{\factorial{p-2}}\vectr{\alpha}^{p}_{n}}+
  \matr{C}\pbrac{\dot{\bar{\vectr{u}}}_{n+1}+\dfrac{\theta_{p-1}{\Delta
        t}^{p-1}}{\factorial{p-1}}\vectr{\alpha}^{p}_{n}}
  +\matr{K}\pbrac{\bar{\vectr{u}}_{n+1}+\dfrac{\theta_{p}{\Delta
        t}^{p}}{p!}\vectr{\alpha}^{p}_{n}}\\
  +\dfrac{\pbrac{\theta_{2}-\theta_{1}}}{2}\fnof{\vectr{g}}{\vectr{u}_{n-1}}+
  \pbrac{1-\theta_{2}}\fnof{\vectr{g}}{\vectr{u}_{n}}+
  \dfrac{\pbrac{\theta_{2}+\theta_{1}}}{2}\fnof{\vectr{g}}{\vectr{u}_{n+1}}+\bar{\vectr{f}}-
  \text{Error}=\vectr{0}
  \label{eqn:secondorderdynamic2}
\end{multline}
with a second order approximation in time or
\begin{multline}
  \matr{M}\pbrac{\ddot{\bar{\vectr{u}}}_{n+1}+\dfrac{\theta_{p-2}{\Delta
        t}^{p-2}}{\factorial{p-2}}\vectr{\alpha}^{p}_{n}}
  +\matr{C}\pbrac{\dot{\bar{\vectr{u}}}_{n+1}+\dfrac{\theta_{p-1}{\Delta
        t}^{p-1}}{\factorial{p-1}}\vectr{\alpha}^{p}_{n}}
  +\matr{K}\pbrac{\bar{\vectr{u}}_{n+1}+\dfrac{\theta_{p}{\Delta
        t}^{p}}{p!}\vectr{\alpha}^{p}_{n}}\\ +\dfrac{\pbrac{\theta_{1}-\theta_{3}}}{6}\fnof{\vectr{g}}{\vectr{u}_{n-2}}
  +\dfrac{\pbrac{\theta_{3}+\theta_{2}-2\theta_{1}}}{2}\fnof{\vectr{g}}{\vectr{u}_{n-1}}
  +\pbrac{1-\dfrac{\pbrac{\theta_{3}+2\theta_{2}-\theta_{1}}}{2}}\fnof{\vectr{g}}{\vectr{u}_{n}} \\
  +\dfrac{\pbrac{\theta_{3}+3\theta_{2}+2\theta_{1}}}{6}\fnof{\vectr{g}}{\vectr{u}_{n+1}}+\bar{\vectr{f}}-
  \text{Error}=\vectr{0}
  \label{eqn:thirdorderdynamic2}
\end{multline}
with a third order approximation in time as
$\fnof{\vectr{u}}{t_{n-2}}=\vectr{u}_{n-2}$,
$\fnof{\vectr{u}}{t_{n-1}}=\vectr{u}_{n-1}$,
$\fnof{\vectr{u}}{t_{n}}=\vectr{u}_{n}$ and
$\fnof{\vectr{u}}{t_{n+1}}=\vectr{u}_{n+1}=\hat{\vectr{u}}_{n+1}+
\dfrac{{\Delta t}^{p}}{p!}\vectr{\alpha}^{p}_{n}$ where
$\hat{\vectr{u}}_{n+1}$ is the \emph{predicted displacement} at the new time
step and is given by
\begin{equation}
  \hat{\vectr{u}}_{n+1}=\gsum{q=0}{p-1}{\dfrac{{\Delta
        t}^{q}}{q!}\symover{q}{\vectr{u}}_{n}}
\end{equation}

Rearranging gives
\begin{multline}
  \fnof{\vectr{\psi}}{\vectr{\alpha}^{p}_{n}}=\pbrac{\dfrac{\theta_{p-2}{\Delta
        t}^{p-2}}{\factorial{p-2}}\matr{M}+\dfrac{\theta_{p-1}{\Delta
        t}^{p-1}}{\factorial{p-1}}\matr{C}+\dfrac{\theta_{p}{\Delta
        t}^{p}}{p!}\matr{K}}\vectr{\alpha}^{p}_{n}
  +\theta_{1}\fnof{\vectr{g}}{\hat{\vectr{u}}_{n+1}+ \dfrac{{\Delta
        t}^{p}}{p!}\vectr{\alpha}^{p}_{n}} \\
  +\pbrac{1-\theta_{1}}\fnof{\vectr{g}}{\vectr{u}_{n}}
  +\pbrac{\matr{M}\ddot{\bar{\vectr{u}}}_{n+1}+\matr{C}\dot{\bar{\vectr{u}}}_{n+1}+\matr{K}\bar{\vectr{u}}_{n+1}
    +\bar{\vectr{f}}}= \vectr{0}
  \label{eqn:firstorderdynamic}
\end{multline}
for a first order approximation or
\begin{multline}
  \fnof{\vectr{\psi}}{\vectr{\alpha}^{p}_{n}}=\pbrac{\dfrac{\theta_{p-2}{\Delta
        t}^{p-2}}{\factorial{p-2}}\matr{M}+\dfrac{\theta_{p-1}{\Delta
        t}^{p-1}}{\factorial{p-1}}\matr{C}+\dfrac{\theta_{p}{\Delta
        t}^{p}}{p!}\matr{K}}\vectr{\alpha}^{p}_{n}
  +\dfrac{\pbrac{\theta_{2}+\theta_{1}}}{2}\fnof{\vectr{g}}{\hat{\vectr{u}}_{n+1}+
    \dfrac{{\Delta t}^{p}}{p!}\vectr{\alpha}^{p}_{n}} \\
  +\pbrac{1-\theta_{2}}\fnof{\vectr{g}}{\vectr{u}_{n}}+\dfrac{\pbrac{\theta_{2}-\theta_{1}}}{2}\fnof{\vectr{g}}{\vectr{u}_{n-1}}
  +\pbrac{\matr{M}\ddot{\bar{\vectr{u}}}_{n+1}+\matr{C}\dot{\bar{\vectr{u}}}_{n+1}+\matr{K}\bar{\vectr{u}}_{n+1}
    +\bar{\vectr{f}}}= \vectr{0}
  \label{eqn:secondorderdynamic}
\end{multline}
for a second order approximation or
\begin{multline}
  \fnof{\vectr{\psi}}{\vectr{\alpha}^{p}_{n}}=\pbrac{\dfrac{\theta_{p-2}{\Delta
        t}^{p-2}}{\factorial{p-2}}\matr{M}+\dfrac{\theta_{p-1}{\Delta
        t}^{p-1}}{\factorial{p-1}}\matr{C}+\dfrac{\theta_{p}{\Delta
        t}^{p}}{p!}\matr{K}}\vectr{\alpha}^{p}_{n}+
  \dfrac{\pbrac{\theta_{3}+3\theta_{2}+2\theta_{1}}}{6}\fnof{\vectr{g}}{\hat{\vectr{u}}_{n+1}+
    \dfrac{{\Delta t}^{p}}{p!}\vectr{\alpha}^{p}_{n}} \\
  +\pbrac{1-\dfrac{\pbrac{\theta_{3}+2\theta_{2}-\theta_{1}}}{2}}\fnof{\vectr{g}}{\vectr{u}_{n}}
  +\dfrac{\pbrac{\theta_{3}+\theta_{2}-2\theta_{1}}}{2}\fnof{\vectr{g}}{\vectr{u}_{n-1}}
  +\dfrac{\pbrac{\theta_{1}-\theta_{3}}}{6}\fnof{\vectr{g}}{\vectr{u}_{n-2}} \\
  +\pbrac{\matr{M}\ddot{\bar{\vectr{u}}}_{n+1}+\matr{C}\dot{\bar{\vectr{u}}}_{n+1}+\matr{K}\bar{\vectr{u}}_{n+1}+
    \bar{\vectr{f}}}= \vectr{0}
  \label{eqn:thirdorderdynamic}
\end{multline}
for a third order approximation.

If we now define the \emph{Amplification matrix} $\matr{A}$ as
\begin{equation}
  \matr{A}=\dfrac{\theta_{p-2}{\Delta t}^{p-2}}{\factorial{p-2}}\matr{M}+
  \dfrac{\theta_{p-1}{\Delta t}^{p-1}}{\factorial{p-1}}\matr{C}+
  \dfrac{\theta_{p}{\Delta t}^{p}}{p!}\matr{K}
\end{equation}
and the right hand side vector $\vectr{b}$ as
\begin{equation}
  \vectr{b}=\matr{M}\ddot{\bar{\vectr{u}}}_{n+1}+\matr{C}\dot{\bar{\vectr{u}}}_{n+1}+
  \matr{K}\bar{\vectr{u}}_{n+1}+\bar{\vectr{f}}
\end{equation}

We can now write \eqnref{eqn:firstorderdynamic} as
\begin{equation}
\fnof{\vectr{\psi}}{\vectr{\alpha}^{p}_{n}}=\matr{A}\vectr{\alpha}^{p}_{n}+
\theta_{1}\fnof{\vectr{g}}{\hat{\vectr{u}}_{n+1}+ \dfrac{{\Delta
      t}^{p}}{p!}\vectr{\alpha}^{p}_{n}}+\pbrac{1-\theta_{1}}\fnof{\vectr{g}}{\vectr{u}_{n}}+\vectr{b}= \vectr{0}
\end{equation}
and \eqnref{eqn:secondorderdynamic} as
\begin{multline}
\fnof{\vectr{\psi}}{\vectr{\alpha}^{p}_{n}}=\matr{A}\vectr{\alpha}^{p}_{n}+
\dfrac{\pbrac{\theta_{2}+\theta_{1}}}{2}\fnof{\vectr{g}}{\hat{\vectr{u}}_{n+1}
  +\dfrac{{\Delta t}^{p}}{p!}\vectr{\alpha}^{p}_{n}} \\
+\pbrac{1-\theta_{2}}\fnof{\vectr{g}}{\vectr{u}_{n}}+
\dfrac{\pbrac{\theta_{2}-\theta_{1}}}{2}\fnof{\vectr{g}}{\vectr{u}_{n-1}}+\vectr{b}=\vectr{0}
\end{multline}
and \eqnref{eqn:thirdorderdynamic} as
\begin{multline}
\fnof{\vectr{\psi}}{\vectr{\alpha}^{p}_{n}}=\matr{A}\vectr{\alpha}^{p}_{n}+
\dfrac{\pbrac{\theta_{3}+3\theta_{2}+2\theta_{1}}}{6}\fnof{\vectr{g}}{\hat{\vectr{u}}_{n+1}
  +\dfrac{{\Delta t}^{p}}{p!}\vectr{\alpha}^{p}_{n}} \\
+\pbrac{1-\dfrac{\pbrac{\theta_{3}+2\theta_{2}-\theta_{1}}}{2}}\fnof{\vectr{g}}{\vectr{u}_{n}}
+\dfrac{\pbrac{\theta_{3}+\theta_{2}-2\theta_{1}}}{2}\fnof{\vectr{g}}{\vectr{u}_{n-1}} \\
+\dfrac{\pbrac{\theta_{1}-\theta_{3}}}{6}\fnof{\vectr{g}}{\vectr{u}_{n-2}}
+\vectr{b}=\vectr{0}
\end{multline}

If $\fnof{\vectr{g}}{\vectr{u}}\equiv\vectr{0}$ then
\eqnref{eqn:firstorderdynamic} or \eqnref{eqn:secondorderdynamic} is linear in
$\vectr{\alpha}^{p}_{n}$ and $\vectr{\alpha}^{p}_{n}$ can be found by solving
the linear equation
\begin{equation}
  \vectr{\alpha}^{p}_{n} =-\inverse{\pbrac{\dfrac{\theta_{p-2}{\Delta t}^{p-2}}{\factorial{p-2}}\matr{M}+
      \dfrac{\theta_{p-1}{\Delta t}^{p-1}}{\factorial{p-1}}\matr{C}+
      \dfrac{\theta_{p}{\Delta
          t}^{p}}{p!}\matr{K}}}\pbrac{\matr{M}\ddot{\bar{\vectr{u}}}_{n+1}+
    \matr{C}\dot{\bar{\vectr{u}}}_{n+1}+\matr{K}\bar{\vectr{u}}_{n+1}+\bar{\vectr{f}}}
\end{equation}
or 
\begin{equation}
  \vectr{\alpha}^{p}_{n} =-\inverse{\matr{A}}\vectr{b}
\end{equation}

If $\fnof{\vectr{g}}{\vectr{u}}$ is not $\equiv\vectr{0}$ then
\eqnref{eqn:firstorderdynamic} or \eqnref{eqn:secondorderdynamic} is nonlinear in $\vectr{\alpha}^{p}_{n}$. To solve this
equation we use Newton's method \ie
\begin{equation}
  \begin{split}
    \text{1.  } & \fnof{\matr{J}}{\vectr{\alpha}^{p}_{n(i)}}.\delta
    \vectr{\alpha}^{p}_{n(i)} = 
    -\fnof{\vectr{\psi}}{\vectr{\alpha}^{p}_{n(i)}} \\
    \text{2.  } & \vectr{\alpha}^{p}_{n(i+1)}=\vectr{\alpha}^{p}_{n(i)}+\delta
    \vectr{\alpha}^{p}_{n(i)}
  \end{split}
\end{equation}
where $\fnof{\matr{J}}{\vectr{\alpha}^{p}_{n}}$ is the Jacobian and is given by
  \begin{align}
    \fnof{\matr{J}}{\vectr{\alpha}^{p}_{n}}&=\dfrac{\theta_{p-2}{\Delta t}^{p-2}}{\factorial{p-2}}\matr{M}+
    \dfrac{\theta_{p-1}{\Delta
      t}^{p-1}}{\factorial{p-1}}\matr{C}+\dfrac{\theta_{p}{\Delta t}^{p}}{p!}\matr{K}+
  \theta_{1}
  \delby{\fnof{\vectr{g}}{\hat{\vectr{u}}_{n+1}+\dfrac{{\Delta
          t}^{p}}{p!}
      \vectr{\alpha}^{p}_{n}}}{\vectr{\alpha}^{p}_{n}}\\
      &=\dfrac{\theta_{p-2}{\Delta t}^{p-2}}{\factorial{p-2}}\matr{M}+
    \dfrac{\theta_{p-1}{\Delta
      t}^{p-1}}{\factorial{p-1}}\matr{C}+\dfrac{\theta_{p}{\Delta t}^{p}}{p!}\matr{K}+
  \theta_{1}\delby{\vectr{u}}{\vectr{\alpha}^{p}_{n}}
  \delby{\fnof{\vectr{g}}{\hat{\vectr{u}}_{n+1}+\dfrac{{\Delta
          t}^{p}}{p!}
      \vectr{\alpha}^{p}_{n}}}{\vectr{u}}\\
      &=\dfrac{\theta_{p-2}{\Delta t}^{p-2}}{\factorial{p-2}}\matr{M}+
    \dfrac{\theta_{p-1}{\Delta
      t}^{p-1}}{\factorial{p-1}}\matr{C}+\dfrac{\theta_{p}{\Delta t}^{p}}{p!}\matr{K}+
  \dfrac{\theta_{1}\Delta t^{p}}{p!}
  \delby{\fnof{\vectr{g}}{\hat{\vectr{u}}_{n+1}+\dfrac{{\Delta
          t}^{p}}{p!}
      \vectr{\alpha}^{p}_{n}}}{\vectr{u}}
  \end{align}
or
\begin{equation}
  \fnof{\matr{J}}{\vectr{\alpha}^{p}_{n}}=\matr{A}+\dfrac{\theta_{1}{\Delta
      t}^{p}}{p!}
  \delby{\fnof{\vectr{g}}{\hat{\vectr{u}}_{n+1}+\dfrac{{\Delta t}^{p}}{p!}\vectr{\alpha}^{p}_{n}}}{\vectr{u}}
\end{equation}
for a first order approximation or by
\begin{equation}
  \fnof{\matr{J}}{\vectr{\alpha}^{p}_{n}}=\dfrac{\theta_{p-2}{\Delta t}^{p-2}}{\factorial{p-2}}\matr{M}+
  \dfrac{\theta_{p-1}{\Delta
      t}^{p-1}}{\factorial{p-1}}\matr{C}+\dfrac{\theta_{p}{\Delta t}^{p}}{p!}\matr{K}+
  \dfrac{\pbrac{\theta_{2}+\theta_{1}}{\Delta t}^{p}}{2p!}
  \delby{\fnof{\vectr{g}}{\hat{\vectr{u}}_{n+1}+\dfrac{{\Delta
          t}^{p}}{p!}
      \vectr{\alpha}^{p}_{n}}}{\vectr{u}}
\end{equation}
or
\begin{equation}
  \fnof{\matr{J}}{\vectr{\alpha}^{p}_{n}}=\matr{A}+\dfrac{\pbrac{\theta_{2}+\theta_{1}}{\Delta
      t}^{p}}{2p!}
  \delby{\fnof{\vectr{g}}{\hat{\vectr{u}}_{n+1}+\dfrac{{\Delta t}^{p}}{p!}\vectr{\alpha}^{p}_{n}}}{\vectr{u}}
\end{equation}
for a second order approximation or by
\begin{equation}
  \fnof{\matr{J}}{\vectr{\alpha}^{p}_{n}}=\dfrac{\theta_{p-2}{\Delta
      t}^{p-2}}{\factorial{p-2}}\matr{M}+ \dfrac{\theta_{p-1}{\Delta
      t}^{p-1}}{\factorial{p-1}}\matr{C}+\dfrac{\theta_{p}{\Delta
      t}^{p}}{p!}\matr{K}+
  \dfrac{\pbrac{\theta_{3}+3\theta_{2}+2\theta_{1}}{\Delta t}^{p}}{6p!}
  \delby{\fnof{\vectr{g}}{\hat{\vectr{u}}_{n+1}+\dfrac{{\Delta t}^{p}}{p!}
      \vectr{\alpha}^{p}_{n}}}{\vectr{\alpha}^{p}_{n}}
\end{equation}
or
\begin{equation}
  \fnof{\matr{J}}{\vectr{\alpha}^{p}_{n}}=\matr{A}+\dfrac{\pbrac{\theta_{3}+3\theta_{2}+2\theta_{1}}{\Delta
      t}^{p}}{6p!}
  \delby{\fnof{\vectr{g}}{\hat{\vectr{u}}_{n+1}+\dfrac{{\Delta
          t}^{p}}{p!}\vectr{\alpha}^{p}_{n}}}{\vectr{\alpha}^{p}_{n}}
\end{equation}
for a third order approximation.

Once $\vectr{\alpha}^{p}_{n}$ has been obtained the values at the next time step can be obtained from
\begin{equation}
  \begin{split}
    \vectr{u}_{n+1} &= \vectr{u}_{n}+\Delta t
    \dot{\vectr{u}}_{n}+\cdots+\dfrac{{\Delta
        t}^{p}}{p!}\vectr{\alpha}^{p}_{n}=\hat{\vectr{u}}_{n+1}+
    \dfrac{{\Delta t}^{p}}{p!}\vectr{\alpha}^{p}_{n}\\
    \dot{\vectr{u}}_{n+1} &= \dot{\vectr{u}}_{n}+\Delta t
    \ddot{\vectr{u}}_{n}+\cdots+\dfrac{{\Delta
        t}^{p-1}}{\factorial{p-1}}\vectr{\alpha}^{p}_{n}=\dot{\hat{\vectr{u}}}_{n+1}+\dfrac{{\Delta
        t}^{p-1}}{\factorial{p-1}}\vectr{\alpha}^{p}_{n} \\
    &\vdots \\
    \symover{p-1}{\vectr{u}}_{n+1} &= \symover{p-1}{\vectr{u}}_{n}+\Delta t\vectr{\alpha}^{p}_{n}
  \end{split}
\end{equation}

For algorithms in which the degree of the polynomial, $p$, is higher than the
order we require the algorithm to be initialised so that the initial velocity
or acceleration can be computed. The initial velocity or acceleration values
can be obtained by substituting the initial displacement or initial
displacement and velocity values into \eqnref{eqn:generaldynamicnonlinear},
rearranging and solving. For example consider the case of a second degree
polynomial and a first order system. Substituing the initial displacement
$\vectr{u}_{0}$ into \eqnref{eqn:generaldynamicnonlinear} gives
\begin{equation}
  \matr{C}\dot{\vectr{u}}_{0}+\matr{K}\vectr{u}_{0}+\fnof{\vectr{g}}{\vectr{u}_{0}}+\bar{\vectr{f}}_{0}=\vectr{0}
\end{equation}
and therefore an approximation to the initial velocity can be found from
\begin{equation}
  \dot{\vectr{u}}_{0}=-\inverse{\matr{C}}\pbrac{\matr{K}\vectr{u}_{0}+\fnof{\vectr{g}}{\vectr{u}_{0}}+\bar{\vectr{f}}_{0}}
\end{equation}

Similarily for a third degree polynomial and a second order system the initial
acceleration can be found from
\begin{equation}
  \ddot{\vectr{u}}_{0}=-\inverse{\matr{M}}\pbrac{\matr{C}\dot{\vectr{u}}_{0}+\matr{K}\vectr{u}_{0}+
    \fnof{\vectr{g}}{\vectr{u}_{0}}+\bar{\vectr{f}}_{0}}
\end{equation}

To evaluate the mean weighted load vector, $\bar{\vectr{f}}$, we need to
evaluate the integral in \eqnref{eqn:meanweightedloadvector}. In a similar
fashion to how \eqnref{eqn:meanweightednonlinearvector} is evaluated we can
calculate
\begin{equation}
  \bar{\vectr{f}}=\theta_{1}\vectr{f}_{n+1}+\pbrac{1-\theta_{1}}\vectr{f}_{n}
  \label{eqn:firstordermeanweightedloadvector}
\end{equation}
for a first order approximation or
\begin{equation}
  \bar{\vectr{f}}=\dfrac{\pbrac{\theta_{2}+\theta_{1}}}{2}\vectr{f}_{n+1}+\pbrac{1-\theta_{2}}\vectr{f}_{n}
  +\dfrac{\pbrac{\theta_{2}-\theta_{1}}}{2}\vectr{f}_{n-1}
  \label{eqn:secondordermeanweightedloadvector}
\end{equation}
for a second order approximation or
\begin{multline}
  \bar{\vectr{f}}=\dfrac{\pbrac{\theta_{3}+3\theta_{2}+2\theta_{1}}}{6}\vectr{f}_{n+1}
  +\pbrac{1-\dfrac{\pbrac{\theta_{3}+2\theta_{2}-\theta_{1}}}{2}}\vectr{f}_{n} \\
  +\dfrac{\pbrac{\theta_{3}+\theta_{2}-2\theta_{1}}}{2}\vectr{f}_{n-1}
  +\dfrac{\pbrac{\theta_{1}-\theta_{3}}}{6}\vectr{f}_{n-2}
  \label{eqn:thirdordermeanweightedloadvector}
\end{multline}
for a third order approximation.

\subsubsection{Special SN11 case, p=1}

For this special case, the mean predicited values are given by
\begin{equation}
   \bar{\vectr{u}}_{n+1} = \vectr{u}_{n}
\end{equation}

The predicted displacement values are given by
\begin{equation}
   \hat{\vectr{u}}_{n+1} = \vectr{u}_{n}
\end{equation}

The amplification matrix is given by
\begin{equation}
  \matr{A}=\matr{C}+\theta_{1}\Delta t \matr{K}
\end{equation}

The right hand side vector is given by
\begin{equation}
  \vectr{b}=\matr{K}\bar{\vectr{u}}_{n+1}+\bar{\vectr{f}}
\end{equation}

The load vector is given by
\begin{equation}
  \bar{\vectr{f}}=\theta_{1}\vectr{f}_{n+1}+\pbrac{1-\theta_{1}}\vectr{f}_{n}
\end{equation}

The nonlinear function is given by
\begin{equation}
  \fnof{\vectr{\psi}}{\vectr{\alpha}^{1}_{n}}=\matr{A}\vectr{\alpha}^{1}_{n}+\theta_{1}\fnof{\vectr{g}}{\hat{\vectr{u}}_{n+1}+ 
    \Delta t\vectr{\alpha}^{1}_{n}}+\pbrac{1-\theta_{1}}\fnof{\vectr{g}}{\vectr{u}_{n}}+\vectr{b}=\vectr{0}
\end{equation}

The Jacobian matrix is given by
\begin{equation}
  \fnof{\matr{J}}{\vectr{\alpha}^{1}_{n}}=\matr{A}+\theta_{1}\Delta t
  \delby{\fnof{\vectr{g}}{\hat{\vectr{u}}_{n+1}+\Delta t\vectr{\alpha}^{1}_{n}}}{\vectr{u}}
\end{equation}

And the time step update is given by
\begin{equation}
    \vectr{u}_{n+1} = \vectr{u}_{n}+\Delta t\vectr{\alpha}^{1}_{n}
\end{equation}

\subsubsection{Special SN21 case, p=2}

For this special case, the mean predicited values are given by
\begin{equation}
  \begin{split}
    \bar{\vectr{u}}_{n+1} &= \vectr{u}_{n}+\theta_{1}\Delta t\dot{\vectr{u}}_{n}\\
    \dot{\bar{\vectr{u}}}_{n+1} &= \dot{\vectr{u}}_{n}
  \end{split}
\end{equation}
where
\begin{equation}
  \dot{\vectr{u}}_{0}=-\inverse{\matr{C}}\pbrac{\matr{K}\vectr{u}_{0}+\fnof{\vectr{g}}{\vectr{u}_{0}}+\bar{\vectr{f}}_{0}}
\end{equation}

The predicted displacement values are given by
\begin{equation}
   \hat{\vectr{u}}_{n+1} = \vectr{u}_{n}+\Delta t\dot{\vectr{u}}_{n}
\end{equation}

The amplification matrix is given by
\begin{equation}
  \matr{A}=\theta_{1}\Delta t\matr{C}+\dfrac{\theta_{2}{\Delta t}^{2}}{2}\matr{K}
\end{equation}

The right hand side vector is given by
\begin{equation}
  \vectr{b}=\matr{C}\dot{\bar{\vectr{u}}}_{n+1}+\matr{K}\bar{\vectr{u}}_{n+1}+\bar{\vectr{f}}
\end{equation}

The load vector is given by
\begin{equation}
  \bar{\vectr{f}}=\dfrac{\pbrac{\theta_{2}+\theta_{1}}}{2}\vectr{f}_{n+1}+\pbrac{1-\theta_{2}}\vectr{f}_{n}
  +\dfrac{\pbrac{\theta_{2}-\theta_{1}}}{2}\vectr{f}_{n-1}
\end{equation}

The nonlinear function is given by
\begin{multline}
  \fnof{\vectr{\psi}}{\vectr{\alpha}^{2}_{n}}=\matr{A}\vectr{\alpha}^{2}_{n}+\dfrac{\pbrac{\theta_{2}+\theta_{1}}}{2}\fnof{\vectr{g}}{\hat{\vectr{u}}_{n+1}+
    \dfrac{{\Delta t}^{2}}{2}\vectr{\alpha}^{2}_{n}}+\\
  \pbrac{1-\theta_{2}}\fnof{\vectr{g}}{\vectr{u}_{n}}+\dfrac{\pbrac{\theta_{2}-\theta_{1}}}{2}\fnof{\vectr{g}}{\vectr{u}_{n-1}}+\vectr{b}=\vectr{0}
\end{multline}

The Jacobian matrix is given by
\begin{equation}
  \fnof{\matr{J}}{\vectr{\alpha}^{2}_{n}}=\matr{A}+\dfrac{\pbrac{\theta_{2}+\theta_{1}}{\Delta t}^{2}}{4}
  \delby{\fnof{\vectr{g}}{\hat{\vectr{u}}_{n+1}+\dfrac{{\Delta t}^{2}}{2}\vectr{\alpha}^{2}_{n}}}{\vectr{u}}
\end{equation}

And the time step update is given by
\begin{equation}
  \begin{split}
    \vectr{u}_{n+1} &= \vectr{u}_{n}+\Delta t\dot{\vectr{u}}_{n} +\dfrac{{\Delta t}^{2}}{2}\vectr{\alpha}^{2}_{n} \\
    \dot{\vectr{u}}_{n+1} &= \dot{\vectr{u}}_{n}+\Delta t\vectr{\alpha}^{2}_{n}
  \end{split}
\end{equation}

\subsubsection{Special SN22 case, p=2}

For this special case, the mean predicited values are given by
\begin{equation}
  \begin{split}
    \bar{\vectr{u}}_{n+1} &= \vectr{u}_{n}+\theta_{1}\Delta t\dot{\vectr{u}}_{n}\\
    \dot{\bar{\vectr{u}}}_{n+1} &= \dot{\vectr{u}}_{n}
  \end{split}
\end{equation}

The predicted displacement values are given by
\begin{equation}
   \hat{\vectr{u}}_{n+1} = \vectr{u}_{n}+\Delta t\dot{\vectr{u}}_{n}
\end{equation}

The amplification matrix is given by
\begin{equation}
  \matr{A}=\matr{M}+\theta_{1}\Delta t\matr{C}+\dfrac{\theta_{2}{\Delta t}^{2}}{2}\matr{K}
\end{equation}

The right hand side vector is given by
\begin{equation}
  \vectr{b}=\matr{C}\dot{\bar{\vectr{u}}}_{n+1}+\matr{K}\bar{\vectr{u}}_{n+1}+\bar{\vectr{f}}
\end{equation}

The load vector is given by
\begin{equation}
  \bar{\vectr{f}}=\dfrac{\pbrac{\theta_{2}+\theta_{1}}}{2}\vectr{f}_{n+1}+\pbrac{1-\theta_{2}}\vectr{f}_{n}
  +\dfrac{\pbrac{\theta_{2}-\theta_{1}}}{2}\vectr{f}_{n-1}
\end{equation}

The nonlinear function is given by
\begin{multline}
  \fnof{\vectr{\psi}}{\vectr{\alpha}^{2}_{n}}=\matr{A}\vectr{\alpha}^{2}_{n}+\dfrac{\pbrac{\theta_{2}+\theta_{1}}}{2}\fnof{\vectr{g}}{\hat{\vectr{u}}_{n+1}+
    \dfrac{{\Delta t}^{2}}{2}\vectr{\alpha}^{2}_{n}}+\\
  \pbrac{1-\theta_{2}}\fnof{\vectr{g}}{\vectr{u}_{n}}+\dfrac{\pbrac{\theta_{2}-\theta_{1}}}{2}\fnof{\vectr{g}}{\vectr{u}_{n-1}}+\vectr{b}=\vectr{0}
\end{multline}

The Jacobian matrix is given by
\begin{equation}
  \fnof{\matr{J}}{\vectr{\alpha}^{2}_{n}}=\matr{A}+\dfrac{\pbrac{\theta_{2}+\theta_{1}}{\Delta t}^{2}}{4}
  \delby{\fnof{\vectr{g}}{\hat{\vectr{u}}_{n+1}+\dfrac{{\Delta t}^{2}}{2}\vectr{\alpha}^{2}_{n}}}{\vectr{u}}
\end{equation}

And the time step update is given by
\begin{equation}
  \begin{split}
    \vectr{u}_{n+1} &= \vectr{u}_{n}+\Delta t\dot{\vectr{u}}_{n} +\dfrac{{\Delta t}^{2}}{2}\vectr{\alpha}^{2}_{n} \\
    \dot{\vectr{u}}_{n+1} &= \dot{\vectr{u}}_{n}+\Delta t\vectr{\alpha}^{2}_{n} 
  \end{split}
\end{equation}

\subsubsection{Special SN32 case, p=3}

For this special case, the mean predicited values are given by
\begin{equation}
  \begin{split}
    \bar{\vectr{u}}_{n+1} &= \vectr{u}_{n}+\theta_{1}\Delta
    t\dot{\vectr{u}}_{n}+\dfrac{\theta_{2}{\Delta
        t}^{2}}{2}\ddot{\vectr{u}}_{n}\\
    \dot{\bar{\vectr{u}}}_{n+1} &=
    \dot{\vectr{u}}_{n}+\theta_{1}\Delta t\ddot{\vectr{u}}_{n}\\
    \ddot{\bar{\vectr{u}}}_{n+1} &= \ddot{\vectr{u}}_{n}
  \end{split}
\end{equation}

The predicted displacement values are given by
\begin{equation}
   \hat{\vectr{u}}_{n+1} = \vectr{u}_{n}+\Delta
   t\dot{\vectr{u}}_{n}+\dfrac{{\Delta t}^{2}}{2}\ddot{\vectr{u}}_{n}
\end{equation}

The amplification matrix is given by
\begin{equation}
  \matr{A}=\theta_{1}{\Delta t}\matr{M}+\dfrac{\theta_{2}{\Delta t}^{2}}{2}\matr{C}+\dfrac{\theta_{3}{\Delta t}^{3}}{6}\matr{K}
\end{equation}

The right hand side vector is given by
\begin{equation}
  \vectr{b}=\matr{M}\ddot{\bar{\vectr{u}}}_{n+1}+\matr{C}\dot{\bar{\vectr{u}}}_{n+1}+\matr{K}\bar{\vectr{u}}_{n+1}+\bar{\vectr{f}}
\end{equation}

The load vector is given by
\begin{multline}
  \bar{\vectr{f}}=\dfrac{\pbrac{\theta_{3}+3\theta_{2}+2\theta_{1}}}{6}\vectr{f}_{n+1}
  +\pbrac{1-\dfrac{\pbrac{\theta_{3}+2\theta_{2}-\theta_{1}}}{2}}\vectr{f}_{n} \\
  +\dfrac{\pbrac{\theta_{3}+\theta_{2}-2\theta_{1}}}{2}\vectr{f}_{n-1}
  +\dfrac{\pbrac{\theta_{1}-\theta_{3}}}{6}\vectr{f}_{n-2}
\end{multline}

The nonlinear function is given by
\begin{multline}
  \fnof{\vectr{\psi}}{\vectr{\alpha}^{3}_{n}}=\matr{A}\vectr{\alpha}^{3}_{n}+\dfrac{\pbrac{\theta_{3}+3\theta_{2}+2\theta_{1}}}{6}
  \fnof{\vectr{g}}{\hat{\vectr{u}}_{n+1}+\dfrac{{\Delta t}^{3}}{6}\vectr{\alpha}^{3}_{n}}+\\
  \pbrac{1-\dfrac{\pbrac{\theta_{3}+2\theta_{2}-\theta_{1}}}{2}}\fnof{\vectr{g}}{\vectr{u}_{n}}
  +\dfrac{\pbrac{\theta_{3}+\theta_{2}-2\theta_{1}}}{2}\fnof{\vectr{g}}{\vectr{u}_{n-1}} \\
  +\dfrac{\pbrac{\theta_{1}-\theta_{3}}}{6}\fnof{\vectr{g}}{\vectr{u}_{n-2}}
  +\vectr{b}=\vectr{0}
\end{multline}

The Jacobian matrix is given by
\begin{equation}
  \fnof{\matr{J}}{\vectr{\alpha}^{3}_{n}}=\matr{A}+\dfrac{\pbrac{\theta_{3}+3\theta_{2}+2\theta_{1}}{\Delta t}^{3}}{36}
  \delby{\fnof{\vectr{g}}{\hat{\vectr{u}}_{n+1}+\dfrac{{\Delta t}^{3}}{6}\vectr{\alpha}^{3}_{n}}}{\vectr{u}}
\end{equation}

And the time step update is given by
\begin{equation}
  \begin{split}
    \vectr{u}_{n+1} &= \vectr{u}_{n}+\Delta t\dot{\vectr{u}}_{n}
    +\dfrac{{\Delta t}^{2}}{2}\ddot{\vectr{u}}_{n}
    +\dfrac{{\Delta t}^{3}}{6}\vectr{\alpha}^{3}_{n} \\
    \dot{\vectr{u}}_{n+1} &= \dot{\vectr{u}}_{n}+\Delta t\ddot{\vectr{u}}_{n}+\dfrac{{\Delta t}^{2}}{2}\vectr{\alpha}^{3}_{n} \\
    \ddot{\vectr{u}}_{n+1} &= \ddot{\vectr{u}}_{n}+\Delta t\vectr{\alpha}^{3}_{n} \\
  \end{split}
\end{equation}

\section{Interface Conditions}

\subsection{Variational principles}

The branch of mathematics concerned with the problem of finding a function for
which a certain integral of that function is either at its largest or smallest
value is called the \emph{calculus of variations}. When scientific laws are formulated in terms of the principles of the calculus
of variations they are termed \emph{variational principles}. 

\subsection{Lagrange Multipliers}

\section{Optimisation Problems}

\subsection{Unconstrained Optimisation}

The general form of a unconstrained optimisation problem is
\begin{equation}
  \begin{aligned}
    \min_{\vect{x}} & & & \fnof{f}{\vect{x}} \\
    \text{subject to} & & & \vect{a}\le\vect{x}\le\vect{b}
  \end{aligned}
\end{equation}

\subsection{Linear Program}

The general form of a linear programming problem is
\begin{equation}
  \begin{aligned}
    \min_{\vect{x}} & & & \transpose{\vect{c}}{\vect{x}} \\
    \text{subject to} & & & \matr{A}\vect{x}=\vect{b} \\
    & & & \vect{x}\ge\vect{0}
  \end{aligned}
\end{equation}

\subsection{Constrained Optimisation}

The general form of a PDE constrained optimisation problem is
\begin{equation}
  \begin{aligned}
    \min_{\vect{x}} & & & \fnof{f}{\vect{x}} \\
    \text{subject to} &&& \vect{a}\le\vect{x}\le\vect{b} \\
    & & &\fnof{\vect{g}}{\vect{x}}=\vect{c} \\
    &&& \fnof{\vect{h}}{\vect{x}}\ge\vect{d}
  \end{aligned}
\end{equation}

\subsection{PDE Constrained Optimisation}

The general form of a PDE constrained optimisation problem is
\begin{equation}
  \begin{aligned}
    \min_{\vect{x},\vect{u}} & & & \fnof{f}{\vect{x},\vect{u}} \\
    \text{subject to} & & &\fnof{\vect{p}}{\vect{x},\vect{u}}=\vect{0} \\
    &&& \vect{a}\le\vect{x}\le\vect{b} \\
    & & &\fnof{\vect{g}}{\vect{x}}=\vect{c} \\
    &&& \fnof{\vect{h}}{\vect{x}}\ge\vect{d}
  \end{aligned}
\end{equation}
