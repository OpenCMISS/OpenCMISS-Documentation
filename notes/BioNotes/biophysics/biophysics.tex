\chapter{Biophysics of Nerve and Muscle}

\section{Nerves}

\subsection{Cable Theory}

\subsubsection{Cable equation}
We will consider a nerve as a special case of a conducting cable.
Consider a cable as is shown in \figref{fig:cable}.
\pstexfigure{biophysics/figs/cable.pstex}{}{A standard cable with axial
  current, $i_{a}$ (\mA), membrane current, $i_{m}$ (\mA/unit length of
  current), axial resistance $r_{a}$ (\ohm/unit length), and cable potential,
  $\fnof{V}{x,t}$ (\mV).}{fig:cable}{}

Ohm's law for the cable is
\begin{equation*}
  \delby{V}{x}=-r_{a}i_{a}
\end{equation*}
and continuity of current is given by
\begin{equation*}
  i_{m}=-\delby{i_{a}}{x}
\end{equation*}
Hence
\begin{equation*}
  \deltwosqby{V}{x}=-r_{a}\delby{i_{a}}{x}=r_{a}i_{m}
\end{equation*}
Now $i_{m}$ can be broken into a capacitive current, $i_{c}$, and an ionic
current $i_{i}$. For nerves and other biological cables the capacitive current
arises as a consequence of a bilipid membrane layer in cells as shown in
\figref{fig:bilipidmembrane}\nocite{zzz-berne:1988}.

The capacitive current arises from the separation of charge $Q$ across the
bilipid membrane, that is
\begin{equation*}
  Q=cV=\goneint{i_{c}}{t} \quad \text{or} \quad i_{c} = c\dby{V}{t}
\end{equation*}
where $c$ is the capacitance of the membrane (typically $c$ is about
1\uFpcmpcm for bilipid membranes).

\incgrfigure{width=10cm}{biophysics/epsfiles/cellmembrane.eps}{}{Structure
  of a phospholipid bilayer. From Physiology by R.M. Berne and M.N. Levy.}
{fig:bilipidmembrane}

The ionic current component of the membrane current comes about due to the 
transport of ions through an ionic channel in the membrane as shown in
\figref{fig:ionicchannel}.

\pstexfigure{biophysics/figs/ionicchannel.pstex}{}{Ionic channel in a cell
  membrane.}{fig:ionicchannel}{}

Substituting the capacitive current we obtain the cable theory equation (Note
that $i_{i}$ would be a leakage current in a normal cable).
\begin{equation}
  \dfrac{1}{r_{a}}\deltwosqby{V}{x}=c\delby{V}{t}+i_{i}
  \label{eqn:cabletheoryequation}
\end{equation}
Two special cases of cable conduction will now be considered:
\subsubsection{1. Steady State Passive Current}
Consider a steady state passive current given by $i_{i}=\dfrac{V}{r_{m}}$. The
cable equation now becomes
\begin{equation*}
  \dtwosqby{V}{x}=\dfrac{r_{a}}{r_{m}}V=\dfrac{V}{\lambda^{2}}\quad\text{where}
  \quad\lambda=\sqrt{\dfrac{r_{m}}{r_{a}}}
\end{equation*}
The solution to this equation is
\begin{equation}
  V(x)=Ae^{-x/\lambda}+Be^{x/\lambda}
  \label{eqn:longsscable}
\end{equation}
Note: for a long cable $B=0$ and this can be approximated as
$V(x)=Ae^{-x/\lambda}$ which gives the relationship shown in
\figref{fig:longcable}. In this case $\lambda$ can be interpreted as a {\em
  space constant}.

\pstexfigure{biophysics/figs/longcable.pstex}{}{Voltage relationship for a long
  cable with a steady state passive leakage current.}{fig:longcable}{}

\subsubsection{2. Constant Conduction Velocity}
If $\theta$ is the constant conduction velocity of the cable we look for
solutions of the form
\begin{equation*}
  \fnof{V}{x,t}=\fnof{f}{x-\theta t}
\end{equation*}
Therefore
\begin{equation*}
  \deltwosqby{V}{x}=f^{''}=\dfrac{1}{\theta^{2}}\deltwosqby{V}{t}
\end{equation*}
The cable equation hence becomes
\begin{equation*}
  \dfrac{1}{r_{a}\theta^{2}}\deltwosqby{V}{t}=c\delby{V}{t}+i_{i}
\end{equation*}
Now let the axial resistance, $r_{a}=\dfrac{R_{i}}{\pi r^{2}}$, where
$r$ is the radius and $R_{i}$ is the specific resistivity of the intra-cellular
fluid. Similarly let the ionic current, $i_{i}=2\pi r I_{i}$, where
$I_{i}$ is the ionic current per unit area of membrane, and let the
capacitance, $c=2\pi r C$, where $C$ is the capacitance per unit area of
membrane. The equation now becomes
\begin{equation}
  \dfrac{r}{2 R_{i} \theta^{2}}\dtwosqby{V}{t}=C\dby{V}{t}+I_{i}
  \label{eqn:constvelocitycable}
\end{equation}
Since $R_{i}, C$ and $I_{i}$ are independent of radius $r$ then
$\theta^{2}$ varies as $r$ or $\theta$ varies as $\sqrt{r}$. This dependence
is shown in \figref{fig:velocityradiusdependance}.

\pstexfigure{biophysics/figs/radiusdependance.pstex}{}{Dependence of conduction
  velocity, $\theta$, on cable radius, $r$.}{fig:velocityradiusdependance}{}

The implication of this dependence is that you can make signals travel faster
by increasing cable radius but at a diminishing rate (see
\figref{fig:velocityradiusdependance}). Typical conduction velocities are: AV
node $\sim$ $0.2$ \mps ($7$ \um fibres); Purkinje fibres $\sim$ $4$ \mps
($50$ \um fibres). Further increases in $\theta$ for large fibres 
require an increase in $r_{m}$. In nerves nature does this by introducing
insulation called myelin on the main fast conducting nerves. This is shown in
\figref{fig:myelinsheath}.  Multiple Sclerosis is characterised by the
destruction of these myelin sheaths.

\incgrfigure{width=10cm}{biophysics/epsfiles/myelinsheath.eps}{}{Myelin
  sheaths around nerves. (a) Schematic drawing of Schwann cells wrapping
  around an axon to form myelin. (b) Cross-section through a myelinated axon
  near a node of Ranvier. From Physiology by R.M. Berne and M.N.
  Levy.}{fig:myelinsheath}

\subsection{Ionic Currents}

\subsubsection{Membrane potential and current flow}
Ionic currents result from the transfer of ions through the cell membrane
between the intra-cellular and extra-cellular fluid as shown in
\figref{fig:cellspace}.

\pstexfigure{biophysics/figs/cellspace.pstex}{}{The cell membrane separating
  the intra- and extra-cellular spaces.}{fig:cellspace}{}

In general the cell-membrane will be permeable only to certain ions in certain
directions. Hence, in general the intra-cellular concentration for an ion will
be different from the extra-cellular concentration for the ion. The
concentrations of common ions in the different cell spaces for different
tissues are given in
\tabrefs{tab:ionconcentrations_skeletal}{tab:ionconcentrations_cardiac}. The
different concentrations set up an electro-chemical potential across the cell
membrane. This potential is given by the Nernst equation
\begin{equation}
  \begin{split}
    E &=  \dfrac{RT}{ZF}\ln\dfrac{\conc{C}{o}}{\conc{C}{i}} \\
    &=  2.303\dfrac{RT}{ZF}\log\dfrac{\conc{C}{o}}{\conc{C}{i}}
  \end{split}
  \label{eqn:nernst}
\end{equation}
where $R$ is the gas constant, $T$ the absolute temperature, $Z$ the valency
of the ion involved and $F$ is Faraday's constant. Note that at $37$ \degC
$\dfrac{RT}{F}$ is approximately $26$ \mV. Hence $2.303\dfrac{RT}{F}$=$60$ \mV
(i.e. $60$ \mV change for a $10$ fold change in concentration). Equilibrium
potentials for various ions are shown in
\tabrefs{tab:ionconcentrations_skeletal}{tab:ionconcentrations_cardiac}.

\begin{table} \centering
  \begin{tabular}{|c|c|c|c|} \hline
    & \emph{Intra-cellular} & \emph{Extra-Cellular} & \emph{Equilibrium} \\
    & \emph{concentration} & \emph{concentration} & \emph{potential} \\ \hline
    \ion{Na}{+} & $12$ \mM & $145$ \mM & $E_{\ion{Na}{+}}=+65$ \mV \\
    \ion{K}{+} & $155$ \mM & $4$ \mM & $E_{\ion{K}{+}}=-95$ \mV \\
    \ion{Ca}{2+} & $<\tento{-7}$ \mM & $1.5$ \mM & $E_{\ion{Ca}{2+}}=+128$ \mV\\
    \ion{Cl}{-} & $4.2$ \mM & $123$ \mM & $E_{\ion{Cl}{-}}=-88$ \mV \\ \hline
  \end{tabular}
  \caption{Intra- and Extra- cellular ion concentrations and equilibrium
    potentials for mammalian skeletal muscle.}
  \label{tab:ionconcentrations_skeletal}
\end{table}

\begin{table} \centering
  \begin{tabular}{|c|c|c|c|} \hline
    & \emph{Intra-cellular} & \emph{Extra-Cellular} & \emph{Equilibrium} \\
    & \emph{concentration} & \emph{concentration} & \emph{potential} \\ \hline
    \ion{Na}{+} & $5$--$34$ \mM & $140$ \mM & $E_{\ion{Na}{+}}=+51$ \mV \\
    \ion{K}{+} & $104$--$180$ \mM & $5.4$ \mM & $E_{\ion{K}{+}}=-85$ \mV \\
    \ion{Ca}{2+} & $0.0003$--$0.001$ \mM & $3$ \mM & $E_{\ion{Ca}{2+}}=+133$ 
    \mV \\
    \ion{Cl}{-} & $8$--$79$ \mM & $100$ \mM & $E_{\ion{Cl}{-}}=-22$ \mV \\ 
    \hline
  \end{tabular}
  \caption{Intra- and Extra- cellular ion concentrations and equilibrium
    potentials for mammalian cardiac muscle.}
  \label{tab:ionconcentrations_cardiac}
\end{table}

Now from Ohm's law the ionic current through a membrane for say, potassium,
will be given by
\begin{equation*}
  \begin{split}
    i_{K} &= \dfrac{1}{R}\Delta V \\
    &= g_{K}(V_{m} - E_{K})
  \end{split}
\end{equation*}
where $g_{K}$ is the membrane \emph{conductance} for potassium ions and $V_{m}$
is the \emph{membrane potential} (difference in potential between the
intra-cellular and extra-cellular spaces). This equation is represented
graphically in \figref{fig:membranecurrent}.

\pstexfigure{biophysics/figs/membranecurrent.pstex}{}{Potassium ionic current
  vs. membrane potential.}{fig:membranecurrent}{}

We should note here that by definition an outward current (from intra- to
extra-cellular space) is defined as a positive current. It should also be
noted that electrical current flows in the direction that the positive charge
carriers move. Hence the potassium current (in which the positive \ion{K}{+}
ions move from inside the cell to outside the cell) is a positive current.
Consider now \figref{fig:membranecurrent}. As the membrane potential increases
above $E_{K}$ the membrane current changes from a negative current to a
positive current, or in order words, the membrane current changes from an
inward current to an outward current. The point $E_{K}$ is hence known as the
\emph{reversal potential} as it is at this point that the current reverses its
direction.

Membrane conductance for an ion depends on the number of conducting channels
(see \figref{fig:ionicchannel}) and their properties or \emph{gating}.

\subsubsection{Gating mechanism}

When a conducting channel is fully open (i.e. all the gates are fully open)
the membrane current is given by
\begin{equation}
  \overline{i_{K}}=\overline{g_{K}}\pbrac{V_{m} - E_{K}}
\end{equation}
where $\overline{g_{K}}$ is constant (i.e. can be used in Ohm's law). Now
consider an intermediate state. If the fraction of gates open (or in their
$\alpha$ state) is $y$ then the fraction of gates closed (or in their
$\beta$ state) is $1-y$. The membrane conductance is hence given by
\begin{equation}
  g_{K}=\overline{g_{K}}y
\end{equation}
and the membrane current by
\begin{equation}
  i_{K}=\overline{i_{K}}y
\end{equation}
Now if the opening rate coefficient is $\alpha_{y}$, and the closing rate
coefficient is $\beta_{y}$ then the gating mechanism can be depicted as in
\figref{fig:gatingmechanism}.

\pstexfigure{biophysics/figs/gatingmechanism.pstex}{}{Gating
  mechanism.}{fig:gatingmechanism}{}

The change in the fraction of gates open is hence governed by
\begin{equation}
  \dby{y}{t}=\alpha_{y}\pbrac{1-y}-\beta_{y}y
\end{equation}
The processes of opening and closing of the gates is called \emph{gate
  kinetics}. The steady state for the gates for this form of gating kinetics
is given by
\begin{equation}
  \dby{y}{t}=\alpha_{y}\pbrac{1-y_{\infty}}-\beta_{y}y_{\infty}\quad
  \Rightarrow\quad y_{\infty}=\dfrac{\alpha_{y}}{\alpha_{y}+\beta_{y}}
\end{equation}

From experiment it has been found that both $\alpha_{y}$ and $\beta_{y}$
depend on the membrane voltage e.g. on depolarisation (increasing membrane
voltage) $\alpha_{y}$ might increase and $\beta_{y}$ might decrease (as is the
case for \ion{K}{+} ions). This is shown graphically in
\figref{fig:voltdepgates}.

\pstexfigure{biophysics/figs/voltdepgates.pstex}{}{Dependence of the gating
  rate constants on membrane voltage for a typical \ion{K}{+} type
  channel.}{fig:voltdepgates}{}

The behaviour of the gates (and hence channel conductance) for a membrane
voltage step is shown in \figref{fig:gatevoltstep}.

\pstexfigure{biophysics/figs/gatevoltstep.pstex}{}{Behaviour of open gates
  (and hence channel conductance) for a membrane voltage clamp step
  experiment.}{fig:gatevoltstep}{}

\subsection{The Squid Axon and Hodgkin-Huxley Equations}

\subsubsection{The squid action potential}

In 1952 A.L. Hodgkin and A.F. Huxley performed voltage clamp experiments (like
the one shown in \figref{fig:gatevoltstep}) on giant squid axons
\cite{hodgkin:1952}. They found that the onset of the \ion{K}{+} current was
sigmoidal in shape not exponential as in \figref{fig:gatevoltstep}. To explain
this the membrane current had to depend on the gates in a non-linear way, i.e.
\begin{equation}
  i_{y}=\overline{i_{y}}y^{\gamma}
\end{equation}
where $\gamma$ is the number of gates (in series) per channel.

For example consider two gates ($\gamma=2$) with $y=\dfrac{1}{2}$ (i.e. half the
gates open). In this case we can average all the gates into one of the four
possible states that the two gates can be in, namely (open, open), (open,
closed), (closed, open) and (closed, closed). For conduction through the
channel both gates have to be in the open state, that is the fraction of
conducting gates is $\dfrac{1}{4}=\pbrac{\dfrac{1}{2}}^{2}=y^{\gamma}$.

Hodgkin and Huxley found that you need $\gamma=4$ gates for \ion{K}{+}. These
gates were called the $n$ gates (or potassium activation variables) and were
found to open with increasing membrane potential. Hence the equation for the
potassium current is
\begin{equation}
  i_{K}=n^{4}\overline{g_{K}}\pbrac{V_{m} - E_{K}}
\end{equation}

For the sodium channels the gating mechanisms and kinetics were found to be a
bit more complicated. It was found that you needed gates which open with
increasing membrane potential \emph{and} gates which close with increasing
membrane potential. The gates which open with increasing potential were called
the $m$ gates (or sodium activation variables) and the gates which close with
increasing potential were called the $h$ gates (or sodium inactivation
variables). The equation for the sodium current is
\begin{equation}
  i_{Na}=m^{3}h\overline{g_{Na}}\pbrac{V_{m} - E_{Na}}
\end{equation}
and the sodium gate voltage relationship for a voltage step is shown in
\figref{fig:sodiumgatevoltstep}.

\pstexfigure{biophysics/figs/sodiumgatevoltstep.pstex}{}{Behaviour of the
  sodium gates (and hence channel conductance) for a voltage clamp step
  experiment.}{fig:sodiumgatevoltstep}{}

It should be noted that the mechanisms for the shutting off of the sodium and
potassium channels are different. For the potassium channels the process is
called \emph{deactivation}. This is characterised by the gates closing only
when the membrane potential drops. For the sodium channels the process is
called \emph{inactivation}. This is characterised by an inactivation gate that
closes when the potential rises. Thus the channel closes even when the
membrane potential does not drop. These two processes are shown in
\figref{fig:de-inactivation}.

\pstexfigure{biophysics/figs/de-inactivation.pstex}{}{(a) Deactivation and (b)
  inactivation of a channel with membrane potential
  changes.}{fig:de-inactivation}{}

Putting the \ion{Na}{+} and \ion{K}{+} channels together we obtain the nerve
\emph{action potential} shown in \figref{fig:squidactionpot}.

\pstexfigure{biophysics/figs/squidactionpot.pstex}{}{The squid action
  potential with the
  sodium and potassium channel conductances.}{fig:squidactionpot}{}

\subsubsection{Propagating action potentials}

When this process happens at one end of a nerve we get a \emph{propagating}
action potential. By raising the membrane potential of one end of the nerve
(or by injecting current) we generate an axial current along the nerve from
the high potential end to the low potential end in accordance with Ohm's law.
This current raises the membrane potential (depolarises) to a point whereby
the sodium channels open.  The point is known as the \emph{threshold voltage}.
Once the sodium channels open the sodium ions flood in and cause an inward
current which raises the membrane potential even higher. This causes further
axial current and a resultant action potential is hence propagated down the
nerve. This process is shown in \figref{fig:propactionpot}. It should be noted
that this process requires ion pumps in the membrane to restore the intra- and
extra-cellular ion concentrations to their resting levels. These pumps require
energy (ATP - adenosine tri-phosphate) to operate. Until the ionic
concentrations in an area of nerve have been returned to their resting states
an action potential will not propagate in this area of nerve. This period
after an action potential has passed when the nerve cannot be activated is
called the \emph{refractory period}.

\pstexfigure{biophysics/figs/propactionpot.pstex}{}{Propagating action
  potential. (a) The nerve is initially at rest at -'ve membrane potential
  with the cell membrane permeable to \ion{K}{+}.  (b) The membrane potential
  is raised (or current is injected into the cell) at the left hand end which
  depolarises (raises the membrane potential of) the cell. (c) The raised
  membrane potential opens the \ion{Na}{+} $m$ gates at the threshold
  potential and further raises the membrane potential to further increase the
  axial current. (d) the action potential is propagated via local current
  loops.}{fig:propactionpot}{}

\subsubsection{The Hodgkin-Huxley equations}

%\enlargethispage{\baselineskip}
The work of Hodgkin and Huxley\footnote{Alan Hodgkin, Andrew Huxley and
  Sir John Eccles were awarded the 1963 Noble prize for Physiology or Medicine
  for the study of the transmission of nerve impulses along a nerve fibre.} can
be summarised in the Hodgkin-Huxley equations for the action potential.
\begin{equation}
  \begin{split}
    \dfrac{1}{r_{a}}\deltwosqby{V_{m}}{x} &= C \delby{V_{m}}{t} + i_{K} 
    + i_{Na} \\
    i_{K} &= n^{4}\overline{g_{K}}\pbrac{V_{m} - E_{K}} \\
    i_{Na} &= m^{3}h\overline{g_{Na}}\pbrac{V_{m} - E_{Na}} \\
    \dby{n}{t} &= \alpha_{n}\pbrac{1-n}+\beta_{n}n \\
    \dby{m}{t} &= \alpha_{m}\pbrac{1-m}+\beta_{m}m \\
    \dby{h}{t} &= \alpha_{h}\pbrac{1-h}+\beta_{h}h
  \end{split}
  \label{eqn:hodgkin-huxley}
\end{equation}
where $\alpha_{n},\beta_{n},\alpha_{m},\beta_{m},\alpha_{h}$ and $\beta_{h}$
are all voltage dependent.

\section{Muscle}

There are three main kinds of muscle found in the body, namely skeletal
muscle, cardiac (heart) muscle and smooth muscle (used, for example, in the
constriction of the arteries).

\subsection{Anatomy}

\subsubsection{Gross anatomy}

The gross structure of skeletal muscle is best though of in terms of a
hierarchy. From the organ down to the cellular level there are three levels in
this hierarchy. These levels are defined by connective tissue (made up of
mainly collagen) that encases each muscle level. The first level is that of the
whole \emph{muscle organ} which is encased by the \emph{epimysium}. The next
level down is called the \emph{fascicle} of the muscle and is encased by the
\emph{perimysium}. The final level is that of the \emph{muscle fibre} or
muscle cell. This level is encased by the \emph{endomysium}. This hierarchy is
shown in \figref{fig:smgrossanatomy}.

\incgrfigure{width=15cm}{biophysics/epsfiles/smgrossanatomy.eps}{}{Gross
  anatomy of skeletal muscle. (a) The hierarchy of internal muscle structures
  and connective tissue that encloses them. (b) Photomicrograph of a cross
  section of skeletal muscle (64 x). From Human Anatomy and Physiology by E.
  N.  Marieb.}{fig:smgrossanatomy}

\subsubsection{Cellular anatomy}
The muscle cell differs a little from the idealised cells of the axons or
nerves. Firstly the terminology applied to muscle cells is a little different.
In muscle cells the cell membrane is known as the sarcolemma. This word
comes from the Greek base word \emph{sarc-} which means flesh. A second
difference is that muscle cells are \emph{multinuclear}, which means they have
more than one nucleus per cell.

\enlargethispage{-\baselineskip}
\enlargethispage{-\baselineskip}
\enlargethispage{-\baselineskip}
\enlargethispage{-\baselineskip}
The anatomy of the muscle is organised in another hierarchy. Each muscle
fibre is made up of a number of smaller units called \emph{myofibrils}. These
myofibrils are in turn made up two different proteins called
\emph{myofilaments}. The two proteins are distinguished by their size and
called the thick and thin filaments. The thick filament is made up of a
protein called \emph{myosin}. The thin filament is made up predominately of a
protein called \emph{actin}. This hierarchy is shown in
\figref{fig:smcellanatomy}\nocite{marieb:1992}.

\incgrfigure{width=15cm}{biophysics/epsfiles/smcellanatomy.eps}{}{Anatomy of
  a muscle cell (fibre). (a) Photomicrograph (x 250) (b) Diagram of muscle
  fibre (c) Portion of a myofibril (d) Enlargement of a sarcomere (e)
  Arrangement of the thick and thin filaments. From Human Anatomy and
  Physiology by E. N.  Marieb.}{fig:smcellanatomy}

Within each myofibril a further part of muscle can be identified. The proteins
actin and myosin are arranged in a regular fashion both longitudinally and
radially throughout the myofibril. In the longitudinal direction these
filaments are arranged so that they overlap in a regular fashion. When viewed
under the microscope this regular overlap generates a series of light and dark
bands. This pattern of light and dark bands gives muscle a striped appearance.
Both skeletal and cardiac (but not smooth) muscle exhibit this striped pattern
and are hence known as \emph{striated} (meaning striped) muscle. These
striations are shown in \figref{fig:smcellanatomy}a.

Looking more closely at each band a line can be seen in the middle of
each band. The line is named the $Z$-line in the light (or $I$-band) and the
$M$-line in the dark (or $A$-band). As these structure are regular and
repeating a further part of muscle can be identified. This unit is known as
the \emph{sarcomere} and is defined as the unit of muscle between two
consecutive $Z$-lines. The typical sarcomere length in resting skeletal muscle
is $2.2$ \um. These bands, lines and the sarcomere are shown in
\figref{fig:smcellanatomy}c--e.

The complete anatomical hierarchy of muscle is shown in
\figref{fig:smorganisation}.

\incgrfigure{height=18cm}{biophysics/epsfiles/smorganisation.eps}{}
{Organisation of skeletal muscle. From Human Anatomy and Physiology by E. N.
  Marieb.}{fig:smorganisation}

\subsubsection{The myofilaments}

We will now look in more detail at the individual myofilaments as the
structures at this level are important when we look at how muscle contracts in
\secref{sec:excitationcontraction}. The thick filament is made of a number of
myosin molecules. Each myosin molecule consists of a long ``tail'' and a large
``head'' as shown in \figref{fig:myofilaments}a. These heads are also known as
\emph{cross-bridges}, the meaning of which will become clear later.

\incgrfigure{height=18cm}{biophysics/epsfiles/smfilaments.eps}{}{Structure
  of the myofilaments. From Human Anatomy and Physiology by E. N. Marieb.}
{fig:myofilaments}


Each myosin molecule is about $12$ \nm in diameter and about $1.6$ \um long.
Groups of myosin molecules arrange themselves so that in the longitudinal
direction of the thick filament the molecules are tail to tail and in the
radial direction the heads are in a helical pattern with a pitch of $42.9$ \nm.
This is shown in \figref{fig:myofilaments}b.

The thin filament is more complicated than the thick filament in that it
contains three main components. The main component is actin. Actin is a
protein that forms a double stranded helix. The overall molecule is about $5$
\nm in diameter and extends $1$ \um either side of the $Z$-line. Within the
groove defined by the double stranded actin helix runs the second component of
the thin filament, a molecule called \emph{tropomyosin}. The final component
of the thin filament is a complex called the \emph{troponin} complex. This
complex is attached along the tropomyosin molecule at regular intervals. The
arrangement of the thin filament is shown in \figref{fig:myofilaments}c and
the overall arrangement of the myofibril in \figref{fig:myofilaments}d.

\subsection{Properties of muscle contraction}
\label{sec:musclecontraction}

\subsubsection{Twitch and tetanus}

When a muscle is stimulated (electrically) it contracts and generates
force. If the muscle is fixed in such a way as then ends cannot move the
muscle is said to be contracting \emph{isometrically}, that is contracting at
constant length. Another mode of contraction is if the muscle contracts
against a constant load. Here the muscle is said to be contracting
\emph{isotonically}, that is contracting at constant tension.

For isometric contraction the way force is developed depends on the mode of
stimulation. A single stimulus produces a \emph{twitch}, in which tension
increases rapidly and then declines. If a second stimulus is applied before
the first twitch has fallen to zero, the peak tension for the second twitch is
higher than that of the first. As the frequency of stimulation is increased
the tension generated rises smoothly and the muscle is said to be in
\emph{tetanus}.  This is shown in \figref{fig:twitchtetanus}.

\pstexfigure{biophysics/figs/twitchtetanus.pstex}{}{Isometric force generation
  in response to stimulus.}{fig:twitchtetanus}{}

The critical frequency for tetanus is about $10$ stimulations per second. The
maximum force developed in tetanus is about $2$--$3$ times greater than that
of a twitch.

\subsubsection{Isometric force-length relationship}

For isometric contraction an experiment can be performed whereby the amount of
force a muscle generates is measured for a number of different lengths of
muscle.  The results for this experiment are shown in the top solid line of
\figref{fig:isometricforcelength}.

\pstexfigure{biophysics/figs/isometricforcelength.pstex}{}{Isometric force
  generation as a result of muscle length.}{fig:isometricforcelength}{}

In addition to this passive muscle (that is muscle that is not contracting)
resists extension beyond the normal resting length for the muscle. It is hence
possible in this experiment to also measure the passive force-length behaviour
of muscle. This is shown in the bottom solid line in
\figref{fig:isometricforcelength}. By subtracting the passive tension from the
total tension we can estimate the amount of tension developed in the muscle
through contraction. This is the dashed line in
\figref{fig:isometricforcelength}. As can be seen this active tension is
maximum for lengths close to the resting length and declines if the muscle
length is either increased or decreased.

\subsubsection{Isometric quick release}

When a muscle in isometric tetanus is suddenly shortened and clamped at a new
fixed length, the tension falls abruptly and sub-sequentially rises to a new
maximum level determined by the isometric force-length relationship.

\pstexfigure{biophysics/figs/isometricquickrel.pstex}{}{Isometric quick release
  experiment.}{fig:isomemtricquickrel}{}

The extent of the shortening (or length step) determines the extent of the
initial fall in tension and, in part, the time course of the subsequent
recovery.

\subsubsection{Isotonic force-velocity relationships}

In an isotonic contraction the muscle is allowed to shorten against a constant
force or load. This experiment is normally performed with the muscle attached
to one end of a lever with a fixed amount of weight on the other end of the
lever. The muscle is then allowed to contract at constant length (i.e.
initially isometric contraction) and generate force. When the force generated
is equal to the applied weight (known as the afterload) the muscle begins to
shorten. It has been found that the velocity of this shortening is constant
for a given afterload and that the magnitude of this velocity depends on the
magnitude of the afterload. This is shown in \figref{fig:isotonic}.

\pstexfigure{biophysics/figs/isotonic.pstex}{}{Isotonic force-length
  relationship.}{fig:isotonic}{}

An experiment can be done whereby the velocity of shortening is measured for a
varying amount of afterload. The results of this experiment are known as the
isotonic force-velocity relationship and are shown in
\figref{fig:isotonicforcevel}. This curve has a characteristic hyperbolic
shape. The maximum force generated occurs when the muscle is contracting
isometrically (hence at zero velocity) and is denoted $T_{0}$. The maximum
velocity occurs when there is no after-load and is denoted $V_{\text{max}}$
(experimentally this is found from extrapolating the curve from low values of
afterload).

\pstexfigure{biophysics/figs/isotonicforcevel.pstex}{}{Isotonic force-velocity
  relationship.}{fig:isotonicforcevel}{}

\Figref{fig:isotonicforcevel} represents what we would expect from experience
with our own muscles. Our muscles are designed to lift small loads quickly or
large loads slowly.

\subsubsection{Isotonic quick release}

With this technique the muscle is first caused to contract isometrically and
is then abruptly released to shorten against constant afterload. Typical
results are shown in \figref{fig:isotonicquickrel}.

\pstexfigure{biophysics/figs/isotonicquickrel.pstex}{}{Isotonic quick release
  experiment.}{fig:isotonicquickrel}{}

One release, the tension falls rapidly to the level of the new
afterload. Initially there is an abrupt shortening which coincides with the
change in tension. Subsequentally, a slower isotonic shortening occurs at a
velocity which is inversely related to the new afterload.

\subsection{Hill model of muscle contraction}

\subsubsection{Hill's three component model}

\citet{hill:1938} proposed a simple mechanical model that was empirically
based.  Hill's three component model lumps the mechanical behaviour of muscle
into three elements:
\begin{enumerate}
\item \textbf{The parallel elastic element}: This represents the potential
  energy stored in relaxed skeletal muscle when it is stretched beyond its
  resting length. It is assumed that this element represents the passive
  elastic structures such as connective tissue.
\item \textbf{The series elastic element}: This is represented as an undamped
  spring. It was assumed that the length of this element is dependent only on
  the force in the muscle and is therefore independent of the kinetics of
  contraction.
\item \textbf{The contractile element}: In this element all the active features
  of contraction are lumped in a ``black-box'' approach. It is assumed that
  this component was freely extensible in the relaxed state and that is was
  ``switched on'' at the onset of contraction. Once fully activated the force
  generated by this contractile element was assumed to be dependent only on its
  instantaneous length and velocity and independent of time.
\end{enumerate}

This model is shown in \figref{fig:Hillmodel}.

\pstexfigure{biophysics/figs/Hillmodel.pstex}{}{Hill model of
  muscle}{fig:Hillmodel}{}

A.V. Hill proposed the following relationship for the force-velocity
relationship for contraction.
\begin{equation}
  \pbrac{V+b}\pbrac{T+a}=b\pbrac{T_{0}+a}
  \label{eqn:Hillforcevelocity}
\end{equation}
where $V$ is the velocity of shortening, $T$ the tension, $T_{0}$ the
isometric tension (at zero velocity) and $a$ and $b$ are parameters. This
equation can be rewritten as
\begin{equation}
  V=\dfrac{b\pbrac{T_{0}-T}}{T+a}
  \label{eqn:Hillvelocity}
\end{equation}
From experiment it has been found that $\dfrac{a}{T_{0}}=0.28$ and $b=0.4$
muscle lengths/second $\sim 12$ \mmps. Hence the maximum velocity of
shortening can be found at the point where $T=0$, i.e.
\begin{equation}
  V_{0}=V_{\text{max}}=\dfrac{bT_{0}}{a}=\dfrac{0.4}{0.28}\sim 1.43 l_{0}/s
\end{equation}
where $l_{0}$ is the resting length of the muscle.

\subsubsection{Limitations of Hill's model}

Hill's model for muscle formed the basis of the understanding of muscle for
about twenty years. It was eventually discarded because the predictions of the
model and experiment differed for two key experiments:
\begin{enumerate}
\item The Hill model failed to predict accurately the time course of the rise
  in force during isometric contraction.
\item The Hill model does not predict the transients following quick isometric
  and isotonic release.
\end{enumerate}

\subsection{Sliding filament theory}

\subsubsection{Sarcomere force-length relation}

In 1954 H.E. Huxley observed that during contraction the size of the $I$-band
diminishes.  They proposed that this could be explained by allowing the
relative sliding of the thick and thin filaments. The myofilaments, themselves
were found not to shorten. This process explains the sarcomere force-length
relation shown in the top part of \figref{fig:sarcoforcelength}. This theory
was known as the \emph{sliding filament theory}.

\pstexfigure{biophysics/figs/sarcoforcelength.pstex}{}{Sarcomere force-length
  relationship.}{fig:sarcoforcelength}{}

The shape of the curve in \figref{fig:sarcoforcelength} can be explained by
the different degree of interaction between the thick and thin filaments (also
shown schematically in the bottom part of \figref{fig:sarcoforcelength}). At
point (a) no force is generated in the sarcomere as there is no interaction
between the thick and thin filaments. At point (b) there is maximum force as
there is maximum interaction. This continues up to point (c) when the thin
filaments start to interact. At point (d) there is steric interference of
the thin filaments limiting the interaction between them and the thick
filaments. At point (e) there is interference of both the thick and thin
filaments resulting in no force production.

\subsubsection{Excitation-Contraction of muscle}
\label{sec:excitationcontraction}

We will now look in a bit more detail as to how muscle contracts in response
to a stimulus. First consider the schematic of the muscle cell as shown in
\figref{fig:smsarcolemma}.

\incgrfigure{width=15cm}{biophysics/epsfiles/smsarcolemma.eps}{}{The muscle
  cell showing cell membrane details. From Human Anatomy and Physiology by E.
  N. Marieb.}{fig:smsarcolemma}

What should be noted here is the network of ``tubes'' called the
\emph{sarcoplasmic reticulum} just under the sarcolemma (cell membrane). This
sarcoplasmic reticulum is ``connected'' to the sarcolemma through the
T-tubules (or transverse tubules) at the \emph{terminal cisterna}. This is
shown schematically in \figref{fig:excitationcontract}.

\pstexfigure{biophysics/figs/excitcontract.pstex}{}{Excitation contraction
  coupling in skeletal muscle.}{fig:excitationcontract}{}

The method of contraction for a muscle twitch is then as follows: An action
potential propagates along the sarcolemma from a neuro-muscular junction and
down the T-tubule to the terminal cisterna of the sarcoplasmic reticulum.  The
change in the membrane potential from the action potential opens voltage
dependent gates and allows a small amount of \ion{Ca}{2+} ions to enter the
sarcoplasmic reticulum. This triggers a much greater release of \ion{Ca}{2+}
into the sarcoplasm (the inside of the cell).  Calcium then interacts with
troponin in the thin filament which then undergoes steric rearrangement.  This
rearrangement allows a myosin head (cross-bridge) to bind to actin in a
reaction that causes the muscle to contract. At the same time the muscle is
contracting the intra-cellular calcium also activates pumps in the
sarcoplasmic reticulum.  These pumps take the calcium out from the cell and
back into the sarcoplasmic reticulum. As the level of intra-cellular calcium
drops the cross-bridges can not re-attach once they (eventually) detach and the
muscle relaxes.

\subsection{Muscle energetics}

\subsubsection{Cross-bridge cycling}

When muscle contracts it does work. This work hence requires a supply of
energy. The main ``currency'' of energy in the cell is adenosine
tri-phosphate or ATP. The equilibrium equation by which ATP generates energy
is 
\begin{equation}
  \text{ATP} \rightleftharpoons \text{ADP} + \text{P} + \text{energy} 
\end{equation}
where P is free phosphate and ADP is adenosine di-phosphate. 

As explained in \secref{sec:excitationcontraction} a muscle twitch occurs
when a myosin cross-bridge attaches to actin and stops when the cross-bridge
detaches. For a tetanised muscle, however, the story is a slightly
different. During tetanus the cross-bridges are continually attaching,
detaching and then re-attaching. This process is known as \emph{cross-bridge
  cycling} and is shown in \figref{fig:smxbridgecycle}.

\incgrfigure{width=15cm}{biophysics/epsfiles/smxbridgecycle.eps}{}
{Cross-bridge energy cycle. From Human Anatomy and Physiology by E. N.
  Marieb.}{fig:smxbridgecycle}

The cross-bridge cycle starts at point (1) in \figref{fig:smxbridgecycle} when
the myosin cross-bridge (with ADP and P) attaches to actin. At this point the
myosin-head is in it's \emph{high-energy} configuration. The myosin head then
rotates to its \emph{intermediate-energy} configuration causing the actin
molecule to slide resulting in contraction. As the myosin head rotates the
myosin tail is stretched generating force. ADP and P are released from the
myosin head during this process.  This is point (2) in
\figref{fig:smxbridgecycle}.  ATP now attaches to the myosin head causing the
cross-bridge to deattach from the actin molecule.  The myosin molecule is now
it's \emph{low-energy} configuration as shown in point (3). Hydrolysis of the
ATP bound to the myosin head now occurs and the cross-bridge cycle moves to
point (4). The cycle is completed when the cross-bridge re-attaches to the
actin molecule. Recalling \figref{fig:isotonicforcevel} we can now interpret
$V_{0}$ as the maximum turnover rate of cross-bridges.

It should be noted that energy (ATP) is required for \emph{relaxation} not
contraction of muscle. This is the reason why \emph{rigor mortis} occurs on
death. With death the body ceases the manufacture of ATP. As the stores of
ATP run down muscles that are contracted cannot be relaxed and hence the
muscle stays in its contracted state.

\subsubsection{Muscle Power}
We will return now to Hill's model of contraction to consider muscle power.
Recall \Eqnref{eqn:Hillvelocity} for velocity, i.e.
\begin{equation*}
  V=\dfrac{b\pbrac{T_{0}-T}}{T+a}
\end{equation*}
Now power=force x velocity, hence
\begin{equation}
  P=Tb\dfrac{T_{0}-T}{T+a}
\end{equation}
Maximum power will occur when $\dby{P}{T}=0$, therefore
\begin{equation*}
  \begin{split}
    0 &= b\dfrac{T_{0}-T}{T+a}+Tb\dfrac{\pbrac{T+a}.1-\pbrac{T_{0}-T}}
    {\pbrac{T+a}^{2}} \\
    &= \dfrac{b\pbrac{T+a}{T_{0}-T}-T\pbrac{T_{0}+a}}{\pbrac{T+a}^{2}}
  \end{split}
\end{equation*}
\begin{equation*}
  \Rightarrow -T^{2}-2aT+aT_{0}=0 \quad \text{or} \quad T^{2}+2aT-aT_{0}=0
\end{equation*}
hence
\begin{equation*}
  \begin{split}
    T &= \dfrac{-2a \pm \sqrt{4a^{2}-4aT_{0}}}{2} \\
    &= \sqrt{a^{2}-aT_{0}}-a \\
    &= 0.32T_{0} \quad \text{with}\quad \dfrac{a}{T_{0}}=0.28
  \end{split}
\end{equation*}
that is maximum power occurs when $T \approx \dfrac{1}{3}T_{0}$ and similarly
maximum velocity occurs when $V \approx \dfrac{1}{3}V_{0}$.

\subsubsection{Heat production}

Consider now the heat produced during isometric tension, as shown in
\figref{fig:isometricheat}.

\pstexfigure{biophysics/figs/isometricheat.pstex}{}{Heat production during
  isometric tension.}{fig:isometricheat}{}

Heat production in muscle can be broken into the following components.
\begin{enumerate}
\item \textbf{Heat production due to isometric contraction}: The heat production
  during isometric tetanus can be further broken into fast and slow, or
  \emph{labile} and \emph{stable} components. Stable heat rate is maximum at
  the normal resting length of the muscle and falls either side of this length
  (i.e. the same trend and the force-length relationship). Stable heat
  production is due to ATP splitting by two-processes: actin-myosin
  interaction (which is proportional to filament overlap) and calcium pumping
  by the sarcoplasmic reticulum. Labile heat, on the other hand, is
  independent of length and hence it is not associated with actin-myosin
  interaction. It is thought labile heat production is due to processes
  associated with the initiation of contraction.
\item \textbf{Heat production due to activation}: When muscle is stretched to a
  length such that there is no actin-myosin interaction stable heat production
  is not zero (in fact it is about 30\% of the stable heat production at
  resting length). This non-zero stable heat production can not be due to
  actin-myosin interaction (as there is no filament interaction) and is
  thought to be due to activation and excitation-contraction coupling.
\item \textbf{Heat production due to shortening}: A.V. Hill\footnote{Archibald
    Hill was awarded the 1922 Nobel prize for Physiology or Medicine for the
    discovery relating to heat production in muscles.} showed that if a muscle
  is allowed to shorten more heat is produced than if the muscle was held at a
  fixed length. Hill showed that this \emph{shortening heat} is dependent only
  on the extent of shortening and is independent of the time during which
  shortening occurs, and the load under which the shortening took place. This
  extra shortening heat is due to the increased velocity of the muscle which
  increases the turnover of cross-bridges and hence increases the amount of
  ATP split and energy/heat produced.
\item \textbf{Relaxation heat}: As shown in \figref{fig:isometricheat} as force
  declines at the termination of a tetanus heat production falls then rises
  again to a shortlived peak. This secondary rise in heat production is termed
  \emph{relaxation heat} and is thought to be due to: (1) dissipation in the
  muscle of work performed on it by external series elasticity; (2)
  dissipation of work on one region of the muscle as it extended by shortening
  in another region; (3) thermoelastic effects.
\item \textbf{Recovery heat}: Hill described this heat around 1922. He showed
  that the initial heat from contraction was followed by a \emph{recovery
    heat} of similar magnitude over the next $20$--$30$ minutes. It is generally
  accepted that this recovery heat is associated with the resynthesis of
  phosphocretaine and restoration of glycogen stores. For muscles which are
  specialised for sustainted activity (e.g. heart muscle) initial and recovery
  heat occurs almost synchronously and cannot be distinguished.
\item \textbf{Resting heat production}: The heat produced by resting (passive)
  muscle is appreciable with the resting heat production of skeletal muscle
  accounting for half of the total body basal metabolic rate. Further, the
  rate of heat production for cardiac muscle is five times greater than that
  of an equivalent mass of skeletal muscle. Resting heat production is
  increased with a muscle is stretched. This is known as the \emph{Feng}
  effect. It is unclear the exact cause of this resting heat but the activity
  of the background sodium-potassium pump has been shown to contribute only a
  small amount to this resting heat production. It is argued that resting heat
  production may be due to protein synthesis.
\end{enumerate}

\section{Fading Memory Model}

We will now move away from biophysical models of muscle (e.g.
\citet{huxley:1957} or \citet{huxley:1971}) and investigate a
continuum model of muscle mechanics. For the moment we will only be concerned
with tetanised muscle.

Under steady state conditions muscle has a tension-length relationship as
shown in \figrefs{fig:isometricforcelength}{fig:sarcoforcelength}. We define
now the length ratio $\lambda$ as the actual muscle length divided by the
resting length i.e. $\lambda$ is $1.0$ at the muscle resting length. We will
derive the model from the point of view of a isometric quick release
experiment i.e. a change in the length of muscle whilst at steady state.

Let $\fnof{Q}{T,T_{0}}$ be some, as yet unspecified, non-linear function of
$T,T_{0}$ such that $\fnof{Q}{T_{0},T_{0}}=0$. Consider now the effect of a
change in length, $\Delta\lambda$, at a time $\tau (< t)$ on the current
response at time $t$. The change in $Q$ (at time $t$) resulting from this length
step (at time $\tau$) is
\begin{equation*}
  \begin{split}
    \Delta Q &= \fnof{\Phi}{t-\tau}\Delta\lambda \\
    &= \fnof{\Phi}{t-\tau}\dby{\fnof{\lambda}{\tau}}{t}dt
  \end{split}
\end{equation*}
where $\fnof{\Phi}{t}$ is the material response function, that is
$\fnof{\Phi}{t-\tau}$ is the response at time $t$ to a stimulus at time
$\tau$. Integrating we have
\begin{equation}
  \fnof{Q}{t}=\gint{-\infty}{t}{\fnof{\Phi}{t-\tau}\fnof{\dot{\lambda}}{\tau}}
  {t}
  \label{eqn:hereditaryintegral}
\end{equation}
This type of integral is known as a hereditary integral.

Now if we assume that the overall response depends more on recent events that
earlier events --- that is the material has a ``fading memory'' --- and if we
assume superposition (i.e. linearity) it is customary to put the material
responds function to be
\begin{equation}
  \fnof{\Phi}{t-\tau}=\dsum_{i=1}^{N}A_{i}e^{-\alpha_{i}t}
\end{equation}
where $\alpha_{i}$ are the rate constants for the
material. \Eqnref{eqn:hereditaryintegral} now becomes
\begin{equation}
  \fnof{Q}{\fnof{T}{t},T_{0}}=\dsum_{i=1}^{N}A_{i}\gint{-\infty}{t}{
    e^{-\alpha_{i}\pbrac{t-\tau}}\fnof{\dot{\lambda}}{\tau}}{\tau}
\end{equation}
We need to use experimental results to determine what form $Q$ has, how many
$N$'s are needed and what the values of $A_{i}$ and $\alpha_{i}$ are. There
are four types of tests that can be used to reveal $Q,\alpha_{i}$ and $A_{i}$:
\begin{enumerate}
\item Step changes in length --- relaxation test (transient response test)
\item Step changes in tension --- creep test (transient response test)
\item Frequency response test
\item Wave propagation test
\end{enumerate}

As the length step is the most common in the literature for single fibre
experiments we will use this to define the model then we can use other tests to
verify the model. 

Consider two length step experiments as shown in \figref{fig:fmlengthstep}.
\pstexfigure{biophysics/figs/fmlengthstep.pstex}{}{Two step-length experiments.}
{fig:fmlengthstep}{}
By using a Heavyside and Dirac delta functions we can put
$\lambda=\Delta\lambda \fnof{H_{0}}{\tau}+\lambda_{0}$ and
$\dot{\lambda}=\Delta\lambda\fnof{\delta}{\tau}$. 

Substituting this into
\Eqnref{eqn:hereditaryintegral} we obtain
\begin{equation*}
  \fnof{Q}{\fnof{T}{t},T_{0}}=\Delta\lambda\dsum_{i=1}^{N}A_{i}\gint{-\infty}{t}
  {e^{-\alpha_{i}\pbrac{t-\tau}}\fnof{\delta}{\tau}}{\tau}
\end{equation*}
Now defining $t=0$ as the instance of the length step we have
\begin{equation*}
  \fnof{Q}{\fnof{T}{t},T_{0}}=\Delta\lambda\dsum_{i=1}^{N}A_{i}e^{-\alpha_{i}t}
\end{equation*}

At $t=0$ (immediately after the step) $T=T_{1_{j}}$, that is minimum tension
is reached. Hence
\begin{equation}
  \fnof{Q}{T_{1_{j}},T_{0}}=\Delta\lambda_{j}\dsum_{i=1}^{N}A_{i}
\end{equation}

Experimentally we find the result as shown in \figref{fig:fmresults}.

\pstexfigure{biophysics/figs/fmresults.pstex}{}{Experimental results of step
  length experiments.}{fig:fmresults}{}

It is found that all these curves superimpose if the isometric points are 
the same (that is all curves are scaled by $T_{0}$). This is a consequence of
the number of available cross-bridges at different lengths (i.e. each
crossbridge is an independent force generator so tension is proportional to
the number of active cross-bridges). We can therefore put
\begin{equation*}
  \fnof{Q}{T_{1},T_{0}}=\fnof{Q}{\dfrac{T_{1}}{T_{0}}}
\end{equation*}
From the experimental data we find we can fit the results with
\begin{equation*}
  \fnof{Q}{T_{1},T_{0}}=\dfrac{T_{1}/T_{0}-1}{T_{1}/T_{0}+a}
\end{equation*}
where $a$ is a parameter (the curvature of the curve). Hence the general form
for the $Q$ function is
\begin{equation}
  \fnof{Q}{T,T_{0}}=\dfrac{T/T_{0}-1}{T/T_{0}+a}  
  \label{eqn:Qdefinition}
\end{equation}
Fitting to data from Huxley and Simmons we find that $a=2$ and $\dsum
A_{i}=81$ (both non-dimensional). 

With $Q$ defined in \Eqnref{eqn:Qdefinition}, the complete tension recovery
following the length step is given by
\begin{equation}
  \dfrac{T/T_{0}-1}{T/T_{0}+a} =\Delta\lambda\dsum_{i=1}^{N} A_{i}
  e^{-\alpha_{i}t}
  \label{eqn:tensionrecovery}
\end{equation}
Consider now a length step of $\Delta\lambda=-0.007$ (i.e. a $0.7$\%
shortening). The results of this experiment are shown in
\figref{fig:fmtimeresponse}.

\pstexfigure{biophysics/figs/fmtimeresponse.pstex}{}{Time response of tension
  recovery following a length step.}{fig:fmtimeresponse}{}

From \figref{fig:fmtimeresponse} the slow recovery is modelled by
\Eqnref{eqn:tensionrecovery} with $\alpha_{1}=32$ \mps and $A_{1}=17$.
The slow recovery phase is associated with cross-bridge cycling. This gives
good agreement with the experimental results for $t>20$ \ms.

For $t<20$ \ms we need a $2^{\text{nd}}$ order process (i.e.
$\alpha_{2},\alpha_{3},A_{2}$ and $A_{3}$), with $\alpha_{2}=1$ \mps,
$A_{2}=26$, $\alpha_{3}=5$ \mps and $A_{3}=42$. This fast recovery phase is
associated with attached cross-bridge head dynamics (head rotation etc.).  The
final model is hence
\begin{equation}
  \dfrac{T/T_{0}-1}{T/T_{0}+a}=\dsum_{i=1}^{3} A_{i}\gint{-\infty}{t}
  {e^{-\alpha_{i}\pbrac{t-\tau}}\fnof{\dot{\lambda}}{\tau}}{\tau}
  \label{eqn:fadingmemory}
\end{equation}
This model gives good agreement with experiment for a large range of
$\Delta\lambda$ (but not too large or fast).

\subsection{Finite duration length step}

In practice the time take for the tension to fall to $T_{1}$ following a
length step is not zero. We will consider now the case when the muscle takes a
finite duration ($\Delta t$) for the tension to drop. The velocity of
shortening is now given by
\begin{equation*}
  \dot{\lambda}=\begin{cases} 
        0 & t < 0 \\
        \Delta\lambda/\Delta t & 0 < t < \Delta t \\
        0 & t > \Delta t
      \end{cases}
\end{equation*}
\Eqnref{eqn:fadingmemory} now becomes
\begin{equation*}
  \begin{split}
    \dfrac{T/T_{0}-1}{T/T_{0}+a} &= \dsum_{i=1}^{3} A_{i}
    \gint{-\infty}{\Delta t}{e^{-\alpha_{i}\pbrac{t-\tau}}
      \dfrac{\Delta\lambda}{\Delta t}}{\tau} \\
    &=\Delta\lambda\dsum_{i=1}^{3}A_{i}e^{-\alpha_{i}t}\dfrac{\pbrac{
        e^{\alpha_{i}\Delta t}-1}}{\alpha_{i}}\Delta t
  \end{split}
\end{equation*}
At $t=\Delta t$ this becomes
\begin{equation}
  \dfrac{T/T_{0}-1}{T/T_{0}+a}=\Delta\lambda\dsum_{i=1}^{3}A_{i}
  \dfrac{\pbrac{1-e^{-\alpha_{i}\Delta t}}}{\alpha_{i}}\Delta t
\end{equation}
The effects of $\Delta t$ on the tension response are shown in
\figref{fig:fmfinitedurationresponse}.

\pstexfigure{biophysics/figs/fmfinitedurationresponse.pstex}{}{Effects of the
  time taken for a length step.}{fig:fmfinitedurationresponse}{}

\subsection{Force step response}
\enlargethispage{\baselineskip}
Consider now the response to a force step. For the moment we will consider the
case were we are initially at the plateau of the force-length relationship
(i.e. $\lambda\approx$1 and $\fnof{T}{\lambda}\approx$constant). The
experimental results are shown in \figref{fig:fmforcestepresponse1}.

\pstexfigure{biophysics/figs/fmforcestepresponse1.pstex}{}{Effects of the time
  taken for a force step.}{fig:fmforcestepresponse1}{}

Consider one time constant only (i.e. $N$=1) and the region where
$\dot{\lambda}$ constant. Hence
\begin{equation}
  \begin{split}
    \dfrac{T/T_{0}-1}{T/T_{0}+a} &= A_{1}\gint{-\infty}{t}
    {e^{-\alpha_{1}\pbrac{t-\tau}}\dot{\lambda}}{\tau} \\
    &=A_{1}e^{-\alpha_{1}t}\dot{\lambda}\gint{-\infty}{t}
    {e^{\alpha_{1}\tau}}{\tau} \\
    &=\dfrac{A_{1}}{\alpha_{1}}\dot{\lambda}
  \end{split}
  \label{eqn:fmtransresponse1}
\end{equation}
This is shown in \figref{fig:fmforcevelocity1}.

\pstexfigure{biophysics/figs/fmforcevelocity1.pstex}{}{Force-velocity
  relationship for the fading memory model.}{fig:fmforcevelocity1}{}

Now if we let $\hat{a}=aT_{0}$ and $\hat{b}=-\dfrac{\alpha_{1}}{A_{1}}$ then
\Eqnref{eqn:fmtransresponse1} becomes
\begin{equation*}
  V=\dfrac{\hat{b}(T_{0}-T)}{T+\hat{a}}
\end{equation*}
which is the same as Hill's equation for velocity, \Eqnref{eqn:Hillvelocity}.

Consider now the case when isometric tension $\fnof{T_{0}}{\lambda}$ varies
with length in the region of interest (i.e. we are on the ascendening or
descending limb of the force-length relationship). We will consider only one
time constant (i.e. $N=1$) and start with
\begin{equation*}
   \dfrac{T/T_{0}-1}{T/T_{0}+a}=A_{1}\gint{-\infty}{t}{e^{-\alpha_{1}
       \pbrac{t-\tau}}\dot{\lambda}}{\tau}
\end{equation*}
We now differentiate with respect to $t$, holding $T$ constant to extract an
expression for $\dot{\lambda}$. To do this we must make use of \emph{Leibnitz's
  rule} for differentiating and integral with respect to one of its
limits. Leibnitz's rule is
\begin{equation*}
  \dby{ }{t}\bbrac{\gint{\fnof{g}{t}}{\fnof{h}{t}}{\fnof{f}{x,t}}{x}}=
  \gint{\fnof{g}{t}}{\fnof{h}{t}}{\delby{\fnof{f}{x,t}}{t}}{x}+
  \fnof{f}{\fnof{h}{t},t}\dby{h}{t}-\fnof{f}{\fnof{g}{t},t}\dby{g}{t}
\end{equation*}
hence
\begin{equation*}
  \dfrac{\pbrac{T/T_{0}+a}.1-\pbrac{T/T_{0}-1}.1}{\pbrac{T/T_{0}+a}^{2}}.
  \dfrac{-T}{T_{0}^{2}}\dby{T_{0}}{\lambda}\dot{\lambda}=\alpha_{1}
  \dfrac{T/T_{0}-1}{T/T_{0}+a} + A_{1}\dot{\lambda}
\end{equation*}
therefore
\begin{equation}
  \dot{\lambda}=\dfrac{\alpha_{1}\dfrac{T/T_{0}-1}{T/T_{0}+a}}
  {A_{1}+\dfrac{1+a}{\pbrac{T/T_{0}+a}^{2}}\dfrac{T}{T_{0}^{2}}
    \dby{T_{0}}{\lambda}}
  \label{eqn:fmtransresponse2}
\end{equation}
Note that \Eqnref{eqn:fmtransresponse2} collapses to
\Eqnref{eqn:fmtransresponse1} when $\dby{T_{0}}{\lambda}=0$. This result is
shown in \figref{fig:fmforcestepresponse2}.

\pstexfigure{biophysics/figs/fmforcestepresponse2.pstex}{}{Effects of the time
  taken for a force step.}{fig:fmforcestepresponse2}{}

It should be noted now that the force-velocity curve depends on $\lambda$ as
shown in \figref{fig:fmforcevelocity2}.

\pstexfigure{biophysics/figs/fmforcevelocity2.pstex}{}{Fading memory
  force-velocity relationship for different extension
  ratios.}{fig:fmforcevelocity2}{}
